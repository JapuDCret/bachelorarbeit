
\subsubsection{Funktionsumfang}


\begin{anf}{anf:2110}{2110}{Schnittstellen-Logging}{Basis.}{f.}{S}
\multicolumn{5}{|p{14.05cm}|}{Im Frontend ist das Aufrufen von Schnittstellen ist mittels einer Logmeldung zu notieren. Hierbei soll vor Aufruf geloggt werden, welche Schnittstelle aufgerufen wird und mit welchen Parametern. Nach dem Aufruf soll bei Erfolg das Ergebnisobjekt geloggt werden, und bei einem Fehler soll dieser notiert werden.} \\
\hline
\end{anf}

\begin{anf}{anf:2111}{2111}{Use-Case-Logging}{Basis.}{f.}{S}
\multicolumn{5}{|p{14.05cm}|}{Tritt im Frontend ein Use-Case auf, soll dieser im Log notiert werden. Beispielsweise soll notiert werden, wenn ein Nutzer das Absenden eines Formular initiiert.} \\
\hline
\end{anf}

\begin{anf}{anf:2120}{2120}{Übertragung von Logs}{Basis.}{f.}{S}
\multicolumn{5}{|p{14.05cm}|}{Logmeldungen des Frontends sind an ein \enquote{Log-Management}-Partnersystem weiterzuleiten.\par Dabei sind Logmeldungen ab dem Log-Level \enquote{DEBUG} und höher zu übertragen.} \\
\hline
\end{anf}

\begin{anf}{anf:2210}{2210}{Error-Monitoring}{Basis.}{f.}{S}
\multicolumn{5}{|p{14.05cm}|}{Wird ein nicht abgefangener Fehler im JavaScript-Kontext geworfen, so ist dieser automatisch zu erfassen und um weitere Attribute zu ergänzen.\par Abgefangene und behandelte Fehler können ebenso erfasst werden, jedoch ist hierbei keine automatische Erfassung gefordert.} \\
\hline
\end{anf}

\begin{anf}{anf:2220}{2220}{Übertragung von Fehlern}{Basis.}{f.}{S}
\multicolumn{5}{|p{14.05cm}|}{Sämtlich erfasste Fehler des Frontends sind an ein \enquote{Error-Monitoring}-Partnersystem weiterzuleiten.} \\
\hline
\end{anf}

\begin{anf}{anf:2310}{2310}{Tracing}{Basis.}{f.}{S}
\multicolumn{5}{|p{14.05cm}|}{Im Frontend sowie im Backend sind Tracing Spans zu erstellen, die Business-Methoden sowie Schnittstellenaufrufe umschließen. Bei einem Schnittstellaufruf sind die Informationen des Spankontextes über einen Traceheader zu übergeben, sodass die subsequent erstellten Spans hiermit assoziiert werden können.} \\
\hline
\end{anf}

\begin{anf}{anf:2311}{2311}{Tracing-Standard}{Leistungs.}{n. f.}{A}
\multicolumn{5}{|p{14.05cm}|}{Das Tracing soll einem gängigen Standard (wie OpenTelemetry oder OpenTracing) folgen.} \\
\hline
\end{anf}

\begin{anf}{anf:2320}{2320}{Übertragung von Tracingdaten}{Basis.}{f.}{S}
\multicolumn{5}{|p{14.05cm}|}{Sämtlich erfasste Tracingdaten von Front- und Backend sind an ein \enquote{Tracing}-Partnersystem weiterzuleiten.} \\
\hline
\end{anf}

\begin{anf}{anf:2410}{2410}{Metriken}{Basis.}{f.}{S}
\multicolumn{5}{|p{14.05cm}|}{Im Frontend sind beispielhaft Metriken zu erheben, wie z. B. die Anzahl an Produkten oder die Anzahl an aufgetretenen} \\
\hline
\end{anf}

\begin{anf}{anf:2411}{2411}{Metrik-Standard}{Begeist.}{n. f.}{A}
\multicolumn{5}{|p{14.05cm}|}{Metriken sollen nach einem gängigen Standard (wie OpenTelemetry) erfasst werden.} \\
\hline
\end{anf}

\begin{anf}{anf:2420}{2420}{Übertragung von Metrikdaten}{Basis.}{f.}{S}
\multicolumn{5}{|p{14.05cm}|}{Sämtlich erfasste Metriken des Frontends sind an ein \enquote{Metrik}-Partnersystem weiterzuleiten.} \\
\hline
\end{anf}

\begin{anf}{anf:2510}{2510}{Session-Replay}{Basis.}{f.}{S}
\multicolumn{5}{|p{14.05cm}|}{Im Frontend sind Daten zwecks Session-Replay zu erheben, welche u. A. Benutzerinteraktionen, Schnittstellaufrufe sowie DOM-Manipulationen enthalten.} \\
\hline
\end{anf}

\begin{anf}{anf:2511}{2511}{Schalter für Session-Replay}{Basis.}{f.}{A}
\multicolumn{5}{|p{14.05cm}|}{Beim ersten Aufruf des Frontends sind keine Session-Replay-Daten zu erheben.\par Stattdessen soll der Nutzer über einen Dialog auswählen können, ob er der Aufnahme zustimmt und erst danach sind diese Daten zu erheben.} \\
\hline
\end{anf}

\begin{anf}{anf:2520}{2520}{Übertragung von Session-Replay-Daten}{Basis.}{f.}{S}
\multicolumn{5}{|p{14.05cm}|}{Sämtlich im Frontend erfasste Daten zum Session-Replay sind an ein \enquote{Session-Replay}-Partnersystem weiterzuleiten.} \\
\hline
\end{anf}

\subsubsection{Eigenschaften}


\begin{anf}{anf:3010}{3010}{Resilienz der Übertragung}{Begeist.}{f.}{S}
\multicolumn{5}{|p{14.05cm}|}{Daten, die der Nachvollziehbarkeit dienen, sollen, wenn möglich, bei einer fehlgeschlagenen Verbindung nicht verworfen werden. Sie sind mindestens 120s vorzuhalten und in dieser Zeit sind wiederholt Verbindungsversuche zu unternehmen.} \\
\hline
\end{anf}

\begin{anf}{anf:3020}{3020}{Batchverarbeitung}{Begeist.}{f.}{S}
\multicolumn{5}{|p{14.05cm}|}{Daten, die der Nachvollziehbarkeit dienen, sind, wenn möglich, gruppiert an externe Systeme zu senden. Hierbei ist eine kurze Aggregationszeit von bis zu 10s akzeptabel.} \\
\hline
\end{anf}

\begin{anf}{anf:3100}{3100}{Anzahl Partnersysteme}{Basis.}{n. f.}{K}
\multicolumn{5}{|p{14.05cm}|}{Die Anzahl an zusätzlichen Partnersystemen, die für die Lösung benötigt werden, ist so gering zu halten wie möglich.} \\
\hline
\end{anf}

\begin{anf}{anf:3200}{3200}{Structured Logging}{Leistungs.}{f.}{A+S}
\multicolumn{5}{|p{14.05cm}|}{Das Logging soll mit einem vordefinierten Format durchgeführt werden. Für ähnliche Funktionsgruppen (wie ein Schnittstellenaufruf) soll das gleiche Format verwendet werden. Ein anwendungsübergreifendes Format ist nicht gefordert.} \\
\hline
\end{anf}

\subsubsection{Partnersysteme}


\begin{anf}{anf:5100}{5100}{Partnersystem \textit{Log-Management}}{Basis.}{f.}{A+S}
\multicolumn{5}{|p{14.05cm}|}{Es existiert ein \enquote{Log-Management}-Partnersystem, zu dem Logmeldungen weitergeleitet werden und welches diese speichert.} \\
\hline
\end{anf}

\begin{anf}{anf:5110}{5110}{Manuelle Analyse \textit{Log-Management}}{Basis.}{f.}{A+S}
\multicolumn{5}{|p{14.05cm}|}{Nutzer des Systems sollen die erfassten Logmeldungen einsehen sowie diese filtern können. Die Filtierung erfolgt auf Basis der Eigenschaften der Logmeldung (bspw. des Log-Levels).} \\
\hline
\end{anf}

\begin{anf}{anf:5200}{5200}{Partnersystem \textit{Error-Monitoring}}{Basis.}{f.}{A+S}
\multicolumn{5}{|p{14.05cm}|}{Es existiert ein \enquote{Error-Monitoring}-Partnersystem, zu dem Fehler weitergeleitet werden und welches diese persistiert.} \\
\hline
\end{anf}

\begin{anf}{anf:5210}{5210}{Manuelle Analyse \textit{Error-Monitoring}}{Basis.}{f.}{A+S}
\multicolumn{5}{|p{14.05cm}|}{Nutzer des Systems sollen die erfassten Fehler einsehen sowie diese filtern können. Die Filtierung erfolgt auf Basis der Eigenschaften der Fehler (bspw. der Fehlername).} \\
\hline
\end{anf}

\begin{anf}{anf:5220}{5220}{Visualisierung \textit{Error-Monitoring}}{Leistungs.}{f.}{A+S}
\multicolumn{5}{|p{14.05cm}|}{Die Fehler sollen bspw. in Histogrammen grafisch dargestellt werden können.} \\
\hline
\end{anf}

\begin{anf}{anf:5230}{5230}{Alerting  \textit{Error-Monitoring}}{Begeist.}{f.}{A+S}
\multicolumn{5}{|p{14.05cm}|}{Bei Auftreten von bestimmten Fehlern oder einer Anzahl von Fehlern soll eine Meldung erzeugt werden können (per E-Mail, Slack, o. Ä.).} \\
\hline
\end{anf}

\begin{anf}{anf:5300}{5300}{Partnersystem \textit{Tracing}}{Basis.}{f.}{A+S}
\multicolumn{5}{|p{14.05cm}|}{Es existiert ein \enquote{Tracing}-Partnersystem, welches die Tracingdaten konsumiert und speichert. Zusammengehörige Spans sind zu Traces zusammenzufassen.} \\
\hline
\end{anf}

\begin{anf}{anf:5310}{5310}{Manuelle Analyse \textit{Tracing}}{Basis.}{f.}{A+S}
\multicolumn{5}{|p{14.05cm}|}{Die erfassten Tracingdaten sind für die Nutzer des Systems einsehbar, sowie können diese gefiltert werden. Die Filtierung erfolgt auf Basis von Eigenschaften der Tracingdaten (wie Name des meldenden Systems).} \\
\hline
\end{anf}

\begin{anf}{anf:5320}{5320}{Visualisierung \textit{Tracing}}{Basis.}{f.}{A+S}
\multicolumn{5}{|p{14.05cm}|}{Das Partnersystem, zu dem die Tracingdaten weitergeleitet werden, soll diese grafisch als Trace-Gantt-Diagramm darstellen können.} \\
\hline
\end{anf}

\begin{anf}{anf:5400}{5400}{Partnersystem \textit{Metriken}}{Leistungs.}{f.}{A+S}
\multicolumn{5}{|p{14.05cm}|}{Es existiert ein \enquote{Metrik}-Partnersystem, zu dem Metriken weitergeleitet werden und welches diese persistiert.} \\
\hline
\end{anf}

\begin{anf}{anf:5410}{5410}{Visualisierung \textit{Metriken}}{Leistungs.}{f.}{A+S}
\multicolumn{5}{|p{14.05cm}|}{Metriken sind grafisch darstellbar, bspw. in Histogrammen.} \\
\hline
\end{anf}

\begin{anf}{anf:5420}{5420}{Alerting \textit{Metriken}}{Begeist.}{f.}{A+S}
\multicolumn{5}{|p{14.05cm}|}{Bei Auftreten von bestimmten Metrikwerten oder Überschreitungen von Schwellen soll eine Meldung erzeugt werden können (per E-Mail, Slack, o. Ä.).} \\
\hline
\end{anf}

\begin{anf}{anf:5500}{5500}{Partnersystem \textit{Session-Replay}}{Basis.}{f.}{A+S}
\multicolumn{5}{|p{14.05cm}|}{Es existiert ein \enquote{Session-Replay}-Partnersystem, zu die Daten zum Session-Replay gesendet werden und welches diese analysiert und speichert.} \\
\hline
\end{anf}

\begin{anf}{anf:5510}{5510}{Nachstellung \textit{Session-Replay}}{Basis.}{f.}{A+S}
\multicolumn{5}{|p{14.05cm}|}{Dieses System soll anhand der Daten jede aufgezeichnete Benutzersitzung in Videoform nachstellen.} \\
\hline
\end{anf}