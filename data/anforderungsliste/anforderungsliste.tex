
\subsubsection{Grundanforderungen}


\begin{anf}{anf:1010}{1010}{Konzept}{B}{f}{A}
\multicolumn{5}{|p{14.05cm}|}{Es wird ein System konzipiert, welches darauf abzielt die Nachvollziehbarkeit einer JavaScript-basierten Webanwendung zu verbessern. Speziell sollen Benutzerinteraktionen und Anwendungsverhalten nachvollziehbarer gemacht werden.} \\
\hline
\end{anf}

\begin{anf}{anf:1020}{1020}{Demoanwendung}{B}{f}{A}
\multicolumn{5}{|p{14.05cm}|}{Eine Demoanwendung ist zu Erstellen und soll dazu dienen, das Konzept darauf anwenden zu können.\par Diese Demoanwendung soll Fehlerverhalten beinhalten, die dann mithilfe der Lösung besser nachvollziehbar zu gestalten sind.} \\
\hline
\end{anf}

\begin{anf}{anf:1030}{1030}{Proof-of-Concept}{B}{f}{A}
\multicolumn{5}{|p{14.05cm}|}{Auf Basis des Konzeptes, ist die Demo-Anwendung zu erweitern.} \\
\hline
\end{anf}

\begin{anf}{anf:1031}{1031}{Bewertung Proof-of-Concept}{B}{f}{A}
\multicolumn{5}{|p{14.05cm}|}{Nach Abschluss der Implementierung des Proof-of-Concepts soll dieser veranschaulicht und bewertet werden. Grundlage hierfür sind diese Anforderungen sowie die, zu identifizierende, Fähigkeit die Fehlerszenarien der Demoanwendung nachvollziehbar zu gestalten.} \\
\hline
\end{anf}

\subsubsection{Funktionsumfang}


\begin{anf}{anf:2010}{2010}{Schnittstellen-Logging}{B}{f}{S}
\multicolumn{5}{|p{14.05cm}|}{Das Aufrufen von Schnittstellen ist mittels einer Logmeldung zu notieren. Hierbei sind relevante Informationen wie Aufrufparameter ebenfalls zu notieren.} \\
\hline
\end{anf}

\begin{anf}{anf:2011}{2011}{Use-Case-Logging}{B}{f}{S}
\multicolumn{5}{|p{14.05cm}|}{Tritt ein Use-Case auf, soll dieser im Log notiert werden. Beispielsweise soll notiert werden, wenn ein Nutzer ein Formular absendet.} \\
\hline
\end{anf}

\begin{anf}{anf:2020}{2020}{Error-Monitoring}{B}{f}{S}
\multicolumn{5}{|p{14.05cm}|}{Tritt ein Fehler auf, der nicht gefangen wurde, so ist dieser automatisch zu erfasst und um weitere Attribute zu ergänzen.\par Sonstige Fehler können auch erfasst werden, aber hierbei ist keine automatische Erfassung gefordert.} \\
\hline
\end{anf}

\begin{anf}{anf:2030}{2030}{Tracing}{B}{f}{S}
\multicolumn{5}{|p{14.05cm}|}{Es werden Tracingdaten ähnlich wie bei OpenTracing und OpenTelemetry erfasst.\par Optimalerweise werden die Tracingdaten mit OpenTelemetry-konformen Komponenten erfasst.} \\
\hline
\end{anf}

\begin{anf}{anf:2040}{2040}{Metriken}{B}{f}{S}
\multicolumn{5}{|p{14.05cm}|}{Es werden Metrikdaten ähnlich wie bei OpenTelemetry erfasst.\par Optimalerweise werden die Tracingdaten mit OpenTelemetry-konformen Komponenten erfasst.} \\
\hline
\end{anf}

\begin{anf}{anf:2050}{2050}{Session-Replay}{B}{f}{S}
\multicolumn{5}{|p{14.05cm}|}{Es sollen Session-Replay-Daten erhoben werden, anhand dessen die Benutzerinteraktionen und das Anwendungsverhalten nachgestellt werden kann. Diese Funktionalität darf jedoch standardmäßig deaktiviert sein.} \\
\hline
\end{anf}

\begin{anf}{anf:2110}{2110}{Übertragung von Logs}{B}{f}{S}
\multicolumn{5}{|p{14.05cm}|}{Ausgewählte Logmeldungen sind an ein Partnersystem weiterzuleiten. Die Auswahl könnte über die Kritikalität, also dem Log-Level, der Logmeldung erfolgen.} \\
\hline
\end{anf}

\begin{anf}{anf:2120}{2120}{Übertragung von Fehlern}{B}{f}{S}
\multicolumn{5}{|p{14.05cm}|}{Sämtlich erfasste Fehler sind an ein Partnersystem weiterzuleiten.} \\
\hline
\end{anf}

\begin{anf}{anf:2130}{2130}{Übertragung von Tracingdaten}{B}{f}{S}
\multicolumn{5}{|p{14.05cm}|}{Sämtlich erfasste Tracingdaten sind an ein Partnersystem weiterzuleiten.} \\
\hline
\end{anf}

\begin{anf}{anf:2140}{2140}{Übertragung von Metrikdaten}{B}{f}{S}
\multicolumn{5}{|p{14.05cm}|}{Sämtlich erfasste Metrikdaten sind an ein Partnersystem weiterzuleiten.} \\
\hline
\end{anf}

\begin{anf}{anf:2150}{2150}{Übertragung von Session-Replay-Daten}{B}{f}{S}
\multicolumn{5}{|p{14.05cm}|}{Sämtlich erfasste Session-Replay-Daten sind an ein Partnersystem weiterzuleiten.} \\
\hline
\end{anf}

\subsubsection{Eigenschaften}


\begin{anf}{anf:3010}{3010}{Resilienz der Übertragung}{S}{f}{S}
\multicolumn{5}{|p{14.05cm}|}{Daten, die der Nachvollziehbarkeit dienen, sollen, wenn möglich, bei einer fehlgeschlagenen Verbindung nicht verworfen werden. Sie sind mindestens 120s vorzuhalten und in dieser Zeit sind wiederholt Verbindungsversuche zu unternehmen.} \\
\hline
\end{anf}

\begin{anf}{anf:3020}{3020}{Batchverarbeitung}{S}{f}{S}
\multicolumn{5}{|p{14.05cm}|}{Daten, die der Nachvollziehbarkeit dienen, sind, wenn möglich, gruppiert an externe Systeme zu senden. Hierbei ist eine kurze Aggregationszeit von bis zu 10s akzeptabel.} \\
\hline
\end{anf}

\begin{anf}{anf:3100}{3100}{Anzahl Partnersysteme}{B}{nf}{K}
\multicolumn{5}{|p{14.05cm}|}{Die Anzahl an zusätzlichen Partnersystemen, die für die Lösung benötigt werden, ist so gering zu halten wie möglich.} \\
\hline
\end{anf}

\begin{anf}{anf:3200}{3200}{Structured Logging}{L}{f}{A+S}
\multicolumn{5}{|p{14.05cm}|}{Das Logging soll mit einem vordefinierten Format durchgeführt werden. Für ähnliche Funktionsgruppen (wie ein Schnittstellenaufruf) soll das gleiche Format verwendet werden. Ein anwendungsübergreifendes Format ist nicht gefordert.} \\
\hline
\end{anf}

\subsubsection{Partnersysteme}


\begin{anf}{anf:5010}{5010}{Partnersystem \textit{Log-Management}}{B}{f}{A+S}
\multicolumn{5}{|p{14.05cm}|}{Es existiert ein Partnersystem, zu dem Logmeldungen weitergeleitet werden. Dieses System soll die Logmeldungen speichern und den Entwicklern und Betreibern eine Einsicht in die erfassten Logmeldungen bieten.} \\
\hline
\end{anf}

\begin{anf}{anf:5020}{5020}{Partnersystem \textit{Error-Monitoring}}{B}{f}{A+S}
\multicolumn{5}{|p{14.05cm}|}{Es existiert ein Partnersystem, zu dem Fehler weitergeleitet werden. Dieses System soll die Fehler speichern und den Entwicklern und Betreibern eine Einsicht in die erfassten Fehler bieten.} \\
\hline
\end{anf}

\begin{anf}{anf:5021}{5021}{Visualisierung \textit{Error-Monitoring}}{L}{f}{A+S}
\multicolumn{5}{|p{14.05cm}|}{Das Partnersystem, zu dem die Fehler weiterzuleiten sind, soll diese grafisch darstellen können.} \\
\hline
\end{anf}

\begin{anf}{anf:5022}{5022}{Alerting  \textit{Error-Monitoring}}{S}{f}{A+S}
\multicolumn{5}{|p{14.05cm}|}{Das Partnersystem, zu dem die Fehler weiterzuleiten sind, soll bei Auftreten von bestimmten Fehlern oder Fehleranzahlen eine Meldung erzeugen (per E-Mail, Slack, o. Ä.).} \\
\hline
\end{anf}

\begin{anf}{anf:5030}{5030}{Partnersystem \textit{Tracing}}{B}{f}{A+S}
\multicolumn{5}{|p{14.05cm}|}{Es existiert ein Partnersystem, zu dem Tracingdaten weitergeleitet werden. Dieses System soll die Fehler speichern und den Entwicklern und Betreibern eine Einsicht in die erfassten Tracingdaten bieten.} \\
\hline
\end{anf}

\begin{anf}{anf:5031}{5031}{Visualisierung \textit{Tracing}}{B}{f}{A+S}
\multicolumn{5}{|p{14.05cm}|}{Das Partnersystem, zu dem die Tracingdaten weitergeleitet werden, soll diese grafisch als Tracing-Wasserfallgraph darstellen können.} \\
\hline
\end{anf}

\begin{anf}{anf:5040}{5040}{Partnersystem \textit{Metriken}}{L}{f}{A+S}
\multicolumn{5}{|p{14.05cm}|}{Es existiert ein Partnersystem, zu dem Metriken weitergeleitet werden. Dieses System soll die Fehler speichern und den Entwicklern und Betreibern eine Einsicht in die erfassten Metriken bieten.} \\
\hline
\end{anf}

\begin{anf}{anf:5041}{5041}{Visualisierung \textit{Metriken}}{L}{f}{A+S}
\multicolumn{5}{|p{14.05cm}|}{Das Partnersystem, zu dem die Metriken weiterzuleiten sind, soll diese grafisch darstellen können.} \\
\hline
\end{anf}

\begin{anf}{anf:5042}{5042}{Alerting \textit{Metriken}}{S}{f}{A+S}
\multicolumn{5}{|p{14.05cm}|}{Das Partnersystem, zu dem die Metriken weiterzuleiten sind, soll bei Auftreten von bestimmten Metrikwerten oder Überschreitungen von Schwellen eine Meldung erzeugen (per E-Mail, Slack, o. Ä.).} \\
\hline
\end{anf}

\begin{anf}{anf:5050}{5050}{Partnersystem \textit{Session-Replay}}{B}{f}{A+S}
\multicolumn{5}{|p{14.05cm}|}{Es existiert ein Partnersystem, zu die Session-Replay-Daten weitergeleitet werden. Dieses System soll anhand dieser Daten eine Benutzersitzung rekreieren können.} \\
\hline
\end{anf}