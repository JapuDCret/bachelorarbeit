% \section{Bewertung der Anforderungen}

Vor der Erstellung des Konzeptes und der eigentlichen Implementierung wurden dafür Anforderungen definiert. Diese sollen nun überprüft werden, ob und in welchem Umfang sie umgesetzt wurden. Dafür werden die Anforderungen tabellarisch und in kurzer Form dargestellt. Um den Grad der Erfüllung zu beschreiben, existiert zudem die Spalte \textit{Erfüllungsgrad}. Dabei handelt es sich um einen Prozentwert von 0\%--100\%, wobei 0\% keine Umsetzung der Anforderung bedeutet und 100\% eine vollständige. Werte, die dazwischen liegen, werden nachgehend genauer erläutert, da hierfür keine allgemeine Beschreibung gewählt werden kann.
	
% textidote: ignore begin
\begingroup
\centering
\setlength{\LTleft}{-20cm plus -1fill}
\setlength{\LTright}{\LTleft}
\begin{longtable}{|p{0.85cm}|p{6.2cm}|p{1.55cm}|p{1.75cm}|p{1.1cm}|p{1.8cm}|}
\hline
Id & Titel & Kano-Modell & Funktions\-art & Quelle & Erfüllungs\-grad \\
\endhead
\hline
\multicolumn{6}{|l|}{Funktionsumfang} \\
\hline
2110 & Schnittstellen-Logging & Basis. & f. & S & 100\% \\
\hline
2111 & Use-Case-Logging & Basis. & f. & S & 100\% \\
\hline
2120 & Übertragung von Logs & Basis. & f. & S & 100\% \\
\hline
2210 & Error-Monitoring & Basis. & f. & S & 100\% \\
\hline
2220 & Übertragung von Fehlern & Basis. & f. & S & 100\% \\
\hline
2310 & Tracing & Basis. & f. & S & 100\% \\
\hline
2311 & Tracing-Standard & Leistungs. & n. f. & A & 100\% \\
\hline
2320 & Übertragung von Tracingdaten & Basis. & f. & S & 100\% \\
\hline
2410 & Metriken & Basis. & f. & S & 100\% \\
\hline
2411 & Metrik-Standard & Begeist. & n. f. & A & 100\% \\
\hline
2420 & Übertragung von Metrikdaten & Basis. & f. & S & 100\% \\
\hline
2510 & Session-Replay & Basis. & f. & S & 100\% \\
\hline
2511 & Schalter für Session-Replay & Basis. & f. & A & 100\% \\
\hline
2520 & Übertragung von Session-Replay-Daten & Basis. & f. & S & 100\% \\
\hline
\multicolumn{6}{|l|}{Eigenschaften} \\
\hline
3010 & Resilienz der Übertragung & Begeist. & f. & S &  \\
\hline
3020 & Batchverarbeitung & Begeist. & f. & S & 100\% \\
\hline
3100 & Anzahl Partnersysteme & Basis. & n. f. & K & 70\% \\
\hline
3200 & Structured Logging & Leistungs. & f. & A+S &  \\
\hline
\multicolumn{6}{|l|}{Partnersysteme} \\
\hline
5100 & Partnersystem \textit{Log-Management} & Basis. & f. & A+S & 100\% \\
\hline
5110 & Manuelle Analyse \textit{Log-Management} & Basis. & f. & A+S & 100\% \\
\hline
5200 & Partnersystem \textit{Error-Monitoring} & Basis. & f. & A+S & 100\% \\
\hline
5210 & Manuelle Analyse \textit{Error-Monitoring} & Basis. & f. & A+S & 100\% \\
\hline
5220 & Visualisierung \textit{Error-Monitoring} & Leistungs. & f. & A+S & 100\% \\
\hline
5230 & Alerting  \textit{Error-Monitoring} & Begeist. & f. & A+S & 0\% \\
\hline
5300 & Partnersystem \textit{Tracing} & Basis. & f. & A+S & 100\% \\
\hline
5310 & Manuelle Analyse \textit{Tracing} & Basis. & f. & A+S & 100\% \\
\hline
5320 & Visualisierung \textit{Tracing} & Basis. & f. & A+S & 100\% \\
\hline
5400 & Partnersystem \textit{Metriken} & Leistungs. & f. & A+S & 100\% \\
\hline
5410 & Visualisierung \textit{Metriken} & Leistungs. & f. & A+S & 100\% \\
\hline
5420 & Alerting \textit{Metriken} & Begeist. & f. & A+S & 0\% \\
\hline
5500 & Partnersystem \textit{Session-Replay} & Basis. & f. & A+S & 100\% \\
\hline
5510 & Nachstellung \textit{Session-Replay} & Basis. & f. & A+S & 100\% \\
\hline
\caption{Tabellarische Bewertung der Anforderungen}
\label{tab:anforderungsbewertung}
\end{longtable}
\endgroup

% textidote: ignore end

Die Anforderung 3010 wurde nicht vollständig erfüllt, da nicht alle Daten, die der Nachvollziehbarkeit dienen, resilient erfasst und übertragen werden. Genauer werden Logs, Metriken und Fehler so behandelt - jedoch nicht Traces. Grund hierfür ist, dass durch den Einsatz von OpenTelemetry-Komponenten dies nicht oder nicht einfach möglich war. Da somit 3 von 4 Datentypen eine Resilienz aufweisen, wurde die Anforderung mit einem Erfüllungsgrad von 75\% bewertet.

Die Anzahl der Systeme gering zu halten, also Anforderung 3100, konnte nicht vollständig umgesetzt werden. Grund hierfür ist, dass zwar auf ein weiteres System zum Error-Monitoring verzichtet werden konnte, aber dennoch 3 zusätzliche Partnersysteme notwendig sind. Somit wurde diese Anforderung als teilweise erfüllt bewertet.

Die Anforderungen 5230 und 5420 wurden nicht erfüllt, da kein Alerting mithilfe von Splunk umgesetzt wurde. Grund hierfür war eine Netzwerkeinschränkung von dem Kuber\-netes\-cluster, die es nicht einfach möglich machte Daten von Splunk aktiv nach außen zu senden. Dadurch das es sich um zwei Begeisterungsmerkmale handelte, wurde ein Fehlen dessen akzeptiert.