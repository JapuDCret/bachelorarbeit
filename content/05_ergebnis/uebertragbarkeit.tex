% \section{Übertragbarkeit}

Neben der Erfüllung der gestellten Anforderungen sowie der Aufdeckung von Fehlerszenarien, sollte die erstellten Ergebnisse eine Übertragbarkeit aufweisen. Übertragbarkeit meint hierbei, ob und wie gut sich die ermittelten Ergebnisse auf andere Projekte übertragen lassen.

Zunächst wurde eine Demoanwendung erstellt, anhand dessen die Entwicklung des Konzeptes und der eigentlichen Implementierung erfolgten. Bei dieser Demoanwendung handelt es sich um eine Single-Page-Application, die mit Angular umgesetzt wurde und auf ein Backend zugreift, das mittels einer Microservice-orientierten Architektur umgesetzt wurde. Setzen Aspekte der erstellten Lösung auf diese Randbedingungen, schränken sie die Übertragbarkeit auf andere Systeme ein.

Das Konzept wurde jedoch nicht starr an die Demoanwendung angelehnt, sondern basiert grundlegend auf den zuvor identifizierten fehlenden Informationen bei SPAs und den ermittelten Technologien, um diese Informationslücke zu füllen. Somit weißt das Konzept eine grundlegend gute Übertragbarkeit auf. Anders ist dies jedoch bei der Umsetzung, denn hier wurden einige Aspekte auf Basis der eingesetzten Frameworks umgesetzt. Beispielsweise basiert das Tracing im Backend auf der Unterstützung von OpenTracing im Eclipse-Microprofile-Framework. Des Weiteren basiert die Erfassung von Logs und Fehlern im Frontend auf Angular spezifischen Bibliotheken oder Komponenten.

Eine direkte Bewertung der Übertragbarkeit erfolgt nicht, stattdessen werden die Annahmen erfasst. Die Erfassung der Annahmen erfolgt in \autoref{tab:annahmen-der-erstellten-loesungen}. Dabei wird jede Annahme jeweils für das Konzept sowie für die Umsetzung bewertet, dabei werden folgende Bewertungsschlüssel verwendet:

\begin{itemize}
	\item \textbf{ja}: Die Annahme muss erfüllt sein.
	\item \textbf{nein}: Die Annahme muss nicht erfüllt sein.
	\item \textbf{nebensächlich}: Damit ist gemeint, dass die Annahme nicht erfüllt sein muss, jedoch dies bspw. zu einer Minderung der Funktionalität führt. Weiterhin kann hiermit auch gemeint sein, dass die Annahme durch einen äquivalenten Ersatz erfüllt werden kann.
\end{itemize}

\begingroup
\centering
\setlength{\LTleft}{-20cm plus -1fill}
\setlength{\LTright}{\LTleft}
\begin{longtable}{|p{10cm}|p{1.9cm}|p{1.9cm}|}
\hline
Annahme & Konzept & Umsetzung \\
\endhead
\hline
Das System kann erweitert werden & ja & ja \\
\hline
Das System kann erweitert werden & ja & ja \\
\hline
Das Frontend ist eine RIA & ja & ja \\
\hline
Das Frontend ist eine SPA & nein & ja \\
\hline
Das Frontend wurde mit Angular erstellt & nein & ja \\
\hline
Das Frontend realisiert einen Wizard & nein & nein \\
\hline
Das Backend folgt einem Microservice-Architektur-Ansatz & nebensächl. & ja \\
\hline
Das Backend wurde mit Java umgesetzt & nein & ja \\
\hline
Das Backend wurde mit Eclipse MicroProfile umgesetzt & nein & nebensächl. \\
\hline
Splunk kann eingesetzt werden & nebensächl. & ja \\
\hline
Jaeger kann eingesetzt werden & nebensächl. & ja \\
\hline
LogRocket kann eingesetzt werden & ja & ja \\
\hline
\caption{Annahmen der erstellten Lösungen}
\label{tab:annahmen-der-erstellten-loesungen}
\end{longtable}
\endgroup