% \chapter{Ergebnis}

\section{Demonstration}

	\textit{Nachdem nun eine Implementierung steht, soll die Erweiterung auf nicht-technische Weise veranschaulicht werden. Hier soll dargestellt werden, wie die Nachvollziehbarkeit nun verbessert worden ist.}
	
\section{Kriterien}

	\textit{Die zuvor definierten Kriterien in 4.1 sollen hier überprüft werden.}
	
\section{Übertragbarkeit}

	\textit{Wie gut lassen sich die ermittelten Ergebnisse im PoC auf andere Projekte im selben Umfeld übertragen?}
	
\section{Einschätzung von anderen Entwicklern (optional)}

	\textit{\textbf{Dieser Abschnitt kann ggf. wegfallen}, wenn nicht genügend Zeit besteht oder der Nutzen nicht den Aufwand gerechtfertigt.}
	
	\textit{In diesem Abschnitt könnten Frontend-Entwicklern mit der Demo-Anwendung mit und ohne die Lösung konfrontiert werden. Daraufhin könnten diese befragt werden, ob}
	
	\begin{enumerate}
		\item \textit{Wie gut entspricht die Demo-Anwendung einem realen Szenario?}
		\item \textit{Sind die vorgestellten Probleme realitätsnah?}
		\item \textit{Wie gut lassen sich die Probleme ohne die Lösung beheben?}
		\item \textit{Wie gut lassen sich die Probleme mit der Lösung beheben?}
		\item \textit{Ist der Lösungsansatz zu komplex?}
		\item \textit{Gibt es Bedenken zum Lösungsansatz?}
	\end{enumerate}