% \chapter{Abschluss}

\section{Zusammenfassung}

In der Arbeit konnte das grundlegende Ziel, einen Ansatz zu erstellen mit dem Betreibern von JavaScript-basierten Webanwendungen eine Nachvollziehbarkeit gewährleistet werden kann, erreicht werden. Die erstellte Lösung, also der Proof-of-Concept, konnte die Mehrheit der gestellten Anforderungen erfüllen sowie konnten zuvor definierte Fehlerszenarien aufgedeckt werden. Weiterhin weist die erstellte Lösung, insbesondere dabei das Konzept, eine Übertragbarkeit auf andere ähnliche Softwareprojekte auf. Zudem konnte kein signifikanter negativer Einfluss für den Nutzer festgestellt werden. Alles in Allem konnte somit ein Ansatz vorgestellt werden, der die gesetzten Ziele erreicht hat, sowie sind die gefundenen Ergebnisse auch in anderen Situationen anwendbar.

\section{Fazit}

Es konnte ein Kernproblem von RIAs festgestellt werden, aber es konnten zudem Methoden und Technologien identifiziert werden, dieses Kernproblem zu beheben. Weiterhin wurden diese Ansätze erfolgreich eingesetzt und boten einen Mehrwert für Entwickler und Betreiber, der sich nur marginal auf die Performance der Webanwendung niederschlug.

\section{Ausblick}

Wie bei der Recherche zum Stand der Technik zu sehen war, gibt es seit 2020 eine neue Entwicklung, die das Feld der Nachvollziehbarkeit bzw. der Observability prägen könnte. Damit ist der Standard OpenTelemetry gemeint, der voraussichtlich kurz nach Abgabe dieser Arbeit veröffentlicht wird. Sollte der Standard von Herstellern adaptiert werden, könnte dies zu einer höheren Auswahl an Technologien führen sowie könnten die erstellten Lösungen umfangreicher werden. Genau abzusehen, was die Zukunft für OpenTelemetry bringt, ist schwierig. Jedoch ist diese Entwicklung in meinen Augen vielversprechend, besonders durch die Unterstützung von führenden Herstellern von Observability-Werkzeugen. Somit sollte dieses Feld in nächster Zeit verfolgt werden.