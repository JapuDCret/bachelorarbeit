% \chapter{Abschluss}

\section{Zusammenfassung}

Ziel der Arbeit war es einen Ansatz zu erstellen, mit dem Betreibern von JavaScript-basierten Webanwendungen eine Nachvollziehbarkeit gewährleistet werden kann. Der Proof-of-Concept, konnte die Mehrheit der gestellten Anforderungen erfüllen sowie zuvor definierte Fehlerszenarien aufgedeckt werden. Weiterhin weist die erstellte Lösung und dabei das Konzept, eine Übertragbarkeit auf andere ähnliche Softwareprojekte auf. Zudem konnte kein signifikanter negativer Einfluss für den Nutzer festgestellt werden. Somit wurde das grundlegende Ziel dieser Arbeit erreicht.

\section{Fazit}

Es konnte ein Kernproblem von Webanwendungen und speziell bei SPAs festgestellt werden, aber es konnten zudem Methoden und Technologien identifiziert werden, die dieses Kernproblem beheben. Weiterhin wurden diese Ansätze erfolgreich eingesetzt und boten einen Mehrwert für Entwickler und Betreiber, der sich nur marginal auf die Performance der Webanwendung niederschlug.

\section{Ausblick}

Wie bei der Recherche zum Stand der Technik zu sehen war, gibt es seit 2020 mit dem OpenTelemetry-Standard eine neue Entwicklung, die das Feld der Nachvollziehbarkeit bzw. der Observability in den nächsten Jahren nachhaltig beeinflussen wird. Sollte der Standard von Herstellern adaptiert werden, könnte dies zu einer höheren Auswahl an Technologien führen, die verwendet werden können, da sie miteinander kompatibel sind. Des Weiteren kann es hierdurch einfacher werden mehrere Observability-Systeme und -Konzepte miteinander zu kombinieren, um eine erhöhte Durchleuchtung zu erreichen. Diese Entwicklung ist in meinen Augen vielversprechend, besonders durch die Unterstützung von führenden Herstellern von Observability-Werkzeugen. Somit sollte dieses Feld in nächster Zeit verfolgt werden.