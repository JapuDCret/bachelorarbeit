% \textit{Hier soll beschrieben werden, was JavaScript-basierte Webanwendungen sind.}

Rich-Internet-Applications (RIA, oder auch Rich-Web-Application) werden oftmals damit assoziiert, dass sie Webanwendungen darstellen, welche Merkmale und Funktionalitäten einer Desktopanwendung besitzen \cite{TenYearsOfRIAs} \cite{NecessityOfMethodologiesToModelRIAs}. Sie besitzen bspw. erweiterte Interaktionsmöglichkeiten (wie Drag-And-Drop), eine detail- und funktionsreiche Benutzeroberfläche mit Fokus auf Benutzbarkeit, zudem eine erhöhte Responsiveness im Vergleich zu klassischen Webanwendungen \cite{TenYearsOfRIAs}.

Erste Ansätze von RIAs gab es bereits ohne eine ausreichende Unterstützung von JavaScript in Browsern. 2002 wurde das Produkt Macromedia Flash MX veröffentlicht, eine an Macromedia Flash (später Adobe Flash) angelehnte Laufzeitumgebungen, die speziell für die Erstellung von RIAs entwickelt wurde \cite{MacromediaFlashMXWhitePaper}, welche zudem den Begriff der Rich-Internet-Application prägte \cite{TenYearsOfRIAs}. Weiterhin wurden RIAs mit einer Vielzahl von Technologien umgesetzt, wie z. B. mit Macromedia Flex (später Adobe Flex nun Apache Flex) oder Java Applets \cite{NecessityOfMethodologiesToModelRIAs} \cite{RIAsTheNextStageOfApplicationDevelopment} \cite{RichInternetApplications} \cite{FinkIntroducingSPAs}. Diese RIAs benötigen extra Umgebungen, die meist via Plugins in den Browser eingebunden werden, ein Beispiel hierfür ist Adobe Flash. Adobe stellte jedoch 2020 den Support für Flash ein \cite{Netlytic}. Heutzutage sind aber JavaScript-basierte RIAs \cite{RIAsTheNextStageOfApplicationDevelopment}, auch Rich-Web-Based-Applications genannt \cite{RichWebBasedApplications} \cite{Netlytic}, die Norm. 

Neben den Vorteilen einer RIA, besitzen diese jedoch auch einige Nachteile. RIAs in jedweder Form stellen eine Herausforderung für Webcrawler dar und erschweren bzw. verhindern die Indexierung der Seite durch Suchmaschinen \cite{CrawlingRIAs}. RIAs leiden zudem oft darunter, dass sie die Funktionen zur Barrierefreiheit in Browsern nicht nutzen. Um dem entgegenzuwirken veröffentlichte das W3C 2011 eine Empfehlung \cite{W3CAccessibleRIAs}, um die Barrierefreiheit auch bei RIAs zu gewährleisten.

\subsection{Single-Page-Applications}
\label{sec:single-page-applications}

Single-Page-Applications (SPA) stellen eine spezielle Art von Rich-Internet-Applications dar. Sie gehen bei der dynamischen Inhaltsdarstellung einen Schritt weiter \cite{SinglePageApplication}: Die gesamte Anwendung wird über ein einziges HTML-Dokument und die darin referenzierten Inhalte erzeugt. Eine Charakteristik sind nicht nur erweiterte Interaktionen, sondern auch komplexe lokale Zustände, die im Client gepflegt werden. Wird beispielsweise eine neue Seite aufgerufen, wird statt einer Dokumentenabfrage via HTTP ein interner Zustand geändert, welcher dann DOM-Manipulationen auslöst, die die Darstellung der Seite ändern.

Für das Bereitstellen einer solchen Anwendung ist meist nur ein simpler Webserver ausreichend und ein oder mehrere Dienste, von denen aus die SPA ihre Inhalte abrufen kann. Populäre Frameworks zur Realisierung von SPAs sind beispielsweise Angular \cite{AngularHomepage}, React \cite{ReactHomepage} oder Vue.js \cite{VueJSHomepage}.

SPAs bieten zudem durch ihre stark clientbasierte Herangehensweise die Möglichkeit, die Anwendung als Offline-Version bereitzustellen. Sind neben der Logik keine externen Daten notwendig oder wurden diese bereits abgerufen und gecached, so kann eine SPA auch \enquote{offline} von Benutzern verwendet werden. Weiterhin steigern SPAs die User Experience (UX), indem sie u. A. schneller agieren, da keine kompletten Seitenaufrufe notwendig sind \cite{ImprovementOfAcedemicServiceBasedOnSPA}.

Dieser grundsätzlich andere Ansatz bringt mit sich auch negative Eigenschaften. Unter anderem werden native Browserfunktionen umgangen, wie die automatisch befüllte Browserhistorie, da keine  HTML-Dokumente angefragt werden. Weiterhin leiden \enquote{virtuelle} Verlinkungen und Buttons darunter, dass sie nicht alle Funktionen unterstützten, die normale HTML-Elemente aufweisen. Um dies und andere verwandte Probleme zu beheben, besitzen die Angular, React und Vue.js teils spezielle Implementierungen oder es muss bspw. die jeweilige Router-Bibliothek herangezogen werden.

Nichtsdestotrotz ist ein jahrelanger Trend des Erfolges von Single-Page-Applications zu erkennen \cite{SinglePageApplication}. Zudem stehen Entwicklern heutzutage eine große Auswahl an erprobten Technologien in diesem Gebiet zur Verfügung, um umfangreiche und nahtlos funktionierende SPAs zu erstellen \cite{TheStateOfJavaScript2020}.