% \textit{Hier soll beschrieben werden, was JavaScript-basierte Webanwendungen sind.}

Rich-Internet-Applications (RIA, oder auch Rich-Web-Application) werden oftmals damit assoziiert, dass sie Webanwendungen darstellen, welche Merkmale und Funktionalitäten einer Desktopanwendung besitzen \cite{TenYearsOfRIAs} \cite{NecessityOfMethodologiesToModelRIAs}. Sie besitzen bspw. erweiterte Interaktionsmöglichkeiten (wie Drag-And-Drop), eine detail- und funktionsreiche Benutzeroberfläche mit Fokus auf Benutzbarkeit sowie bieten sie meist eine erhöhte Responsiveness im Vergleich zu klassischen Webanwendungen \cite{TenYearsOfRIAs}.

RIAs begründen ihre Anfänge nicht etwa erst mit der Standardisierung von Ajax, sondern erste Ansätze gab es bereits ohne die notwendige JavaScript-Unterstützung. 2002 wurde das Produkt Macromedia Flash MX von Macromedia veröffentlicht, welches eine an Macromedia Flash (später Adobe Flash) angelehnte Laufzeitumgebungen war, die speziell für RIAs entwickelt wurde \cite{MacromediaFlashMXWhitePaper}. Der Begriff der Rich-Internet-Application wurde mit Macromedia Flash MX geprägt \cite{TenYearsOfRIAs}. Weiterhin wurden RIAs mit einer Vielzahl von Technologien umgesetzt, wie z. B. mit Macromedia Flex (später Adobe Flex nun Apache Flex) oder Java Applets \cite{NecessityOfMethodologiesToModelRIAs} \cite{RIAsTheNextStageOfApplicationDevelopment} \cite{RichInternetApplications} \cite{FinkIntroducingSPAs}.

Neben den Vorteilen einer RIA, besitzen diese jedoch auch einige Nachteile. RIAs in jedweder Form stellen eine Herausforderung für Webcrawler dar und erschweren oder verhindern so, die Indexierung der Seite durch Suchmaschinen \cite{CrawlingRIAs}. Manche RIAs benötigen extra Umgebungen, die meist via Plugins in den Browser eingebunden werden, jedoch wurde 2020 der Support für Adobe Flash eingestellt \cite{Netlytic}. Heutzutage sind aber Ajax-basierte RIAs \cite{RIAsTheNextStageOfApplicationDevelopment} die Norm, sog. Rich-Web-Based-Applications \cite{RichWebBasedApplications} \cite{Netlytic}. Neben diesen Aspekten leiden jedoch RIAs meist darunter, dass sie die Konzepte zur Barrierefreiheit in Browsern nicht nutzen. Hierfür gibt es jedoch seit 2011 eine Empfehlung des W3C \cite{W3CAccessibleRIAs}, um die Barrierefreiheit auch bei RIAs zu gewährleisten.

\subsection{Single-Page-Applications}
\label{sec:single-page-applications}

Single-Page-Applications (SPAs) stellen eine spezielle Form von Rich-Web-Based-Applications dar. Sie gehen bei der dynamischen Inhaltsdarstellung jedoch einen Schritt weiter \cite{SinglePageApplication}: Die gesamte Anwendung wird über ein einziges HTML-Dokument und die darin referenzierten Inhalte erzeugt. Im Client sind nun nicht nur erweiterte Interaktionen eine Charakteristik, sondern der Client wird erhält einen lokalen Zustand, der gepflegt wird. Wird beispielsweise eine neue Seite aufgerufen, wird anstatt einer Dokumentenabfrage via HTTP ein interner Zustand geändert, welcher dann DOM-Manipulationen auslöst, die die Seite ändern.

Für das Bereitstellen einer solchen Anwendung ist meist nur ein simpler Webserver ausreichend und ein oder mehrere Dienste, von denen aus die SPA ihre Inhalte abrufen kann. Populäre Frameworks zur Realisierung von SPAs sind beispielsweise Angular \cite{AngularHomepage}, React \cite{ReactHomepage} oder Vue.js \cite{VueJSHomepage}.

SPAs bieten zudem durch ihre stark clientbasierte Herangehensweise die Möglichkeit, die Anwendung als Offline-Version bereitzustellen. Sind neben der Logik keine externen Daten notwendig oder wurden diese bereits abgerufen und gecached, so kann eine SPA auch \enquote{offline} von Benutzern verwendet werden. Weiterhin steigern SPAs die User Experience (UX), indem sie u. A. schneller agieren, da keine kompletten Seitenaufrufe notwendig sind \cite{ImprovementOfAcedemicServiceBasedOnSPA}.

Durch diesen grundsätzlich anderen Ansatz gibt es aber auch negative Eigenschaften. Unter anderem werden native Browserfunktionen umgangen, wie die automatisch befüllte Browserhistorie, denn es werden keine neuen HTML-Dokumente angefragt. Weiterhin leiden \enquote{virtuelle} Verlinkungen und Buttons darunter, dass sie nicht alle Funktionen unterstützten, die normale HTML-Elemente aufweisen. Um dies und andere verwandte Probleme zu beheben, besitzen die zuvor genannten Frameworks spezielle Implementierungen oder ggf. muss eine zusätzliche Bibliothek herangezogen werden, wie z. B. die jeweiligen Router-Bibliotheken.

Nichtsdestotrotz ist ein jahrelanger Trend von der Einführung von Single-Page-Applications zu erkennen \cite{SinglePageApplication}, hinzukommend stehen heutzutage eine große Auswahl an erprobten Technologien in diesem Gebiet zur Verfügung \cite{TheStateOfJavaScript2020}.

% In dieser Arbeit werden SPAs untersucht, denn einerseits fallen diese in das Interessengebiet der Open Knowledge, anderseits gibt es aber auch einige Eigenheiten, die die Nachvollziehbarkeit reduzieren. Beispielsweise gehen durch die starke Trennung von Client und Server auch Kontextinformationen verloren. Zudem wird die Anwendung beim Client größer und komplexer, welches das Potenzial von Ungereimtheiten erhöht.