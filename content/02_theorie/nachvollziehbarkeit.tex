Nachvollziehbarkeit bedeutet allgemein, dass über ein resultierendes Verhalten eines Systems auch interne Zustände nachvollzogen werden können. Dies ist keine neue Idee, sondern fand bereits 1960 im Gebiet der Kontrolltheorie starke Bedeutung \cite{OnTheGeneralTheoryOfControlSystems}. Nach Freedman \cite{TestabilityOfSoftwareComponents} und Scrocca \etal \cite{EnablingEventDrivenObservability} lässt sich diese Definition auch auf Softwaresysteme übertragen.
	
In dieser Arbeit wird sich mit der Nachvollziehbarkeit in speziellen Situation befasst, nämlich wenn die Stakeholder das Verhalten einer Webapplikation und die Interaktionen eines Nutzers verstehen möchten.% Dies bedeutet auch, dass keine allgemeine Nachvollziehbarkeit das Ziel ist, denn z.B. das Nutzer die internen Zustände eines Systems nachvollziehen können, ist nicht das Ziel.
	
\subsection{Nutzen}
	
Tritt beispielsweise ein Softwarefehler (Bug) bei einem Nutzer auf, aber die Stakeholder erhalten nicht ausreichende Informationen, so kann der Bug ignoriert werden oder gering priorisiert und in Vergessenheit geraten. Dies geschah im Jahr 2013, als Khalil Shreateh eine Sicherheitslücke bei Facebook fand und diesen bei Facebooks Bug-Bounty-Projekt Whitehat meldete \cite{FacebookBugBounyHunt}. Sein Fehlerreport wurde aufgrund mangelnder Informationen abgelehnt:
	
\begin{quotation}
Unfortunately your report [...] did not have enough technical information for us to take action  on  it. We  cannot  respond  to  reports  which  do  not contain enough detail to allow us to reproduce an issue.
\end{quotation}
	
Durch den Bug konnte Shreateh auf die private Profilseite von Nutzern schreiben, ohne dass er mit ihnen vernetzt war. Um Aufmerksamkeit auf das Sicherheitsproblem zu erregen, hinterließ er eine Nachricht auf Facebooks Gründer und CEO Mark Zuckerbergs Profilseite. Erst hiernach nahm sich Facebooks Team dem Problem an.
	
\subsection{Nachvollziehbarkeit bei SPAs}
	
Wie zuvor in \autoref{sec:clientbased-webapps} ``\nameref{sec:clientbased-webapps}`` geschildert, gibt es bei Webapplikationen und insbesondere Singe-Page-Applications besondere Barrikaden, die es den Stakeholdern erschwert das Verhalten einer Applikation und die Interaktionen eines Nutzers nachzuvollziehen.
	
Bei SPAs ist die Hauptursache, dass die eigentliche Applikation nur beim Client läuft und nur gelegentlich mit einem Backend kommuniziert.
	
Im nächsten Kapitel werden Methoden und Konzepte beschrieben, wie man in Softwareprojekten die Nachvollziehbarkeit verbessern kann.

% In dieser Arbeit werden SPAs untersucht, denn einerseits fallen diese in das Interessengebiet der Open Knowledge, anderseits gibt es aber auch einige Eigenheiten, die die Nachvollziehbarkeit reduzieren. Beispielsweise gehen durch die starke Trennung von Client und Server auch Kontextinformationen verloren. Zudem wird die Applikation beim Client größer und komplexer, welches das Potenzial von Ungereimtheiten erhöht.