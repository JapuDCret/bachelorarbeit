\section{Browserumgebung}

\textit{Hier soll eine Beschreibung der Umgebung erstellt werden, mit Nennung der speziellen Eigenschaften (Keywords: Sandbox, CORS, Logging)}

\section{JavaScript-basierte Webapplikationen}
\textit{Hier soll beschrieben werden, was JavaScript-basierte Webapplikationen sind.}

Diese Ausarbeitung konzentriert sich, wie im Titel beschrieben, auf JavaScript-basierte Webapplikationen. Weiterhin wird sich auf sogenannte Single-Page-Applications (SPAs) konzentriert, welche eine Submenge der JavaScript-basierten Webapplikationen darstellen. Um allen Lesern eine gleiche Grundkenntnis zu ermöglichen, werden diese Konzepte nun kurz vorgestellt.

\textit{..Erklärung zu JavaScript-basierten Webapplikationen..}

\textit{..Erklärung zu SPAs..}

Bekannte Frameworks, um SPAs zu Erstellen, sind beispielsweise Angular \cite{AngularHomepage}, React \cite{ReactHomepage} oder Vue.js \cite{VueJSHomepage}.
	
\newpage

% \section{Instandhaltung und Support}

\section{Softwarebetrieb}

Diese Arbeit konzentriert sich auf Software, die sich in der Betriebsphase befindet. Gängige Software-Entwicklungszyklen und ihre Definition dieser Phase werden folgend beschrieben.

\subsection{Klassisches Vorgehen}

	\begin{wrapfigure}[14]{r}{0.45\linewidth}
		\centering
		\vspace{-\baselineskip}
		\includegraphics[width=\linewidth]{img/02_theorie/software-life-cycle.png}
		\caption{Lebenszyklus einer Software}
		\label{fig:software-development-life-cycle}
		\source{Eigene Darstellung von \cite{ASimulationModelWaterfallSoftware}}
	\end{wrapfigure}
	
	In vielen Modellen über den Lebenszyklus einer Software wird die Phase während der Betreibung oftmals \enquote{Maintenance} genannt \cite{ManagingTheComplexityOfWebSystemsDevelopment} \cite{ASimulationModelWaterfallSoftware}, in der Instandhaltung und Support den Alltag bestimmen. Sie ist nach Zelkowitz \etal \cite{PrinciplesOfSoftwareEngineeringAndDesign} für rund zwei Drittel der Entwicklungskosten verantwortlich, begründet durch exponentielle Steigung \cite{ExtremeProgrammingExplained}.
	
	Das Wasserfallmodell \cite{ASimulationModelWaterfallSoftware} sowie das V-Modell XT \cite{WaterfallVsVModelVsAgile} sehen vor, dass in dieser Phase die Software funktionstüchtig gehalten wird und dass die Anforderungen an die Software erfüllt sind. Bei nicht-erfüllten Anforderungen oder Fehlern, werden diese behoben. Jedoch ein kontinuierlicher Verbesserungsprozess ist in dieser Phase nicht vorgesehen.
	
%	Es werden immer bessere Methoden entwickelt, um Probleme in Software - oder auch Bugs - zu verringern. Jedoch erhöht sich zugleich die Komplexität von Software, was zur Ursache hat, dass es mehr Nährboden für Bugs gibt \cite{TrackingDownSoftwareBugsAnomalyDetection}. De-facto sind Bugs ein unvermeidbarer Bestandteil einer Software und müssen daher erwartet und gehandhabt werden \cite{TheMythicalManMonth}.
%	
%	Wenn nun ein Bug auffällt, sei es durch einen Nutzer oder auch zufällig einem Stakeholder, muss entschieden werden, ob dieser zu beheben ist. Wenn eine Behebung angestrebt wird, benötigt der Stakeholder meistens Rahmeninformationen \cite{WhatMakesAGoodBugReport} um den Bug ggf. zu reproduzieren und die Situation nachzuvollziehen. Desto mehr Verständnis der Stakeholder über das Problem erhält, desto schneller und präziser kann er die Ursache aufdecken. Die Ermöglichung der schnellen Verständnis über ein Problem, wird in dieser Arbeit \textbf{Nachvollziehbarkeit} genannt.

\subsection{Agiles Vorgehen}

	\begin{wrapfigure}[11]{l}{0.45\linewidth}
		\centering
		\vspace{-\baselineskip}
		\includegraphics[width=\linewidth]{img/02_theorie/devops-life-cycle.png}
		\caption{DevOps Toolchain}
		\label{fig:devops-life-cycle}
		\source{Wikimedia Commons \cite{DevOpsLifeCycle}}
	\end{wrapfigure}
	
	Bei agilen Ansätzen wird der Betrieb meist nicht abgegrenzt von der normalen Entwicklung. In dieser Phase werden weiterhin Anforderungen erhoben und diese Stück für Stück umgesetzt \cite{WaterfallVsVModelVsAgile}. Vorteilhaft dabei ist, dass auf neue Wünsche oder Auffälligkeiten sehr einfach reagiert werden kann. Es wird eine kontinuierliche Verbesserung angestrebt.
	
	Um den kontinuierlichen Entwicklungs- und Deploymentprozess reibungslos ablaufen zu lassen, werden Ansätze wie DevOps \cite{DevOps} verfolgt (vgl. \autoref{fig:devops-life-cycle}).

\newpage

\section{Nachvollziehbarkeit}

	\textit{Hier soll die Nachvollziehbarkeit allgemein beschrieben werden und warum sie erstrebenswert ist.}

	Sie beschäftigt sich mit der Informationserfassung und -aufbereitung, um das Verhalten eines Systems und die Interaktionen der Nutzer für die Stakeholder verständlich zu machen. Sie ist getrennt von der Anstrebung einer Reproduzierbarkeit nach der wissenschaftlichen Methode anzusehen.
	
	{\color{red}\textit{\lipsum[1]}}
	
	Tritt ein Problem bei einem Nutzer auf, aber die Stakeholder erhalten nicht ausreichende Informationen, so kann der Bug ignoriert werden oder in Vergessenheit geraten. Dies geschah im Jahr 2013, als Khalil Shreateh eine Sicherheitslücke bei Facebook fand und bei Facebooks Bug-Bounty-Projekt Whitehat meldete \cite{FacebookBugBounyHunt}. Sein Fehlerreport wurde aufgrund mangelnder Informationen abgelehnt:
	
	\begin{quotation}
	Unfortunately your report [...] did not have enough technical information for us to take action  on  it. We  cannot  respond  to  reports  which  do  not contain enough detail to allow us to reproduce an issue.
	\end{quotation}

\subsection{Nachvollziehbarkeit bei Webapplikationen}

	\textit{Hier sollen die Besonderen Hürden bei Webapplikationen hervorgehoben werden (indirekte Kommunikation, keinen Zugriff auf Logs, etc.)}