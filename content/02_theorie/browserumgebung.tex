Web Browser haben sich seit der Veröffentlichung von Mosaic, einer der ersten populären Browser, im Jahr 1993 stark weiterentwickelt. Das Abrufen und Anzeigen von statischen HTML-Dokumenten wurde mit Hilfe von JavaScript um interaktive und später dynamische Inhalte erweitert. Heutzutage können in Browsern komplexe Webapplikationen realisiert werden, welche zudem unabhängig von einem speziellen Browser entwickelt werden können. Durch diese Entwicklung und die breiten Anwendungsfälle, besitzt die Umgebung ``Browser`` besondere Eigenschaften, welche nachfolgend beschrieben werden.

\subsection{JavaScript}

Als JavaScript 1997 veröffentlicht und in den NetScape Navigator integriert wurde, gab es die berechtigen Bedenken, dass das Öffnen einer Webseite dem Betreiber erlaubt Code auf dem System eines Nutzers auszuführen. Damit dies nicht eintritt, wurde der JavaScript Ausführungskontext in eine virtuelle Umgebung integriert, einer Sandbox. \cite{LearningJavaScript}

Die JavaScript-Sandbox bei Browsern schränkt ein, dass unter anderem kein Zugriff auf das Dateisystem erfolgen kann. Auch Zugriff auf native Bibliotheken oder Ausführung von nativem Code ist nicht möglich \cite{TheSpyInTheSandbox}. Browser bieten dafür aber einige Schnittstellen an, die es erlauben z.B. Daten beim Client zu speichern oder auch Videos abzuspielen.

\nomenclature[Fachbegriff]{CORS}{Cross-Origin Resource Sharing}
\nomenclature[Fachbegriff]{Ajax}{Asynchronous JavaScript and XML}
\nomenclature[Fachbegriff]{W3C}{World Wide Web Consortium}
\nomenclature[Fachbegriff]{XHR}{XMLHttpRequest}

Microsoft nahm 1999 im Internet Explorer 5.0 eine neue Methode in ihre JavaScript-Umgebung auf, um den Funktionsumfang zu erweitern: Ajax (Asynchronous JavaScript and XML). Ajax erlaubt die Datenabfrage von Webservern mittels JavaScript. Hierdurch wird ermöglicht, dass Inhalte auf Webseiten dynamisch abgefragt und dargestellt wurden, zuvor war hierfür ein erneuter Seitenaufruf notwendig. Das Konzept wurde kurz darauf von allen damals gängigen Browser übernommen. Erst jedoch mit der Standardisierung 2006 durch das W3C \cite{TheXMLHttpRequestObject} fand die Methode Anklang und Einsatz bei Entwicklern und ist seitdem der Grundstein für unser dynamisches und interaktives Web \cite{TheStoryOfXMLHTTP}.

Webapplikationen wurden nun immer beliebter, aber Entwickler klagten darüber, dass Browser die Abfragen von JavaScript nur auf dem bereitstellenden Webserver, also ``same-origin``, erlauben\cite{CrossSiteXHRWithCORS}. Im selben Jahr der Standardisierung von Ajax, wurde ein erster Entwurf zur Ermöglichung und Absicherung von Abrufen domänenfremder Ressourcen eingereicht \cite{AuthorizingCORS}, das sogenannte Cross-Origin Resource Sharing.

%\subsubsection{Fehleranfälligkeit}
%
%Die Sprache JavaScript stellt an sich auch eine Besonderheit der Umgebung dar. Denn anders als z.B. C, C++, Java gilt sie als fehleranfällig \cite{FastReproducingWebApplicationErrors}. Dies

\subsection{Sicherheitsvorkehrungen}

% Zusätzlich zu der Sandbox gibt es weitere Vorkehrungen, die definieren welche Daten innerhalb der JavaScript Umgebung abgerufen und mit welchen Diensten kommuniziert werden darf \cite{LearningJavaScript}. Zwei wichtige dieser Vorkehrungen, die eine Webapplikationen nutzen kann bzw. beachten muss, werden folgend erklärt.

\subsubsection{Content-Security-Policy}

\nomenclature[Fachbegriff]{CSP}{Content-Security-Policy}
\nomenclature[Fachbegriff]{XSS}{Cross-Site-Scripting}
\nomenclature[Fachbegriff]{CDN}{Content Delivery Network}

Eine Content-Security-Policy definiert, welche Ressourcen (also Bilder, Skripte, etc.) von der Webapplikation aus geladen werden dürfen und über welches Protokoll. Dies dient dem Schutz vor Cross-Site-Scripting, indem eine Webapplikation beschränken kann, welche Aufrufe von ihr aus getätigt werden dürfen. Wird beispielsweise eine JavaScript-Bibliothek aus einer externen Quelle (wie z.B. ein CDN) benutzt und die Bibliothek wird zu späterer Zeit böswillig ausgetauscht oder modifiziert, kann durch die Härtungsmaßnahme einer Content-Security-Policy gewährleistet werden, dass keine Daten an unbekannte Server geschickt werden dürfen \cite{MDNContentSecurityPolicy}.

\subsubsection{Cross-Origin Resource Sharing (CORS)}

Das Konzept von CORS stellt sicher, dass eine Webapplikation keine Ressourcen von Webservern anfragt, die nicht dem bereitstellenden Webserver entsprechen \cite{MDNCORS}. Folgendes Beispiel soll den Nutzen und den generellen Ablauf von CORS näher erläutern:

\begin{quotation}
Ein Nutzer ruft eine Webapplikation auf, welche unter \texttt{localhost:3000} bereitgestellt wird. Diese Webapplikation sendet, für den Nutzer unwissend, Anfragen an einen Facebook Dienst. Dies lässt der Browser aber nur zu, wenn der Dienst von Facebook explizit bestätigt, dass eine Anfrage von \texttt{localhost:3000} diese Aktion ausführen darf.

Dafür sendet der Browser eine OPTIONS-Anfrage, in der die Herkunft (``Origin``) der Anfrage notiert ist. Der Facebook Dienst antwortet daraufhin mit den entsprechenden CORS-Headern und gibt somit an, ob die Anfrage von dieser Origin aus erlaubt ist. Nun prüft der Browser, ob die von Facebook übermittelten Origins übereinstimmen. Ist dies nicht der Fall, wird die Anfrage im Browser blockiert und im JavaScript Kontext schlägt dieser fehl. Details zum Fehler werden der JavaScript Umgebung nicht verfügbar gemacht.
\end{quotation}

\subsection{Logdaten}

Ähnlich wie bei anderen Umgebungen gibt es eine standardisierte Log- bzw. Konsolenausgabe für die JavaScript Umgebung \cite{MDNConsole}. Diese Ausgabe ist aber für den Standard-Benutzer eher unbekannt und der Zugriff darauf sowie die Funktionen dessen können je nach Browser variieren. Deswegen kann in den meisten Fällen nicht darauf gehofft werden, dass die Nutzer dieses Log bereitstellen. Zusätzlich ist es durch die zuvor beschrieben Härtungsmaßnahmen von Browsern nicht möglich das Log in eine Datei zu schreiben.

Eine Automatisierung der Logdatenerhebung ist zudem auch nicht trivial, denn die Daten, welche in die Ausgabe geschrieben werden, sind nicht aus der JavaScript Umgebung aus lesbar. Alternativ können die Daten selber erhoben oder abgefangen werden. Hierbei besteht aber weiterhin die Hürde, wie die Daten an die Stakeholder gelangen.

\subsection{Fernzugriff}

Ein weiterer Punkt, der die Umgebung ``Browser`` von anderen unterscheidet, ist dass die Stakeholder sich normalerweise nicht auf die Systeme der Nutzer schalten können. Bei Expertenanwendungen ginge dies vielleicht, aber wenn eine Webapplikation für den offenen Markt geschaffen ist, sind die Nutzer zahlreich und unbekannt.

Weiterhin gibt es standardmäßig keine Funktionalität wie z.B. das Remote Application Debugging \cite{JavaDebugWireProtocol}, welches Java unterstützt.
