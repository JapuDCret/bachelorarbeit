% \textit{Hier soll beschrieben werden, was JavaScript-basierte Webapplikationen sind.}

% Diese Ausarbeitung konzentriert sich, wie im Titel beschrieben, auf JavaScript-basierte Webapplikationen. Weiterhin wird sich auf sogenannte Single-Page-Applications (SPAs) konzentriert, welche eine Submenge der JavaScript-basierten Webapplikationen darstellen. Um allen Lesern eine gleiche Grundkenntnis zu ermöglichen, werden diese Konzepte nun kurz vorgestellt.

%In diesem Abschnitt wird erläutert, mit welcher der Art von Anwendungen sich diese Arbeit beschäftigt und welche Eigenschaften diese besitzen.

\subsection{JavaScript-basierte Webapplikationen}

Eine JavaScript-basierte Webapplikation, ist eine Webapplikation, in der die Hauptfunktionalitäten über JavaScript realisiert werden. Dies umfasst unter anderem Interaktivität und dynamische Inhaltsdarstellung. Hierbei werden meist nur Grundgerüste in HTML und gegebenenfalls auch CSS bereitgestellt, und die eigentlichen Inhalte werden dynamisch mit JavaScript erstellt. Die Inhalte werden überwiegend über zusätzliche Schnittstellen der Webapplikation bereitgestellt.

\subsection{Single-Page-Applications}

Single-Page-Applications (SPAs) sind eine Submenge der JavaScript-basierten Webapplikationen und gehen bei der dynamischen Inhaltsdarstellung einen Schritt weiter. Logische Seiten werden nicht über eigene HTML-Dateien bereitgestellt, sondern dynamisch von der Anwendung aus erzeugt. Für das Bereitstellen einer solchen Applikation, ist daher nur ein simpler Webserver nötig und ein oder mehrere Dienste, von dem die SPA ihre Inhalte abrufen kann. Populäre Frameworks, um SPAs zu Erstellen, sind beispielsweise Angular \cite{AngularHomepage}, React \cite{ReactHomepage} oder Vue.js \cite{VueJSHomepage}.

In dieser Arbeit werden SPAs untersucht, denn einerseits fallen diese in das Interessengebiet und das aktuelle Projektumfeld der Open Knowledge, anderseits gibt es aber auch einige Eigenheiten, die die Nachvollziehbarkeit reduzieren. Beispielsweise gehen durch die starke Trennung von Client und Server auch Kontextinformationen verloren. Zudem wird die Applikation beim Client größer und komplexer, welches mit einem erhöhten Potenzial für Ungereimtheiten einher geht.