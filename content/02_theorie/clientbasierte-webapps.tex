% \textit{Hier soll beschrieben werden, was JavaScript-basierte Webanwendungen sind.}

% Diese Ausarbeitung konzentriert sich, wie im Titel beschrieben, auf JavaScript-basierte Webanwendungen. Weiterhin wird sich auf sogenannte Single-Page-Applications (SPAs) konzentriert, welche eine Submenge der JavaScript-basierten Webanwendungen darstellen. Um allen Lesern eine gleiche Grundkenntnis zu ermöglichen, werden diese Konzepte nun kurz vorgestellt.

%In diesem Abschnitt wird erläutert, mit welcher der Art von Anwendungen sich diese Arbeit beschäftigt und welche Eigenschaften diese besitzen.

\subsection{JavaScript-basierte Webanwendungen}

Eine JavaScript-basierte Webanwendung, ist eine Webanwendung, die in JavaScript realisiert wurde und bei jener der Browser als Laufzeitumgebung verwendet wird. Dies umfasst unter anderem Interaktivität und dynamische Inhaltsdarstellung. Hierbei werden meist nur Grundgerüste der Anwendung in HTML und CSS bereitgestellt, jedoch werden die eigentlichen Inhalte dynamisch mit JavaScript erstellt. Die Inhalte werden auch über zusätzliche Schnittstellen der Webanwendung zu Partnersystemen bereitgestellt.

\subsection{Single-Page-Applications}
\label{subsec:singe-page-applications}

Single-Page-Applications (SPAs) sind eine Teilmenge der JavaScript-basierten Webanwendungen und gehen bei der dynamischen Inhaltsdarstellung einen Schritt weiter. Die gesamte Anwendung wird über ein einziges HTML-Dokument und die darin referenzierten Inhalte erzeugt. Oftmals wird auf Basis dessen ein erheblicher Teil der Logik auf Clientseite umgesetzt, was die Anwendung zu einem Rich- bzw. Fat-Client macht.

Für das Bereitstellen einer solchen Anwendung, ist meist nur ein simpler Webserver ausreichend und ein oder mehrere Dienste, von dem aus die SPA ihre Inhalte abrufen kann. Populäre Frameworks sind beispielsweise Angular \cite{AngularHomepage}, React \cite{ReactHomepage} oder Vue.js \cite{VueJSHomepage} zur Realisierung von SPAs.

SPAs bieten zudem durch ihre clientbasierte Herangehensweise Stakeholdern die Möglichkeit, die Anwendung als Offline-Version bereitzustellen. Sind neben der Logik keine externen Daten notwendig oder wurden diese bereits abgerufen und gecached, so kann eine SPA auch \enquote{offline} von Benutzern verwendet werden. Weiterhin steigern SPAs die User Experience (UX), indem sie u. A. schneller agieren, da keine kompletten Seitenaufrufe notwendig sind \cite{ImprovementOfAcedemicServiceBasedOnSPA}.

Durch diesen grundsätzlich anderen Ansatz, gibt es aber auch negative Eigenschaften. Unter anderem werden native Browserfunktionen umgangen, wie die automatisch befüllte Browserhistorie, denn es werden keine neuen HTML-Dokumente angefragt. Weiterhin leiden \enquote{virtuelle} Verlinkungen und Buttons darunter, dass sie nicht alle Funktionen unterstützten, die normale HTML-Elemente aufweisen. Um dies und andere verwandte Probleme zu beheben, besitzen die zuvor genannten Frameworks spezielle Implementierungen oder ggf. muss eine zusätzliche Bibliothek herangezogen werden, wie z. B. die jeweiligen Router-Bibliotheken.

Nichtsdestotrotz ist ein jahrelanger Trend von der Einführung von Single-Page-Applications zu erkennen, heutzutage steht eine große Auswahl an erprobten Technologien zur Verfügung \cite{TheStateOfJavaScript2020}.

% In dieser Arbeit werden SPAs untersucht, denn einerseits fallen diese in das Interessengebiet der Open Knowledge, anderseits gibt es aber auch einige Eigenheiten, die die Nachvollziehbarkeit reduzieren. Beispielsweise gehen durch die starke Trennung von Client und Server auch Kontextinformationen verloren. Zudem wird die Anwendung beim Client größer und komplexer, welches das Potenzial von Ungereimtheiten erhöht.