% \textit{Hier soll beschrieben werden, was JavaScript-basierte Webapplikationen sind.}

% Diese Ausarbeitung konzentriert sich, wie im Titel beschrieben, auf JavaScript-basierte Webapplikationen. Weiterhin wird sich auf sogenannte Single-Page-Applications (SPAs) konzentriert, welche eine Submenge der JavaScript-basierten Webapplikationen darstellen. Um allen Lesern eine gleiche Grundkenntnis zu ermöglichen, werden diese Konzepte nun kurz vorgestellt.

%In diesem Abschnitt wird erläutert, mit welcher der Art von Anwendungen sich diese Arbeit beschäftigt und welche Eigenschaften diese besitzen.

\subsection{JavaScript-basierte Webapplikationen}

Eine JavaScript-basierte Webapplikation, ist eine Webapplikation, in der die Hauptfunktionalitäten über JavaScript realisiert werden. Dies umfasst unter anderem Interaktivität und dynamische Inhaltsdarstellung. Hierbei werden meist nur Grundgerüste in HTML und gegebenenfalls auch CSS bereitgestellt, und die eigentlichen Inhalte werden dynamisch mit JavaScript erstellt. Die Inhalte werden überwiegend über zusätzliche Schnittstellen der Webapplikation bereitgestellt.

\subsection{Single-Page-Applications}
\label{subsec:singe-page-applications}

Single-Page-Applications (SPAs) sind eine Submenge der JavaScript-basierten Webapplikationen und gehen bei der dynamischen Inhaltsdarstellung einen Schritt weiter. Die gesamte Anwendung wird über ein einziges HTML-Dokument und die darin referenzierten Inhalte erzeugt. Auf Basis dessen wird oftmals viel der Logik im Clientteil angesiedelt, welches die Anwendung in einen Rich- bzw. Fat-Client verwandelt.

Für das Bereitstellen einer solchen Applikation, ist unter anderem nur ein simpler Webserver notwendig und ein oder mehrere Dienste, von dem aus die SPA ihre Inhalte abrufen kann. Populäre Frameworks, um SPAs zu Erstellen, sind beispielsweise Angular \cite{AngularHomepage}, React \cite{ReactHomepage} oder Vue.js \cite{VueJSHomepage}.

Neben der Eigenschaft eine Alternative zu anderen Webapplikationen zu sein, bieten SPAs zudem den Stakeholdern die Möglichkeit diese als Offline-Version zur Verfügung zu stellen. Grund dafür ist, dass eine SPA bereits jedwede Logik enthält, sind zusätzlich keine externen Daten notwendig, so kann eine SPA simpel als Offline-Version bereitgestellt werden. Weiterhin steigern SPAs die User Experience (UX), indem sie unter anderem schneller agieren, da keine kompletten Seitenabrufe notwendig sind \cite{ImprovementOfAcedemicServiceBasedOnSPA}.

Durch diesen brachial anderen Ansatz, gibt es aber auch negative Eigenschaften. Unter anderem werden native Browserfunktionen umgangen, wie die automatisch befüllte Browserhistorie oder auch Aktionen zu Verlinkungen, wie Scrollrad-Klicks oder Lesezeichen, die mit ``virtuelle`` Links und Buttons nicht möglich sind. Dies ist zu Teilen auch bei JavaScript-basierten Webapplikationen der Fall. Um diese Funktionalitäten wiederherzustellen sind für die zuvor genannten Frameworks Angular, React und auch Vue.js zusätzliche Bibliotheken notwendig.

Nichtsdestotrotz kann ein jahrelanger Trend von der Adaption von Single-Page-Applications erkannt werden und eine große Auswahl an erprobten Technologien stehen heutzutage zur Verfügung \cite{StateofJS2019}.

% In dieser Arbeit werden SPAs untersucht, denn einerseits fallen diese in das Interessengebiet der Open Knowledge, anderseits gibt es aber auch einige Eigenheiten, die die Nachvollziehbarkeit reduzieren. Beispielsweise gehen durch die starke Trennung von Client und Server auch Kontextinformationen verloren. Zudem wird die Applikation beim Client größer und komplexer, welches das Potenzial von Ungereimtheiten erhöht.