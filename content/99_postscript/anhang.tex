\section{Studien zur Browserkompatibilität}
\label{sec:studien-zur-browser-kompatibilitaet}

Im \autoref{sec:browserumgebung} wurde die Anzahl an Studien zur Browserkompatiblität dargestellt. Die Daten hierfür wurden über die Literatursuchmaschine \enquote{Google Scholar} am 25.02.2021 abgerufen. Dabei wurde für jedes Jahr eine eigene Suche durchgeführt und die Ergebnisanzahl notiert, wobei jede Suche eingeschränkt wurde, dass das Veröffentlichungsdatum im jew. Jahr lag. Für die Suche wurde folgender Suchterm benutzt:
\begin{verbatim}
"cross browser" compatibility|incompatibility|inconsistency|XBI
\end{verbatim}

\begin{wraptable}[10]{r}{0.42\linewidth}
\centering
\vspace{-\baselineskip}
\begin{tabular}{|l|l|}
  \hline
  Jahr & Suchtreffer \\
  \hline
  2015 & 272 \\
  \hline
  2016 & 228 \\
  \hline
  2017 & 208 \\
  \hline
  2018 & 204 \\
  \hline
  2019 & 172 \\
  \hline
  2020 & 167 \\
  \hline
\end{tabular}
\caption{Suchtreffer zu Studien über Browserkompatibilität}
	\label{tab:suchtreffer-metastudie}
\end{wraptable}

\def\lc{\left\lceil}   
\def\rc{\right\rceil}

Die jahresbezogene Trefferanzahl (vgl. \autoref{tab:suchtreffer-metastudie}) soll aufdecken, ob ein Trend in der Literatur zu erkennen ist. Ein schwacher, aber vorhandener Negativtrend konnte festgestellt werden.

Weiterhin lässt sich dieselbe Thematik bei Google Trends \cite{GoogleTrendsCrossBrowserCompatibility} untersuchen. Hierbei wurden die Suchtrends für den Suchterm \texttt{cross browser compatability} abgerufen und geplotted (vgl. \autoref{fig:google-search-trends_cross-browser-compatability}). Dabei lässt sich ebenso ein Negativtrend erkennen.

\begin{figure}[H]
	\centering
	\includegraphics[width=0.84\linewidth]{img/99_postscript/google-trends_cross-browser-compatability.png}
	\caption{Google Trends zur Browserkompatibilität, angereichert mit \cite{MicrosoftIEandEdgeLifecycleFAQ}}
	\label{fig:google-search-trends_cross-browser-compatability}
\end{figure}

\section{Weitere Demonstrationen}

\subsection{LogRocket}
\label{sec:demo-logrocket}

Eine Aufzeichnung des Session-Replays von LogRocket zu einer beispielhaften Sitzung, kann unter \texttt{Anhang/session-replay-via-logrocket.mp4} betrachtet werden. Das eigentliche Session-Replay erfolgt interaktiv im Browser und Daten wie der DOM, die Konsole und auch Netzwerkaufrufe können vom Entwickler inspiziert werden.

\subsection{Splunk}
\label{sec:demo-splunk}

Neben Logs wurden in Splunk zudem auch Metriken und Fehler erhoben. Bei den aufgezeichneten Metriken handelt es sich um Beispiele für den in der Demo umgesetzten Anwendungsfall: Die Anzahl an Produkten im Warenkorb und die aufgetretenen Fehler in einer Sitzung. Die Metriken wurden auf einer Seite in Splunk visuell dargestellt, welche in \autoref{fig:splunk_metric-overview} zu sehen ist.

\begin{figure}[H]
	\centering
	\includegraphics[width=1.00\linewidth]{img/99_postscript/splunk_metric-overview.png}
	\caption{Metrikübersicht in Splunk}
	\label{fig:splunk_metric-overview}
\end{figure}

Um Fehler zu visualisieren und gruppiert darzustellen, wurde in Splunk eine eigene Seite erstellt, die in \autoref{fig:splunk_error-dashboard} zu betrachten ist. In der Fehlerübersicht lassen sich ein Histogramm, ein Top-Chart sowie die letzten 50 Events im Detail betrachten. Durch diese Seite lassen sich bspw. erhöhte Fehleraufkommen feststellen und die Häufigkeit von unterschiedlichen Fehlern vergleichen.

\begin{figure}[H]
	\centering
	\includegraphics[width=1.00\linewidth]{img/99_postscript/splunk_error-dashboard.png}
	\caption{Fehlerübersicht in Splunk}
	\label{fig:splunk_error-dashboard}
\end{figure}

\subsection{Jaeger}
\label{sec:demo-jaeger}

In Jaeger lassen sich neben dem Trace-Gantt-Diagramm auch Traces miteinander vergleichen, sodass Unterschiede in den Aufrufen entdeckt werden können. In den Abbildungen \ref*{fig:jaeger_comparison-all} und \ref*{fig:jaeger_comparison-zoom} ist ein solcher Vergleich zu sehen. Hierbei werden Traces verglichen, die durch eine Anfrage von Warenkorb- und Übersetzungsdaten von Frontend aus resultierten. Dabei divergieren die beiden Traces in der Antwort vom Übersetzungsdienst, denn es wird bei einem Trace dort ein Fehler geworfen.

\begin{figure}[H]
	\centering
	\includegraphics[width=1.00\linewidth]{img/99_postscript/jaeger_comparison-all.png}
	\caption{Tracevergleich in Jaeger}
	\label{fig:jaeger_comparison-all}
\end{figure}

\begin{figure}[H]
	\centering
	\includegraphics[width=1.00\linewidth]{img/99_postscript/jaeger_comparison-zoom.png}
	\caption{Tracevergleich in Jaeger (Zoom)}
	\label{fig:jaeger_comparison-zoom}
\end{figure}

Der von Jaeger automatisch erstellte Dienst-Abhängigkeits-Graph kann folgend in \autoref{fig:jaeger_service-dependency-graph} betrachtet werden.

\begin{figure}[H]
	\centering
	\includegraphics[width=1.00\linewidth]{img/99_postscript/jaeger_service-dependency-graph.png}
	\caption{Dienst-Abhängigkeits-Graph in Jaeger}
	\label{fig:jaeger_service-dependency-graph}
\end{figure}