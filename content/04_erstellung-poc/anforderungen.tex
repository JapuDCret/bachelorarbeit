%\section{Anforderungen}
%\textit{Hier soll definiert werden, welche Kriterien der PoC erfüllen soll und wenn möglich sollen zusätzlich Möglichkeiten zur Überprüfung dieser Kriterien definiert werden.}
	
\subsection{Definitionen}
	
Um die Anforderungen systematisch zu kategorisieren, werden folgend zwei Modelle vorgestellt, mit denen die Anforderungen kategorisiert werden.

Beim ersten handelt es sich um das Kano-Modell \cite{KanoModell} der Kundenzufriedenheit kategorisiert (vgl. \autoref{tab:merkmale-nach-dem-kano-modell}).
	
\begin{table}[H]
\begin{tabular}{ |p{1.2cm}|p{2.75cm}|p{9.55cm}| }
	\hline
	Kürzel & Titel & Beschreibung \\
	\hline
	\textbf{B} & Basis\-merkmal & Merkmale, die als selbstverständlich angesehen werden. Eine Erfüllung erhöht kaum die Zufriedenheit, jedoch eine Nichterfüllung führt zu starker Unzufriedenheit \\
	\hline
	\textbf{L} & Leistungs\-merkmal & Merkmale, die der Kunde erwartet und bei nicht Vorhandensein in Unzufriedenheit äußert. Ein Vorhandensein erzeugt Zufriedenheit, beim übertreffen umso mehr. \\
	\hline
	\textbf{S} & Begeisterungs\-merkmal & Merkmale, mit der man sich von der Konkurrenz herabsetzen kann und die den Nutzenfaktor steigern. Sind sie vorhanden, steigern sie die Zufriedenheit merklich. \\
	\hline
	\textbf{U} & Unerhebliches Merkmal & Für den Kunden belanglos, ob vorhanden oder nicht \\
	\hline
	\textbf{R} & Rückweisungs\-merkmal & Diese Merkmale führen bei Vorhandensein zu Unzufriedenheit, sind jedoch beim Fehlen unerheblich \\
	\hline
\end{tabular}
 \captionsetup{justification=centering}
  \caption{Merkmale nach dem Kano-Modell der Kundenzufriedenheit}
   \label{tab:merkmale-nach-dem-kano-modell}
\end{table}

Neben der Unterscheidung nach dem Kano-Modell werden die Anforderungen in funktionale und nicht-funktionale Anforderungen \cite{FunktionaleUndNichtFunktionaleAnforderungen} aufgeteilt (vgl. \autoref{tab:kategorien-der-anforderungsliste}).

\begin{table}[H]
\begin{tabular}{ |p{1.25cm}|p{3cm}|p{9.25cm}| }
	\hline
	Kürzel & Titel & Beschreibung \\
	\hline
	f & funktional & Beschreiben Anforderungen, welche ein Produkt ausmachen und von anderen differenzieren (``Was soll das Produkt können?{``}). Sie sind sehr spezifisch für das jeweilige Produkt. Ein Beispiel: Das Frontend fragt Daten für X vom Partnersystem\#1 über eine SOAP-API ab, etc.\\
	\hline
	nf & nicht-funktional & Beschreiben Leistungs- sowie Qualitätsanforderungen und Randbedingungen (``Wie soll das Produkt sich verhalten?{``}). Sie sind meist unspezifisch und in gleicher Form auch in unterschiedlichsten Produkten vorzufinden. Beispiele sind: Benutzbarkeit, Verfügbarkeit, Antwortzeit, etc. Zur Überprüfung sind oftmals messbare, vergleichbare und reproduzierbare Definitionen notwendig. \\
	\hline
\end{tabular}
 \captionsetup{justification=centering}
  \caption{Kategorien der Anforderungen}
   \label{tab:kategorien-der-anforderungsliste}
\end{table}
	
\subsection{Anforderungsanalyse}

\textit{Wie werden die Anforderungen erhoben?}

Die primäre Quelle von Anforderungen, stellen die Stakeholder selbst dar. Die Stakeholder dieser Arbeit sind Christian Wansart und Stephan Müller von Open Knowledge. Sie betreuen die Arbeit und haben ein starkes Interesse, dass ein sinnvolles und übertragbares Ergebnis aus der Arbeit entspringt, um es z.B. bei Kunden anwenden zu können.

\textit{{\color{red}Hier: Erstellung von einer Klassifizierung der Anforderungsquellen}}
	
\subsection{Anforderungsliste}

\textit{Was sind die Anforderungen?}

\begin{table}[H]
\begin{tabular}{ |p{1.25cm}|p{5.5cm}|p{2.25cm}|p{2.1cm}|p{1.25cm}| }
\hline
Id           & Name         & Kano-Modell  & Funktionsart & Quelle       \\
\textit{1234} & \textit{Dummy} & \textit{Dummy} & \textit{nf} & \textit{S} \\
\hline
\multicolumn{5}{|l|}{\textit{Hier soll eine Beschreibung hin.}} \\
\hline
\end{tabular}
\end{table}