%\section{Anforderungen}
%\textit{Hier soll definiert werden, welche Kriterien der PoC erfüllen soll und wenn möglich sollen zusätzlich Möglichkeiten zur Überprüfung dieser Kriterien definiert werden.}

Das zu erstellende Proof-of-Concept soll einige Rahmenbedingungen erfüllen. In diesem Abschnitt werden diese Bedingungen näher beschrieben.
	
\subsection{Definitionen}
	
Um die Anforderungen systematisch einzuordnen, werden sie auf Basis von zwei Modellen kategorisiert, welche folgend vorgestellt werden.

Beim ersten Modell handelt es sich um das Kano-Modell \cite{KanoModell} der Kundenzufriedenheit, welches in  \autoref{tab:merkmale-nach-dem-kano-modell} erläutert wird.

\begin{table}[H]
\begin{tabular}{ |p{1.15cm}|p{2.75cm}|p{9.6cm}| }
	\hline
	Kürzel & Titel & Beschreibung \\
	\hline
	\textbf{B} & Basis\-merkmal & Merkmale, die als selbstverständlich angesehen werden. Eine Erfüllung erhöht kaum die Zufriedenheit, jedoch eine Nichterfüllung führt zu starker Unzufriedenheit \\
	\hline
	\textbf{L} & Leistungs\-merkmal & Merkmale, die der Kunde erwartet und bei nicht Vorhandensein in Unzufriedenheit äußert. Ein Vorhandensein erzeugt Zufriedenheit, beim Übertreffen umso mehr. \\
	\hline
	\textbf{S} & Begeisterungs\-merkmal & Merkmale, die eine Herabsetzung von der Konkurrenz ermöglichen und die den Nutzenfaktor steigern. Sind sie vorhanden, steigern sie die Zufriedenheit merklich. \\
	\hline
	\textbf{U} & Unerhebliches Merkmal & Für den Kunden belanglos, ob vorhanden oder nicht. \\
	\hline
	\textbf{R} & Rückweisungs\-merkmal & Diese Merkmale führen bei Vorhandensein zu Unzufriedenheit, sind jedoch beim Fehlen unerheblich. \\
	\hline
\end{tabular}
 \captionsetup{justification=centering}
  \caption{Merkmale nach dem Kano-Modell der Kundenzufriedenheit}
   \label{tab:merkmale-nach-dem-kano-modell}
\end{table}

Neben der Unterscheidung nach dem Kano-Modell werden die Anforderungen in funktionale und nicht-funktionale Anforderungen \cite{FunktionaleUndNichtFunktionaleAnforderungen} aufgeteilt (vgl. \autoref{tab:kategorien-der-anforderungsliste}).

\begin{table}[H]
\begin{tabular}{ |p{1.15cm}|p{2.75cm}|p{9.6cm}| }
	\hline
	Kürzel & Titel & Beschreibung \\
	\hline
	f & funktional & Beschreiben Anforderungen, welche ein Produkt ausmachen und von anderen differenzieren (\enquote{Was soll das Produkt können?}). Sie sind sehr spezifisch für das jeweilige Produkt. Ein Beispiel: Das Frontend fragt Daten für X vom Partnersystem 1 über eine SOAP-API ab, etc.\\
	\hline
	nf & nicht-funktional & Beschreiben Leistungs- und Qualitätsanforderungen und Randbedingungen (\enquote{Wie soll das Produkt sich verhalten?}). Sie sind meist unspezifisch und in gleicher Form auch in unterschiedlichsten Produkten vorzufinden. Beispiele sind: Benutzbarkeit, Verfügbarkeit, Antwortzeit, etc. Zur Überprüfung sind oftmals messbare, vergleichbare und reproduzierbare Definitionen notwendig. \\
	\hline
\end{tabular}
 \captionsetup{justification=centering}
  \caption{Kategorien der Anforderungen}
   \label{tab:kategorien-der-anforderungsliste}
\end{table}
	
\subsection{Anforderungsanalyse}

Die Anforderungen, welche von der zu erstellende Lösung gefordert werden, ergaben sich durch den Einfluss verschiedener Quellen. Die primäre Quelle an Anforderungen stellen die Stakeholder dieser Arbeit, Christian Wansart und Stephan Müller, dar. Als Stakeholder betreuen sie die Arbeit und haben ein eigenes Interesse, dass aus der Arbeit ein erfolgreiches und übertragbares Ergebnis resultiert.

Neben den Stakeholdern ergeben sich auch Anforderungen direkt aus der Forschungsfrage selbst und den Bestrebungen des Autors. Die Quellen werden in den Anforderungen mit einem Kürzel angegeben, wie z. B. \texttt{A} für Autor, zu sehen in \autoref{tab:quellen-der-anforderungen}.

Eine dritte Quelle von Anforderungen ergibt sich aus der Problemstellung des Kunden der Open Knowledge, welche in der Motivation angesprochen wurde. Die beiden Stakeholder brachten neben ihren eigenen Bestrebungen auch die Rahmenbedingungen und Wünsche des Kunden mit ein. Aus dieser Kommunikation ergaben sich somit weitere Anforderungen, welche einen realitätsnahen Charakter haben.

Anforderungen können auch eine Kombination von mehreren Quellen besitzen, wenn die Anforderung aus einer gemeinsamen Bestrebung oder Diskussion entstand.

\begin{table}[H]
\begin{tabular}{ |p{1.15cm}|p{1.9cm}|p{10.45cm}| }
	\hline
	Kürzel & Titel & Beschreibung \\
	\hline
	A & Autor & Hiermit ist der Autor dieser Arbeit gemeint. \\
	\hline
	S & Stakeholder & Die beiden Stakeholder Christian Wansart und Stephan Müller \\
	\hline
	K & Kunde & Ein Kunde der Open Knowledge, ein Direktversicherer. \\
	\hline
\end{tabular}
 \captionsetup{justification=centering}
  \caption{Quellen der Anforderungen}
   \label{tab:quellen-der-anforderungen}
\end{table}
	
\subsection{Anforderungsliste}

Um die Anforderungen strukturiert zu erfassen, werden sie ähnlich einer Karteikarte, wie in \autoref{tab:beispiel-anforderung} zu sehen, dargestellt. Hierbei erhält jede Anforderung eine Kategorisierung nach dem Kano-Modell, ob sie funktional oder nicht-funktional ist und aus welcher Anforderungsquelle sie entstammt. Jede Anforderung erhält zudem eine eindeutige Id, die nachfolgend in der Arbeit zur Referenzierung dient.

\begin{table}[H]
\begin{tabular}{ |p{1.25cm}|p{5.5cm}|p{2.25cm}|p{2.1cm}|p{1.25cm}| }
\hline
Id            & Name          & Kano-Modell   & Funktionsart  & Quelle        \\
\textit{1234} & \textit{Dummy} & \textit{S} & \textit{nf} & \textit{S} \\
\hline
\multicolumn{5}{|l|}{\textit{Hier wird die Anforderung beschrieben.}} \\
\hline
\end{tabular}
 \captionsetup{justification=centering}
  \caption{Beispiel einer Anforderung}
   \label{tab:beispiel-anforderung}
\end{table}

% use generated list

% textidote: ignore begin

\subsubsection{Grundanforderungen}


\subsubsection{Funktionsumfang}


\begin{anf}{anf:1010}{1010}{Schnittstellen-Logging}{B}{f}{S}
\multicolumn{5}{|p{14.05cm}|}{Das Aufrufen von Schnittstellen soll mittels einer Logmeldung notiert werden. Hierbei sollen relevante Informationen wie Aufrufparameter mit notiert werden.} \\
\hline
\end{anf}

\begin{anf}{anf:1011}{1011}{Use-Case-Logging}{B}{f}{S}
\multicolumn{5}{|p{14.05cm}|}{Tritt ein Use-Case auf, soll dieser im Log notiert werden. Beispielsweise soll notiert werden, dass ein Nutzer einen neuen Datensatz angelegt möchte und wenn der diesen anlegt.} \\
\hline
\end{anf}

\begin{anf}{anf:1100}{1100}{Logübertragung}{B}{f}{S}
\multicolumn{5}{|p{14.05cm}|}{Ausgewählte Logmeldungen sollen an ein Partnersystem weitergeleitet werden. Die Auswahl könnte über die Kritikalität, also dem Log-Level, der Logmeldung erfolgen.} \\
\hline
\end{anf}

\subsubsection{Eigenschaften}


\begin{anf}{anf:2010}{2010}{Resilienz der Übertragung}{S}{f}{S}
\multicolumn{5}{|p{14.05cm}|}{Daten die der Nachvollziehbarkeit dienen, sollen wenn möglich bei einer fehlgeschlagenen Verbindung nicht verworfen werden. Sie sollen mindestens 120s vorgehalten werden und in dieser Zeit sollen wiederholt Verbindungsversuche unternommen werden.} \\
\hline
\end{anf}

\begin{anf}{anf:2020}{2020}{Batchverarbeitung}{S}{f}{S}
\multicolumn{5}{|p{14.05cm}|}{Daten, die der Nachvollziehbarkeit dienen, sollen wenn möglich gruppiert an externe Systeme gesendet werden. Hierbei ist eine kurze Aggregationszeit von bis zu 10s akzeptabel.} \\
\hline
\end{anf}

\subsubsection{Partnersysteme}

% textidote: ignore end