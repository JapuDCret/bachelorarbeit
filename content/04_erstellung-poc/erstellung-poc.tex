% \chapter{Beispielhafte Integration}
	
\section{Kriterien}

	\textit{Hier soll definiert werden, welche Kriterien der PoC erfüllen soll und wenn möglich sollen zusätzlich Möglichkeiten zur Überprüfung dieser Kriterien definiert werden.}
	
\section{Vorstellung der Demoanwendung}

	\textit{In diesem Abschnitt soll die Demoanwendung vorgestellt werden, anhand dessen das Proof-of-Concept erstellt wird. Damit das Proof-of-Concept erstellt werden kann, muss die Demoanwendung die zuvor beschriebenen Probleme aufweisen, hierbei sollen die Probleme möglichst realitätsnah sein und nicht frei erfunden.}
	
\section{Konzept}
	
	\subsection{Architektur}

	\textit{Hier soll die grobe Architektur geplant werden, welche Komponente es gibt und wie diese kommunizieren sollen.}
	
	\subsection{Datenverarbeitung}
		
		\subsubsection{Erhebung}
		\textit{Wie werden die Daten erhoben (Nennung der verwendeten Methoden!)?}
		\textit{Wie gelangen die Daten an eine auswertende Komponente?}
		
		\subsubsection{Auswertung}
		\textit{Wie werden die Daten zusammengefasst und ausgewertet?}
		\textit{Wie gelangt das Ergebnis an die darstellende Komponente?}
		
		\subsubsection{Visualisierung}
		\textit{Wie werden den Stakeholdern die Informationen präsentiert?}

\section{Implementierung}

	\textit{Auf Basis des Konzeptes soll nun eine Implementierung erfolgen.}

	\subsection{Technologie-Stack}