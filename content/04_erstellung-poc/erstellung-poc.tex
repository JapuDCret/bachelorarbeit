% \chapter{Beispielhafte Integration}
	
\section{Anforderungen}
%\section{Anforderungen}
%\textit{Hier soll definiert werden, welche Kriterien der PoC erfüllen soll und wenn möglich sollen zusätzlich Möglichkeiten zur Überprüfung dieser Kriterien definiert werden.}
	
\subsection{Definitionen}
	
Um die Anforderungen systematisch zu kategorisieren, werden folgend zwei Modelle vorgestellt, mit denen die Anforderungen kategorisiert werden.

Beim ersten handelt es sich um das Kano-Modell \cite{KanoModell} der Kundenzufriedenheit kategorisiert (vgl. \autoref{tab:merkmale-nach-dem-kano-modell}).
	
\begin{table}[H]
\begin{tabular}{ |p{1.2cm}|p{2.75cm}|p{9.55cm}| }
	\hline
	Kürzel & Titel & Beschreibung \\
	\hline
	\textbf{B} & Basis\-merkmal & Merkmale, die als selbstverständlich angesehen werden. Eine Erfüllung erhöht kaum die Zufriedenheit, jedoch eine Nichterfüllung führt zu starker Unzufriedenheit \\
	\hline
	\textbf{L} & Leistungs\-merkmal & Merkmale, die der Kunde erwartet und bei nicht Vorhandensein in Unzufriedenheit äußert. Ein Vorhandensein erzeugt Zufriedenheit, beim übertreffen umso mehr. \\
	\hline
	\textbf{S} & Begeisterungs\-merkmal & Merkmale, mit der man sich von der Konkurrenz herabsetzen kann und die den Nutzenfaktor steigern. Sind sie vorhanden, steigern sie die Zufriedenheit merklich. \\
	\hline
	\textbf{U} & Unerhebliches Merkmal & Für den Kunden belanglos, ob vorhanden oder nicht \\
	\hline
	\textbf{R} & Rückweisungs\-merkmal & Diese Merkmale führen bei Vorhandensein zu Unzufriedenheit, sind jedoch beim Fehlen unerheblich \\
	\hline
\end{tabular}
 \captionsetup{justification=centering}
  \caption{Merkmale nach dem Kano-Modell der Kundenzufriedenheit}
   \label{tab:merkmale-nach-dem-kano-modell}
\end{table}

Neben der Unterscheidung nach dem Kano-Modell werden die Anforderungen in funktionale und nicht-funktionale Anforderungen \cite{FunktionaleUndNichtFunktionaleAnforderungen} aufgeteilt (vgl. \autoref{tab:kategorien-der-anforderungsliste}).

\begin{table}[H]
\begin{tabular}{ |p{1.25cm}|p{3cm}|p{9.25cm}| }
	\hline
	Kürzel & Titel & Beschreibung \\
	\hline
	f & funktional & Beschreiben Anforderungen, welche ein Produkt ausmachen und von anderen differenzieren (``Was soll das Produkt können?{``}). Sie sind sehr spezifisch für das jeweilige Produkt. Ein Beispiel: Das Frontend fragt Daten für X vom Partnersystem\#1 über eine SOAP-API ab, etc.\\
	\hline
	nf & nicht-funktional & Beschreiben Leistungs- sowie Qualitätsanforderungen und Randbedingungen (``Wie soll das Produkt sich verhalten?{``}). Sie sind meist unspezifisch und in gleicher Form auch in unterschiedlichsten Produkten vorzufinden. Beispiele sind: Benutzbarkeit, Verfügbarkeit, Antwortzeit, etc. Zur Überprüfung sind oftmals messbare, vergleichbare und reproduzierbare Definitionen notwendig. \\
	\hline
\end{tabular}
 \captionsetup{justification=centering}
  \caption{Kategorien der Anforderungen}
   \label{tab:kategorien-der-anforderungsliste}
\end{table}
	
\subsection{Anforderungsanalyse}

\textit{Wie werden die Anforderungen erhoben?}

Die primäre Quelle von Anforderungen, stellen die Stakeholder selbst dar. Die Stakeholder dieser Arbeit sind Christian Wansart und Stephan Müller von Open Knowledge. Sie betreuen die Arbeit und haben ein starkes Interesse, dass ein sinnvolles und übertragbares Ergebnis aus der Arbeit entspringt, um es z.B. bei Kunden anwenden zu können.

\textit{{\color{red}Hier: Erstellung von einer Klassifizierung der Anforderungsquellen}}
	
\subsection{Anforderungsliste}

\textit{Was sind die Anforderungen?}

\begin{table}[H]
\begin{tabular}{ |p{1.25cm}|p{5.5cm}|p{2.25cm}|p{2.1cm}|p{1.25cm}| }
\hline
Id           & Name         & Kano-Modell  & Funktionsart & Quelle       \\
\textit{1234} & \textit{Dummy} & \textit{Dummy} & \textit{nf} & \textit{S} \\
\hline
\multicolumn{5}{|l|}{\textit{Hier soll eine Beschreibung hin.}} \\
\hline
\end{tabular}
\end{table}
	
\section{Vorstellung der Demoanwendung}
%\section{Anforderungen}
%\textit{In diesem Abschnitt soll die Demoanwendung vorgestellt werden, anhand dessen das Proof-of-Concept erstellt wird. Damit das Proof-of-Concept erstellt werden kann, muss die Demoanwendung die zuvor beschriebenen Probleme aufweisen, hierbei sollen die Probleme möglichst realitätsnah sein und nicht frei erfunden.}

Wie in \autoref{anf:1020} beschrieben, ist eine Demoanwendung zu erstellen, auf Basis dessen das Konzept anzuwenden ist und somit praktisch umgesetzt werden kann. Dieser Abschnitt beschäftigt sich mit der Vorstellung der Demoanwendung und der repräsentativen Aufgabe, die diese übernimmt.

In der Motivation wurde ein konkretes Problem eines Kunden der Open Knowledge genannt. Damit die Demoanwendung realistisch eine , wird sie in Grundzügen die Webanwendung des Direktversicherers nachahmen. Bei der Webanwendung handelt es sich um eine mit Angular erstellte SPA, die den Nutzer verschiedene teils dynamische Formulare ausfüllen lässt und am Ende diese Daten an einen Dienst sendet und ein Ergebnis erhält, welches dargestellt wird. Während der Eingabe der Formulare werden einzelne Werte gegen Dienste validiert. Um die gewünschte Demoanwendung zu definieren, wird im folgenden Abschnitt das Verhalten festgelegt.

\subsection{Verhaltensdefinition}

Mit den beiden Stakeholdern, also Christian Wansart und Stephan Müller, die beide am Projekt für den Kunden involviert sind, wurde diese Verhaltensdefinition erstellt. Diesen Ansatz der Definition der Software anhand des Verhaltens nennt man Behavior Driven Development (BDD). Um die BDD-Definition festzuhalten wurde sie in der gängigen Gherkin \cite{Gherkin} Syntax geschrieben. Die Syntax ist natürlich zu lesen, und folgend werden alle gewünschen Features der Demoanwendung in der Gherkin-Syntax beschrieben.

\lstinputlisting[
  language = gherkin,
   caption = Demoanwendung: Gherkin Definition zum Feature \enquote{Warenkorb},
captionpos = b,
     label = lst:demoanwendung-gherkin-warenkorb
]{content/04_erstellung-poc/warenkorb-gherkin/1-warenkorb.feature}

\lstinputlisting[
  language = gherkin,
   caption = Demoanwendung: Gherkin Definition zum Feature \enquote{Rechnungsadresse},
captionpos = b,
     label = lst:demoanwendung-gherkin-rechnungsadresse
]{content/04_erstellung-poc/warenkorb-gherkin/2-rechnungsadresse.feature}

\lstinputlisting[
  language = gherkin,
   caption = Demoanwendung: Gherkin Definition zum Feature \enquote{Lieferadresse},
captionpos = b,
     label = lst:demoanwendung-gherkin-lieferadresse
]{content/04_erstellung-poc/warenkorb-gherkin/3-lieferadresse.feature}

\lstinputlisting[
  language = gherkin,
   caption = Demoanwendung: Gherkin Definition zum Feature \enquote{Zahlungsdaten},
captionpos = b,
     label = lst:demoanwendung-gherkin-zahlungsdaten
]{content/04_erstellung-poc/warenkorb-gherkin/4-zahlungsdaten.feature}

\lstinputlisting[
  language = gherkin,
   caption = Demoanwendung: Gherkin Definition zum Feature \enquote{Bestellung abschließen},
captionpos = b,
     label = lst:demoanwendung-gherkin-rechnungsadresse
]{content/04_erstellung-poc/warenkorb-gherkin/5-bestellung_abschließen.feature}

\subsection{Frontend}

\begin{figure}[H]
	\centering
	\includegraphics[width=0.75\linewidth]{img/04_erstellung-poc/demoanwendung_vorstellung_01-warenkorb.png}
	\caption{Demoanwendung: Startseite \enquote{Warenkorb}}
	\label{fig:demoanwendung_vorstellung_01-warenkorb}
\end{figure}

\begin{figure}[H]
	\centering
	\includegraphics[width=0.75\linewidth]{img/04_erstellung-poc/demoanwendung_vorstellung_02-rechnungsadresse.png}
	\caption{Demoanwendung: Seite \enquote{Rechnungsadresse}}
	\label{fig:demoanwendung_vorstellung_02-rechnungsadresse}
\end{figure}

\begin{figure}[H]
	\centering
	\includegraphics[width=0.75\linewidth]{img/04_erstellung-poc/demoanwendung_vorstellung_03-lieferdaten.png}
	\caption{Demoanwendung: Seite \enquote{Lieferdaten}}
	\label{fig:demoanwendung_vorstellung_03-lieferdaten}
\end{figure}

\begin{figure}[H]
	\centering
	\includegraphics[width=0.75\linewidth]{img/04_erstellung-poc/demoanwendung_vorstellung_04-zahlungsdaten.png}
	\caption{Demoanwendung: Seite \enquote{Zahlungsdaten}}
	\label{fig:demoanwendung_vorstellung_04-zahlungsdaten}
\end{figure}

\begin{figure}[H]
	\centering
	\includegraphics[width=0.75\linewidth]{img/04_erstellung-poc/demoanwendung_vorstellung_05-bestelluebersicht.png}
	\caption{Demoanwendung: Seite \enquote{Bestellübersicht}}
	\label{fig:demoanwendung_vorstellung_05-bestelluebersicht}
\end{figure}

\begin{figure}[H]
	\centering
	\includegraphics[width=0.75\linewidth]{img/04_erstellung-poc/demoanwendung_vorstellung_06-bestellbestaetigung.png}
	\caption{Demoanwendung: Finale Seite \enquote{Bestellbestätigung}}
	\label{fig:demoanwendung_vorstellung_06-bestellbestaetigung}
\end{figure}

\subsection{Backend}

\begin{figure}[H]
	\centering
	\includegraphics[width=1.0\linewidth]{img/04_erstellung-poc/demoanwendung_k8s-deployment.png}
	\caption{Demoanwendung: Kubernetes-Architektur-Diagramm, Quelle: Eigene Darstellung}
	\label{fig:demoanwendung_k8s-deployment}
\end{figure}
	
% \newpage
	
\section{Konzept}
	
	\subsection{Datenverarbeitung}

	Auf Basis der zuvor vorgestellten Methoden und Praktiken wird nun eine sinnvolle Kombination für das Frontend konzeptioniert, die als Ziel hat, die Nachvollziehbarkeit nachhaltig zu erhöhen. Es werden die Grunddisziplinen Datenerhebung, -auswertung und -präsentation unterschieden und nacheinander beschrieben. Danach und darauf aufbauend wird eine grobe Architektur vorgestellt, die diese Ansätze in ein Gesamtbild bringt.
		
	\subsubsection{Erhebung}
	
	Wie zuvor in \autoref{sec:nachvollziehbarkeit-bei-spas} beschrieben, erhalten Betreiber und Entwickler im Normalfall nur unzureichende Information über das Anwendungsverhalten oder die getätigten Nutzerinteraktionen bei einer SPA. Aus diesem Grund sollen explizit weitere Daten erhoben werden, um die Nachvollziehbarkeit zu erhöhen.
	
	Wie aus den Erkenntnissen von FAME (\autoref{sec:fame}) und Kaiju (\autoref{sec:kaiju}) zu deuten ist, gibt es durch die Verknüpfung von verschiedenen Datenkategorien einen Mehrwert für die Verständnis von Betreibern und Entwicklern. Deshalb sollen in der Lösung die 4 Datenkategorien \enquote{Logs}, \enquote{Metriken}, \enquote{Traces} und \enquote{Fehler} erhoben und an Partnersysteme weitergeleitet werden.
	
	Neben diesen Daten sollen auch die Benutzerinteraktionen aufgezeichnet werden. Hierfür soll jedoch kein tiefer gehendes Real-User-Monitoring zur Verwendung kommen, stattdessen soll ein Session-Replay-Mechanismus eingesetzt werden. RUM wird nicht gefordert, da es für die Verständnisgewinnung der Benutzerinteraktionen weniger aussagekräftig ist als Session-Replay. Bei Session-Replay werden die Benutzerinteraktionen im Kontext dargestellt und nicht gesondert oder abstrahiert, was für eine Nachvollziehbarkeit hinderlich sein kann. Beim Session-Replay wird jedoch jedwede Ein- und Ausgaben der Webanwendung aufgezeichnet, damit dies nicht ein zu großes Datenvolumen erzeugt und um den Nutzer nicht konstant zu überwachen, sollte sie standardmäßig abgeschaltet sein und nur auf expliziten Nutzerwunsch aktiviert werden.
	
	\vspace{-0.75\baselineskip}
	\subsubsection{Auswertung}
	\vspace{-0.50\baselineskip}
	
	Die genaue Auswertung ist Teil der Implementierung und diese Disziplin sieht keine direkten Vorgaben vor. Jedoch sind die gemeldeten Daten mit Kontextinformationen anzureichern, wenn möglich. Diese umfassen bspw. Zeitstempel, User-Agent, IP, Browser.
	
	\vspace{-0.75\baselineskip}
	\subsubsection{Präsentation}
	\vspace{-0.50\baselineskip}
	
	Um ein zufriedenstellendes Ergebnis zu gewährleisten, sollte die Lösung die Daten auf folgende Weise den Betreibern und Entwicklern präsentieren:
	
	\begin{enumerate}
		\item Logdaten..
		\begin{enumerate}
			\item ..lassen sich einsehen.
			\item ..lassen sich basierend auf ihren Eigenschaften filtern.
		\end{enumerate}
		\item Fehler..
		\begin{enumerate}
			\item ..lassen sich einsehen,
			\item ..lassen sich basierend auf ihren Eigenschaften filtern,
			\item ..lassen sich gruppieren.
			\item Fehlergruppen lassen sich in Graphen visualisieren (bspw. Histogramm der Häufigkeit).
		\end{enumerate}
		\item Metriken..
		\begin{enumerate}
			\item ..lassen sich in Graphen visualisieren.
		\end{enumerate}
		\item Traces..
		\begin{enumerate}
			\item ..lassen sich einsehen,
			\item ..lassen sich basierend auf ihren Eigenschaften filtern,
			\item ..lassen sich als ein Trace-Gantt-Diagramm darstellen.
		\end{enumerate}
		\item Session-Replay-Daten
		\begin{enumerate}
			\item Mithilfe der Session-Replay-Daten soll eine videoähnliche Nachstellung einer Sitzung erstellt werden.
		\end{enumerate}
	\end{enumerate}
	
	\subsection{Architektur}
	
	Auf Basis der zuvor beschriebenen Grunddisziplinen wird nun eine beispielhafte Umsetzung dessen konzipiert. Genauer wird eine Architektur vorgeschlagen, welche auf die zuvor betrachteten Methoden und Praktiken zurückgreift, um eine verbesserter Nachvollziehbarkeit zu erreichen. Speziell wird im Folgeabschnitt zudem vorgeschlagen, welche Werkzeuge oder Technologien zum Einsatz kommen sollen und wie diese Komponenten miteinander kommunizieren.
	
	 Die genaue Erhebung der Daten ist Teil der Implementierung und wird hier nicht näher bestimmt. Jedoch ergeben sich aus der zuvor definierten Anforderungen zur Erhebung bereits Datenkategorien, welche von einem entsprechenden Partnersystem zu konsumieren und verarbeiten sind. Es wird zwar nicht auf spezielle Werkzeuge oder Technologien eingegangen, aber es lassen sich bereits Partnersysteme bestimmen, auf Basis der zuvor identifizierten Funktionsbereiche. Hierbei wurde zudem versucht möglichst viele Bereiche über die gleichen Partnersysteme abzubilden (vgl. \autoref{anf:3100}), die Machbarkeit einer solchen Verknüpfung basiert auf den Ergebnissen von \autoref{chap:methoden-und-praktiken}.
	 
	 So soll für die Verarbeitung von Fehler-, Log-, und Metrikdaten ein einzelnes Partnersystem verantwortlich sein, welches ermöglicht diese zu konsumieren, speichern, durchsuchen und diese zu visualisieren. Grund hierfür ist, dass die Daten gemeinsame Eigenschaften besitzen, die eine gemeinsame Verarbeitung erlauben. In der \autoref{fig:grobe-architektur} ist dieses System als \enquote{Log- und Monitoringplattform} vorzufinden.
	
	Um Traces zu konsumieren und den Betreibern und Entwicklern aufbereitet zu visualisieren, soll ein weiteres Partnersystem eingesetzt werden. Dieses System wurde als notwendig empfunden, da kein Werkzeug identifiziert werden konnte, welches neben Traces noch andere Datenkategorien zufriedenstellend abdecken kann. Auf Basis von OpenTelemetry könnten sich jedoch in Zukunft Technologien entwickeln, welches alle 3 Datenkategorien von OpenTelemetry unterstützt: Metriken, Traces und Logs. Es wurde sich zudem gegen eine weitreichende Monitoringplattform, wie z. B. New Relic oder Dynatrace, entschieden, denn hier wurde identifiziert, dass diese nicht ausreichend flexibel für verschiedene Projekte sind und zudem auch nicht erlauben, dass einzelne Komponenten ausgetauscht oder entfernt werden können.
	
	Es ist zudem ein drittes System notwendig, um die gewünschte Funktionalität des Session-Replays einzubinden. Session-Replay ist ein spezielles und sehr konkretes Aufgabengebiet und es konnte kein Werkzeug identifiziert werden, welches sich nicht nur auf dieses Gebiet spezialisiert.
	
\begin{figure}[H]
	\centering
	\includegraphics[width=1.00\linewidth]{img/04_erstellung-poc/konzept-simple.png}
	\caption{Grobe Architektur}
	\label{fig:grobe-architektur}
\end{figure}

%Wie in der Datenerfassung erwähnt, werden die einzelnen Datentypen unterschiedlich erhoben und besitzen somit auch andere Eigenschaften. Wie bei Big Data \cite{ZikopoulosUnderstandingBigData}, lassen sich auch hier die Eigenschaften Volume, Velocity und Variety identifizieren. Der Aspekt Volume ist weniger präsent, denn die Datenmengen sind nicht vergleichbar mit echten Big-Data-Anwendungen. Genau ist dies nicht prognostizierbar, aber in der Evaluierung des Stands der Technik, ließ sich ein Datendurchsatz von 1 MB/min feststellen - somit stellt dies im Frontend keine Herausforderung dar, jedoch in den verarbeitenden Systemen kann dies natürlich durch eine große Menge an Frontends multipliziert werden, was jedoch nicht im Fokus der Arbeit steht.
%
%Eine Variety der Daten, also Unterschiedlichkeit der Datenstruktur, ist definitiv vorhanden und entspringt den verschiedenen Funktionsgebieten. Auch innerhalb derselben Datenströme kann eine Variety identifiziert werden, denn bspw. sind Logmeldungen sehr individuell, sie folgen meist nicht streng einem Format und enthalten unterschiedliche Mengen an Informationen.
%
%Der Aspekt des Velocity ist zudem auch sehr wichtig und eine Visualisierung dessen für das vorhergehende Konzept findet sich in \autoref{fig:grobe-architektur-datendurchsatz}.
%	
%\begin{figure}[H]
%	\centering
%	\includegraphics[width=0.75\linewidth]{img/04_erstellung-poc/konzept-datendurchsatz.png}
%	\caption{Grobe Architektur mit hervorgehobenem Datendurchsatz}
%	\label{fig:grobe-architektur-datendurchsatz}
%\end{figure}

	\subsection{Technologie-Stack}
	\label{sec:technologie-stack}
	
	Auf Basis der zuvor erstellten Architektur wird sich nun für spezielle Technologien entschieden, mit der diese Architektur umgesetzt werden soll. Weiterhin wird behandelt, wie die Daten vom Frontend aus erhoben werden sollen und wie sie an die Partnersysteme gelangen.
	
	Für die \enquote{Log- und Monitoringplattform} wurde sich für Splunk entschieden, denn auf Basis der Evaluierung konnte festgestellt werden, dass Splunk die drei gewünschten Datenkategorien Logs, Metriken und Fehler zufriedenstellend unterstützt. Es wurde sich gegen New Relic und Dynatrace entschieden, da es diesen Werkzeugen an Flexibilität fehlt und sie viele Funktionen anbieten, die für die Lösung nicht notwendig sind. Eine weitere Alternative ist der Elastic Stack, welcher jedoch nicht näher evaluiert wurde. Somit wird Splunk eingesetzt, aber für eine äquivalente Lösung kann Splunk durch ein gleichwertiges Werkzeug ausgetauscht werden.
	
	Für das Partnersystem, welches sich mit den Tracedaten befasst, wurde sich für Jaeger entschieden. Die Entscheidung wurde auf der Basis getroffen, dass Jaeger einen moderner Tracingdienst darstellt, welcher zudem quelloffen entwickelt wird. Weiterhin ist in Zukunft eine Unterstützung des OpenTelemetry-Standards geplant, was durch die standardisierte Schnittstelle die Anbindung an bestehende Systeme vereinfachen wird. Des Weiteren erfüllt Jaeger alle aufgestellten Kriterien und erzeugt zudem ein Abbild der Systemarchitektur auf Basis der Traces.
	
	Um die Session-Replay-Funktionalität einzubringen wird in diesem Konzept LogRocket vorgeschlagen. LogRockets Nachstellung einer Sitzung ist nicht nur wie gefordert videoähnlich, sondern auch interaktiv. Die gesamte HTML-Struktur wird nachgestellt und kann so zu jedem Zeitpunkt begutachtet werden.
	
	Der sich daraus ergebende Technologiestack, angewandt auf die Architektur, ist in \autoref{fig:architektur-technologien} zu betrachten. Wie bei Splunk erwähnt sind auch die anderen Werkzeuge durch gleichwertige Werkzeuge ersetzbar, so können individuelle Anpassungen erfolgen und betriebliche Gegebenheiten zu berücksichtigen.
	
\begin{figure}[H]
	\centering
	\includegraphics[width=1.00\linewidth]{img/04_erstellung-poc/konzept-technologien.png}
	\caption{Architektur mit speziellen Technologien}
	\label{fig:architektur-technologien}
\end{figure}

	\subsection{Übertragbarkeit}
	
	Übertragbarkeit beschäftigt sich mit der Eigenschaft eines Ansatzes in verschiedene Situationen anwendbar zu sein. Ein Ansatz ist nicht übertragbar, wenn zu vielen Annahmen über die Situation getroffen werden.
	
	Das zuvor definierte Konzept wurde getrennt von der Demoanwendung erstellt und ist somit nicht auf dessen Eigenschaften beschränkt. Das Konzept nimmt dennoch an, dass es sich um eine Webanwendung handelt, auf das es anzuwenden ist. Weiterhin wird angenommen, dass Quellcodeänderungen vorgenommen werden können und das Partnersysteme hinzugefügt sowie angebunden werden können.
	
	Es wird jedoch nicht angenommen, dass die Partnersysteme exakt von den vorgeschlagenen Technologien realisiert werden. Vielmehr wurde die Funktionsgruppen definiert, zusammengefasst und darauf basierend eine Auswahl getroffen. Sowohl die Zusammenfassung der Funktionsgruppen als auch die Auswahl der eigentlichen Technologien sind individuell änderbar, sodass ein angepasstes aber äquivalent hilfreiches Konzept resultiert.
	
	Somit lässt sich abschließend betrachten, dass das Konzept eine akzeptable Übertragbarkeit aufweist. Eine tiefergehende Nachbetrachtung der Übertragbarkeit erfolgt in \autoref{sec:uebertragbarkeit}, im Anschluss an die Implementierung. Hierbei kann es zu Abweichungen zu dieser Bewertung der Übertragbarkeit kommen, aufgrund von Implementierungsdetails oder einem geänderten Vorgehen. Auf Basis des beleuchteten Konzeptes, wird im nächsten Abschnitt die eigentliche Umsetzung beschrieben.

\section{Implementierung}
\subsection{Backend}

Wie in \autoref{subsec:demoanwendung-backend} beschrieben, wurden die Dienste mit Eclipse MicroProfile umgesetzt. Neben den standardmäßig enthaltenen Bibliotheken, gibt es hierbei aber auch unterstützte optionale Bibliotheken, wie Implementierungen von \texttt{OpenAPI}, \texttt{OpenTracing}, \texttt{Fault Tolerance} und vieler weiterer \cite{EclipseMicroprofile}.

Um Traces von den Microservices zu sammeln, wurde die OpenTracing Implementierung sowie ein Jaeger-Client \cite{JaegerClient} zum Exportieren der Daten hinzugezogen. Mit dieser Anbindung lassen sich per Annotation (vgl. \autoref{lst:implementierung-traced-example}) alle zu tracenden Businessmethoden definieren, die dann automatisch getraced und über den Jaeger-Client an Jaeger gesendet werden. Bei jedem Microservice wurde diese Annotation dann an relevante Methoden geschrieben und der Jaeger-Client konfiguriert, was automatisch zu der Übertragung von verteilten Traces in Jaeger führte.

\lstinputlisting[
  language = java,
     style = java-eclipse,
basicstyle = {\footnotesize\fontfamily{pcr}\selectfont},
   caption = Beispielhafter Einsatz der @Traced-Annotation,
captionpos = b,
     label = lst:implementierung-traced-example
]{content/04_erstellung-poc/implementierung-code/TracedExample_OrderService.java}

\begin{wrapfigure}[8]{r}{0.45\textwidth}
\centering
\vspace{-\baselineskip}
\includegraphics[width=\linewidth]{img/04_erstellung-poc/implementierung_jaeger-trace-example.png}
\caption{Ausschnitt des Traces zu \autoref{lst:implementierung-traced-example}}
\label{fig:implementierung_jaeger-trace-example}
\end{wrapfigure}

In Jaeger erzeugt der o. g. Quellcode die in \autoref{fig:implementierung_jaeger-trace-example} zu sehenden Spans. Neben den Traces werden keine weiteren Daten von Backend-Komponenten erhoben, da das Hauptaugenmerk der Arbeit im Frontend liegt.

\subsection{Frontend}

\subsubsection{Traces und Metriken}

Das Frontend erhebt ebenso wie das Backend Traces, aber zusätzlich werden auch Metriken, Logmeldungen und Fehler erhoben und gemeldet. Traces und Metriken werden auf Basis von OpenTelemetry JavaScript Komponenten \cite{OpenTelemetryJS} erhoben. Genauer werden diese Komponenten in einem Angular Modul (siehe \autoref{lst:app-observability}) initialisiert und der restlichen Anwendung über \enquote{providers} zur Verfügung gestellt. Hierbei wird der SPA ein \texttt{Tracer} bereitgestellt, mit dem Spans aufgezeichnet werden können, ein \texttt{Meter}, welcher es erlaubt Metriken zu erstellen, und eine \texttt{requestCounter}-Metrik, welches die Aufzeichnung der Aufrufanzahl schnittstellenübergreifend erlaubt.

\lstinputlisting[
  language = JavaScript,
     style = ES6,
basicstyle = {\footnotesize\fontfamily{pcr}\selectfont},
   caption = Quellcode des Moduls \enquote{app-observability.module.ts},
captionpos = b,
     label = lst:app-observability
]{content/04_erstellung-poc/implementierung-code/app-observability.module.ts}

Im \autoref{lst:shopping-cart-datasource} ist die Benutzung des zur Verfügung gestellten \texttt{Tracers} zu sehen, hierbei wird ein Span erstellt und bei Schnittstellenaufrufen an die jeweiligen Services übergeben.

\lstinputlisting[
  language = JavaScript,
     style = ES6,
basicstyle = {\footnotesize\fontfamily{pcr}\selectfont},
   caption = Datenquelle zum Abrufen und Zusammenführen der Artikeldaten,
captionpos = b,
     label = lst:shopping-cart-datasource
]{content/04_erstellung-poc/implementierung-code/tracing_shopping-cart-datasource.ts}

Beispielhaft im Dienst zum Abrufen der Übersetzungsdaten (vgl. \autoref{lst:localization.service}) wird der übergebene Span als Elternspan benutzt. Bei dem eigentlichen HTTP-Aufruf wird zudem ein HTTP-Header \texttt{uber-trace-id} angereichert, den der dort laufende Jaeger-Client interpretiert \cite{JaegerClient} und daraus die Beziehung zu den Frontend-Spans herstellt. Zusätzlich zum Tracing wird hierbei auch die Metrik \texttt{requestCounter} inkrementiert.

\lstinputlisting[
  language = JavaScript,
     style = ES6,
basicstyle = {\footnotesize\fontfamily{pcr}\selectfont},
   caption = Service zum Abrufen der Übersetzungsdaten,
captionpos = b,
     label = lst:localization.service
]{content/04_erstellung-poc/implementierung-code/tracing_localization.service.ts}

Es wurde sich für die OpenTelemetry Implementierung für Tracing und Metriken im Frontend entschieden, da wie in \autoref{subsec:opentelemetry} beschrieben, OpenTelemetry einen vielversprechenden Standard darstellt. Weiterhin konnte keine Bibliothek identifiziert werden, die die Traces erhebt und direkt nach Jaeger sendet. Es gibt zwar einen Jaeger-Client für Node.js\footnote{Jaeger-Client für Node.js: https://github.com/jaegertracing/jaeger-client-node}, jedoch befindet sich das browserkompatible Pendant\footnote{Jaeger-Client für Browser: https://github.com/jaegertracing/jaeger-client-javascript/} seit 2017 in den Startlöchern. Weiterhin existiert ein OTel Exporter für Jaeger\footnote{OTel Jaeger Exporter: https://github.com/open-telemetry/opentelemetry-js/tree/main/packages/opentelemetry-exporter-jaeger}, welcher jedoch auch nur mit Node.js funktioniert.

Die gesammelten OTel Tracingdaten werden über einen Standard-Exporter an das \enquote{Backend4Frontend} gesendet, welcher diese dann in ein Jaeger-konformes Format umwandelt und sie dann subsequent an Jaeger übertragt. Die Metrikdaten werden jedoch bereits im Frontend konvertiert, in ein Splunk-kompatibles Logformat. Nach der Konvertierung werden die Daten an den \texttt{SplunkForwardingService} übergeben, welcher im folgenden Abschnitt näher beschrieben wird.

\subsubsection{Logging}

Das Logging im Frontend wurde über das npm \cite{npm} Paket \texttt{ngx-logger}\footnote{ngx-logger auf GitHub: https://github.com/dbfannin/ngx-logger} realisiert, welches eine speziell an Angular angepasste Logging-Lösung darstellt. Da dieses Pakete extra an Angular angepasst ist, lässt es sich ohne großen Aufwand als Modul einbinden, vgl. \autoref{lst:logging_app.module}.

\lstinputlisting[
  language = JavaScript,
     style = ES6,
basicstyle = {\footnotesize\fontfamily{pcr}\selectfont},
   caption = Ausschnitt des Hauptmoduls \texttt{app.module.ts},
captionpos = b,
     label = lst:logging_app.module
]{content/04_erstellung-poc/implementierung-code/logging_app.module.ts}

Wie in den vorherigen Codebeispielen zum Tracing zu sehen war, kann ein \texttt{NGXLogger} im Konstruktor von Komponenten und Diensten injected werden. Logmeldungen die hiermit erfasst werden, werden je nach Konfiguration und Loglevel der jeweiligen Meldung in die Browserkonsole geschrieben. Über einen \texttt{NGXLoggerMonitor} lassen sich die Logmeldungen anzapfen, wie in \autoref{lst:splunk-logging-monitor} zu sehen ist. Hierbei werden die Logmeldungen in ein Splunkformat übertragen und dann über den \texttt{SplunkForwardingService} an das \enquote{Backend4Frontend} übertragen. Eine direkte Übertragung an Splunk ist nicht möglich, da Splunk nicht mit kompatiblen CORS-Headern antwortet. Das Backend4Frontend reichert neben dem Weiterleiten auch die Meldungen mit Kontextinformationen, wie der User-IP, der Browserversion usw. an.

\lstinputlisting[
  language = JavaScript,
     style = ES6,
basicstyle = {\footnotesize\fontfamily{pcr}\selectfont},
   caption = Implementierung des \texttt{NGXLoggerMonitor}-Interfaces,
captionpos = b,
     label = lst:splunk-logging-monitor
]{content/04_erstellung-poc/implementierung-code/splunk-logging-monitor.ts}

\subsubsection{Fehler}

Die ErrorHandler-Hook von Angular übermittelt aufgetretene und unbehandelte Fehler an den \texttt{SplunkForwardingErrorHandler}. Weiterhin ist der ErrorHandler \texttt{Injectable} in andere SPA Klassen, wo er bspw. bei den Schnittstellen dazu benutzt wird, dass auch behandelte Fehler an Splunk zu übermitteln.

Wird ein Fehler gemeldet, werden zunächst die Fehlerinformationen in einen Splunkdatensatz konvertiert und dann über den zuvor behandelten \texttt{SplunkForwardingService} an Splunk weitergeleitet. Neben diesem Verhalten wird zusätzlich auch der Fehler an LogRocket übermittelt, damit dieser im Video des Session-Replays gesondert angezeigt wird.

\lstinputlisting[
  language = JavaScript,
     style = ES6,
basicstyle = {\footnotesize\fontfamily{pcr}\selectfont},
   caption = ErrorHandler zum Abfangen und Weiterleiten von aufgetretenen Fehlern,
captionpos = b,
     label = lst:splunk-forwarding-error-handler
]{content/04_erstellung-poc/implementierung-code/splunk-forwarding-error-handler.ts}

\subsubsection{Session-Replay}

LogRocket wird standardmäßig nicht aktiv, außer wenn der Nutzer explizit der Aufzeichnung zustimmt \citationneeded{}. Dies bedeutet jedoch auch, dass bis zur Zustimmung keine Sitzungsdaten aufgenommen wurden. Wenn der Nutzer zustimmt, wird, wie im \autoref{lst:session-replay_checkout.component} zu sehen ist, LogRocket initialisiert und mit identifizierenden Daten angereichert. Nach dem Warenkorbdialog und nach der Eingabe der Rechnungsadresse werden zudem weitere identifizierende Daten an LogRocket übermittelt. Die Aufnahme der Sitzung läuft größtenteils autonom, lediglich behandelte Fehler müssen LogRocket manuell übermittelt werden.

\lstinputlisting[
  language = JavaScript,
     style = ES6,
basicstyle = {\footnotesize\fontfamily{pcr}\selectfont},
   caption = Initialisierung von LogRocket in der Hauptkomponente,
captionpos = b,
     label = lst:session-replay_checkout.component
]{content/04_erstellung-poc/implementierung-code/session-replay_checkout.component.ts}


	\textit{Auf Basis des Konzeptes soll nun eine Implementierung erfolgen.}