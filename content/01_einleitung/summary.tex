\section*{\thispagestyle{empty}Kurzfassung}
% Hier ist eine Kurfassung der wichtigsten Ergebnisse und Erkenntnisse der Arbeit im Sinne einer „Management Summary“ zu erstellen, die vom Umfang her nicht mehr als eine Seite umfassen soll.
% vgl. http://inf.fh-dortmund.de/~pa_data/Strukturvorlage.pdf
	
Im Projektalltag führen Probleme in JavaScript-basierten Webanwendungen zu kostspieligen Ausfällen, die oft durch fehlende Informationen länger dauern als vergleichbare Fehler in Backendsystemen. Aus diesem Grund wird in dieser Arbeit das Problem der Nachvollziehbarkeit bei diesen Webanwendungen untersucht.
	
Das Ziel der Arbeit ist es, eine Erkenntnis hervorzubringen, warum Entwickler und Betreiber von einer Webanwendung weniger relevante Informationen erhalten. Weiterhin ist zu ergründen, wie dieser Informationslücke entgegenwirkt werden kann, gibt es hierbei bewährte Ansätze und wie sieht der Stand der Technik aus. Letztendlich ist ein Proof-of-Concept hervorzubringen, anhand dessen die gefundenen Ansätze anzuwenden und zu veranschaulichen sind.

Um das Ziel der Arbeit zu erreichen wurde zunächst die Ausgangssituation beschrieben, um festzustellen welche besonderen Eigenschaften diese Webanwendungen besitzen. Folgend wurde die Nachvollziehbarkeit als solche ergründet und welche Methoden existieren, um eine bessere Nachvollziehbarkeit zu erreichen. Darauf aufbauend beleuchtet die Arbeit zudem den Stand der Technik, sowie wurden kriteriengeleitet einige Technologien ausgewählt, um diese beim Proof-of-Concept zu verwenden.

Der Proof-of-Concept wurde auf Basis einer Demoanwendung erstellt, bei welcher es sich hauptsächlich um eine mit Angular erstellte SPA handelt. Bei der SPA handelt es sich um einen Checkout-Wizard, welcher eingebaute Fehlerszenarien enthält. Für den Proof-of-Concept wurde diese SPA in verschiedenen Aspekten erweitert, um bspw. Logs, Metriken, Traces und Fehler zu protokollieren. Diese Daten wurden an Partnersysteme (Splunk, Jaeger, LogRocket) weitergeleitet und somit den Entwicklern sowie Betreibern aufbereitet zur Verfügung gestellt.

Letztendlich wurde die erstellte Lösung durch zuvor definierte Anforderungen und aufzudeckende Fehlerszenarien überprüft und es konnte festgestellt werden, dass der geforderte Mehrwert erreicht werden konnte. Weiterhin konnte eine Übertragbarkeit auf andere ähnliche Softwareprojekte identifiziert werden, sodass die erstellte Lösung auch hier anwendbar ist.

\newpage{}

\section*{\thispagestyle{empty}Abstract}
% Hier ist die Kurzfassung in englischer Sprache erneut als Abstract zu verfassen. Auch diese Übersetzung sollte nicht länger als eine Seite sein. Diese Kurzfassung ist ein Muss.
% vgl. http://inf.fh-dortmund.de/~pa_data/Strukturvorlage.pdf

Failures in web applications that rely on JavaScript often result in costly outages. The time to fix can also be magnified compared to similar issues with backend applications, caused by missing crucial information needed to debug and fix the issue. To combat this, this thesis explores the problem of observability of web applications.

The aim of the work is therefore to bring forth an understanding, why relevant information is missing. Furthermore, it is to be explored, how this information can be obtained and if there are tried and tested methods to achieve this. Based on this gathered understanding a proof of concept is to be developed, that uses the findings and illustrates their purpose.

To achieve the aim of the work first and foremost the environment is investigated, e. g. what are the characteristics of web applications that differ from backend applications. After that, the term ``observability`` is defined and examined. The state of the art regarding ``observability`` is studied, resulting in the knowledge of current methods and technologies. Based on these findings, some technologies are selected for use in the proof of concept.

The proof of concept is developed based on a demo application, that is in its core an Angular SPA. The SPA is a wizard for a webshop checkout, which also contains built-in issues. This demo application was then extended to yield more information, such as logs, metrics, traces and errors. This data was then forwarded to systems, that utilized them to enable developers and operators to ``observe`` the SPA. These systems were Splunk, Jaeger and LogRocket.

Finally, the created solution was verified based on requirements and its ability to reveal crucial information about the built-in issues. It was found that the solution yielded the required information and also fulfilled the vast majority of the previously defined requirements. Additionally, it was determined that the created solution is applicable to other similar software projects.