% a hack, to make the Motivation fit into one page..
\vspace{-\baselineskip}

%In diesem Unterkapitel sollten folgende Punkte behandelt werden:
%\begin{itemize}
%	\item	Was ist das Problem
%	\item 	Problemgeschichte?
%\end{itemize}
\section{Motivation}

Die Open Knowledge GmbH ist als branchenneutraler Softwaredienstleister in einer Vielzahl von Branchen wie Automotive, Logistik, Telekommunikation und Versicherungs- und Finanzwirtschaft aktiv. Zu den zahlreichen Kunden der Open Knowledge GmbH gehört auch ein führender deutscher Direktversicherer. 

Ein Direktversicherer bietet Versicherungsprodukte seinen Kunden ausschließlich im Direktvertrieb, d. h. vor allem über das Internet und zusätzlich auch über Telefon, Fax oder B
rief an. Im Unterschied zum klassischen Versicherer verfügt ein Direktversicherer jedoch über keinen Außendienst oder Geschäftsstellen, bei denen Kunden eine persönliche Beratung bekommen können. Da das Internet der primäre Vertriebskanal ist, gehört heute ein umfassender Online-Auftritt zum Standard. Dieser besteht typischerweise aus einem Kundenportal mit der Möglichkeit Angebote für Versicherungsprodukte berechnen und abschließen zu können, sowie persönliche Daten und Verträge einzusehen.

\nomenclature[Fachbegriff]{Serverseitiges Rendering}{Die darzustellenden Inhalte, werden beim Server generiert und der Client stellt diese dar. Beispielsweise sind Anwendungen mit PHP oder auch eine Java Web Application}
\nomenclature[Fachbegriff]{Clientseitiges Rendering}{Der Server stellt dem Client lediglich die Logik und die notwendigen Daten bereit, die eigentliche Inhaltsgenerierung geschieht im Client. Für ein Beispiel siehe \autoref{subsec:singe-page-applications}}

% i.d.R. getrennt: https://www.scribbr.de/wissenschaftliches-schreiben/abkuerzungen/
Während in der Vergangenheit Online-Auftritte i. d. R. als Webapplikation mit serverseitigen Rendering realisiert wurden, sind heutzutage Javascript-basierte Webapplikation mit clientseitigem Rendering die Norm. Bei einer solchen Webapplikation befindet sich die gesamte Logik mit Ausnahme der Berechnung des Angebots und der Verarbeitung der Antragsdaten im Browser des Nutzers.

Im produktiven Einsatz kommt es auch bei gut getesteten Webapplikationen hin und wieder vor, dass es zu unvorhergesehenen Fehlern in der Berechnung oder Verarbeitung kommen kann. Liegt die Ursache für den Fehler im Browser, z. B. aufgrund einer ungültigen Wertkombination, ist dies eine Herausforderung. Während bei Server-Anwendungen Fehlermeldungen in den Log-Dateien einzusehen sind, gibt es für den Betreiber der Anwendung i. d. R. keine Möglichkeit die notwendigen Informationen über den Nutzer und seine Umgebung abzurufen. Noch wichtiger ist, dass er mitbekommt, wenn ein Nutzer ein Problem bei der Bedienung der Anwendung hat. Ohne eine aktive Benachrichtigung durch den Nutzer, sowie detaillierte Informationen, ist es dem Betreiber nicht möglich, Kenntnis über das Problem zu erlangen, geschweige denn dieses nachzustellen.

Dies stellt ein Kernproblem von  Webapplikationen dar \cite{ClientSideMonitoringOfDistributedSystems}. Im Rahmen der Arbeit soll daher ein Proof-of-Concept konzipiert und umgesetzt werden, welcher dieses Kernproblem am Beispiel einer Demoanwendung löst.

%\begin{itemize}
%	\item 	Was soll mit der Arbeit erreicht werden? Welche Ziele werden angestrebt?
%			Möglichst kurz und präzise geplante Ergebnisse umreißen. Daran werden
%			Ihre Resultate am Ende gemessen!
%\end{itemize}
\section{Zielsetzung}

Das grundlegende Ziel dieser Arbeit soll es sein, den Betreibern einer JavaScript-basierten Webapplikation die Möglichkeit zu geben das Verhalten ihrer Applikation und die Interaktionen von Nutzern. Diese Nachvollziehbarkeit soll insbesondere bei Fehlerfällen u. Ä. gewährleistet sein, aber auch in sonstigen Fällen soll eine Nachvollziehbarkeit möglich sein. Eine vollständige Überwachung der Applikation und des Nutzers (wie bpsw. bei Werbe-Tracking) sind jedoch nicht vorgesehen. Daraus ergibt sich die Forschungsfrage:

\begin{quotation}
	Wie sieht ein Ansatz aus, um bei clientseitigen JavaScript-basierten Webapplikation den Betreibern eine Nachvollziehbarkeit zu gewährleisten?
\end{quotation}

Vom Leser wird eine Grundkenntnis der Informatik in Theorie oder Praxis erwartet, aber es sollen keine detaillierten Erfahrungen in der Webentwicklung vom Leser erwartet werden. Daher sind das Projektumfeld und seine besonderen Eigenschaften zu erläutern.

Die anzustrebende Lösung soll ein Proof-of-Concept sein, welches anhand einer bestehenden Demoanwendung erstellt werden soll. Die Demoanwendung soll repräsentativ eine abgespeckte JavaScript-basierte Webapplikation darstellen, bei der die zuvor benannten Hürden zur Nachvollziehbarkeit bestehen.

Vor der eigentlichen Lösungserstellung soll jedoch die theoretische Seite beleuchtet werden, indem die Nachvollziehbarkeit sowie Methoden und Praktiken zur Erreichung dieser beschrieben werden. Es gilt aktuelle Literatur und den Stand der Technik zu erörtern, in Bezug auf die Forschungsfrage. Beim Stand der Technik sollen Technologien aus Fachpraxis und Literatur näher betrachtet werden und beschrieben werden.

Weiterhin gilt es zu beleuchten, wie die Auswirkungen für die Nutzer der Webapplikation sind. Wurde die Leistung der Webapplikation beeinträchtigt (erhöhte Ladezeit, erhöhte Datenlast)? Werden mehr Daten von ihm erhoben und zu welchem Zweck?

Am Ende der Ausarbeitung soll überprüft werden, ob und wie die Forschungsfrage beantwortet wurde. Auch die Übertragbarkeit der erstellten Lösung (PoC) und Ergebnisse gilt es hierbei näher zu betrachten.

\subsection{Abgrenzung}

\nomenclature[Fachbegriff]{PoC}{Proof-of-Concept}

Die Demoanwendung wird als Single-Page-Application (SPA) realisiert, denn hier bewegt sich das Projektumfeld von der Open Knowledge GmbH. Bei der Betrachtung der Datenerhebung und -verarbeitung ist eine volle Konformität mit der DSGVO nicht zu prüfen.

Bei der Erörterung zum Stand der Technik sollen detaillierte Produktvorstellungen nicht das Ziel sein, auch ist keine umfassende Gegenüberstellung anzustreben.

\pagebreak

%\begin{itemize}
%	\item 	Wie wird vorgegangen, um das Ziel zu erreichen?
%	\item 	Warum ist die Arbeit so gegliedert, wie sie gegliedert ist?
%	\item 	Welche Aspekte werden nicht behandelt und warum?
%\end{itemize}
\section{Vorgehensweise}

\vspace{-0.5\baselineskip}

Zur Vorbereitung eines Proof-of-Concepts wird zunächst die Ausgangssituation geschildert. Speziell wird auf die Herausforderungen der Umgebung \enquote{Browser} eingegangen, besonders in Hinblick auf die Verständnisgewinnung zu Interaktionen eines Nutzers und des Verhaltens der Applikation. Des Weiteren wird die Nachvollziehbarkeit als solche formal beschrieben und was sie im Projektumfeld genau bedeutet.

Darauf aufbauend werden allgemeine Methoden vorgestellt, mit der die Betreiber und Entwickler eine bessere Nachvollziehbarkeit erreichen können. Dabei werden die Besonderheiten der Umgebung beachtet und es wird erläutert, wie diese Methoden in der Umgebung zum Einsatz kommen können. Hiernach sind Ansätze aus der Literatur und Fachpraxis zu erörtern, welche eine praktische Realisierung der zuvor vorgestellten Methoden darstellen.

Auf Basis des detaillierten Verständnisses der Problemstellung und der Methoden wird nun ein Proof-of-Concept erstellt. Ziel soll dabei sein, die Nachvollziehbarkeit einer Webapplikation zu verbessern. Der Proof-of-Concept erfolgt auf Basis einer Demoanwendung, die im Rahmen dieser Arbeit erstellt wird.

Ist ein Proof-of-Concept nun erstellt, wird analysiert, welchen Einfluss es auf die Nachvollziehbarkeit hat und ob die gewünschten Ziele erreicht wurden (vgl. Zielsetzung).

\vspace{-0.5\baselineskip}

\section{Open Knowledge GmbH}

\vspace{-0.5\baselineskip}

%{\color{red}TODO: Dieser Abschnitt muss noch überarbeitet werden}

Die Bachelorarbeit wird im Rahmen einer Werkstudententätigkeit innerhalb der Open Knowledge GmbH erstellt. Der Standortleiter des Standortes Essen, Dipl. Inf. Stephan Müller, übernimmt die Zweitbetreuung.

Die Open Knowledge GmbH ist ein branchenneutrales mittelständisches Dienstleistungsunternehmen mit dem Ziel bei der Analyse, Planung und Durchführung von Softwareprojekten zu unterstützen. Das Unternehmen wurde im Jahr 2000 in Oldenburg, dem Hauptsitz des Unternehmens, gegründet und beschäftigt heute 74 Mitarbeiter. Mitte 2017 wurde in Essen der zweite Standort eröffnet, an dem 13 Mitarbeiter angestellt sind.

Die Mitarbeiter von Open Knowledge übernehmen in Kundenprojekten Aufgaben bei der Analyse über die Projektziele und der aktuellen Ausgangssituationen, der Konzeption der geplanten Software, sowie der anschließenden Implementierung. Die erstellten Softwarelösungen stellen Individuallösungen dar und werden den Bedürfnissen der einzelnen Kunden entsprechend konzipiert und implementiert. Technisch liegt die Spezialisierung bei der Mobile- und bei der Java Enterprise Entwicklung, bei der stets moderne Technologien und Konzepte verwendet werden. Die Geschäftsführer als auch diverse Mitarbeiter der Open Knowledge GmbH sind als Redner auf Fachmessen wie der Javaland oder als Autoren in Fachzeitschriften wie dem Java Magazin vertreten.

\pagebreak
