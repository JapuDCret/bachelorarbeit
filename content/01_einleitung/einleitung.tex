% a hack, to make the Motivation fit into one page..
%\vspace{-\baselineskip}

%In diesem Unterkapitel sollten folgende Punkte behandelt werden:
%\begin{itemize}
%	\item	Was ist das Problem
%	\item 	Problemgeschichte?
%\end{itemize}
\section{Motivation}

%\vspace{-\baselineskip}

Die Open Knowledge GmbH ist als branchenneutraler Softwaredienstleister in einer Vielzahl von Branchen wie Automotive, Logistik, Telekommunikation und Versicherungs- und Finanzwirtschaft aktiv. Zu den zahlreichen Kunden der Open Knowledge GmbH gehört auch ein führender deutscher Direktversicherer. 

Ein Direktversicherer bietet seinen Kunden Versicherungsprodukte ausschließlich im Direktvertrieb, d. h. vor allem über das Internet und zusätzlich auch über Telefon, Fax oder Brief an. Im Unterschied zum klassischen Versicherer verfügen Direktversicherer über keinen Außendienst oder Geschäftsstellen, bei denen Kunden eine persönliche Beratung erhalten. Da das Internet der primäre Vertriebskanal ist, ist heute ein umfassender Online-Auftritt die Norm. Dieser besteht typischerweise aus einem Kundenportal mit der Möglichkeit Angebote für Versicherungsprodukte berechnen und abschließen zu können, sowie persönliche Daten und Verträge einsehen und ändern zu können.

\nomenclature[Fachbegriff]{Serverseitiges Rendering}{Die darzustellenden Inhalte, werden beim Server generiert und der Client stellt diese dar. Beispielsweise sind Anwendungen mit PHP oder auch eine Java Web Application}
\nomenclature[Fachbegriff]{Clientseitiges Rendering}{Der Server stellt dem Client lediglich die Logik und die notwendigen Daten bereit, die eigentliche Inhaltsgenerierung geschieht im Client. Beispiel siehe \autoref{sec:single-page-applications}}

Während in der Vergangenheit Online-Auftritte i. d. R. als Webanwendung mit serverseitigem Rendering realisiert wurden, sind heutzutage JavaScript-basierte Webanwendungen mit clientseitigem Rendering die weit verbreitet \cite{ShiftToClientSideWebApplications}. Bei einer solchen Webanwendung befindet sich ein Großteil der Anwendungslogik im Browser des Nutzers, z. B. erfolgt bei einem Wizard erst bei Absenden eine Interaktion mit einem Server.

Im produktiven Einsatz kommt es auch bei gut getesteten Webanwendungen vor, dass bspw. unvorhergesehene Fehler in der Berechnung oder Verarbeitung auftreten. Liegt die Ursache für den Fehler im Browser, z. B. aufgrund einer ungültigen Wertkombination, steht der Betreiber vor einer Herausforderung. Bei Server-Anwendungen können Fehlermeldungen in Log-Dateien geprüft werden, ein Betreiber einer Webanwendung hat i. d. R. keine Möglichkeit die notwendigen Informationen über den Nutzer und seine Umgebung abzurufen. Hinzu kommt, dass der Betreiber gar nicht mitbekommt, wenn ein Nutzer auf ein Problem bei der Bedienung der Anwendung hat. Ohne eine aktive Benachrichtigung durch den Nutzer, sowie detaillierte Informationen, ist es dem Betreiber nicht möglich, Kenntnis über das Problem zu erlangen, geschweige denn dieses nachzustellen.

Dies stellt ein Kernproblem von Webanwendungen dar \cite{ClientSideMonitoringOfDistributedSystems}. Im Rahmen der Arbeit soll daher ein Proof-of-Concept konzipiert und umgesetzt werden, welcher dieses Kernproblem am Beispiel einer Demoanwendung löst.

%\begin{itemize}
%	\item 	Was soll mit der Arbeit erreicht werden? Welche Ziele werden angestrebt?
%			Möglichst kurz und präzise geplante Ergebnisse umreißen. Daran werden
%			Ihre Resultate am Ende gemessen!
%\end{itemize}
\section{Zielsetzung}

Das grundlegende Ziel dieser Arbeit ist es, den Betreibern einer JavaScript-basierten Webanwendung die Möglichkeit zu geben, das Verhalten ihrer Anwendung und die Interaktionen von Nutzern nachzuvollziehen. Diese soll insbesondere bei Fehlerfällen gewährleistet sein, ist aber auch in anderen Situationen erstrebenswert, z. B. wenn die Betreiber nachvollziehen wollen, welche Interaktionen der Nutzer getätigt hat. Eine vollständige Überwachung der Anwendung und des Nutzers (wie bspw. bei Werbe-Tracking) ist hingegen nicht vorgesehen. Daraus ergibt sich die Forschungsfrage:

\begin{quotation}
	\textit{Wie sieht ein Ansatz aus, um den Betreibern von JavaScript-basierten Webanwendungen eine Nachvollziehbarkeit zu gewährleisten?}
\end{quotation}

% Vom Leser wird eine Grundkenntnis der Informatik in Theorie oder Praxis erwartet, aber es sollen keine detaillierten Erfahrungen in der Webentwicklung vom Leser erwartet werden. Daher sind das Projektumfeld und seine besonderen Eigenschaften zu erläutern.

Zur Vorbereitung der Entwicklung des Proof-of-Concepts ist der Stand der Technik zu recherchieren. Die daraus resultierenden Erkenntnisse sind bei der Erstellung zu beachten und anzuwenden.

Die anzustrebende Lösung ist ein Proof-of-Concept, welcher eine, zu erstellende, Demoanwendung erweitert. Die Demoanwendung steht repräsentativ für eine JavaScript-basierte Webanwendung, bei der die zuvor benannten Hürden zur Nachvollziehbarkeit bestehen.

Weiterhin gilt es zu beleuchten, welche Auswirkungen der Einsatz von Methoden zur Nachvollziehbarkeit für Nutzer der Webanwendung hat. Wird bpsw. die Leistung der Webanwendung beeinträchtigt (erhöhte Ladezeit, erhöhte Datenlast).

Am Ende der Ausarbeitung soll überprüft werden, ob und wie die Forschungsfrage beantwortet wurde. Auch die Übertragbarkeit der erstellten Lösung und Ergebnisse gilt es hierbei näher zu betrachten.

\subsection{Abgrenzung}

\nomenclature[Fachbegriff]{PoC}{Proof-of-Concept}
\nomenclature[Fachbegriff]{SPA}{Single-Page-Application}

Die Demoanwendung wird als Single-Page-Application (SPA) \cite{SinglePageApplication} realisiert. Bei der Datenerhebung und -verarbeitung werden datenschutzrechtliche Aspekte nicht näher zu betrachtet. Darüber hinaus liegt bei der Recherche zum Stand der Technik eine allgemein bewertende Gegenüberstellung von Technologien nicht im Fokus der Arbeit.

\pagebreak

%\begin{itemize}
%	\item 	Wie wird vorgegangen, um das Ziel zu erreichen?
%	\item 	Warum ist die Arbeit so gegliedert, wie sie gegliedert ist?
%	\item 	Welche Aspekte werden nicht behandelt und warum?
%\end{itemize}
\section{Vorgehensweise}

\vspace{-0.25\baselineskip}

Für die Realisierung eines Proof-of-Concepts zur Erhöhung der Nachvollziehbarkeit einer bestehenden Anwendung, wird zunächst die theoretische Seite des Forschungsfeldes beleuchtet. Hierzu gehört eine nähere Betrachtung der Umgebung \enquote{Browser}, von Webanwendungen weiter gilt es den der \enquote{Nachvollziehbarkeit} zu definieren und im Hinblick auf SPAs zu erläutern.

Darauf aufbauend werden aktuelle Ansätze zur verbesserten Nachvollziehbarkeit recherchiert und beschrieben. Speziell sollen hierbei übergreifende und allgemein angewandte Methoden ausgearbeitet und beschrieben werden. Weiterhin ist darauffolgend der Stand der Technik aus Wirtschaft und Literatur zu recherchieren und vorzustellen.

Bevor der Proof-of-Concept implementiert wird, soll ein Konzept erstellt werden, welches darlegt, wie der Proof-of-Concept eine verbesserte Nachvollziehbarkeit erreicht. Folgend ist auf Basis des Konzeptes die Demoanwendung zu erweitern, welches den Proof-of-Concept darstellt. Im Anschluss an die Implementierung gilt es diese kritisch zu bewerten, einerseits ob die Forschungsfrage beantwortet werden konnte und andererseits in Aspekten wie Übertragbarkeit und Auswirkungen für den Nutzer.

\vspace{-0.25\baselineskip}

\section{Open Knowledge GmbH}

\vspace{-0.50\baselineskip}

Die Bachelorarbeit wird im Rahmen einer Werkstudententätigkeit innerhalb der Open Knowledge GmbH erstellt. Der Leiter des Standortes Essen, Dipl.-Inf. Stephan Müller, übernimmt die Zweitbetreuung. Neben Stephan Müller ist Christian Wansart ein Stakeholder der hier zu erstellenden Lösung, er ist bei Open Knowledge angestellt.

Die Open Knowledge GmbH ist ein branchenneutrales mittelständisches Dienstleistungsunternehmen mit dem Ziel bei der Analyse, Planung und Durchführung von Softwareprojekten zu unterstützen. Das Unternehmen wurde im Jahr 2000 in Oldenburg gegründet und beschäftigt heute über 80 Mitarbeiter. Mitte 2017 wurde in Essen der zweite Standort eröffnet, an dem derzeit 13 Mitarbeiter angestellt sind.

Die Mitarbeiter von Open Knowledge übernehmen in Kundenprojekten Aufgaben bei der Analyse über die Projektziele und der aktuellen Ausgangssituationen, der Konzeption der geplanten Software, sowie der anschließenden Implementierung. Die erstellten Softwarelösungen stellen Individuallösungen dar und werden den Bedürfnissen der einzelnen Kunden entsprechend konzipiert und implementiert. Technisch liegt die Spezialisierung bei der Mobile- und bei der Java Enterprise Entwicklung, bei der stets moderne Technologien und Konzepte verwendet werden. Trotz seiner Größe hat Open Knowledge eine starke Außendarstellung in Fachpublikationen sowie auf Fachkonferenzen. So sind sowohl die Geschäftsführer als auch diverse Mitarbeiter von Open Knowledge bspw. als Redner auf der Javaland oder als Autoren wie dem Java Magazin vertreten.

\pagebreak
