% a hack, to make the Motivation fit into one page..
%\vspace{-\baselineskip}

%In diesem Unterkapitel sollten folgende Punkte behandelt werden:
%\begin{itemize}
%	\item	Was ist das Problem
%	\item 	Problemgeschichte?
%\end{itemize}
\section{Motivation}

%\vspace{-\baselineskip}

Die Open Knowledge GmbH ist als branchenneutraler Softwaredienstleister in einer Vielzahl von Branchen wie Automotive, Logistik, Telekommunikation und Versicherungs- und Finanzwirtschaft aktiv. Zu den zahlreichen Kunden der Open Knowledge GmbH gehört auch ein führender deutscher Direktversicherer. 

Ein Direktversicherer bietet Versicherungsprodukte seinen Kunden ausschließlich im Direktvertrieb, d. h. vor allem über das Internet und zusätzlich auch über Telefon, Fax oder Brief an. Im Unterschied zum klassischen Versicherer verfügt ein Direktversicherer jedoch über keinen Außendienst oder Geschäftsstellen, bei denen Kunden eine persönliche Beratung erhalten. Da das Internet der primäre Vertriebskanal ist, ist heute ein umfassender Online-Auftritt die Norm. Dieser besteht typischerweise aus einem Kundenportal mit der Möglichkeit Angebote für Versicherungsprodukte berechnen und abschließen zu können, sowie persönliche Daten und Verträge einsehen und ändern zu können.

\nomenclature[Fachbegriff]{Serverseitiges Rendering}{Die darzustellenden Inhalte, werden beim Server generiert und der Client stellt diese dar. Beispielsweise sind Anwendungen mit PHP oder auch eine Java Web Application}
\nomenclature[Fachbegriff]{Clientseitiges Rendering}{Der Server stellt dem Client lediglich die Logik und die notwendigen Daten bereit, die eigentliche Inhaltsgenerierung geschieht im Client. Beispiel siehe \autoref{sec:single-page-applications}}

Während in der Vergangenheit Online-Auftritte i. d. R. als Webanwendung mit serverseitigem Rendering realisiert wurden, sind heutzutage JavaScript-basierte Webanwendungen mit clientseitigem Rendering die Norm \cite{ShiftToClientSideWebApplications}. Bei einer solchen Webanwendung befindet sich ein Großteil der Logik im Browser des Nutzers, bspw. wird der Nutzer durch einen komplexen Wizard geführt und erst bei Absenden des letzten Formulars geschieht eine Interaktion mit einem Server.

Im produktiven Einsatz kommt es auch bei gut getesteten Webanwendungen hin und wieder vor, dass es zu unvorhergesehenen Fehlern in der Berechnung oder Verarbeitung kommt. Liegt die Ursache für den Fehler im Browser, z. B. aufgrund einer ungültigen Wertkombination, ist dies eine Herausforderung. Während bei Server-Anwendungen Fehlermeldungen in den Log-Dateien einzusehen sind, gibt es für den Betreiber der Anwendung i. d. R. keine Möglichkeit die notwendigen Informationen über den Nutzer und seine Umgebung abzurufen. Noch wichtiger ist, dass er mitbekommt, wenn ein Nutzer ein Problem bei der Bedienung der Anwendung hat. Ohne eine aktive Benachrichtigung durch den Nutzer, sowie detaillierte Informationen, ist es dem Betreiber nicht möglich, Kenntnis über das Problem zu erlangen, geschweige denn dieses nachzustellen.

Dies stellt ein Kernproblem von  Webanwendungen dar \cite{ClientSideMonitoringOfDistributedSystems}. Im Rahmen der Arbeit soll daher ein Proof-of-Concept konzipiert und umgesetzt werden, welcher dieses Kernproblem am Beispiel einer Demoanwendung löst.

%\begin{itemize}
%	\item 	Was soll mit der Arbeit erreicht werden? Welche Ziele werden angestrebt?
%			Möglichst kurz und präzise geplante Ergebnisse umreißen. Daran werden
%			Ihre Resultate am Ende gemessen!
%\end{itemize}
\section{Zielsetzung}

Das grundlegende Ziel dieser Arbeit soll es sein, den Betreibern einer JavaScript-basierten Webanwendung die Möglichkeit zu geben, das Verhalten ihrer Anwendung und die Interaktionen von Nutzern nachzuvollziehen. Diese Nachvollziehbarkeit soll insbesondere bei Fehlerfällen u. Ä. gewährleistet sein, ist aber auch in sonstigen Fällen zu erstreben, wie z. B. wenn die Betreiber nachvollziehen wollen, welche Interaktionen der Nutzer getätigt hat. Eine vollständige Überwachung der Anwendung und des Nutzers (wie bspw. bei Werbe-Tracking) sind jedoch nicht vorgesehen. Daraus ergibt sich die Forschungsfrage:

\begin{quotation}
	Wie sieht ein Ansatz aus, um den Betreibern von JavaScript-basierten Webanwendungen eine Nachvollziehbarkeit zu gewährleisten?
\end{quotation}

Vom Leser wird eine Grundkenntnis der Informatik in Theorie oder Praxis erwartet, aber es sollen keine detaillierten Erfahrungen in der Webentwicklung vom Leser erwartet werden. Daher sind das Projektumfeld und seine besonderen Eigenschaften zu erläutern.

Die anzustrebende Lösung soll ein Proof-of-Concept sein, welches eine, zu erstellende, Demoanwendung erweitert. Die Demoanwendung soll repräsentativ eine abgespeckte JavaScript-basierte Webanwendung darstellen, bei der die zuvor benannten Hürden zur Nachvollziehbarkeit bestehen.

Weiterhin gilt es zu beleuchten, wie die Auswirkungen für die Nutzer der Webanwendung sind. Wurde die Leistung der Webanwendung beeinträchtigt (erhöhte Ladezeit, erhöhte Datenlast)? Werden mehr Daten von ihm erhoben und zu welchem Zweck?

Am Ende der Ausarbeitung soll überprüft werden, ob und wie die Forschungsfrage beantwortet wurde. Auch die Übertragbarkeit der erstellten Lösung (PoC) und Ergebnisse gilt es hierbei näher zu betrachten.

\subsection{Abgrenzung}

\nomenclature[Fachbegriff]{PoC}{Proof-of-Concept}

Die Demoanwendung wird als Single-Page-Application (SPA) \cite{SinglePageApplication} realisiert, denn hier bewegt sich das Projektumfeld von der Open Knowledge GmbH. Bei der Datenerhebung und -verarbeitung sind datenschutzrechtliche Aspekte nicht näher zu betrachten. Bei der Betrachtung von Technologien aus der Wirtschaft ist eine bewertende Gegenüberstellung nicht das Ziel.

\pagebreak

%\begin{itemize}
%	\item 	Wie wird vorgegangen, um das Ziel zu erreichen?
%	\item 	Warum ist die Arbeit so gegliedert, wie sie gegliedert ist?
%	\item 	Welche Aspekte werden nicht behandelt und warum?
%\end{itemize}
\section{Vorgehensweise}

\vspace{-0.25\baselineskip}

Um das Ziel dieser Arbeit, also ein Proof-of-Concept zu erstellen, welches die Nachvollziehbarkeit einer bestehenden Anwendung erhöht, zu erreichen, wird zunächst die theoretische Seite des Forschungsfeldes beleuchtet. Hierzu gehört eine nähere Betrachtung der Umgebung \enquote{Browser}, von Webanwendungen, sowie gilt es die Nachvollziehbarkeit zu definieren und im Hinblick auf SPAs zu erläutern.

Darauf aufbauend sind aktuelle Ansätze zur verbesserten Nachvollziehbarkeit zu recherchieren und zu beschreiben. Speziell sollen hierbei die allgemeinen übergreifenden Methoden und die tatsächlichen angewandten Praktiken differenziert beschrieben werden. Hierbei ist u. A. der Stand der Technik aus Wirtschaft und Literatur zu erläutern, um darauffolgend und auf Basis dessen ein Proof-of-Concept zu erstellen.

Bevor jedoch der PoC implementiert wird, soll ein Konzept erstellt werden, welches darlegt, wie der PoC eine verbesserte Nachvollziehbarkeit erreicht. Ist nun das Konzept erstellt, gilt es dieses auf eine SPA anzuwenden und das Proof-of-Concept zu erstellen. Im Anschluss an die Implementierung gilt es diese kritisch zu bewerten, einerseits ob die Forschungsfrage beantwortet werden konnte und andererseits in Aspekten wie Übertragbarkeit und Auswirkungen für den Nutzer.

\vspace{-0.25\baselineskip}

\section{Open Knowledge GmbH}

\vspace{-0.50\baselineskip}

Die Bachelorarbeit wird im Rahmen einer Werkstudententätigkeit innerhalb der Open Knowledge GmbH erstellt. Der Leiter des Standortes Essen, Dipl.-Inf. Stephan Müller, übernimmt die Zweitbetreuung. Neben Stephan Müller ist Christian Wansart ein Stakeholder der hier zu erstellenden Lösung, er ist bei der Open Knowledge angestellt.

Die Open Knowledge GmbH ist ein branchenneutrales mittelständisches Dienstleistungsunternehmen mit dem Ziel bei der Analyse, Planung und Durchführung von Softwareprojekten zu unterstützen. Das Unternehmen wurde im Jahr 2000 in Oldenburg gegründet und beschäftigt heute 74 Mitarbeiter. Mitte 2017 wurde in Essen der zweite Standort eröffnet, an dem 13 Mitarbeiter angestellt sind.

Die Mitarbeiter von Open Knowledge übernehmen in Kundenprojekten Aufgaben bei der Analyse über die Projektziele und der aktuellen Ausgangssituationen, der Konzeption der geplanten Software, sowie der anschließenden Implementierung. Die erstellten Softwarelösungen stellen Individuallösungen dar und werden den Bedürfnissen der einzelnen Kunden entsprechend konzipiert und implementiert. Technisch liegt die Spezialisierung bei der Mobile- und bei der Java Enterprise Entwicklung, bei der stets moderne Technologien und Konzepte verwendet werden. Die Geschäftsführer als auch diverse Mitarbeiter der Open Knowledge GmbH sind als Redner auf Fachmessen wie der Javaland oder als Autoren in Fachzeitschriften wie dem Java Magazin vertreten.

\pagebreak
