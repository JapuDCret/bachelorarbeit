%In diesem Unterkapitel sollten folgende Punkte behandelt werden:
%\begin{itemize}
%	\item	Was ist das Problem
%	\item 	Problemgeschichte?
%\end{itemize}
\section{Motivation}

\nomenclature[Fachbegriff]{GUI}{Graphical-User-Interface, Grafische Benutzeroberfläche}
\nomenclature[Fachbegriff]{Log}{Englisch für ``Protokoll``}
\nomenclature[Fachbegriff]{Stakeholder}{In dieser Arbeit werden Betreiber und Entwickler einer Webapplikation als Stakeholder bezeichnet}

Entwickler und Betreiber von JavaScript-basierten Webapplikationen werden fast alltäglich mit Verständnisproblemen konfrontiert. Sie müssen beispielsweise zu Zwecken der Problembehebung oder Verbesserung nachvollziehen können, wie ein Nutzer die Applikation verwendet und wie sich diese dabei verhält. Dies ist in der Umgebung von Browsern deutlich erschwert, als in anderen Projekten mit Nutzerinteraktion. Denn die notwendigen Informationen liegen beim Nutzer, für den Betreiber ist lediglich die Kommunikation zum Backend sichtbar.

Backend-Aufrufe sind jedoch nur bedingt hilfreich, weil sie meist ein Ergebnis eines abgeschlossenen Prozesses darstellen, aber keine Informationen dazu liefern, wie der Prozess abgeschlossen wurde.

Deswegen beschäftigt sich die Arbeit mit der Erreichung von Nachvollziehbarkeit in JavaScript-basierten Webapplikationen und zielt darauf ab die zuvor genannten Hürden zu meistern.

%In der Welt der Webapplikationen gibt es die Hürde, dass meist keine direkte Kommunikation zwischen Nutzern und den Verantwortlichen besteht. Die Verantwortlichen werden in dieser Arbeit mit \textbf{Stakeholder} bezeichnet und umfassen Betreiber und Entwickler.
%
%Die Stakeholder müssen aber das Nutzer- und das Anwendungsverhalten verstehen und nachvollziehen können. Nur so kann eine effiziente Instandhaltung und erfolgreiche Weiterentwicklung der Anwendung gewährleistet werden. Bei JavaScript-basierten Webapplikationen gibt es zusätzlich das Hindernis, dass Kontextinformationen, wie z.B. Logs, nur auf dem Client verfügbar sind. Diese Hürden müssen überkommen werden, es muss eine \textbf{Nachvollziehbarkeit} erreicht werden.

\section{Problemstellung}

%Die besonderen Hürden bei GUIs und speziell bei JavaScript-basierten Webapplikationen, welche die Nachvollziehbarkeit einschränken, gilt es zu bewältigen.

%\begin{wrapfigure}[12]{r}{0.33\linewidth}
%	\centering
%	\vspace{-\baselineskip}
%	\includegraphics[width=\linewidth]{img/instagram-feedback/instagram-feedback.jpg}
%	\caption{Formular aus der Instagram \cite{Instagram} Android App}
%	\label{fig:instagram-feedback-example}
%\end{wrapfigure}

%Aufgrund dieser Bedingungen werden bei Webprojekten Probleme im Feld erwartet. Zur Behebung dieser Mängel benötigen die Stakeholder Informationen. Für Nutzer gibt es daher oftmals Formulare um diese Auffälligkeiten zu melden (Beispiel siehe rechts). Die Einbindung solcher Formulare ist zeit- und kostengünstig, kann aber nur erfolgreich sein, wenn die Nutzer verständliches und informatives Feedback geben können und wollen.

%Diese Hürden für Stakeholder umfassen folgendes:
%\begin{enumerate}
%	\item Kontextinformationen sind nicht einsehbar und
%	\item Nutzerverhalten ist nicht bekannt.
%\end{enumerate}
%
%Problemberichte sind eine gängige Wahl, um den Stakeholdern eine  Verständnishilfe zu bieten (vgl. ~\autoref{fig:instagram-feedback-example}). Bettenburg \etal \cite{WhatMakesAGoodBugReport} fanden jedoch bei Problemberichten eine Diskrepanz, zwischen dem was die Stakeholder als hilfreich empfanden und dem was Nutzer ihnen als Bericht lieferten.
%
%Dies lässt folgern, dass die Informationserhebung nicht allein von den Nutzern aus geschehen sollte. Es ist ein übergreifendes Konzept anzustreben, welches die Informationslücke ausgleicht und eine Nachvollziehbarkeit für die Stakeholder ermöglicht.

Eine Problematik ergibt sich nun aus der Einschränkung, dass den Entwicklern und Betreibern die Möglichkeit fehlt, an notwendige Informationen zu gelangen.

Aufgespalten finden sich aus der Problematik vier Kernprobleme:
\begin{enumerate}
	\item Welche Informationen gibt es und wie helfen diese bei der Nachvollziehbarkeit?
	\item Wie können diese Informationen erfasst werden?
	\item Wie können diese Informationen an die Stakeholder gelangen?
	\item Wie können diese Informationen für die Stakeholder verständlich aufbereitet werden?
\end{enumerate}

%\begin{itemize}
%	\item 	Was soll mit der Arbeit erreicht werden? Welche Ziele werden angestrebt?
%			Möglichst kurz und präzise geplante Ergebnisse umreißen. Daran werden
%			Ihre Resultate am Ende gemessen!
%\end{itemize}
\section{Zielsetzung}

\nomenclature[Fachbegriff]{SPA}{Single Page Application}
\nomenclature[Fachbegriff]{DSGVO}{Datenschutz Grundverordnung}

Ziel dieser Arbeit ist es, eine Möglichkeit zu schaffen, dass die Stakeholder die Interaktionen eines Nutzers und das Verhalten einer Webapplikation nachvollziehen können. Dieses Ziel wird unter dem Begriff ``Nachvollziehbarkeit`` zusammengefasst.

Folgende Fragen sollen im Zuge der Ausarbeitung beantwortet werden:

% Feedback von Stephan: Gefällt mir in der neuen Fassung gut. Ggf. kannst du zu 1 noch 2-3 Unterpunkte hinzufügen worauf du genauer eingehen willst -> z.B. Fehler, "Nutzerverhalten" (wie navigiert der User durch die Anwendung), ...

\begin{enumerate}
	\item Was bedeutet Nachvollziehbarkeit?
	\item Warum ist Nachvollziehbarkeit wichtig?
	\item Wie kann eine gute Nachvollziehbarkeit erreicht werden?
	\begin{enumerate}
		\item Was wird hierzu benötigt?
		\item Wie kann Nutzerverhalten nachvollzogen werden?
		\item Wie können Fehler nachgestellt werden?
	\end{enumerate}
	\item Wie ist eine Webapplikation zu erweitern um dies zu erreichen? \\ (Hierbei sollen Projekte von Open Knowledge untersucht werden.)
	\begin{enumerate}
		\item Was für Technologien helfen hierbei?
		\item Was sind die Auswirkungen für den Nutzer?
		\begin{enumerate}
			\item Wird die Leistung der Webapplikation beeinträchtigt?
			\item Wie wird mit seinen Daten umgegangen (Stichwort DSGVO)?
		\end{enumerate}
	\end{enumerate}
\end{enumerate}


\subsection{Abgrenzung}

% Feedback von Christian: ja finde ich auch gut. Ich war zwischendurch am überlegen, in welchem Rahmen man das betrachten kann, da Datenschutz und Datensicherheit ja ein großes Thema ist. Also, ob man da nicht ggf. eine Abgrenzung machen sollte.

Bei der Betrachtung von Webapplikationen, sollen nur jene betrachtet werden, die dynamisch mit JavaScript erzeugt werden - auch Single-Page-Applications (SPAs) genannt.

Bei der Implementierung soll ein Proof-of-Concept (PoC) bzw. eine Demonstration der vorgestellten Methoden und Konzepte erstellt werden. Eine allgemeingültige Lösung übersteigt den Rahmen der Bachelorarbeit und ist nicht anzustreben.

\nomenclature[Fachbegriff]{PoC}{Proof-of-Concept}

Im Zuge der Implementierung der Erweiterung, soll beleuchtet werden, welche Daten erhoben werden und wie mit diesen umgegangen wird. Eine Analyse der Datenverarbeitung in Hinblick auf vollste Konformität mit der DSGVO wird jedoch nicht Bestandteil sein.

\pagebreak

%\begin{itemize}
%	\item 	Wie wird vorgegangen, um das Ziel zu erreichen?
%	\item 	Warum ist die Arbeit so gegliedert, wie sie gegliedert ist?
%	\item 	Welche Aspekte werden nicht behandelt und warum?
%\end{itemize}
\section{Vorgehensweise}

Um das Ziel, die Erstellung eines Proof-of-Concept (PoC), zu erreichen, wird vorerst die Literatur geprüft und die Ausgangssituation erörtert und beschrieben. Die Nachvollziehbarkeit und ihr Nutzen werden vorgestellt.

%Es ist dem Leser zu vermitteln, was die theoretischen Grundlagen sind und wie die der Nachvollziehbarkeit definiert wird. Es gilt zu erörtern, warum die Nachvollziehbarkeit erstrebenswert ist und wie sehr sie bereits Beachtung findet.

Durch das gewonnene Verständnis über die Ausgangssituation, werden nun Methoden aufgezeigt, welche eine Nachvollziehbarkeit unterstützen. Methoden wie Fehlerberichte, Logging, Monitoring und ggf. andere sind zu betrachten.

Auf Basis des detaillierten Verständnisses der Problemstellung und der Methoden wird nun die Erstellung eines PoC Fokus sein. Das PoC erfolgt auf Basis bestehender Webapplikation der Open Knowledge GmbH. Sie soll als Ziel haben, die Nachvollziehbarkeit zu erhöhen.

Ist ein PoC nun erstellt, wird analysiert, welchen Einfluss es auf die Nachvollziehbarkeit hat und ob die gewünschten Ziele erreicht wurden.

% Anfangs soll identifiziert werden, was alles für einen Nutzer ein Problem darstellen kann. Es muss sich bei den Problemen nicht nur um Laufzeitfehler o.Ä. handeln, sondern Logikfehler oder auch Verständnisprobleme führen zu einer Einschränkung der Nutzbarkeit. Hier soll eine Analyse aus der Literatur und ggf. einer Umfrage erstellt werden, um daraufhin grobe Problembilder zu klassifizieren.

% Danach soll erörtert werden, welche Ursachen es für häufige Problemszenarien gibt. Darauf aufbauend könnten Aussagen getroffen werden, wie diese Szenarien bereits vorher vermeidet werden können oder die Häufigkeit reduziert werden kann.

%Durch das gewonnene Verständnis über diese Probleme, soll nun ein Konzept erarbeitet werden. Dieses Konzept soll darauf abzielen zu den einzelnen Problembildern jeweils einen Ansatz zu finden, diese aufzuzeigen und den Stakeholdern Informationen zu liefern, die bei der Behebung notwendig sind (bspw. Geräteinformationen, Sitzungsinformationen, Logs, etc.).

% Zunächst soll ein Konzept erstellt werden, welches die Komponenten definiert und beschreibt und die Interaktion zwischen den Komponenten umfasst. Auf Basis der Konzeptes soll nun die Implementierung der Erweiterung erfolgen.

% Da nun ein Basisverständnis gewonnen wurde, sollen etablierte Technologien wie Google Cloud \cite{GoogleCloudErrorReporting}, Dynatrace \cite{DynatraceDigitalExperienceMonitoring}, Sentry \cite{SentryForJavaScript} und LogRocket \cite{LogRocket} näher betrachtet werden. Durch die Erörterung des Stands der Technik sollen folgend Empfehlungen ausgesprochen werden, für welche Projekte welche Technologie oder Kombination von Technologien sinnvoll ist. Sollte keine Technologie als angemessen betrachtet werden, so soll basierend auf den Anforderungen ein Vorschlag gemacht werden, wie so eine Technologie aussehen könnte.

%\newpage

\section{Open Knowledge GmbH}

%{\color{red}TODO: Dieser Abschnitt muss noch überarbeitet werden}

Die Bachelorarbeit wird unter anderem im Rahmen einer Werkstudententätigkeit innerhalb der Open Knowledge GmbH erstellt. Der Standortleiter des Standortes Essen, Dipl. Inf. Stephan Müller, übernimmt die Zweitbetreuung.

Die Open Knowledge GmbH ist ein brancheneutrales mittelständisches Dienstleistungsunternehmen mit dem Ziel bei der Analyse, Planung und Durchführung von Softwareprojekten zu unterstützen. Das Unternehmen wurde im Jahr 2000 in Oldenburg, dem Hauptsitz des Unternehmens, gegründet und beschäftigt heute 74 Mitarbeiter. Mitte 2017 wurde der zweite Standort in Essen eröffnet an dem aktuell 13 Mitarbeiter angestellt sind.

Die Mitarbeiter von Open Knowledge übernehmen in Kundenprojekten Aufgaben bei der Analyse über die Projektziele und der aktuellen Ausgangssituationen, der Konzeption der geplanten Software, sowie der anschließenden Implementierung. Die erstellten Softwarelösungen stellen Individuallösungen dar und werden den Bedürfnissen der einzelnen Kunden entsprechend konzipiert und implementiert. Technisch liegt die Spezialisierung bei der Mobile- und bei der Java Enterprise Entwicklung, bei der stets moderne Technologien und Konzepte verwendet werden. Aufgrund der großen Expertise in den Bereichen Technologien und Konzepte sind sowohl die Geschäftsführer als auch diverse Mitarbeiter der Open Knowledge GmbH als Redner auf Fachmessen wie der Javaland oder Autoren in Fachzeitschriften wie dem Java Magazin vertreten.% \cite{VincentOpenKnowledge} % Auch aufgrund der medialen Präsenz konnte die Open Knowledge GmbH in den letzten Jahre zahlreiche Projekte für namhafte Kunden, wie z.B. Vodafone, die HUK-Coburg, die Daimler AG und die DB Schenker AG übernehmen. \cite{VincentOpenKnowledge}

%Durch das breite Dienstleistungsangebot und die Branchenneutralität übernimmt die Open Knowledge GmbH Projekte aus der Automobilindustrie, der Luft- und Raumfahrt, dem Bankwesen und dem Versicherungswesen innerhalb des nationalen und europäischen Raums.
%
%Entsprechend der Philosophie ``Offenkundig Gut`` und dem Namen des Unternehmens wird stets versucht das benötigte Wissen zur Erstellung und Wartung der Software mit dem Kunden zu teilen, sodass die Kunden nach Abschluss der Projekte in der Lage sind, die Software selbstständig zu Pflegen und die Projektarbeit von OK damit abgeschlossen ist \cite{OpenKnowledgePhilosophie}.

\pagebreak
