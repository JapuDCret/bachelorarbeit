% a hack, to make the Motivation fit into one page..
\vspace{-\baselineskip}

%In diesem Unterkapitel sollten folgende Punkte behandelt werden:
%\begin{itemize}
%	\item	Was ist das Problem
%	\item 	Problemgeschichte?
%\end{itemize}
\section{Motivation}

Die Open Knowledge GmbH ist als branchenneutraler Softwaredienstleister in einer Vielzahl von Branchen wie Automotive, Logistik, Telekommunikation und Versicherungs- und Finanzwirtschaft aktiv. Zu den zahlreichen Kunden der Open Knowledge GmbH gehört auch ein führender deutscher Direktversicherer. 

Ein Direktversicherer bietet Versicherungsprodukte seinen Kunden ausschließlich im Direktvertrieb, d. h. vor allem über das Internet und zusätzlich auch über Telefon, Fax oder Brief an. Im Unterschied zum klassischen Versicherer verfügt ein Direktversicherer jedoch über keinen Außendienst oder Geschäftsstellen, wo Kunden eine persönliche Beratung bekommen können. Da das Internet der primäre Vertriebskanal ist, gehört heute ein umfassender Online-Auftritt zum Standard. Dieser besteht typischerweise aus einem Kundenportal mit der Möglichkeit Angebote für Versicherungsprodukte berechnen und abschließen zu können, sowie persönliche Daten und Verträge einzusehen.

\nomenclature[Fachbegriff]{Serverseitiges Rendering}{Die darzustellenden Inhalte, werden beim Server generiert und der Client stellt diese dar. Beispielsweise sind Anwendungen mit PHP oder auch eine Java Web Application}
\nomenclature[Fachbegriff]{Clientseitiges Rendering}{Der Server stellt dem Client lediglich die Logik und die notwendigen Daten bereit, die eigentliche Inhaltsgenerierung geschieht im Client. Für ein Beispiel siehe \autoref{subsec:singe-page-applications}}

% i.d.R. getrennt: https://www.scribbr.de/wissenschaftliches-schreiben/abkuerzungen/
Während in der Vergangenheit Online-Auftritte i. d. R. als Webapplikation mit serverseitigen Rendering realisiert wurden, sind heutzutage Javascript-basierte Webapplikation mit clientseitigem Rendering der Standard. Bei einer solchen Webapplikation befindet sich die gesamte Logik mit Ausnahme der Berechnung des Angebots und der Verarbeitung der Antragsdaten im Browser des Nutzers.

Im produktiven Einsatz kommt es auch bei gut getesteten Webapplikationen hin und wieder vor, dass es zu unvorhergesehenen Fehlern in der Berechnung oder Verarbeitung kommen kann. Liegt die Ursache für den Fehler im Browser, z. B. aufgrund einer ungültigen Wertkombination, steht man vor einer Herausforderung. Während man bei Server-Anwendungen Fehlermeldungen in den Log-Dateien einsehen kann, gibt es für den Betreiber der Anwendung i. d. R. keine Möglichkeit die notwendigen Informationen über den Nutzer und seine Umgebung abzurufen. Noch wichtiger ist, dass er mitbekommt, wenn ein Nutzer ein Problem bei der Bedienung der Anwendung hat. Ohne eine aktive Benachrichtigung durch den Nutzer, sowie detaillierte Informationen, ist es dem Betreiber nicht möglich Kenntnis über das Problem zu erlangen, geschweige denn dieses nachzustellen.

Dies stellt ein Kernproblem von  Webapplikationen dar \cite{ClientSideMonitoringOfDistributedSystems}. Im Rahmen der Arbeit soll daher ein Proof-of-Concept konzipiert und umgesetzt werden, welcher dieses Kernproblem am Beispiel einer Demoanwendung löst.

%\begin{itemize}
%	\item 	Was soll mit der Arbeit erreicht werden? Welche Ziele werden angestrebt?
%			Möglichst kurz und präzise geplante Ergebnisse umreißen. Daran werden
%			Ihre Resultate am Ende gemessen!
%\end{itemize}
\section{Zielsetzung (neu)}

Es sollen keine detaillierten Erfahrungen in der Webentwicklung vom Leser erwartet werden. Damit dies gewährleistet werden kann, ist unter anderem das Projektumfeld und seine besonderen Eigenschaften zu erklären.

Das grundlegende Ziel dieser Arbeit soll es sein, den Betreibern einer JavaScript-basierten Webapplikation die Möglichkeit zu geben das Verhalten ihrer Applikation und die Interaktionen von Nutzern in speziellen Situationen nachzuvollziehen. Diese speziellen Situationen sollen sich auf Fehlerfälle und für die Betreiber interessante Situationen konzentrieren, die jeweils einen Auslösepunkt in der Applikation haben. Eine vollständige Überwachung der Applikation und des Nutzers (wie bpsw. bei Werbe-Tracking) sind nicht vorgesehen.

Die anzustrebende Lösung soll ein Proof-of-Concept sein, welches anhand einer bestehenden Demo-Anwendung erstellt werden soll. Die Demo-Anwendung soll repräsentativ eine abgespeckte JavaScript-basierte Webapplikation darstellen, bei der die zuvor benannten Hürden zur Nachvollziehbarkeit bestehen.

Vor der eigentlichen Lösungserstellung soll jedoch die theoretische Seite beleuchtet werden, indem die Nachvollziehbarkeit sowie Methoden und Praktiken zur Erreichung dieser beschrieben werden. Es gilt aktuelle Literatur und den Stand der Technik zu erörtern, in Bezug auf die Forschungsfrage.

Bei der Lösungserstellung gilt es zudem zu beleuchten, wie die Auswirkungen für die Nutzer der Webapplikation sind. Wurde die Leistung der Webapplikation beeinträchtigt (erhöhte Ladezeit, erhöhte Datenlast)? Werden mehr Daten von ihm erhoben und zu welchem Zweck (Stichwort DSGVO)?

Am Ende der Ausarbeitung soll die Forschungsfrage für die vorgestellte Demo-Anwendung beantwortet worden sein und für den Leser ist erkennbar, wie er die verwendeten Technologien, Methoden und Praktiken selber anwenden kann.

\subsection{Abgrenzung}

\nomenclature[Fachbegriff]{PoC}{Proof-of-Concept}

Bei der Betrachtung von Webapplikationen, wird sich auf Single-Page-Applications (SPAs) konzentriert, denn hier bewegt sich das Projektumfeld von der Open Knowledge GmbH. Bei der Betrachtung der Datenerhebung und -verarbeitung ist eine volle Konformität mit der DSGVO nicht zu prüfen.

\newpage

\section{Zielsetzung (alt)}

\nomenclature[Fachbegriff]{SPA}{Single Page Application}
\nomenclature[Fachbegriff]{DSGVO}{Datenschutz Grundverordnung}

Ziel dieser Arbeit ist es, eine Möglichkeit zu schaffen, dass die Stakeholder die Interaktionen eines Nutzers und das Verhalten einer Webapplikation nachvollziehen können. Dieses Ziel wird unter dem Begriff ``Nachvollziehbarkeit`` zusammengefasst. Folgende Fragen sollen im Zuge der Ausarbeitung beantwortet werden:

% Feedback von Stephan: Gefällt mir in der neuen Fassung gut. Ggf. kannst du zu 1 noch 2-3 Unterpunkte hinzufügen worauf du genauer eingehen willst -> z. B. Fehler, "Nutzerverhalten" (wie navigiert der User durch die Anwendung), ...

\begin{enumerate}
	\item Wie sieht das Projektumfeld aus?
	\item Was bedeutet Nachvollziehbarkeit (bei Webapplikationen)?
	\item Warum ist Nachvollziehbarkeit wichtig?
	\item Wie kann eine gute Nachvollziehbarkeit erreicht werden?
	\begin{enumerate}
		\item Was wird hierzu benötigt?
		\item Wie kann Nutzerverhalten nachvollzogen werden?
		\item Wie können Fehler nachgestellt werden?
	\end{enumerate}
	\item Wie ist eine Webapplikation zu erweitern, um dies zu erreichen? \\ (Hierbei wird eine Demoanwendung untersucht.)
	\begin{enumerate}
		\item Was für Technologien helfen hierbei?
		\item Was sind die Auswirkungen für den Nutzer?
		\begin{enumerate}
			\item Wird die Leistung der Webapplikation beeinträchtigt?
			\item Wie wird mit seinen Daten umgegangen (Stichwort DSGVO)?
		\end{enumerate}
	\end{enumerate}
\end{enumerate}


\subsection{Abgrenzung}

\nomenclature[Fachbegriff]{PoC}{Proof-of-Concept}

Bei der Betrachtung von Webapplikationen, wird sich auf Single-Page-Applications (SPAs) konzentriert, denn hier bewegt sich das Projektumfeld der Open Knowledge. Bei der Implementierung soll ein Proof-of-Concept für eine Demoanwendung das Ziel sein, eine allgemeingültige Lösung ist nicht anzustreben. Bei der Betrachtung der Datenerhebung und -verarbeitung ist eine volle Konformität mit der DSGVO nicht zu prüfen.

\pagebreak

%\begin{itemize}
%	\item 	Wie wird vorgegangen, um das Ziel zu erreichen?
%	\item 	Warum ist die Arbeit so gegliedert, wie sie gegliedert ist?
%	\item 	Welche Aspekte werden nicht behandelt und warum?
%\end{itemize}
\section{Vorgehensweise}

Zur Vorbereitung eines Proof-of-Concepts wird zunächst die Ausgangssituation geschildert. Speziell wird auf die Herausforderungen der Umgebung ``Browser`` eingegangen, besonders in Hinblick auf die Verständnisgewinnung zu Interaktionen eines Nutzers und des Verhaltens der Applikation. Des Weiteren wird die Nachvollziehbarkeit als solche formal beschrieben und was sie im Projektumfeld genau bedeutet.

% Um das Ziel, die Erstellung eines Proof-of-Concept (PoC), zu erreichen, wird vorerst die Literatur geprüft und die Ausgangssituation erörtert und beschrieben. Die Nachvollziehbarkeit und ihr Nutzen werden vorgestellt.

%Es ist dem Leser zu vermitteln, was die theoretischen Grundlagen sind und wie die der Nachvollziehbarkeit definiert wird. Es gilt zu erörtern, warum die Nachvollziehbarkeit erstrebenswert ist und wie sehr sie bereits Beachtung findet.

Darauf aufbauend werden allgemeine Methoden vorgestellt, mit der die Stakeholder eine bessere Nachvollziehbarkeit erreichen können. Dabei werden die Besonderheiten der Umgebung beachtet und es wird erläutert, wie diese Methoden in der Umgebung zum Einsatz kommen können.

% Durch das gewonnene Verständnis über die Ausgangssituation, werden nun Methoden aufgezeigt, welche eine Nachvollziehbarkeit unterstützen. Methoden wie Fehlerberichte, Logging, Monitoring und ggf. andere sind zu betrachten.

Auf Basis des detaillierten Verständnisses der Problemstellung und der Methoden wird nun ein Proof-of-Concept erstellt. Ziel soll dabei sein, die Nachvollziehbarkeit einer Webapplikation zu verbessern. Das Proof-of-Concept erfolgt auf Basis einer bestehenden Webapplikation der Open Knowledge GmbH. 

Ist ein PoC nun erstellt, wird analysiert, welchen Einfluss es auf die Nachvollziehbarkeit hat und ob die gewünschten Ziele erreicht wurden (vgl. Zielsetzung).

% Anfangs soll identifiziert werden, was alles für einen Nutzer ein Problem darstellen kann. Es muss sich bei den Problemen nicht nur um Laufzeitfehler o.Ä. handeln, sondern Logikfehler oder auch Verständnisprobleme führen zu einer Einschränkung der Nutzbarkeit. Hier soll eine Analyse aus der Literatur und ggf. einer Umfrage erstellt werden, um daraufhin grobe Problembilder zu klassifizieren.

% Danach soll erörtert werden, welche Ursachen es für häufige Problemszenarien gibt. Darauf aufbauend könnten Aussagen getroffen werden, wie diese Szenarien bereits vorher vermeidet werden können oder die Häufigkeit reduziert werden kann.

%Durch das gewonnene Verständnis über diese Probleme, soll nun ein Konzept erarbeitet werden. Dieses Konzept soll darauf abzielen zu den einzelnen Problembildern jeweils einen Ansatz zu finden, diese aufzuzeigen und den Stakeholdern Informationen zu liefern, die bei der Behebung notwendig sind (bspw. Geräteinformationen, Sitzungsinformationen, Logs, etc.).

% Zunächst soll ein Konzept erstellt werden, welches die Komponenten definiert und beschreibt und die Interaktion zwischen den Komponenten umfasst. Auf Basis der Konzeptes soll nun die Implementierung der Erweiterung erfolgen.

% Da nun ein Basisverständnis gewonnen wurde, sollen etablierte Technologien wie Google Cloud \cite{GoogleCloudErrorReporting}, Dynatrace \cite{DynatraceDigitalExperienceMonitoring}, Sentry \cite{SentryForJavaScript} und LogRocket \cite{LogRocket} näher betrachtet werden. Durch die Erörterung des Stands der Technik sollen folgend Empfehlungen ausgesprochen werden, für welche Projekte welche Technologie oder Kombination von Technologien sinnvoll ist. Sollte keine Technologie als angemessen betrachtet werden, so soll basierend auf den Anforderungen ein Vorschlag gemacht werden, wie so eine Technologie aussehen könnte.

%\newpage

\section{Open Knowledge GmbH}

%{\color{red}TODO: Dieser Abschnitt muss noch überarbeitet werden}

Die Bachelorarbeit wird im Rahmen einer Werkstudententätigkeit innerhalb der Open Knowledge GmbH erstellt. Der Standortleiter des Standortes Essen, Dipl. Inf. Stephan Müller, übernimmt die Zweitbetreuung.

Die Open Knowledge GmbH ist ein branchenneutrales mittelständisches Dienstleistungsunternehmen mit dem Ziel bei der Analyse, Planung und Durchführung von Softwareprojekten zu unterstützen. Das Unternehmen wurde im Jahr 2000 in Oldenburg, dem Hauptsitz des Unternehmens, gegründet und beschäftigt heute 74 Mitarbeiter. Mitte 2017 wurde in Essen der zweite Standort eröffnet, an dem 13 Mitarbeiter angestellt sind.

Die Mitarbeiter von Open Knowledge übernehmen in Kundenprojekten Aufgaben bei der Analyse über die Projektziele und der aktuellen Ausgangssituationen, der Konzeption der geplanten Software, sowie der anschließenden Implementierung. Die erstellten Softwarelösungen stellen Individuallösungen dar und werden den Bedürfnissen der einzelnen Kunden entsprechend konzipiert und implementiert. Technisch liegt die Spezialisierung bei der Mobile- und bei der Java Enterprise Entwicklung, bei der stets moderne Technologien und Konzepte verwendet werden. Die Geschäftsführer als auch diverse Mitarbeiter der Open Knowledge GmbH sind als Redner auf Fachmessen wie der Javaland oder als Autoren in Fachzeitschriften wie dem Java Magazin vertreten. % Aufgrund der großen Expertise in den Bereichen Technologien und Konzepte sind sowohl die Geschäftsführer als auch diverse Mitarbeiter der Open Knowledge GmbH als Redner auf Fachmessen wie der Javaland oder Autoren in Fachzeitschriften wie dem Java Magazin vertreten.% \cite{VincentOpenKnowledge} % Auch aufgrund der medialen Präsenz konnte die Open Knowledge GmbH in den letzten Jahre zahlreiche Projekte für namhafte Kunden, wie z. B. Vodafone, die HUK-Coburg, die Daimler AG und die DB Schenker AG übernehmen. \cite{VincentOpenKnowledge}

%Durch das breite Dienstleistungsangebot und die Branchenneutralität übernimmt die Open Knowledge GmbH Projekte aus der Automobilindustrie, der Luft- und Raumfahrt, dem Bankwesen und dem Versicherungswesen innerhalb des nationalen und europäischen Raums.
%
%Entsprechend der Philosophie ``Offenkundig Gut`` und dem Namen des Unternehmens wird stets versucht das benötigte Wissen zur Erstellung und Wartung der Software mit dem Kunden zu teilen, sodass die Kunden nach Abschluss der Projekte in der Lage sind, die Software selbstständig zu Pflegen und die Projektarbeit von OK damit abgeschlossen ist \cite{OpenKnowledgePhilosophie}.

\pagebreak
