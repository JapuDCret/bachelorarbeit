%In diesem Unterkapitel sollten folgende Punkte behandelt werden:
%\begin{itemize}
%	\item	Was ist das Problem
%	\item 	Problemgeschichte?
%\end{itemize}
\section{Motivation}

Die Open Knowledge GmbH betreut einen großen deutschen Direktversicherer. Direktversicherer bedeutet, dass die Vertreibung der Produkte nicht über Vermittler, sondern entweder klassisch per Telefon oder heutzutage überwiegend im Netz stattfindet. Dementsprechend ist ein solider Internetauftritt für den Kunden wichtig und Open Knowledge unterstützt hierbei in allen Phasen der Softwareentwicklung.

Eine der Webapplikation des Direktversicherers ermöglicht seinen Kunden (nachfolgend Nutzer genannt), dass sie selber ihre Daten eingeben können und darauf basierend Versicherungspreise erhalten, die auf der Situation der Nutzer basieren. Diese Webapplikation ist clientbasiert und beinhaltet. Diese Eingaben werden an einen Server übertragen, welcher daraufhin mit Preisen für Vertrag basierend auf den Eingaben antwortet.

Im Betrieb kommt es aber vor, dass zu manchen Eingaben der Server keine Preise berechnen kann und einen Fehler liefert. Der Fehler konnte auf das Frontend eingeschränkt werden, denn es wurden nicht erlaubte Wertkombinationen gesendet. Bei einigen dieser invaliden Eingaben, kann nicht nachvollzogen werden, wie ein Kunde zu diesem Zustand gelangt ist. Es gibt derzeit keine Möglichkeit die Informationen des Clients zu erhalten. Im Idealfall gibt es Informationen vom Backend, die bei der Anfrage mitgesendet wurden (Eingaben, User-Agent, IP), welche aber nur bedingt hilfreich sind.

% ggf. Trennung zu Problemstellung hier

Basierend auf den begrenzten Daten, ist es den Stakeholdern bisher nicht gelungen alle dieser Probleme nachzustellen. Dafür fehlen ihnen jene Informationen, die beim Nutzer im Browser vorliegen. Beispielsweise kann nicht nachvollzogen werden, welche Interaktionen der Nutzer vorgenommen hat oder wie genau seine Browserumgebung aussieht. Es handelt sich hierbei um keine Expertenanwendung, sondern um ein öffentliches Produkt im Internet. Es gibt keinen direkten Kommunikationsweg von den Nutzern zu den Stakeholdern. Das Einholen von Feedback der Nutzer ist nicht primär anzustreben, denn wie Bettenburg \etal \cite{WhatMakesAGoodBugReport} herausfanden, sind solche Berichte meist nicht ausreichend hilfreich für die Stakeholder.

Daher ist eine automatische Informationserfassung anzustreben, die den Stakeholdern Verständnis über das Nutzerinteraktionen und das Applikationsverhalten ermöglicht. Letztendlich soll aus Sicht der Stakeholder eine Nachvollziehbarkeit das Ziel sein.

Dabei ist zu beachten, dass das Projektumfeld sich im Web bewegt und das während einer Lösungserstellung mit den Limitierungen von Browsern umgegangen werden muss. Des Weiteren müssen gesetzliche Vorgaben und firmeninterne Vorgaben und Einschränkungen beachtet werden.



%\begin{itemize}
%	\item 	Was soll mit der Arbeit erreicht werden? Welche Ziele werden angestrebt?
%			Möglichst kurz und präzise geplante Ergebnisse umreißen. Daran werden
%			Ihre Resultate am Ende gemessen!
%\end{itemize}
\section{Zielsetzung}

\nomenclature[Fachbegriff]{SPA}{Single Page Application}
\nomenclature[Fachbegriff]{DSGVO}{Datenschutz Grundverordnung}

Ziel dieser Arbeit ist es, eine Möglichkeit zu schaffen, dass die Stakeholder die Interaktionen eines Nutzers und das Verhalten einer Webapplikation nachvollziehen können. Dieses Ziel wird unter dem Begriff ``Nachvollziehbarkeit`` zusammengefasst.

Folgende Fragen sollen im Zuge der Ausarbeitung beantwortet werden:

% Feedback von Stephan: Gefällt mir in der neuen Fassung gut. Ggf. kannst du zu 1 noch 2-3 Unterpunkte hinzufügen worauf du genauer eingehen willst -> z.B. Fehler, "Nutzerverhalten" (wie navigiert der User durch die Anwendung), ...

\begin{enumerate}
	\item Was bedeutet Nachvollziehbarkeit?
	\item Warum ist Nachvollziehbarkeit wichtig?
	\item Wie kann eine gute Nachvollziehbarkeit erreicht werden?
	\begin{enumerate}
		\item Was wird hierzu benötigt?
		\item Wie kann Nutzerverhalten nachvollzogen werden?
		\item Wie können Fehler nachgestellt werden?
	\end{enumerate}
	\item Wie ist eine Webapplikation zu erweitern um dies zu erreichen? \\ (Hierbei sollen Projekte von Open Knowledge untersucht werden.)
	\begin{enumerate}
		\item Was für Technologien helfen hierbei?
		\item Was sind die Auswirkungen für den Nutzer?
		\begin{enumerate}
			\item Wird die Leistung der Webapplikation beeinträchtigt?
			\item Wie wird mit seinen Daten umgegangen (Stichwort DSGVO)?
		\end{enumerate}
	\end{enumerate}
\end{enumerate}


\subsection{Abgrenzung}

% Feedback von Christian: ja finde ich auch gut. Ich war zwischendurch am überlegen, in welchem Rahmen man das betrachten kann, da Datenschutz und Datensicherheit ja ein großes Thema ist. Also, ob man da nicht ggf. eine Abgrenzung machen sollte.

Bei der Betrachtung von Webapplikationen, sollen nur jene betrachtet werden, die dynamisch mit JavaScript erzeugt werden - auch Single-Page-Applications (SPAs) genannt.

\nomenclature[Fachbegriff]{PoC}{Proof-of-Concept}

Bei der Implementierung soll ein Proof-of-Concept für eine spezielle Webapplikation das Ziel sein, eine vollständig allgemeingültige Lösung ist nicht anzustreben. Weiterhin soll beleuchtet werden, welche Daten erhoben werden und wie mit diesen umgegangen wird. Vollste Konformität mit der DSGVO soll jedoch nicht das primäre Ziel sein.

\pagebreak

%\begin{itemize}
%	\item 	Wie wird vorgegangen, um das Ziel zu erreichen?
%	\item 	Warum ist die Arbeit so gegliedert, wie sie gegliedert ist?
%	\item 	Welche Aspekte werden nicht behandelt und warum?
%\end{itemize}
\section{Vorgehensweise}

Zur Vorbereitung eines Proof-of-Concepts, soll zunächst die Ausgangssituation geschildert werden. Speziell sollen auf die Herausforderungen der Umgebung ``Browser`` eingegangen werden, besonders in Hinblick auf die Verständnisgewinnung zu Interaktionen eines Nutzers und des Verhaltens der Applikation. Des Weiteren gilt es die Nachvollziehbarkeit als solche zu erläutern.

% Um das Ziel, die Erstellung eines Proof-of-Concept (PoC), zu erreichen, wird vorerst die Literatur geprüft und die Ausgangssituation erörtert und beschrieben. Die Nachvollziehbarkeit und ihr Nutzen werden vorgestellt.

%Es ist dem Leser zu vermitteln, was die theoretischen Grundlagen sind und wie die der Nachvollziehbarkeit definiert wird. Es gilt zu erörtern, warum die Nachvollziehbarkeit erstrebenswert ist und wie sehr sie bereits Beachtung findet.

Darauf aufbauend werden allgemeine Methoden vorgestellt, mit der die Stakeholder eine bessere Nachvollziehbarkeit erreichen können. Dabei werden die Besonderheiten der Umgebung beachtet und es wird erläutert, wie diese Methoden in der Umgebung zum Einsatz kommen können.

% Durch das gewonnene Verständnis über die Ausgangssituation, werden nun Methoden aufgezeigt, welche eine Nachvollziehbarkeit unterstützen. Methoden wie Fehlerberichte, Logging, Monitoring und ggf. andere sind zu betrachten.

Auf Basis des detaillierten Verständnisses der Problemstellung und der Methoden wird nun ein Proof-of-Concept erstellt. Ziel soll dabei sein, die Nachvollziehbarkeit einer Webapplikation zu verbessern. Das Proof-of-Concept erfolgt auf Basis einer bestehenden Webapplikation der Open Knowledge GmbH. 

Ist ein PoC nun erstellt, wird analysiert, welchen Einfluss es auf die Nachvollziehbarkeit hat und ob die gewünschten Ziele erreicht wurden.

% Anfangs soll identifiziert werden, was alles für einen Nutzer ein Problem darstellen kann. Es muss sich bei den Problemen nicht nur um Laufzeitfehler o.Ä. handeln, sondern Logikfehler oder auch Verständnisprobleme führen zu einer Einschränkung der Nutzbarkeit. Hier soll eine Analyse aus der Literatur und ggf. einer Umfrage erstellt werden, um daraufhin grobe Problembilder zu klassifizieren.

% Danach soll erörtert werden, welche Ursachen es für häufige Problemszenarien gibt. Darauf aufbauend könnten Aussagen getroffen werden, wie diese Szenarien bereits vorher vermeidet werden können oder die Häufigkeit reduziert werden kann.

%Durch das gewonnene Verständnis über diese Probleme, soll nun ein Konzept erarbeitet werden. Dieses Konzept soll darauf abzielen zu den einzelnen Problembildern jeweils einen Ansatz zu finden, diese aufzuzeigen und den Stakeholdern Informationen zu liefern, die bei der Behebung notwendig sind (bspw. Geräteinformationen, Sitzungsinformationen, Logs, etc.).

% Zunächst soll ein Konzept erstellt werden, welches die Komponenten definiert und beschreibt und die Interaktion zwischen den Komponenten umfasst. Auf Basis der Konzeptes soll nun die Implementierung der Erweiterung erfolgen.

% Da nun ein Basisverständnis gewonnen wurde, sollen etablierte Technologien wie Google Cloud \cite{GoogleCloudErrorReporting}, Dynatrace \cite{DynatraceDigitalExperienceMonitoring}, Sentry \cite{SentryForJavaScript} und LogRocket \cite{LogRocket} näher betrachtet werden. Durch die Erörterung des Stands der Technik sollen folgend Empfehlungen ausgesprochen werden, für welche Projekte welche Technologie oder Kombination von Technologien sinnvoll ist. Sollte keine Technologie als angemessen betrachtet werden, so soll basierend auf den Anforderungen ein Vorschlag gemacht werden, wie so eine Technologie aussehen könnte.

%\newpage

\section{Open Knowledge GmbH}

%{\color{red}TODO: Dieser Abschnitt muss noch überarbeitet werden}

Die Bachelorarbeit wird im Rahmen einer Werkstudententätigkeit innerhalb der Open Knowledge GmbH erstellt. Der Standortleiter des Standortes Essen, Dipl. Inf. Stephan Müller, übernimmt die Zweitbetreuung.

Die Open Knowledge GmbH ist ein brancheneutrales mittelständisches Dienstleistungsunternehmen mit dem Ziel bei der Analyse, Planung und Durchführung von Softwareprojekten zu unterstützen. Das Unternehmen wurde im Jahr 2000 in Oldenburg, dem Hauptsitz des Unternehmens, gegründet und beschäftigt heute 74 Mitarbeiter. Mitte 2017 wurde in Essen der zweite Standort eröffnet, an dem 13 Mitarbeiter angestellt sind.

Die Mitarbeiter von Open Knowledge übernehmen in Kundenprojekten Aufgaben bei der Analyse über die Projektziele und der aktuellen Ausgangssituationen, der Konzeption der geplanten Software, sowie der anschließenden Implementierung. Die erstellten Softwarelösungen stellen Individuallösungen dar und werden den Bedürfnissen der einzelnen Kunden entsprechend konzipiert und implementiert. Technisch liegt die Spezialisierung bei der Mobile- und bei der Java Enterprise Entwicklung, bei der stets moderne Technologien und Konzepte verwendet werden. Aufgrund der großen Expertise in den Bereichen Technologien und Konzepte sind sowohl die Geschäftsführer als auch diverse Mitarbeiter der Open Knowledge GmbH als Redner auf Fachmessen wie der Javaland oder Autoren in Fachzeitschriften wie dem Java Magazin vertreten.% \cite{VincentOpenKnowledge} % Auch aufgrund der medialen Präsenz konnte die Open Knowledge GmbH in den letzten Jahre zahlreiche Projekte für namhafte Kunden, wie z.B. Vodafone, die HUK-Coburg, die Daimler AG und die DB Schenker AG übernehmen. \cite{VincentOpenKnowledge}

%Durch das breite Dienstleistungsangebot und die Branchenneutralität übernimmt die Open Knowledge GmbH Projekte aus der Automobilindustrie, der Luft- und Raumfahrt, dem Bankwesen und dem Versicherungswesen innerhalb des nationalen und europäischen Raums.
%
%Entsprechend der Philosophie ``Offenkundig Gut`` und dem Namen des Unternehmens wird stets versucht das benötigte Wissen zur Erstellung und Wartung der Software mit dem Kunden zu teilen, sodass die Kunden nach Abschluss der Projekte in der Lage sind, die Software selbstständig zu Pflegen und die Projektarbeit von OK damit abgeschlossen ist \cite{OpenKnowledgePhilosophie}.

\pagebreak
