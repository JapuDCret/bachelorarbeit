% \chapter{Methoden und Praktiken}

%\textit{In diesem Kapitel soll beschrieben werden, wie eine Nachvollziehbarkeit in Webanwendungen erreicht werden kann. Spezielle Methoden und Praktiken sollen vorgestellt und beleuchtet werden.}
% \textit{Hier könnte unter anderem \textbf{OpenTelemetry} betrachtet werden.}

\section{Methoden für eine bessere Nachvollziehbarkeit}
\label{sec:methoden}

Nachvollziehbarkeit bietet einen wichtigen Mehrwert für Entwickler und Betreiber von Webanwendungen. Wie aber kann eine verbesserte Nachvollziehbarkeit bei einer Webanwendung erreicht werden? In diesem Abschnitt werden einige Methoden vorgestellt mit denen diesen Ziel erreicht werden kann.

\subsection{Fehlerberichte}

\begin{wrapfigure}[16]{r}{0.4\textwidth}
\centering
\includegraphics[width=\linewidth]{img/instagram-feedback/instagram-feedback.jpg}
\caption{Fehlerbericht in der Instagram App \cite{Instagram}}
\label{fig:instagram-bug-report}
\end{wrapfigure}

Fehlerberichte sind ein klassisches Mittel, um den Nutzer aufzufordern selber Bericht über ein Problem zu erstatten (vgl \autoref{fig:instagram-bug-report}). Hiermit können Fehler, aber auch unverständliche Workflows, aufgedeckt werden. Weiterhin können Informationen des Nutzers ermittelt werden, wie es hierzu gekommen ist und warum es ein Problem darstellt, vorausgesetzt er gibt dies an. Fehlerberichte sind ein einfach einzusetzendes Mittel, welches keine oder keine starke Integration in sonstige Teile der Anwendung benötigt.

Konträr zu diesen Vorteilen stehen die von Bettenburg \etal \cite{WhatMakesAGoodBugReport} gesammelten Ergebnisse über die Effektivität von Fehlerberichten. Nutzer meldeten Informationen und Details, die sich für die Entwickler als nicht allzu hilfreich herausstellten. Diese Diskrepanz kann u. A. dadurch erklärt werden, dass Nutzer im Regelfall kein technisches Verständnis des Systems vorweisen.

\subsection{Die Grundpfeiler der Observability}

Da Fehlerberichte nicht ausreichend sind, um den Entwicklern eine angemessene Nachvollziehbarkeit zu gewährleisten, sind zusätzliche Konzepte notwendig, um dies zu erreichen. Nach Sridharan \etal \cite{DistributedSystemsObservability} sowie \cite{TraefikLogsRequestTracingAndMetrics} \cite{IntrospectiveOfTheCloudManagementToolbox} \cite{MultilevelObservabilityInCloudOrchestration} existieren drei Grundpfeiler der Observability, die in ihrer Funktion einzigartig sind und sich gegenseitig ergänzen: Logging, Metriken und Tracing.

\subsubsection{Logging}

%\textit{Folgende Fragen sollen zur Methode beantwortet werden}
%\begin{enumerate}
%	\item \textit{Gibt es Besonderheiten zu Logging in anderen Projekten (Backend vs. Frontend)?}
%	\item \textit{Wie können Logs an einen auswertenden Stakeholder gelangen?}
%	\item \textit{Welches Verhalten kann hiermit aufgedeckt/nachvollziehbar gemacht werden?}
%\end{enumerate}

Logging bezeichnet die systematische Protokollierung von Softwareprozessen und ihren internen Zuständen \cite{LearningToLog}. Diese Protokolle, auch Logs genannt, helfen Betreibern und Entwicklern nach der Ausführung einer Anwendung nachvollziehen zu können, wie die genaue Verarbeitung war. Die daraus resultierende Nachvollziehbarkeit setzt jedoch voraus, dass ausreichend detaillierte Logmeldungen in die Anwendung integriert sind und sie verstanden werden können \cite{LearningToLog}.

Logs stellen die primäre oder in manchen Fällen die einzige Methode dar, wie Betreiber und Entwickler das Verhalten einer Anwendung in einer Produktivumgebung nachvollziehen können  \cite{EventLogsForTheAnalysisOfSoftwareFailures} \cite{LearningToLog}. Gerade in Fehlersituationen können Logs kritische Informationen bereitstellen. Wie in \autoref{sec:logdaten} beschrieben ist die Erhebung von Logmeldungen in einer JavaScript-basierten Webanwendung nicht trivial und findet daher kaum Anklang.

Logmeldungen erfolgen meist textbasiert und in einem menschenlesbaren Format. Wenn nun ein Aggregator Informationen aus einer großen Menge von Logs extrahiert, ist so ein Format hinderlich, da es nicht effizient analysiert werden kann. Um dem entgegen zu wirken, kommt Structured-Logging ins Spiel. Bei Structured-Logging \cite{StructuredAndInteroperableLogging} werden die Logmeldungen in einem definierten Format erzeugt, i. d. R. in Form von Key-Value Paaren. Durch die feste Definition des Formates, z. B. als JSON-Objekt, wird bei der Loganalyse ermöglicht, effizient die notwendigen Daten zu extrahieren.

Structured-Logging ermöglicht, dass auf Basis der Protokolle komplexe Datenanalysen durchgeführt werden können \cite{StructuredAndInteroperableLogging}. Mit diesen lassen sich auch bei großen Datenmengen situationsrelevante Informationen entlocken \cite{StructuredLoggingCraftingUsefulMessageContent}. So lassen sich auch aus Logmeldungen gesonderte Informationen wie Metriken extrahieren.

\subsubsection{Metriken}

Metriken sind numerische Repräsentationen von Daten, die in einer Zeitspanne aufgenommen wurden. Mithilfe von Metriken können mathematische Konzepte verwendet werden, um Verständnis zu gewinnen und Vorhersagen zu treffen \cite{DistributedSystemsObservability}. Metriken sind zudem optimal für effiziente Datenbankabfragen sowie eine Langzeitspeicherung geeignet. Dies ist durch die einheitliche Struktur und die hauptsächlich numerischen Werte bedingt.

\begin{wrapfigure}[10]{r}{0.35\textwidth}
\centering
\includegraphics[width=\linewidth]{img/03_methoden/prometheus-metric-sample.png}
\caption{Struktur eines Prometheus-Metrik-Datensatzes \cite{DistributedSystemsObservability}}
\label{fig:prometheus-metric-datensatz}
\end{wrapfigure}

Beispielsweise identifiziert Prometheus \cite{Prometheus} Metriken über einen eindeutigen Namen und Schlüsselwertpaare (vgl. \autoref{fig:prometheus-metric-datensatz}) und speichert die assoziierten Daten mit einem Zeitstempel und einem Fließkommawert. Zudem können Technologien rund um die Metrikerhebung einfacher mit großen Datenmengen umgehen, da die Metriken aggregierbar sind \cite{DistributedSystemsObservability}.

Durch die Kompatibilität, mathematische Konzepte auf Metriken anwenden zu können, eignen sie sich zudem dafür, die Verfügbarkeit von Komponenten überprüfbar zu gestalten und erlauben so eine Übersicht der \enquote{Gesundheit} eines Gesamtsystems zu veranschaulichen \cite{MultilevelObservabilityInCloudOrchestration} \cite{DistributedSystemsObservability}. Metriken besitzen jedoch auch Grenzen in dem was sie aufdecken können, nämlich lässt sich nur mithilfe von Metriken nicht das Verhalten eines Systems zu einem exakten Zeitpunkt nachvollziehen. Hierbei helfen wiederrum Logs und bieten dabei detaillierte Einsicht in einzelnen Situationen. Damit die Kommunikation zwischen Komponenten oder Systemen nachvollziehbar wird, besonders aus Sicht eines einzelnen ursprünglichen Aufrufs, sind weder Logs noch Metriken ausreichend zielführend - aus diesem Grund entwickelte sich das Tracing \cite{MultilevelObservabilityInCloudOrchestration} \cite{DistributedSystemsObservability}.

\subsubsection{Tracing}
\label{sec:tracing}

Tracing beschäftigt sich mit dem Aufzeichnen von Kommunikationsflüssen in Softwaresystemen \cite{TowardsPerformanceToolingInteroperability}. Hierbei werden die Kommunikationsflüsse innerhalb einer Anwendung bzw. innerhalb eines Systems erfasst. Tracing zeichnet aber auch die Kommunikationsflüsse bei verteilten Systemen auf, um diese, meist komplexen Zusammenhänge, zu veranschaulichen. Das Tracing von verteilten Systemen wird als \enquote{Distributed-Tracing} bezeichnet. Ein herstellerunabhängiger Standard, der sich aus diesem Gebiet entwickelt hat, ist OpenTracing \cite{OpenTracing}.

OpenTracing bildet diese Kommunikationsflüsse über zwei grundlegende Objekte ab: Traces und Spans. Ein Span besitzt einen Anfangs- und einen Endzeitpunkt und \textit{umspannt} meist eine Methodenaufruf. Bei einer Webanwendung kann dies eine JavaScript-Funktion oder ein durch den Nutzer hervorgerufener Eventfluss sein. Werden innerhalb eines Spans weitere Spans erstellt, z. B. durch einen Methodenaufruf, dann werden diese als Kindspans des ursprünglichen Spans assoziiert. Ein Trace ist eine Menge von Spans, die alle über eine einzelne logische Aktion - wie z. B. den Druck einer Taste - ausgelöst wurden oder resultieren. Ein Trace lässt sich einerseits über die kausalen Beziehungen zwischen den Spans visualisieren (vgl. \autoref{fig:otel-causal-relationship}), oder auch über die zeitliche Reihenfolge der einzelnen Spans (vgl. \autoref{fig:otel-temporal-relationship}).

Ein verteilter Trace, oftmals \enquote{Distributed-Trace} genannt, ist ein Trace, der sich aus den Spans verschiedener Systeme zusammensetzt, die miteinander kommunizieren. Hierbei werden die Traceinformationen über zusätzliche Felder bei existierenden Aufrufen propagiert, wie z. B. dem Einfügen eines Trace-Headers bei HTTP-Anfragen. Die dann an ein Tracesystem gemeldeten Spans gehen somit über die Grenzen von Anwendungen, Prozessen und Netzwerken hinaus und bilden einen Distributed-Trace \cite{OpenTracingSpecification}. Auf Basis von reellen Aufrufen können die tatsächlichen Zusammenhänge der einzelnen Systeme miteinander nachempfunden werden.

%\begin{minipage}{.49\textwidth}
%	\centering
%	\includegraphics[width=\linewidth]{img/03_methoden/otel_causal-relationship.png}
%	\captionof{figure}{Kausale Beziehung zwischen Spans. Eigene Darstellung.}
%	\label{fig:otel-causal-relationship}
%\end{minipage}%
%\hspace{.06\textwidth}
%\begin{minipage}{.49\textwidth}
%	\centering
%	\includegraphics[width=\linewidth]{img/03_methoden/otel_temporal-relationship}
%	\captionof{figure}{Zeitliche Beziehung zwischen Spans. Eigene Darstellung.}
%	\label{fig:otel-temporal-relationship}
%\end{minipage}
 
\begin{figure}[H]
	\centering
	\includegraphics[width=0.75\linewidth]{img/03_methoden/otel_causal-relationship.png}
	\caption{Kausale Beziehung zwischen Spans. Eigene Darstellung.}
	\label{fig:otel-causal-relationship}
\end{figure}
 
\begin{figure}[H]
	\centering
	\includegraphics[width=0.75\linewidth]{img/03_methoden/otel_temporal-relationship}
	\caption{Zeitliche Beziehung zwischen Spans. Eigene Darstellung.}
	\label{fig:otel-temporal-relationship}
\end{figure}

\subsection{OpenTelemetry}
\label{subsec:opentelemetry}
% \subsection{OpenTelemetry}
% \label{subsec:opentelemetry}

Auf Basis dieser der drei Grundpfeiler Logging, Metriken und Tracing haben sich einige Technologien entwickelt. Viele dieser Ansätze sind proprietär und nicht miteinander kompatibel, weswegen das Bedürfnis einer Standardisierung entstand. Um die hier angesiedelten Technologien zu vereinheitlichen enstanden u. A. OpenTracing, OpenCensus \cite{OpenCensus} sowie OpenTelemetry, die jeweils darauf abzielen herstellerunabhängige Observability-Konzepte zu definieren.

OpenTelemetry (OTel) ist ein sich derzeit\footnote{Ein erster (General-Availability-)Release der Spezifikation ist für die erste Hälfte 2021 geplant \cite{OpenTelemetryGARelease} (Stand 01.03.2021)} entwickelnder Standard, welcher als Ziel hat, das Erfassen, Weiterleiten und Verarbeiten von Tracing-, Metrik- und Logdaten\footnotemark{} herstellerunabhängig zu ermöglichen. OTel entwickelte sich aus dem Zusammenschluss der Teams hinter den beiden Standards OpenTracing und OpenCensus  \cite{UseNixDistributiveTracing}. Microsoft, Google, führende Unternehmen und Entwickler von Observability-Technologien sowie die Cloud-Native-Computing-Foundation (CNCF) arbeiten an der Entwicklung des OTel-Standards \cite{DistributedTracingInPractice} \cite{OpenTelemetryCommunityMembers}. OTel versucht nicht nur die bisherige Landschaft zu vereinigen, sondern definiert eine zukunftsorientierte Architektur, die aus unterschiedlichen Komponenten besteht, und definiert wie diese miteinander kommunizieren \cite{DistributedTracingInPractice}.

\nomenclature[Fachbegriff]{OTel}{OpenTelemetry}
\nomenclature[Fachbegriff]{CNCF}{Cloud-Native-Computing-Foundation}
\footnotetext{Die Entwicklung einer Logging-Spezifikation ist im Gange \cite{OpenTelemetryLoggingSpecification}.}

Der OpenTelemetry-Standard definiert einige Komponenten, die spezielle Aufgabengebiete abdecken und standardisiert mit anderen Komponenten kommunizieren. Diese Komponenten werden nachfolgend anhand des Beispiels eines Tracing-Spans sowie der \autoref{fig:otel-components} erläutert.

\begin{itemize}
	\item \textbf{API}: Die API stellt die öffentlich sichtbare Schnittstelle der \enquote{low-level} OTel-Verarbeitung (des SDKs) dar, die Zielgruppe umfasst Entwickler sowie instrumentierende Bibliotheken. Mit der API kann ein Entwickler einen Trace initialisieren, darauf aufbauend einen Span erzeugen und diesen Span starten.
	
	\item \textbf{Instrumentation-Library}: Bei dieser Bibliothek handelt es sich um eine spezifische Anbindung der OTel-API bspw. an ein Framework (wie JAX-RS oder Angular). Teilweise erlauben solche Bibliotheken auch eine automatische Erfassung der Daten, sodass bei relevanten Methoden (wie Schnittstellaufrufen) ein Span erzeugt wird.
	
	\item \textbf{SDK}: Das SDK stellt das Herzstück der Logik bei OTel dar und beinhaltet die interne Verarbeitung der Daten. Ein erzeugter Span wird hier mit seinen Kontextinformationen zwischengespeichert und bei Beendigung des Spans wird dieser an angebundene Exporter übergeben.
	
	\item \textbf{Exporter}: Exporter sind spezifische Anbindungen, welche Daten im OTel-Format annehmen und für Gegenstellen aufbereiten und an diese transportieren. Die Gegenstelle kann entweder eine Datensenke darstellen oder auch eine weiterverarbeitende Komponente sein. Im Beispiel (vgl. \autoref{fig:otel-components}) wird der Span nicht in ein anderes Format überführt, da er an einen OTel-Colletor gesendet wird.
	
	\item \textbf{Collector}: Kollektoren sind eine OTel-Schnittstelle, um unterschiedliche OTel-Daten anzunehmen und diese an weitere Systeme mithilfe von Exportern zu übergeben. Bei dem Exporter in diesem Beispiel werden die Daten in das passende Format des Telemetry-Backends überführt.
	
	\item \textbf{Telemetry-Backend}: Ein Telemetry-Backend stellt die Datensenke der OTel-Daten dar und bietet den Entwicklern und Betreibern eine Visualisierung der gesammelten Daten. Beispiele hierfür sind z. B. Jaeger \cite{Jaeger} oder Prometheus \cite{Prometheus}.
\end{itemize}
 
\begin{figure}[H]
	\centering
	\includegraphics[width=1.00\linewidth]{img/03_methoden/otel_components.png}
	\caption{Komponenten von OpenTelemetry, eigene Darstellung auf Basis von \cite{OTelSpecification}}
	\label{fig:otel-components}
\end{figure}

% \section{Praktiken}

% \subsection{Fachpraxis}

\subsection{Application-Performance-Monitoring (APM)}

In der Praxis haben sich einige Technologien entwickelt und etabliert, welche die Nachvollziehbarkeit von Anwendungsverhalten und Nutzerinteraktion ermöglichen oder verbessern. Auf Basis der zuvor vorgestellten Methoden sowie neuer Ansätze haben sich in der Wirtschaft eine Reihe von Praktiken entwickelt. Einer dieser Ansätze ist das Application-Performance-Monitoring (APM, teils auch -Management).

Eine exakte Definition von APM lässt sich nicht nennen, denn es existiert kein Konsens, welche Eigenschaften und Funktionen ein APM umfasst. Ahmed \etal \cite{StudyingTheEffectivenessOfAPMTools}, Heger \etal \cite{APMStateOfTheArtAndChallenges}, Rabl \etal \cite{SolvingBigDataChallengesForAPM} sowie Dynatrace \cite{DynatraceAPM} definieren, dass ein APM eine Menge an Methoden, Techniken und Werkzeugen umfasst, die das System konstant überwachen und Aufschluss über den Zustand geben, sodass die Verfügbarkeit des Systems sichergestellt werden kann. Eine andere Definition vertreten Santos Filipe \cite{ClientSideMonitoringOfDistributedSystems}, New Relic \cite{NewRelicAPM} und Gartner \cite{GartnerMagicQuadrantForAPM}, welche APM weniger allgemeingültig, sondern über explizite Teilaspekte definieren, die es erfüllen muss, um als konformes APM bezeichnet zu werden. Aber auch innerhalb dieser Definition findet sich kein Konsens bzgl. der zu erfüllenden Aspekte, damit ein Monitoring-System zu einem APM wird.

In dieser Arbeit wird auf Basis der ersten und eher allgemeingültigen Definition ein APM definiert: Ein APM befasst sich mit dem Beobachten eines Softwaresystems und der Gewinnung von relevanten Daten aus diesem System zur näheren Analyse, um Rückschlüsse auf die Gesundheit des Systems zu ermöglichen und so die Verfügbarkeit sicherzustellen. Um dies zu erreichen, lassen sich 5 Fachgebiete einteilen, die unterschiedliche Aspekte eines Softwaresystems aufdecken \cite{ASurveyOfCloudMonitoringTools} \cite{GartnerMagicQuadrantForAPM} \cite{ResearchAndApplicationOfOperatingMonitoring}:

\begin{enumerate}
	\item Infrastruktur-Monitoring (IM)
	\item Application-and-Service-Monitoring (ASM)
	\item Real-User-Monitoring (RUM)
	\item Error-Monitoring
	\item Distributed-Tracing
\end{enumerate}

\subsubsection{Infrastructure-Monitoring (IM)}

Infrastructure-Monitoring beschäftigt sich hauptsächlich mit der Überwachung der Infrastruktur. Hierbei wird die Verfügbarkeit von Netzwerkressourcen sowie die Auslastung von Hard- und Softwareressourcen überwacht. Dieses Monitoring kann ohne Anpassungen der Software erfolgen und stellt somit ein Beispiel für Black-Box-Monitoring dar \cite{ClientSideMonitoringOfDistributedSystems}. Zum Beispiel ist die Überwachung von CPU- und Speicherausnutzung eines Systems ein Teil des Infrastructure-Monitoring.

\subsubsection{Application-and-Service-Monitoring (ASM)}

Bei Application-and-Service-Monitoring (ASM) handelt es sich um White-Box-Monitoring. Das bedeutet, dass die Softwarekomponenten angepasst werden müssen, sodass innerhalb der Laufzeitumgebungen Daten gesammelt werden können. Beispielsweise werden die Antwortzeit von Schnittstellaufrufen protokolliert und systematisch überwacht. Auf Basis der Daten lassen sich Abweichungen, von einzelnen Systemen oder vom aktuellen Gesamtsystem zu einem vorherigen Zeitpunkt feststellen.

\subsubsection{Real-User-Monitoring (RUM)}

\nomenclature[Fachbegriff]{UI}{User-Interface}

Real-User-Monitoring beschäftigt sich mit dem Mitschneiden von Nutzerinteraktionen und Umgebungseigenschaften einer Benutzeroberfläche \cite{IdentifyingWebPerformanceDegradations}. Um diese Daten zu ermitteln ist eine Änderung der Software für die Benutzeroberfläche notwendig, welches RUM zu einem White-Box-Monitoring macht. RUM wird jedoch nicht dazu verwendet, die Interaktionen eines einzelnen Nutzers aufzudecken, sondern Aufschluss über die gesamte Nutzerschaft der Anwendung zu erhalten. Die Daten werden oftmals nach den Interaktionen oder auch nach Umgebungseigenschaften, wie dem Browser der Nutzer gruppiert. Dadurch lassen sich Probleme bei der User-Experience \cite{AConceptLatticeForRecognitionOfUserProblems}, aber auch Performanceprobleme der Anwendung feststellen und sowie deren Ursache (z. B. die Umgebung des Nutzers) \cite{IdentifyingWebPerformanceDegradations}.

\subsubsection{Error-Monitoring}

Error-Monitoring konzentriert sich auf das Erfassen und Melden von Fehlern \cite{CrashbinCrashMonitoring}. Es lässt sich sowohl als White-Box- sowie als Black-Box-Monitoring umsetzen, da über existierende Protokollierung bereits Fehler festgestellt werden können. Hierzu kann es sinnvoll sein eine Software anzupassen, also White-Box-Monitoring einzusetzen, um mehr Kontextinformationen zu den Fehlern zu erfassen. Error-Monitoring wird oftmals eng mit Issue-Management verbunden, um aufgetretene Fehler und deren Behebung nachzuhalten \cite{CrashbinCrashMonitoring}.

\subsubsection{Distributed-Tracing}

Beim Distributed-Tracing handelt es sich um die fortgeschrittene Art des Tracings, welches systemübergreifend den Ablauf von Abfragen protokolliert (vgl. \autoref{sec:tracing}). Diese Art von Monitoring gibt, anders als die zuvor beschriebenen Arten, keine Einsicht in einzelne Komponenten, sondern veranschaulicht die resultierenden Interaktionen einer Abfrage.

\subsection{Log-Management}

Neben dem APM gibt es zudem weitere Funktionalitäten, wie z. B. das Log-Management, die Technologien in der Praxis vorweisen. Log-Management umfasst die Erfassung, Speicherung, Verarbeitung und Analyse von Logdaten von Anwendungen. Neben diesen Funktionen bieten solche Werkzeuge oftmals fundierte Suchfunktionen und Visualisierungsmöglichkeiten \cite{DesignLogManagementSystem}. Um die Daten aus einer Anwendung heraus zu exportieren, gibt es meist eine Vielzahl an Integrationen für Frameworks und Logbibliotheken.

Einer der wichtigsten Aspekte des Log-Managements, ist die Fähigkeit mit großen Datenmengen umzugehen und dabei den Nutzern zu ermöglichen mit diesen Daten zu arbeiten, Analysen durchzuführen und auch alte Datensätze abrufen zu können \cite{LoggingAndLogManagement}. Damit die teils enormen Datenmengen den Entwicklern und Betreibern zur Verfügung stehen, aber gleichzeitig nicht das System beeinträchtigen, sind spezielle Architekturkonzepte erforderlich. Beispielsweise werden selten abgerufene oder alte Daten in einen Langzeitspeicher überführt, welcher für die Speicherung optimiert ist, aber im Gegenzug keine zeitlich effizienten Ergebnisse liefern kann \cite{LoggingAndLogManagement}.

\subsection{Session-Replay}

Session-Replay beschreibt das Vorgehen, eine Sitzung eines Nutzers nachzustellen, so als ob sie gerade passiert \cite{NoBoundariesExfiltrationBySessionReplayScripts}. Hierbei können einzelne Aspekte der Anwendung nachgestellt werden, bspw. der Kommunikationsablauf oder die DOM-Manipulationen. Um eine realitätsnahe und entsprechend hilfreiche Nachstellung zu erstellen, sind viele Aspekte aufzuzeichnen. Ein realitätsnahes Session-Replay zeichnet eine enorme Datenmenge für jede Nutzersitzung auf und benötigt besonders beim Einsatz in Browsern eine effiziente Kommunikation, um die User-Experience (UX) nicht negativ zu beeinflussen \cite{AdvancedWebAnalyticsToolForMouseTracking} \cite{LogRocketPerformance}.

\begin{wrapfigure}[11]{r}{0.45\textwidth}
\vspace{-2.0\baselineskip}
\centering
\includegraphics[width=\linewidth]{img/03_methoden/timelapse_figure5.png}
\caption{Mitschneiden von DOM-Events, Abb. aus \cite{TimelapsePaper}}
\label{fig:timelapse_figure5}
\smallskip\par
\includegraphics[width=\linewidth]{img/03_methoden/timelapse_figure6.png}
\caption{Abspielen von DOM-Events, Abb. aus \cite{TimelapsePaper}}
\label{fig:timelapse_figure6}
\end{wrapfigure}

Bereits 2013 entwickelten Burg \etal \cite{TimelapsePaper} mit \enquote{Timelapse} ein Framework, um Benutzersitzungen bei Webanwendungen aufzunehmen und wiederzugeben. Timelapse unterscheidet sich zu gängigen Session-Replay-Ansätzen dahingehend, dass die Wiedergabe keine vereinfachte Nachstellung der Anwendung ist. Stattdessen wird die JavaScript-Eventloop abgekapselt und es werden die Aufrufe von und zu der Eventloop mitgeschnitten (vgl. \autoref{fig:timelapse_figure5}).

Beim Abspielen werden die Aufrufe dann in derselben Reihenfolge an die Eventloop übergeben (vgl. \autoref{fig:timelapse_figure6}). Somit erfolgt eine exakte Wiederholung einer Sitzung, welches eine detaillierte Nachvollziehbarkeit ermöglicht. Leider benötigt dieser Ansatz eine gepatchte Version von WebKit \cite{WebKit}, weswegen auch Zugriff auf das System des Endnutzers vorausgesetzt wird. Zur Veröffentlichung des Berichtes bedeutete dies, dass die Browser Safari, Chrome und Opera unterstützt wurden - jedoch benutzt heute nur noch der Safari Webkit. Aufgrund der Notwendigkeit den Browser der Nutzer zu modifizieren und der Tatsache, dass es seit mehr als 5 Jahren\footnote{Timelapse GitHub Repo \url{https://github.com/burg/replay-staging/}} nicht mehr gepflegt wird, scheidet Timelapse für die hier angestrebte Lösung aus. Die vorgestellten Konzepte stellen jedoch nützliche Kernprinzipien für das Session Replay im Allgemeinen dar.

\section{Werkzeuge und Technologien}
\label{sec:werkzeuge-und-technologien}
%\section{Werkzeuge und Technologien}
%\label{sec:werkzeuge-und-technologien}

%\textit{Basierend auf dem Grundwissen über die Methoden und Praktiken, soll nun der Stand der Technik erörtert werden. Hierbei sollen Werkzeuge und Technologien und ihre Ansätze hervorgehoben werden und mit Hilfe welcher Methoden sie welches Ziel erreichen.}
%
%\textit{Wie in der Zielsetzung definiert sollen hier zwei bis drei Technologien näher vorgestellt werden.}
%
%\textit{Weiterhin könnte beleuchtet werden, wie ähnliche Herausforderungen bei anderen „Fat-Client“-Lösungen (also nicht SPAs) angegangen werden, und kann man hier vielleicht etwas lernen oder übertragen (und wenn nicht, warum nicht)?}

Um die gewünschte Lösung, also ein Proof-of-Concept, zu erstellen, ist zuvor der Stand der Technik zu erörtern. In diesem Abschnitt wird versucht einen repräsentativen Durchschnitt aktueller Technologien vorzustellen, diese zu kategorisieren, dann auf zuvor definierten Kriterien zu bewerten und dann anschließend im Proof-of-Concept zu verwenden.

\subsection{Recherche}

Damit das gewünschte Ziel dieses Abschnitts erreicht wird, wurde neben verfügbarer Literatur auch auf etablierte Plattformen im Gebiet der Gegenüberstellung von Technologien gesetzt. Speziell wurden hierbei Gartner\footnote{Gartner ist ein global agierendes Forschungs- und Beratungsunternehmen im Bereich der IT \cite{GartnerDefinition}} sowie StackShare\footnote{StackShare (\url{https://stackshare.io}) ist eine Vergleichsseite für Entwicklerwerkzeuge und Technologien, die auf Basis von Nutzereingaben Vergleiche erzeugt \cite{StackshareDefinition}} herangezogen. Die identifizierten Technologien werden im nachfolgenden Abschnitt veranschaulicht und kategorisiert. 

Mithilfe von Gartners \enquote{Magic Quadrant for APM} \cite{GartnerMagicQuadrantForAPM} konnte festgestellt werden, dass folgende APM-Werkzeuge zu den führenden Technologien dieser Kategorie angehören: \textit{AppDynamics} \cite{AppDynamics}, \textit{Dynatrace} (ehemals ruxit) \cite{Dynatrace}, \textit{New Relic} \cite{NewRelic}, \textit{Broadcom DX APM} \cite{BroadcomDXAPM}, \textit{Splunk APM} \cite{SplunkAPM} sowie \textit{Datadog} \cite{Datadog}. Bestätigt werden einige dieser Nennungen in der Bewertung bei StackShare \cite{StackShareAPM}, insbesondere New Relic und Datadog werden oft eingesetzt und positiv bewertet. Hinzukommend wird hierbei die Application Insights \cite{AzureApplicationInsights} des \textit{Azure Monitors} von Microsoft in den Top 6 genannt.

Mart{\'i}nez \etal \cite{ComparisonOfE2ETestingToolsForMicroservices} fanden in ihrer Evaluierung von Werkzeugen bei der Unterstützung von E2E-Tests, dass die beiden OpenSource-Technologien \textit{Jaeger} \cite{Jaeger} und \textit{Zipkin} \cite{Zipkin} aktiv dabei helfen können Fehlerszenarien in Microservice-Architekturen besser nachzuvollziehen. Li \etal \cite{ServiceMeshChallengesStateOfTheArt} beschreiben, wie mit \textit{Prometheus} \cite{Prometheus}, Jaeger, Zipkin und \textit{Fluentd} \cite{Fluentd} eine Datenanalyse von Microservices ermöglicht werden kann. Weiterhin beschreiben Picoreti \etal \cite{MultilevelObservabilityInCloudOrchestration} eine Observability-Architektur, die auf Fluentd, Prometheus und \textit{Zipkin} basiert.

Bei StackShares Gegenüberstellung von Error-Monitoring-Produkten \cite{StackShareExceptionMonitoring} stehen drei Technologie hervor: \textit{Sentry} \cite{Sentry}, \textit{TrackJS} \cite{TrackJS} sowie \textit{Rollbar} \cite{Rollbar}. Sentry und TrackJS waren zudem auch bei der Gegenüberstellung der Monitoring-Lösungen \cite{StackShareMonitoring} gelistet.

StackShare bezeichnet Session-Replay als \enquote{User-Feedback-as-a-Service} und hierbei \cite{StackShareUserFeedbackAsAService} lassen sich ebenfalls drei etablierte Produkte identifizieren: \textit{Inspectlet} \cite{Inspectlet}, \textit{FullStory} \cite{FullStory} und \textit{LogRocket} \cite{LogRocket}. Während jedoch Inspectlet und FullStory hauptsächlich darauf abzielen, dass die User-Experience nachvollzogen werden kann, konzentriert sich LogRocket auf technische Informationen, die für Entwickler von Bedeutung sind \cite{Webalyt}. Gartner bietet zudem eine Übersicht \cite{GartnerWebAndMobileAppAnalytics} über Produkte im \enquote{Web and Mobile App Analytics Market} an, hierbei findet sich \textit{Google Analytics} \cite{GoogleAnalytics}, \textit{Adobe Analytics} \cite{AdobeAnalytics} sowie LogRocket auf den obersten Positionen.

\subsection{Übersicht}

Folgend werden in der \autoref{tab:technologie-uebersicht} die gefundenen Technologien näher veranschaulicht. Hierbei wird untersucht, welche Funktionalitäten die jeweilige Technologie vorweist, auf Basis der Produktbeschreiben der Hersteller. Genauer werden folgende, zuvor identifizierte, Funktionalitäten unterschieden und den Technologien zugeordnet: IM, ASM, RUM, Error-Monitoring, Log-Management, (Distributed-)Tracing sowie Session-Replay. Um anzugeben, wie der Funktionsumfang der jeweiligen Funktionalität ist, wird das Vorhandensein mit folgenden 4 Schlüsseln angegeben:

\begin{enumerate}
	\item \texttt{ja}: Die Funktionalität ist vorhanden und der Funktionsumfang entspricht der Definition.
	\item \texttt{ja(*)}: Die Funktionalität ist vorhanden, aber sie ist nicht vergleichbar umfangreich wie andere Technologien.
	\item \texttt{eingeschränkt}: Die Funktionalität ist nur unter bestimmten Voraussetzungen vorhanden oder ist nur teilweise implementiert.
	\item \textit{keine Angabe}: Die Funktionalität ist nicht vorhanden.
\end{enumerate}

\pagebreak

\hvFloat[rotAngle=90,nonFloat=true,capWidth=w]%
{table}%
{
\begin{tabular}{|p{2.25cm}|p{1.5cm}|p{2.0cm}|p{3.0cm}|p{3.0cm}|p{1.5cm}|p{2.5cm}|}
\hline
Technologie & APM & RUM & Error-Mo\-ni\-tor\-ing & Log-Management & Tracing & Session-Replay \\
\hline
Adobe Analytics &  & gruppiert & teils &  &  &  \\
\hline
AppDynamics & ja & gruppiert & ja &  & ja &  \\
\hline
Broadcom DX APM & ja &  & teils & ja & ja &  \\
\hline
DataDog & ja & gruppiert & ja & ja & ja &  \\
\hline
Dynatrace & ja & gruppiert & ja & ja & ja &  \\
\hline
Elastic Stack & möglich & möglich & möglich & ja &  &  \\
\hline
Fluentd &  &  &  & ja &  &  \\
\hline
FullStory &  & ja & teils &  &  & ja \\
\hline
Google Analytics &  & gruppiert & teils &  &  &  \\
\hline
Graylog &  &  &  & ja &  &  \\
\hline
Inspectlet &  & ja & teils &  &  & ja \\
\hline
Jaeger &  &  &  &  & ja &  \\
\hline
LogRocket &  & ja & ja & teils &  & ja \\
\hline
New Relic & ja & gruppiert & ja & ja & ja &  \\
\hline
Papertrail &  &  &  & ja &  &  \\
\hline
\end{tabular}
}
{Übersicht der untersuchten Technologien, Teil 1}
{tab:technologie-uebersicht-teil1}

\hvFloat[rotAngle=90,nonFloat=true,capWidth=w]%
{table}%
{
\begin{tabular}{|p{2.25cm}|p{1.5cm}|p{2.0cm}|p{3.0cm}|p{3.0cm}|p{1.5cm}|p{2.5cm}|}
\hline
Technologie & APM & RUM & Error-Mo\-ni\-tor\-ing & Log-Management & Tracing & Session-Replay \\
\hline
Prometheus & ja &  &  &  &  &  \\
\hline
Rollbar &  & bei \mbox{Fehlern} & ja &  &  & teils \\
\hline
Sentry &  & bei \mbox{Fehlern} & ja &  &  &  \\
\hline
Splunk APM (SignalFX) & ja &  & ja &  & ja &  \\
\hline
Splunk \mbox{Enterprise} & möglich & möglich & möglich & ja &  &  \\
\hline
TrackJS &  & bei \mbox{Fehlern} & ja &  &  &  \\
\hline
Zipkin &  &  &  &  & ja &  \\
\hline
\end{tabular}
}
{Übersicht der untersuchten Technologien, Teil 2}
{tab:technologie-uebersicht-teil2}

\subsection{Kategorisierung}

Damit die Veranschaulichung übersichtlicher wird, werden die Technologien folgend auf Basis gemeinsamer Funktionalitäten kategorisiert. Diese Kategorien ähneln oft den Gruppierungen der Quellen, jedoch wurde die Kategorisierung unabhängig dessen erstellt, sondern auf Basis der eigens evaluierten Funktionalitäten. Daraus resultierend ergaben sich 6 Funktionskategorien, in die die Technologien grob einzuordnen sind:

\begin{enumerate}
	\item APM-Plattformen
	\begin{itemize}
		\item Zu APM-Plattformen gehören allen voran Technologien, bei denen das Application-and-Service-Monitoring sowie das Infrastructure-Monitoring Kernfunktionalitäten darstellen. Jedoch begrenzen sich bis auf eines der Werkzeuge keine nur auf diese beiden Aspekte, sondern können meist mehrere andere Funktionalitäten vorweisen. Darunter am häufigsten sind  Aspekte des Error-Monitorings, des Log-Managements sowie eines Distributed-Tracings. Neben technischen Aspekten bilden viele dieser Tools mithilfe von ASM und RUM auch Einsichten in die geschäftliche Leistung der Anwendung. Auf Basis von RUM werden teils Nutzerverhalten gruppiert visualisiert, um die Nutzerschaft besser verstehen zu können - eine Ansicht einer einzelnen Nutzersitzung wie beim Session-Replay ist jedoch nicht Teil dessen.
	\end{itemize}
	\item Log-Plattformen
	\begin{itemize}
		\item Als Log-Plattformen werden alle Technologien bezeichnet, die eine Verarbeitung von Logdaten als ihre Kernfunktionalität verstehen. Weiterhin sind hier nahezu alle Werkzeuge dazu in der Lage, den Entwicklern und Betreibern eine detaillierte Analyse der Logdaten zu ermöglichen. Weiterhin können oftmals auf Basis dieser Daten auch visuelle Darstellungen erstellt werden. Mit diesen Visualisierungen können Aspekte eines IM, ASM, RUM oder Error-Monitorings nachgestellt werden. Neben diesen Funktionalitäten steht aber auch ein effizientes Persistenzkonzept im Vordergrund, damit mit den enormen Datenmengen aus unterschiedlichen Systemen umgegangen werden kann \cite{TowardsAutomatedLogParsingForLargeScale}.
	\end{itemize}
	\item Distributed-Tracing-Systeme
	\begin{itemize}
		\item Hiermit werden jene Technologien beschrieben, die ein Distributed-Tracing ermöglichen. Hierbei steht oftmals eine effiziente Architektur im Vordergrund, welche explizit auf die enormen Datenmengen angepasst sind, die beim Distributed-Tracing anfallen können \cite{DapperInfrastructure}.
	\end{itemize}
	\item Error-Tracking
	\begin{itemize}
		\item Die Kategorie Error-Tracking zeichnet sich dadurch aus, dass die Technologien hier die Erhebung und Visualisierung von Fehlerdaten als Kernfunktionalität besitzen. Weiterhin besitzen viele dieser Werkzeuge ein detailliertes Issue-Management, mit dem sich Teams organisieren können, um Fehler zu beheben und Arbeiten nachzuhalten.
	\end{itemize}
	\pagebreak
	\item Session-Replay-Dienste
	\begin{itemize}
		% ersterem/letzterem wird klein geschrieben
		% see https://www.gutefrage.net/frage/ersterem-gross-oder-klein
		\item Die Technologien der Kategorie Session-Replay zeichnen Nutzersitzungen auf und stellen diese Betreibern und Entwicklern in nachgestellter Videoform bereit. Hierbei lässt sich eine geschäftliche und eine technische Repräsentation unterscheiden. Bei ersterem werden Nutzersitzungen teils gruppiert und als Heatmaps dargestellt, bei letzterem werden detaillierte technische Informationen mitgeschnitten und dargestellt \cite{Webalyt}.
	\end{itemize}
	\item Web-Analytics
	\begin{itemize}
		\item Die letzte Kategorie beschäftigt sich mit Web-Analytics-Technologien. Diese beschäftigen sich mit der Evaluierung der Performance einer Webanwendung, sei es im geschäftlichen oder auch im technischen Sinne \cite{APracticalEvaluationOfWebAnalytics} \cite{WebAnalyticsAnHourADay}. Allgemeiner lässt sich anhand der Charakteristika sagen, dass Web-Analytics eine sehr spezifische Untermenge von APM-Plattformen darstellt.
	\end{itemize}
\end{enumerate}

Folgend werden in der \autoref{tab:technologie-kategorisierung} die Technologien gruppiert nach ihrer Kategorie dargestellt. Da nicht alle Kategorien gleich hilfreich für die hier angestrebte Lösung sind, findet im \autoref{subsec:technologie-vorauswahl} eine Vorauswahl statt, welche Funktionskategorien näher betrachtet werden sollen.

\hvFloat[rotAngle=90,nonFloat=true,capWidth=w]%
{table}%
{
\begin{tabular}{|p{2.25cm}|p{1.5cm}|p{2.0cm}|p{3.0cm}|p{3.0cm}|p{1.5cm}|p{2.5cm}|}
\hline
Technologie & APM & RUM & Error-Mo\-ni\-tor\-ing & Log-Management & Tracing & Session-Replay \\
\hline
\multicolumn{7}{|p{15.75cm}|}{Observability-Plattformen} \\
\hline
AppDynamics & ja & gruppiert & ja &  & ja &  \\
\hline
Dynatrace & ja & gruppiert & ja & ja & ja &  \\
\hline
New Relic & ja & gruppiert & ja & ja & ja &  \\
\hline
Broadcom DX APM & ja &  & teils & ja & ja &  \\
\hline
Splunk APM (SignalFX) & ja &  & ja &  & ja &  \\
\hline
DataDog & ja & gruppiert & ja & ja & ja &  \\
\hline
\multicolumn{7}{|p{15.75cm}|}{Log-Management} \\
\hline
Papertrail &  &  &  & ja &  &  \\
\hline
Elastic Stack & möglich & möglich & möglich & ja &  &  \\
\hline
Fluentd &  &  &  & ja &  &  \\
\hline
Splunk \mbox{Enterprise} & möglich & möglich & möglich & ja &  &  \\
\hline
Graylog &  &  &  & ja &  &  \\
\hline
\multicolumn{7}{|p{15.75cm}|}{Tracing} \\
\hline
Jaeger &  &  &  &  & ja &  \\
\hline
Zipkin &  &  &  &  & ja &  \\
\hline
\end{tabular}
}
{Übersicht der untersuchten Technologien, Teil 1}
{tab:technologie-uebersicht-teil1}

\hvFloat[rotAngle=90,nonFloat=true,capWidth=w]%
{table}%
{
\begin{tabular}{|p{2.25cm}|p{1.5cm}|p{2.0cm}|p{3.0cm}|p{3.0cm}|p{1.5cm}|p{2.5cm}|}
\hline
Technologie & APM & RUM & Error-Mo\-ni\-tor\-ing & Log-Management & Tracing & Session-Replay \\
\hline
\multicolumn{7}{|p{15.75cm}|}{Metriken} \\
\hline
Prometheus & ja &  &  &  &  &  \\
\hline
\multicolumn{7}{|p{15.75cm}|}{Error-Monitoring} \\
\hline
Sentry &  & bei \mbox{Fehlern} & ja &  &  &  \\
\hline
TrackJS &  & bei \mbox{Fehlern} & ja &  &  &  \\
\hline
Rollbar &  & bei \mbox{Fehlern} & ja &  &  & teils \\
\hline
\multicolumn{7}{|p{15.75cm}|}{Session-Replay} \\
\hline
Inspectlet &  & ja & teils &  &  & ja \\
\hline
FullStory &  & ja & teils &  &  & ja \\
\hline
LogRocket &  & ja & ja & teils &  & ja \\
\hline
\multicolumn{7}{|p{15.75cm}|}{Web-Analytics} \\
\hline
Google Analytics &  & gruppiert & teils &  &  &  \\
\hline
Adobe Analytics &  & gruppiert & teils &  &  &  \\
\hline
\end{tabular}
}
{Übersicht der untersuchten Technologien, Teil 2}
{tab:technologie-uebersicht-teil2}

\subsection{Vorauswahl}
\label{subsec:technologie-vorauswahl}

Wie zuvor beschrieben eignen sich die unterschiedlichen Funktionskategorien teils mehr teils weniger für die, in dieser Arbeit angestrebte Lösung: Ein Proof-of-Concept, welches die Nachvollziehbarkeit einer Webanwendung von Anwendungsverhalten und Nutzerinteraktionen für \textbf{Betreiber und Entwickler} verbessert.

APM-Plattformen bieten durch ihr IM und ASM aufschlussreiche Einsichten in die Leistung und Verfügbarkeit einer Anwendung, helfen aber nicht oder kaum bei der Aufdeckung von einzelnen Problemen. Sie bieten hilfreiche Informationen aus wirtschaftlicher und operativer Sichtweise, weniger aber bieten sie technische Informationen. Ein weiterer Grund gegen die Nutzung von Werkzeugen des Application-Performance-Monitorings ist die hier existierende Marktmacht von proprietären Lösungen, die teilweise sehr unflexibel in den Anpassungsmöglichkeiten sind, können dafür aber vorgefertigte Dashboards liefern, die Arbeit abnehmen können. Um diese Diskrepanz näher zu untersuchen wurden New Relic und Dynatrace beispielhaft näher evaluiert, indem jeweils die Testversion mit einer minimalen Beispielanwendung getestet wurde. Hierbei konnte festgestellt werden, dass die bereitgestellten Informationen, aus Sicht eines Entwicklern, nicht ausreichend Aufschluss bereiten. Es fehlten die detaillierte Einsicht in einzelne Fehlerszenarien oder auch Nutzersitzungen, viel eher wurden gruppierte Informationen bereitgestellt, die die grobe \enquote{Gesundheit} des Systems widerspiegeln. Aus diesen Gründen, gerade aufgrund der divergierenden Zielgruppen, werden APM-Plattformen nicht näher betrachtet.

Hinzukommend und aus einem ähnlichen Grund, wird die Kategorie Web-Analytics nicht näher behandelt. Werkzeuge im Web-Analytics-Bereich legen den Fokus sehr stark auf einer Überprüfung von wirtschaftlichen und operativen Eigenschaften, nicht jedoch den in dieser Arbeit erwünschten Zielen. Die übrig gebliebenen Kategorien werden im nächsten Unterabschnitt näher betrachtet und kriteriengeleitet bewertet.

\subsection{Kriterien}

Um in dem Proof-of-Concept auf die hier vorgestellten Technologien zurückgreifen zu können, werden diese auf Basis verschiedener Kriterien bewertet. Diese Bewertung stützt sich auf öffentlich verfügbare Informationen, die die Hersteller der jeweiligen Technologie selber veröffentlicht haben. Die Kriterien werden folgend näher beschrieben.

\begin{enumerate}
	\item Kostenfrei
	\begin{itemize}
		\item Mit dem Kriterium \enquote{Kostenfrei} soll bewertet werden, ob eine kostenfreie Variante dieser Technologie existiert. Existiert eine kostenfreie Variante, wird \texttt{ja} angegeben und ansonsten \texttt{nein}. Existiert jedoch eine kostenfreie Version, die entweder von den Funktionalitäten oder der zeitlichen Nutzung beschränkt ist, wird diese mit \texttt{f. beschränkt} bzw. \texttt{z. beschränkt} angegeben.
	\end{itemize}

	\item Support für Webanwendungen
	\begin{itemize}
		\item Es ist zu bewerten, ob eine Unterstützung für das Senden von Daten von Webanwendungen existiert. Dies kann in der Form eines Agenten\footnotemark{} oder einer Schnittstelle sein. Ist die Schnittstelle nicht direkt aus einem Browserkontext ansprechbar, aber es ist eine Schnittstelle vorhanden, so ist diese Technologie mit \texttt{möglich} zu bewerten. Ist jedoch keine Schnittstelle vorhanden, die das Senden von eigenen Daten ermöglicht, so ist die Technologie mit \texttt{nein} zu bewerten.
		\footnotetext{Ein Agent ist eine Bibliothek, die die jeweiligen Daten (wie Klicks, Ladezeiten für Ressourcen und DOM-Events, usw.), eigenständig sammelt und an ein Partnersystem übertragt \cite{SolvingBigDataChallengesForAPM}}
	\end{itemize}

	\item OnPremise und SaaS
	\begin{itemize}
		\item Mit diesen zwei Kriterien soll bewertet werden, wie diese Technologie eingesetzt werden kann. Ist sie in einer eigenen Infrastruktur aufsetzbar, also \enquote{On-Premise}, oder existiert die Technologie als buchbarer Dienst z. B. in der Cloud, also als Software-as-a-Service (SaaS).
	\end{itemize}
	
	\pagebreak

	% TODO
	\item Standardisierung
	\begin{itemize}
		\item Setzt die jeweilige Technologie auf etablierte Standards, wie OpenTracing bei Distributed-Tracing? Falls nicht, sind einzelne Komponenten (z. B. zur Instrumentalisierung) quelloffen oder öffentlich spezifiziert, sodass diese ausgetauscht oder angepasst werden können.
	\end{itemize}

	\item Multifunktional
	\begin{itemize}
		\item Mit dem Kriterium \enquote{Multifunktional} ist zu bewerten, ob neben der Kernfunktionalität einer Technologie diese auch eine Menge an weiteren Funktionalitäten vorweisen kann. Eine Technologie kann dies vorweisen, wenn sie min. zwei nicht nah verwandte Funktionalitäten nahezu vollständig besitzt (\texttt{ja} oder \texttt{ja(*)}). Nah verwandt sind hierbei IM und ASM.
	\end{itemize}

	\item Zielgruppe
	\begin{itemize}
		\item Es ist einzuordnen welche Zielgruppen hauptsächlich von dieser Technologie profitieren. Folgende wesentliche Zielgruppen werden differenziert:
		\begin{itemize}
			\item Projektmanager
			\item Fachabteilung
			\item Entwickler
		\end{itemize}
	\end{itemize}
\end{enumerate}

Auf Basis dieser Kriterien werden die Technologien jeweils nach Kategorie bewertet sowie wird schließlich, je nach Funktionskategorie, die präferierte Technologie ausgewählt.

\subsection{Bewertung und Auswahl}
\label{subsec:bewertung-und-auswahl}

In der Kategorie \enquote{Log-Plattformen} findet sich auf Basis der Bewertung wenig Varianz zwischen den Technologien (vgl. \autoref{tab:technologie-bewertung-log-plattformen}), lediglich Fluentd sticht hervor, aber dies ist dadurch erklärbar, dass Fluentd kein vollständiges Log-Management darstellt, sondern es sich um einen Logaggregator handelt \cite{FluentdAggregator}. Somit ist Fluentd nur als verwandte Technologie anzusehen und fällt somit als Präferenz aus. Der Elastic-Stack eignet sich durch die hohe Flexiblität und der Komponente Logstash auch dazu, Log-Management mit ihr zu betreiben  \cite{ThreatIdentificationFromAccessLogsUsingElasticStack} \cite{DesignLogManagementSystem}. Papertrail, Splunk sowie Graylog lassen sich als klassiche Log-Management-Werkzeuge verstehen, indem dass sie speziell auf diese Funktionskategorie angepasst sind. Graylog sowie der Elastic-Stack sind quelloffen, aber auch als SaaS verfügbar. Bei einem gewünschten OnPremise-Deployment kann lediglich nur Papertrail nicht eingesetzt werden, denn dies wird nicht unterstüzt. Letztendlich lässt sich sagen, dass keine dieser 4 Technologien ausschließende Eigenschaften besitzt, sie sind allesamt geeignet für die in dieser Arbeit angestrebte Lösung.

Es wurde sich für \textbf{Splunk} entschieden, da es bereits bei OpenKnowledge im Einsatz ist. Wie aber zuvor erwähnt, eignen sich die anderen Technologien ähnlich gut und eine erneute Auswahl mit anderen situationsbedingten Kriterien könnte variieren.

\input{data/tools-und-werkzeuge/tools-und-werkzeuge_bewertung-log-plattformen.tex}

Im Gebiet der Distributed-Tracing-Systeme gibt es auch nur wenige oberflächliche Unterschiede (vgl. \autoref{tab:technologie-bewertung-distributed-tracing-systeme}), sowohl Jaeger als auch Zipkin sind quelloffen, sowie sind sie weit verbreitet im Einsatz \cite{AnalysisOfDistributedTracingSystemsEffectOnPerformance}. Jaeger scheint für neue Projekten attraktiver zu sein und findet dort mehr Einsatz, wie in StackShares Gegenüberstellung zu sehen ist \cite{StackShareJaegerVsZipkin}. Teilweise ist dies erklärbar durch die Ergebnisse, die Mart{\'i}nez \etal \cite{ComparisonOfE2ETestingToolsForMicroservices} herausfanden: Jaeger zeigt mehr hilfreiche Informationen an und kann diese schneller bereitstellen als Zipkin. Zudem entwickelt Jaeger aktiv eine Unterstützung des neuen OpenTelemetry-Standards \cite{JaegerOpenTelemetry}, jedoch findet sich bei Zipkin keine vergleichbare Entwicklung. Aus diesen Gründen ist \textbf{Jaeger} hierbei das Werkzeug der Wahl.

\begin{table}[H]%
\centering
\addtolength{\leftskip}{-2cm}
\addtolength{\rightskip}{-2cm}
\begin{tabular}{|p{3.05cm}|p{1.8cm}|p{1.7cm}|p{1.2cm}|p{1.3cm}|p{1.7cm}|p{1.3cm}|p{2.6cm}|}
\hline
Technologie & Kostenfrei & Support f. Webanw. & On \mbox{Premise} & SaaS & Standard. & Multif. & Zielgruppe \\
\hline
Jaeger & ja & eingeschr. & ja & nein & ja & nein & Entwickler \\
\hline
Zipkin & ja & eingeschr. & ja & nein & eingeschr. & nein & Entwickler \\
\hline
\end{tabular}
\caption{Bewertung der Technologien der Kategorie \enquote{Distributed-Tracing-Systeme}}
\label{tab:technologie-bewertung-distributed-tracing-systeme}
\end{table}


Die Auswahl in der Kategorie \enquote{Error-Tracking} ist etwas diverser (vgl. \autoref{tab:technologie-bewertung-error-tracking}), denn hier weisen manche Technologien Funktionalitäten auf, die sonst in dem Gebiet fremd sind. Bespielweise bieten Airbrake und Raygun neben einem Error-Monitoring zudem Aspekte eines APM, sodass die Anwendung/das System auch im Normalbetrieb überprüft werden kann. Jedoch sind diese APM-Funktionalitäten nicht so ausgereift, wie bei spezialisierten APM-Lösungen. Airbrake und Raygun sind lediglich als SaaS-Produkte verfügbar, währenddessen Sentry, Rollbar und Bugsnag auch als OnPremise-Lösung verfügbar sind. Sentry ist zudem vollständig quelloffen verfügbar\footnote{Sentry GitHub Repo: \url{https://github.com/getsentry/sentry}} und entwickelt aktiv mit der Community auf GitHub \cite{GitHub}. Weiterhin ist Sentry das einzige identifizierte Werkzeug, welches eine nicht zeitlich begrenzte Version der SaaS-Lösung zur Verfügung stellt. Eine stark aussagekräftige Entscheidung kann jedoch nicht getroffen werden, da alle Werkzeuge hier adäquat die Bedingungen eines guten Error-Monitoring-Werkzeugs erfüllen. Dennoch wird sich an dieser Stelle für \textbf{Sentry} entschieden, auf Basis der zuvor nahe gelegten Gründe.

\begin{table}[H]%
\centering
\addtolength{\leftskip}{-2cm}
\addtolength{\rightskip}{-2cm}
\begin{tabular}{|p{3.05cm}|p{1.8cm}|p{1.7cm}|p{1.2cm}|p{1.3cm}|p{1.7cm}|p{1.3cm}|p{2.6cm}|}
\hline
Technologie & Kostenfrei & Support f. Webanw. & On \mbox{Premise} & SaaS & Standard. & Multif. & Zielgruppe \\
\hline
Sentry & f. begrenzt & ja & ja & ja & eingeschr. & nein & Fachabteilung, Entwickler \\
\hline
TrackJS & z. und f. begrenzt & ja & nein & ja & nein & nein & Fachabteilung, Entwickler \\
\hline
Rollbar & z. und f. begrenzt & ja & ja & ja & nein & nein & Fachabteilung, Entwickler \\
\hline
Airbrake & z. und f. begrenzt & ja & nein & ja & nein & ja & Fachabteilung, Entwickler \\
\hline
Bugsnag & z. und f. begrenzt & ja & ja & ja & eingeschr. & nein & Fachabteilung, Entwickler \\
\hline
Raygun & z. und f. begrenzt & ja & nein & ja & nein & ja & Fachabteilung, Entwickler \\
\hline
\end{tabular}
\caption{Bewertung der Technologien der Kategorie \enquote{Error-Tracking}}
\label{tab:technologie-bewertung-error-tracking}
\end{table}


In der Beschreibung zur Kategorie \enquote{Session-Replay} wurde erwähnt, dass einige dieser Werkzeuge eine eher geschäfts- und andere eher eine entwicklerorientierte Session-Replay-Funktionalität vorweisen (vgl. \autoref{tab:technologie-bewertung-session-replay-dienste}). Genauer ist FullStory fast ausschließlich für das Nachvollziehen von User-Experience konzipiert, währenddessen LogRocket sehr detaillierte und sehr technische Informationen liefert \cite{Webalyt}. Inspectlet lässt sich als Mischung dieser beiden Sichten verstehen, bietet aber z. B. nicht alle Informationen an, die LogRocket darstellt  \cite{Webalyt}. Da die hier angestrebte Lösung auf Betreiber und insbesondere Entwickler abzielt, wird sich hiermit für \textbf{LogRocket} entschieden.

\begin{table}[H]%
\centering
\addtolength{\leftskip}{-2cm}
\addtolength{\rightskip}{-2cm}
\begin{tabular}{|p{3.05cm}|p{1.8cm}|p{1.7cm}|p{1.2cm}|p{1.3cm}|p{1.7cm}|p{1.3cm}|p{2.6cm}|}
\hline
Technologie & Kostenfrei & Support f. Webanw. & On \mbox{Premise} & SaaS & Standard. & Multif. & Zielgruppe \\
\hline
Inspectlet & f. begrenzt & ja & nein & ja & nein & ja & Projektmanager, Fachabteilung, Entwickler \\
\hline
FullStory & f. begrenzt & ja & nein & ja & nein & ja & Projektmanager, Fachabteilung, Entwickler \\
\hline
LogRocket & f. begrenzt & ja & ja & ja & nein & ja & Fachabteilung, Entwickler \\
\hline
\end{tabular}
\caption{Bewertung der Technologien der Kategorie \enquote{Session-Replay-Dienste}}
\label{tab:technologie-bewertung-session-replay-dienste}
\end{table}


\subsection{Vorstellung der Technologien}

\subsubsection{Splunk}
\label{subsec:splunk}

Splunk bietet neben seiner akquirierten APM-Lösung (ehem. SignalFX) eine Log-Plattform an: Splunk Enterprise (nachfolgend auch nur Splunk genannt). Dies stellt das Kernprodukt von Splunk dar und ist eine der führenden Plattformen auf dem Markt \cite{ThreatIdentificationFromAccessLogsUsingElasticStack}. Diese Lösung wird als OnPremise sowie auch als SaaS angeboten, dies erlaubt ein Flexibilität im Deployment.

Um Splunk zu testen wurde die SaaS- sowie die OnPremise-Lösung aufgesetzt, jeweils in der kostenlosen Version. Splunk bietet keine JavaScript-Bibliotheken, die das Senden von Daten an den Dienst vereinfachen - jedoch wird eine ansprechbare HTTP-Schnittstelle angeboten und in der Dokumentation beschrieben, der sog. HTTP Event Collector (HEC) \cite{SplunkHEC}. Der HEC ist jedoch standardmäßig nicht von einem Browserkontext aus verwendbar, da er mit ablehnenden CORS-Headern antwortet. Grund hierfür ist, dass der HEC nicht für dieses Szenario konzipiert wurde. Um dies zu umgehen, wurde ein Proxydienst eingerichtet, der die Daten des Frontends entgegennimmt, diese anreichert und dann an Splunk weiterleitet.

Folgend konnten in Splunk jedwede Loginformationen eingesehen werden, sowie Fehlerdaten, welche zusätzlich an Splunk gemeldet wurden. Innerhalb von Splunk können mit der eigenen Search Processing Language (SPL) \cite{SplunkSPL} Abfragen durchgeführt werden (vgl. \autoref{fig:splunk_search-processing-language}). Mit dieser Sprache lassen sich, ähnlich wie bei SQL, einzelne Werte oder Listen abrufen aber auch neue komplexe Strukturen durch Unterabfragen generieren \cite{SplunkSQLtoSPL}.

\begin{figure}[H]
	\centering
	\includegraphics[width=\linewidth]{img/03_methoden/splunk_search-processing-language.png}
	\caption{Abfragebeispiel in Splunk aus \cite{SplunkSPL}}
	\label{fig:splunk_search-processing-language}
\end{figure}

\subsubsection{Jaeger}
\label{subsec:jaeger}

Jaeger wurde 2017 als ein OpenSource-Projekt der CNCF gestartet \cite{Jaeger}. Es ist ein System für verteiltes Tracing und bietet Funktionalitäten zur Datensammlung, --verarbeitung, und --speicherung bis hin zur Visualisierung. Jaeger unterstützt und implementiert den Standard OpenTracing, unterstützt aber auch Datenformate anderer Hersteller (wie z. B. Zipkin \cite{Zipkin}). Eine Unterstützung des OpenTelemetry-Standards ist derzeit im Gange. Weiterhin kann Jaeger dazu benutzt werden, Metriken nach Prometheus \cite{Prometheus} zu exportieren, einem weiteren CNCF-Projekt zur Speicherung und Visualisierung von Daten.

\begin{wrapfigure}[11]{r}{0.40\textwidth}
\centering
\includegraphics[width=\linewidth]{img/03_methoden/jaeger_dependency-graph.png}
\caption{Dienst-Abhängigkeits-Graph. Quelle: Eigene Darstellung}
\label{fig:jaeger-ui_dependency-graph}
\end{wrapfigure}

Jaeger spezialisiert sich auf Tracing und bietet hierfür eine skalierbare Infrastruktur zur Speicherung und Analyse der Daten. Die Traces werden als angereicherte Trace-Gantt-Diagramme dargestellt, wie in \autoref{fig:jaeger-ui_trace-detail-view} zu sehen ist. Hierbei sind sowohl hierarchische als auch zeitliche Beziehungen visualisiert. Wie bei OpenTracing und OpenTelemetry besteht ein Trace aus mehreren Spans, welche meist eine Methode umschließen. Zu den einzelnen Spans lassen sich weitere Informationen anzeigen, wenn vorhanden, wie bspw. Logmeldungen oder Kontextinformationen.

Anhand der Traces generiert Jaeger zudem automatisch eine Architektur, indem die Beziehungen zwischen Diensten zu sehen ist, dies in \autoref{fig:jaeger-ui_dependency-graph} zu betrachten.

\begin{figure}[H]
	\centering
	\includegraphics[width=\linewidth]{img/03_methoden/jaeger_trace-detail-view.png}
	\caption{Trace-Detailansicht. Eigener Screenshot aus Jaeger}
	\label{fig:jaeger-ui_trace-detail-view}
\end{figure}

\subsubsection{Sentry}
\label{subsec:sentry}

Sentry \cite{Sentry} ist ein SaaS-Produkt der Functional Software Inc., welches sich auf das Error-Monitoring spezialisiert. Die Kernfunktionalitäten beschränken sich auf das Error-Monitoring, auch wenn von anderen Praktiken einige Aspekte präsent sind, stellen diese keine eigens abgeschlossene Funktionalität dar.

Neben einer kommerziellen Version, stellt Sentry auch eine unbegrenzt kostenlos nutzbare Version bereit, welche im Rahmen dieser Arbeit evaluiert wurde. Der Quellcode für das Backend von Sentry wurde zudem veröffentlicht und Sentry bietet darüber hinaus auch eine OnPremise-Lösung an, die auf Docker basiert \cite{SentrySelfHosted}. Um von Webanwendungen Fehler zu erfassen und an Sentry zu melden, bietet Sentry bei NPM \cite{NPM} quelloffene Pakete an \cite{SentryJSGithub}. Dabei werden u. A. Anbindungen für folgende Technologien bzw. Frameworks bereitgestellt: JavaScript, Angular, React und Vue.js.

Wird ein Fehler gemeldet, erstellt Sentry hierzu ein \enquote{Issue}, also einen Problembericht. In diesem Problembericht sind detaillierte Informationen zum Fehler zu finden, wie den Stacktrace, den Zeitstempel, die Nutzerumgebung (Browser, Version, etc.) und auch einen Ausschnitt der zuletzt aufgetretenen Logmeldungen in der Browserkonsole (vgl. \autoref{fig:sentry_issue-details}). Zudem schneidet Sentry jegliche Nutzerinteraktionen mit und stellt diese in dem Problembericht mit dar (vgl. \autoref{fig:sentry_issue-event-breadcrumbs}). Treten Fehler gleichen Ursprungs auf, fasst Sentry diese im selben Problembericht zusammen, zusätzlich kann jede einzelne Fehlerinstanz näher betrachtet werden.

Die angebotenen Fehlerinformationen von Sentry sind zahlreich und helfen beim Nachvollziehen besser als Logs und Traces allein, jedoch mangelt es an einer ganzheitlichen Nachvollziehbarkeit, d. h. wenn kein Fehlerfall eingetreten ist, bietet Sentry hierfür auch keine Informationen.

\begin{figure}[H]
	\centering
	\includegraphics[width=1.00\linewidth]{img/03_methoden/sentry_issue-details.png}
	\caption{Kerninformation eines Issues. Eigener Screenshot aus Sentry}
	\label{fig:sentry_issue-details}
\end{figure}

\begin{figure}[H]
	\centering
	\includegraphics[width=1.00\linewidth]{img/03_methoden/sentry_issue-event-breadcrumbs.png}
	\caption{Verlauf der Userinteraktionen. Eigener Screenshot aus Sentry}
	\label{fig:sentry_issue-event-breadcrumbs}
\end{figure}

\subsubsection{LogRocket}
\label{subsec:logrocket}

LogRocket \cite{LogRocket} ist ein SaaS-Produkt des gleichnamigen Unternehmens und konzentriert sich auf detailliertes Session-Replay von JavaScript-basierten Clientanwendungen, um Probleme identifizieren, nachvollziehen und lösen zu können. Anders als vergleichbare Session-Replay-Technologien sind Entwickler die primäre Zielgruppe, nicht das Marketingteam o. Ä. \cite{Webalyt}.

LogRocket bietet eine kostenlose Testversion des SaaS-Produktes an, welche für die Evaluierung verwendet wurde. Zur Datenerhebung wird das Paket \texttt{logrocket} bei NPM \cite{NPM} angeboten, welches nach der Initialisierung eigenständig die notwendigen Daten sammelt. Mithilfe dieser Daten wird die gesamte Sitzung des Nutzers nachgestellt. Hierbei ist die Anwendung, die Nutzerinteraktionen, die Netzwerkaufrufe sowie das DOM zu sehen. Die Reproduktion wird videoähnlich aufbereitet und erlaubt ein präzises Nachvollziehen der zeitlichen Reihenfolge und Bedeutung (vgl. \autoref{fig:logrocket-session-replay-example}).

Neben dem JavaScript-SDK bietet LogRocket quelloffenene Plugins für folgende Bibliotheken: Redux, React, MobX, Vuex, ngrx, React Native. LogRocket ist zudem als OnPremise-Lösung verfügbar. Zusätzlich bietet LogRocket auch eine Integration für andere Tools, wie z. B. Sentry. Bei der Sentry-Integration wird bei einem gemeldeten Fehler direkt auf das \enquote{Video} in LogRocket verlinkt, sodass der Fehler genau betrachtet werden kann.

\begin{figure}[H]
	\centering
	\includegraphics[width=\linewidth]{img/03_methoden/logrocket_session-replay-example-cropped.png}
	\caption{Ausschnitt eines Session Replays. Eigener Screenshot aus LogRocket}
	\label{fig:logrocket-session-replay-example}
\end{figure}

\vspace{\baselineskip}

Auf dieser Basis ist im folgenden Kapitel der eigentliche Proof-of-Concept zu entwerfen und zu implementieren. Da manche Technologien bzw. Kategorien Überschneidungen in den Funktionalitäten vorweisen (bspw. Splunk und Sentry), kommt es ggf. dazu, dass nicht alle hier identifizierten Technologien im Proof-of-Concept zum Einsatz kommen. Vor dem Proof-of-Concept wird jedoch zunächst die Demoanwendung, auf der das Konzept angewendet werden soll, erstellt und beschrieben.