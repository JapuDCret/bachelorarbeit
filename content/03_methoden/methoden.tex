% \chapter{Methoden und Praktiken}

\textit{In diesem Kapitel soll beschrieben werden, wie eine Nachvollziehbarkeit in Webapplikationen erreicht werden kann. Spezielle Methoden und Praktiken sollen vorgestellt und beleuchtet werden.}
% \textit{Hier könnte unter anderem \textbf{OpenTelemetry} betrachtet werden.}

\section{Methoden}

\subsection{Logging}

\textit{Folgende Fragen sollen zur Methode beantwortet werden}
\begin{enumerate}
	\item \textit{Gibt es Besonderheiten zu Logging in anderen Projekten (Backend vs. Frontend)?}
	\item \textit{Wie können Logs an einen auswertenden Stakeholder gelangen??}
	\item \textit{Welches Verhalten kann hiermit aufgedeckt/nachvollziehbar gemacht werden?}
\end{enumerate}

\subsection{Monitoring}

\textit{Folgende Fragen sollen zur Methode beantwortet werden}
\begin{enumerate}
	\item \textit{Welche Anwendungseigenschaften sind zu monitoren?}
	\item \textit{Welches Verhalten kann hiermit aufgedeckt/nachvollziehbar gemacht werden?}
\end{enumerate}

\subsection{Metriken}

\textit{Folgende Fragen sollen zur Methode beantwortet werden}
\begin{enumerate}
	\item \textit{Welche Metriken können definiert?}
	\item \textit{Wie können Metriken definiert werden?}
	\item \textit{Welches Verhalten kann hiermit aufgedeckt/nachvollziehbar gemacht werden?}
\end{enumerate}

\subsection{Tracing}

\textit{Folgende Fragen sollen zur Methode beantwortet werden}
\begin{enumerate}
	\item \textit{Welche Nutzerinteraktionen sind zu tracen?}
	\item \textit{Welches Verhalten kann hiermit aufgedeckt/nachvollziehbar gemacht werden?}
\end{enumerate}

\subsection{Fehlerberichte}

\textit{Folgende Fragen sollen zur Methode beantwortet werden}
\begin{enumerate}
	\item \textit{Was genau sind Fehlerberichte (=Bug-Reports) }
	\item \textit{Welches Verhalten kann hiermit aufgedeckt/nachvollziehbar gemacht werden?}
\end{enumerate}

\section{Werkzeuge und Technologien}

\textit{Basierend auf dem Grundwissen über die Methoden und Praktiken, soll nun der Stand der Technik erörtert werden. Hierbei sollen Werkzeuge und Technologien und ihre Ansätze hervorgehoben werden und mit Hilfe welcher Methoden sie welches Ziel erreichen.}

\textit{Wie in der Zielsetzung definiert sollen hier zwei bis drei Technologien vorgestellt werden.}