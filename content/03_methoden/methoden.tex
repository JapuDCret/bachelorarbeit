% \chapter{Methoden und Praktiken}

\textit{In diesem Kapitel soll beschrieben werden, wie eine Nachvollziehbarkeit in Webapplikationen erreicht werden kann. Spezielle Methoden und Praktiken sollen vorgestellt und beleuchtet werden.}
% \textit{Hier könnte unter anderem \textbf{OpenTelemetry} betrachtet werden.}

\section{Methoden}

\subsection{Logging}

\textit{Folgende Fragen sollen zur Methode beantwortet werden}
\begin{enumerate}
	\item \textit{Gibt es Besonderheiten zu Logging in anderen Projekten (Backend vs. Frontend)?}
	\item \textit{Wie können Logs an einen auswertenden Stakeholder gelangen??}
	\item \textit{Welches Verhalten kann hiermit aufgedeckt/nachvollziehbar gemacht werden?}
\end{enumerate}

%\subsection{Monitoring}
%
%\textit{Folgende Fragen sollen zur Methode beantwortet werden}
%\begin{enumerate}
%	\item \textit{Welche Anwendungseigenschaften sind zu monitoren?}
%	\item \textit{Welches Verhalten kann hiermit aufgedeckt/nachvollziehbar gemacht werden?}
%\end{enumerate}

\subsection{Metriken}

\textit{Folgende Fragen sollen zur Methode beantwortet werden}
\begin{enumerate}
	\item \textit{Welche Metriken können definiert?}
	\item \textit{Wie können Metriken definiert werden?}
	\item \textit{Welches Verhalten kann hiermit aufgedeckt/nachvollziehbar gemacht werden?}
\end{enumerate}

\subsection{Tracing}

\textit{Folgende Fragen sollen zur Methode beantwortet werden}
\begin{enumerate}
	\item \textit{Welche Nutzerinteraktionen sind zu tracen?}
	\item \textit{Welches Verhalten kann hiermit aufgedeckt/nachvollziehbar gemacht werden?}
\end{enumerate}

\subsection{Fehlerberichte}

\textit{Folgende Fragen sollen zur Methode beantwortet werden}
\begin{enumerate}
	\item \textit{Was genau sind Fehlerberichte (=Bug-Reports) }
	\item \textit{Welches Verhalten kann hiermit aufgedeckt/nachvollziehbar gemacht werden?}
\end{enumerate}

\section{Praktiken aus der Fachpraxis}

\textit{Neben den allgemeinen Methoden weisen einige Technologien in sich geschlossene Funktionsbereiche auf, die einen eigenen Ansatz darstellen. Diese Ansätze sollen hier getrennt von der jeweiligen Technologie beschrieben werden. (Beispielsweise ``Real User Monitoring``)}

\section{Werkzeuge und Technologien}

\textit{Basierend auf dem Grundwissen über die Methoden und Praktiken, soll nun der Stand der Technik aus Fachpraxis und Literatur erörtert werden. Hierbei sollen Werkzeuge und Technologien und ihre Ansätze hervorgehoben werden und mit Hilfe welcher Methoden sie welches Ziel erreichen.}

\textit{In diesem Abschnitt werden die Technologien vom Autor evaluiert und dann beschrieben. Es soll jedoch keine Gegenüberstellung erfolgen, sondern einfach eine Evaluierung, dieses Wissen soll dann später in der Erstellung des Lösungsansatzes helfen.}

\textit{Weiterhin könnte beleuchtet werden, wie ähnliche Herausforderungen bei anderen „Fat-Client“-Lösungen (also nicht SPAs) angegangen werden, und ob man hier etwas lernen oder übertragen kann (und wenn nicht, warum nicht)?}

\section{Fachpraxis}

\textit{Eine geringe Menge (3-8) an Technologien soll hierbei evaluiert werden, im Idealfall durch praktischen Einsatz.}

\section{Literatur}

\begin{enumerate}
	\item \textit{Gibt es hierzu Ansätze in der Literatur?}
	\item \textit{Wie sehen diese aus, welchen Zweck erfüllen sie?}
	\item \textit{Sind sie vergleichbar mit Ansätzen aus der Fachpraxis?}
\end{enumerate}