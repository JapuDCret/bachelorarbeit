% \chapter{Methoden und Praktiken}

\textit{In diesem Kapitel soll beschrieben werden, wie eine Nachvollziehbarkeit in Webapplikationen erreicht werden kann. Spezielle Methoden und Praktiken sollen vorgestellt und beleuchtet werden.}
% \textit{Hier könnte unter anderem \textbf{OpenTelemetry} betrachtet werden.}

\section{Methoden}

\subsection{Logging}

\textit{Folgende Fragen sollen zur Methode beantwortet werden}
\begin{enumerate}
	\item \textit{Gibt es Besonderheiten zu Logging in anderen Projekten (Backend vs. Frontend)?}
	\item \textit{Wie können Logs an einen auswertenden Stakeholder gelangen??}
	\item \textit{Welches Verhalten kann hiermit aufgedeckt/nachvollziehbar gemacht werden?}
\end{enumerate}

%\subsection{Monitoring}
%
%\textit{Folgende Fragen sollen zur Methode beantwortet werden}
%\begin{enumerate}
%	\item \textit{Welche Anwendungseigenschaften sind zu monitoren?}
%	\item \textit{Welches Verhalten kann hiermit aufgedeckt/nachvollziehbar gemacht werden?}
%\end{enumerate}

\subsection{Metriken}

\textit{Folgende Fragen sollen zur Methode beantwortet werden}
\begin{enumerate}
	\item \textit{Welche Metriken können definiert?}
	\item \textit{Wie können Metriken definiert werden?}
	\item \textit{Welches Verhalten kann hiermit aufgedeckt/nachvollziehbar gemacht werden?}
\end{enumerate}

\subsection{Tracing}

\textit{Folgende Fragen sollen zur Methode beantwortet werden}
\begin{enumerate}
	\item \textit{Welche Nutzerinteraktionen sind zu tracen?}
	\item \textit{Welches Verhalten kann hiermit aufgedeckt/nachvollziehbar gemacht werden?}
\end{enumerate}

\subsection{Fehlerberichte}

\textit{Folgende Fragen sollen zur Methode beantwortet werden}
\begin{enumerate}
	\item \textit{Was genau sind Fehlerberichte (=Bug-Reports) }
	\item \textit{Welches Verhalten kann hiermit aufgedeckt/nachvollziehbar gemacht werden?}
\end{enumerate}

\section{Werkzeuge und Technologien}

\textit{Basierend auf dem Grundwissen über die Methoden und Praktiken, soll nun der Stand der Technik erörtert werden. Hierbei sollen Werkzeuge und Technologien und ihre Ansätze hervorgehoben werden und mit Hilfe welcher Methoden sie welches Ziel erreichen.}

\textit{Wie in der Zielsetzung definiert sollen hier zwei bis drei Technologien näher vorgestellt werden.}

\textit{Weiterhin könnte beleuchtet werden, wie ähnliche Herausforderungen bei anderen „Fat-Client“-Lösungen (also nicht SPAs) angegangen werden, und kann man hier vielleicht etwas lernen oder übertragen (und wenn nicht, warum nicht)?}

In der Fachpraxis haben sich einige Technologien über die Jahre entwickelt und etabliert, die eine verbesserte Nachvollziehbarkeit als Ziel haben. Es lassen sich zudem verschiedene Funktionskategorien festlegen, auf die sich die jeweiligen Technologien konzentrieren. Für das Projektumfeld von Webapplikationen konnten folgende Kategorien identifiziert werden.

\subsection{Kategorien}

\subsubsection{System Monitoring}

System Monitoring beschäftigt sich mit der Überwachung der notwendigen Systeme und Dienste in Bezug auf Hardware- und Softwareressourcen. Es handelt sich hierbei um ein projektunabhängiges Monitoring, welches sicherstellen soll, dass die Infrastruktur funktionstüchtig bleibt.

\subsubsection{Log Management}

Log Management umfasst die Erfassung, Speicherung, Verarbeitung und Analyse von Logdaten von Anwendungen. Weiterhin bieten Werkzeuge hierbei oftmals Such- und Meldefunktionen.

\subsubsection{Application Performance Monitoring (APM)}

Beim Application Performance Monitoring werden Daten innerhalb von Applikationen gesammelt, die Rückschlüsse auf die Perfomanz von bspw. Transaktionen geben sollen \cite{StudyingTheEffectivenessOfAPMTools}. Mit diesen Daten können dann Regressionen der Performanz, in Aspekten wie Zeitaufwand oder Ressourcennutzung, festgestellt werden.

\subsubsection{Real User Monitoring (RUM)}

Real User Monitoring beschäftigt sich mit dem Mitschneiden von allen Nutzerinteraktionen mit bspw. einer Webapplikation. Hiermit lässt sich nachvollziehen, wie ein Nutzer die Anwendung verwednet. RUM kann dazu verwendet werden um Herauszufinden, wie ein Nutzer zu einem Zustand gelangt ist. Aber es können auch ineffiziente Klickpfade hierdurch festgestellt werden und darauf basierend UX Verbesserungen vorgenommen werden.

\subsubsection{Synthetic Monitoring}

Beim Synthetic Monitoring werden Endnutzerszenarien simuliert, um zu prüfen und sicherzustellen, dass diese Szenarien wie gewünscht ablaufen. Hierbei kann auf Aspekte wie Funktionalität, Verfügbarkeit und auch verstrichene Zeit kontrolliert werden.

\subsubsection{Error/Crash Monitoring}

Das Error Monitoring konzentriert sich auf das Erfassen und Melden von Fehlern. Es werden oftmals neben dem eigentlichen Fehler auch Aspekte vom RUM und Logging gemeldet, um mehr Kontextinformationen zu liefern.

\subsection{Literatur und Fachpraxis}

Wie zuvor genannt gibt es einige Technologien, die auf eine verbesserte Nachvollziehbarkeit abzielen. Die zuvor genannte Kategorisierung ist hilfreich bei Funktionalitäten, aber nicht um gesamte Technologien einzuordnen. Denn meistens weisen Technologien eine Vielzahl der Funktionsumfänge auf. Folgend werden einige Technologien kurz vorgestellt und am Ende wird eine Übersicht erstellt, indem für den Leser erkennbar wird, welche Technologie welchen Funktionsumfang aufweist.

\subsubsection{Literatur}

\subsubsection{Fachpraxis}

\subsubsubsection{AppDynamics}

\subsubsubsection{New Relic}

\subsubsubsection{Dynatrace}

\subsubsubsection{Sentry}

\subsubsubsection{Honeycomb}

\subsubsubsection{Elastic Stack}

\subsubsubsection{Splunk}

\subsubsection{Kategorisierung}

\begin{wraptable}[13]{r}{0.55\linewidth}
\centering
\vspace{-\baselineskip}
\begin{tabularx}{\linewidth}{|l|X|}
  \hline
  Symbol & Bedeutung \\
  \hline
  K & Eine \textbf{K}ernfunktionalität, bei der der spezielle Funktionsumfang hoch und ausgereift ist. \\
  \hline
  F & Eine \textbf{F}unktionalität, die auch unterstützt wird. \\
  \hline
  T & Eine \textbf{T}eilfunktionalität, sie ist nicht funktionsreich und/oder nicht so ausgereift. \\
  \hline
   & Eine leere Zelle bedeutet, dass diese Funktionalität nicht unterstützt wird. \\
  \hline
\end{tabularx}
\caption{Legende zur Kategorisierung}
\label{tab:legende-zur-kategorisierung}
\end{wraptable}

Um diese Technologien besser einschätzen und bewerten zu können, werden sie folgend kategorisiert. In jeder Kategorie erhält die jeweilige Technologie eine Bewertung entsprechend nach \autoref{tab:legende-zur-kategorisierung}. Die Einordnung erfolgt auf der Einschätzung des Autors in Bezug auf eigener Erfahrung sowie Literatur und den Produktbeschreibungen.

Bei Technologien, die selber Daten erheben (alle Kategorien außer Log Management), wurde jeweils die Webintegration betrachtet. 

Nun kann man in \autoref{tab:kategorisierung-der-vorgestellten-technologien} betrachten, welchen Fokus die jeweiligen Technologien im Detail besitzen.

\begin{table}[H]
\centering
\begin{tabular}{|l|l|l|l|l|l|l|}
  \hline
  Technologie & \makecell{System\\Monitoring} & \makecell{Log\\Management} & APM & RUM & \makecell{Synthetic\\Monitoring} & \makecell{Error\\Monitoring} \\
  \hline
  AppDynamics   & x & x & x & x & x & x \\
  \hline
  New Relic     & K & T & K & K & F & F \\
  \hline
  Dynatrace     & K &   & K & K & F & F \\
  \hline
  Sentry        &   &   &   &   &   & K \\
  \hline
  Honeycomb     & x & x & x & x & x & x \\
  \hline
  Elastic Stack & T & K & T &   &   & T \\
  \hline
  Splunk        &   & K &   &   &   &   \\
  \hline
\end{tabular}
\caption{Kategorisierung der vorgestellten Technologien}
\label{tab:kategorisierung-der-vorgestellten-technologien}
\end{table}