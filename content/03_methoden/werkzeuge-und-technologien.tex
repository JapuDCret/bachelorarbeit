%\section{Werkzeuge und Technologien}
%\label{sec:werkzeuge-und-technologien}

%\textit{Basierend auf dem Grundwissen über die Methoden und Praktiken, soll nun der Stand der Technik erörtert werden. Hierbei sollen Werkzeuge und Technologien und ihre Ansätze hervorgehoben werden und mit Hilfe welcher Methoden sie welches Ziel erreichen.}
%
%\textit{Wie in der Zielsetzung definiert sollen hier zwei bis drei Technologien näher vorgestellt werden.}
%
%\textit{Weiterhin könnte beleuchtet werden, wie ähnliche Herausforderungen bei anderen „Fat-Client“-Lösungen (also nicht SPAs) angegangen werden, und kann man hier vielleicht etwas lernen oder übertragen (und wenn nicht, warum nicht)?}

Um die gewünschte Lösung, also ein Proof-of-Concept, zu erstellen, ist zuvor der Stand der Technik zu erörtern. In diesem Abschnitt wird versucht einen repräsentativen Durchschnitt aktueller Technologien vorzustellen, diese zu kategorisieren und dann auf zuvor definierten Kriterien zu bewerten.

\subsection{Recherche}

Um einen repräsentativen Durchschnitt aktueller Technologien zu erhalten, wurde neben verfügbarer Literatur auch auf etablierte Plattformen im Gebiet der Gegenüberstellung von Technologien gesetzt. Speziell wurden hierbei Gartner\footnote{Gartner ist ein global agierendes Forschungs- und Beratungsunternehmen im Bereich der IT \cite{GartnerDefinition}} sowie StackShare\footnote{StackShare (\url{https://stackshare.io}) ist eine Vergleichsseite für Entwicklerwerkzeuge und Technologien, die auf Basis von Nutzereingaben Vergleiche erzeugt \cite{StackshareDefinition}} herangezogen. Die identifizierten Technologie werden im nachfolgenden Abschnitt veranschaulicht und kategorisiert. 

Mithilfe von Gartners \enquote{Magic Quadrant for APM} \cite{GartnerMagicQuadrantForAPM} konnte festgestellt werden, dass folgende APM-Werkzeuge zu den führenden Technologien dieser Kategorie angehören: AppDynamics \cite{AppDynamics}, Dynatrace (ehemals ruxit) \cite{Dynatrace}, New Relic \cite{NewRelic}, Broadcom DX APM \cite{BroadcomDXAPM}, Splunk APM \cite{SplunkAPM} sowie Datadog\cite{Datadog}. Bestätigt werden einige dieser Nennungen in der Bewertung bei StackShare \cite{StackShareAPM}, insbesondere New Relic und Datadog werden oft eingesetzt und positiv bewertet. Hinzukommend wird hierbei die Application Insights \cite{AzureApplicationInsights} des Azure Monitors von Microsoft in den Top 6 genannt.

%Neben Splunks akquirierter APM-Lösung (ehemals SignalFX), ist Splunk u. A. für sein Log-Management-Produkt Splunk Enterprise \cite{SplunkEnterprise} bekannt, welches ebenso Erwähnung in der Bewertung bei StackShare \cite{StackShareAPM} erhielt. Ebenso finden sich die Komponenten des Elastic Stacks \cite{ElasticStack} (ehemals ELK-Stack) bei der Übersicht \cite{StackShareMonitoring} über Monitoring-Werkzeuge: Elasticsearch, Kibana, Beats und Logstash. In StackShares Gegenüberstellung von Log-Management-Lösungen \cite{StackShareLogManagement} finden sich neben Splunk und dem Elastic Stack zudem Papertrail \cite{Papertrail}, Fluentd \cite{Fluentd} und Graylog \cite{Graylog}.

Mart{\'i}nez \etal \cite{ComparisonOfE2ETestingToolsForMicroservices} fanden in Ihrer Evaluierung von Werkzeugen bei der Unterstützung von E2E-Tests, dass die beiden OpenSource-Technologien Jaeger \cite{Jaeger} und Zipkin \cite{Zipkin} aktiv dabei helfen können Fehlerszenarien in Microservice-Architekturen besser nachzuvollziehen. Li \etal \cite{ServiceMeshChallengesStateOfTheArt} beschreiben, wie mit Prometheus \cite{Prometheus}, Jaeger, Zipkin und Fluentd \cite{Fluentd} eine Datenanalyse von Microservices ermöglicht werden kann. Weiterhin beschreiben Picoreti \etal \cite{MultilevelObservabilityInCloudOrchestration} eine Observability-Architektur, die auf Fluentd, Prometheus und Zipkin basiert.

Bei StackShares Gegenüberstellung von Error-Monitoring-Produkten \cite{StackShareExceptionMonitoring} stehen drei Technologie hervor: Sentry \cite{Sentry}, TrackJS \cite{TrackJS} sowie Rollbar \cite{Rollbar}. Sentry und TrackJS waren zudem auch bei der Gegenüberstellung der Monitoring-Lösungen \cite{StackShareMonitoring} gelistet.

StackShare bezeichnet Session-Replay als \enquote{User-Feedback-as-a-Service} und hierbei \cite{StackShareUserFeedbackAsAService} lassen sich ebenfalls drei etablierte Produkte identifizieren: Inspectlet \cite{Inspectlet}, FullStory \cite{FullStory} und LogRocket \cite{LogRocket}. Während jedoch Inspectlet und FullStory hauptsächlich darauf abzielen, dass die User-Experience nachvollzogen werden kann, konzentriert sich LogRocket auf technische Informationen, die für Entwickler von Bedeutung sind \cite{Webalyt}. Gartner bietet eine Übersicht \cite{GartnerWebAndMobileAppAnalytics} über Produkte im \enquote{Web and Mobile App Analytics Market} an, hierbei findet sich Google Analytics \cite{GoogleAnalytics}, Adobe Analytics \cite{AdobeAnalytics} sowie LogRocket auf den obersten Positionen.

\subsection{Übersicht}

Folgend in Tabellen \ref{tab:technologie-uebersicht-teil1} und \ref{tab:technologie-uebersicht-teil2} werden die gefundenen Technologien näher veranschaulicht. Hierbei wird untersucht, welche Funktionalitäten die jeweilige Technologie vorweist, auf Basis der Produktbeschreiben der Hersteller. Genauer werden folgende Funktionalitäten unterschieden und den Technologien zugeordnet: APM, RUM, Error-Monitoring, Log-Management, (Distributed-)Tracing sowie Session-Replay.

\hvFloat[rotAngle=90,nonFloat=true,capWidth=w]%
{table}%
{
\begin{tabular}{|p{2.25cm}|p{1.5cm}|p{2.0cm}|p{3.0cm}|p{3.0cm}|p{1.5cm}|p{2.5cm}|}
\hline
Technologie & APM & RUM & Error-Mo\-ni\-tor\-ing & Log-Management & Tracing & Session-Replay \\
\hline
Adobe Analytics &  & gruppiert & teils &  &  &  \\
\hline
AppDynamics & ja & gruppiert & ja &  & ja &  \\
\hline
Broadcom DX APM & ja &  & teils & ja & ja &  \\
\hline
DataDog & ja & gruppiert & ja & ja & ja &  \\
\hline
Dynatrace & ja & gruppiert & ja & ja & ja &  \\
\hline
Elastic Stack & möglich & möglich & möglich & ja &  &  \\
\hline
Fluentd &  &  &  & ja &  &  \\
\hline
FullStory &  & ja & teils &  &  & ja \\
\hline
Google Analytics &  & gruppiert & teils &  &  &  \\
\hline
Graylog &  &  &  & ja &  &  \\
\hline
Inspectlet &  & ja & teils &  &  & ja \\
\hline
Jaeger &  &  &  &  & ja &  \\
\hline
LogRocket &  & ja & ja & teils &  & ja \\
\hline
New Relic & ja & gruppiert & ja & ja & ja &  \\
\hline
Papertrail &  &  &  & ja &  &  \\
\hline
\end{tabular}
}
{Übersicht der untersuchten Technologien, Teil 1}
{tab:technologie-uebersicht-teil1}

\hvFloat[rotAngle=90,nonFloat=true,capWidth=w]%
{table}%
{
\begin{tabular}{|p{2.25cm}|p{1.5cm}|p{2.0cm}|p{3.0cm}|p{3.0cm}|p{1.5cm}|p{2.5cm}|}
\hline
Technologie & APM & RUM & Error-Mo\-ni\-tor\-ing & Log-Management & Tracing & Session-Replay \\
\hline
Prometheus & ja &  &  &  &  &  \\
\hline
Rollbar &  & bei \mbox{Fehlern} & ja &  &  & teils \\
\hline
Sentry &  & bei \mbox{Fehlern} & ja &  &  &  \\
\hline
Splunk APM (SignalFX) & ja &  & ja &  & ja &  \\
\hline
Splunk \mbox{Enterprise} & möglich & möglich & möglich & ja &  &  \\
\hline
TrackJS &  & bei \mbox{Fehlern} & ja &  &  &  \\
\hline
Zipkin &  &  &  &  & ja &  \\
\hline
\end{tabular}
}
{Übersicht der untersuchten Technologien, Teil 2}
{tab:technologie-uebersicht-teil2}

\subsection{Kategorisierung}

Damit die Veranschaulichung übersichtlicher wird, werden die Technologien folgend auf Basis gemeinsamer Funktionalitäten kategorisiert. Diese Kategorien ähneln oft den Gruppierungen der Quellen, jedoch wurde die Kategorisierung unabhängig dessen erstellt, sondern auf Basis der eigens evaluierten Funktionalitäten. Daraus resultierend ergaben sich 7 Funktionskategorien, in die die Technologien grob einzuordnen sind:

\begin{enumerate}
	\item Performance-Monitoring
	\begin{itemize}
		\item Zur Kategorie des Performance-Monitorings gehören allen voran Technologien, bei denen das Application-Performance-Monitoring eine Kernfunktionalität darstellt. Jedoch begrenzen sich keine dieser Tools nur auf APM, sondern können meist mehrere andere Funktionalitäten vorweisen. Meist vertreten sind jedoch Aspekte eines Error-Monitoring, eines Log-Managements sowie eines Distributed Tracings. Neben technischen Aspekten bilden viele dieser Tools mithilfe von APM und RUM auch Einsichten in die geschäftliche Leistung der Anwendung. Auf Basis von RUM werden teils Nutzerverhalten gruppiert visualisiert, um die Nutzerschaft besser verstehen zu können - eine Ansicht einer einzelnen Nutzersitzung wie beim Session-Replay ist jedoch nicht Teil dessen.
	\end{itemize}
	\item Log-Management
	\begin{itemize}
		\item Im Log-Management sind alle Technologien zusammengefasst, die eine Verarbeitung von Logdaten als ihre Kernfunktionalität verstehen. Weiterhin sind hier nahezu alle Werkzeuge dazu in der Lage, den Entwicklern und Betreibern eine detaillierte Analyse der Logdaten zu ermöglichen. Weiterhin können oftmals auf Basis dieser Daten auch visuelle Darstellungen erstellt werden. Mit diesen Visualisierungen können Aspekte eines APM, RUM oder Error-Monitorings nachgestellt werden. Neben diesen Funktionalitäten steht aber auch ein effizientes Persistenzkonzept im Vordergrund, damit mit den enormen Datenmengen aus unterschiedlichen Systemen umgegangen werden kann \cite{TowardsAutomatedLogParsingForLargeScale}.
	\end{itemize}
	\item Tracing
	\begin{itemize}
		\item In der Kategorie Tracing sind jene Technologien angesiedelt, die sich nur auf Distributed Tracing konzentrieren. Hierbei steht oftmals eine effiziente Architektur im Vordergrund, welche explizit auf die enormen Datenmengen angepasst sind, die beim Distributed Tracing anfallen können \cite{DapperInfrastructure}.
	\end{itemize}
	\item Metriken
	\begin{itemize}
		\item Zur Kategorie Metriken gehören die Technologien, die nicht die fortgeschrittenen Eigenschaften der Kategorie Performance-Monitoring vorweisen, sondern sich größtenteils oder ausschließlich mit sammeln, verarbeiten und darstellen von Metrikdaten beschäftigen.
	\end{itemize}
	\item Error-Monitoring
	\begin{itemize}
		\item Die Kategorie Error-Monitoring zeichnet sich dadurch aus, dass die Technologien hier die Erhebung und Visualisierung von Fehlerdaten als Kernfunktionalität besitzen. Weiterhin besitzen viele dieser Werkzeuge ein detailliertes Issue-Management, mit dem sich Teams organisieren können, um Fehler zu beheben und Arbeiten nachzuhalten.
	\end{itemize}
	\item Session-Replay
	\begin{itemize}
		\item Die Technologien der Kategorie Session-Replay zeichnen Nutzersitzungen auf und stellen diese Betreibern und Entwicklern in nachgestellter Videoform bereit. Hierbei lässt sich jedoch eine geschäftliche und eine technische Repräsentation unterscheiden, beim ersteren werden teils Nutzersitzungen gruppiert und als Heatmaps dargestellt und beim letzteren werden detaillierte technische Informationen mitgeschnitten sowie dargestellt \cite{Webalyt}.
	\end{itemize}
	\item Web-Analytics
	\begin{itemize}
		\item Die letzte Kategorie beschäftigt sich mit Web-Analytics Technologien. Diese beschäftigen sich mit der Evaluierung der Performance einer Webanwendung, sei es im geschäftlichen oder auch im technischen Sinne \cite{APracticalEvaluationOfWebAnalytics} \cite{WebAnalyticsAnHourADay}. Allgemeiner lässt sich anhand der Charakteristika sagen, dass Web-Analytics eine sehr spezifische Untermenge des Performance-Monitorings darstellt.
	\end{itemize}
\end{enumerate}

Folgend werden in den Tabellen \ref{tab:technologie-kategorisierung-teil1} und \ref{tab:technologie-kategorisierung-teil2} die Technologien gruppiert nach ihrer Kategorie dargestellt. Da nicht alle Kategorien gleich hilfreich für die hier angestrebte Lösung sind, findet im \autoref{subsec:technologie-vorauswahl} eine Vorauswahl statt, welche Funktionskategorien näher betrachtet werden sollen.

\begingroup
\centering
\setlength{\LTleft}{-20cm plus -1fill}
\setlength{\LTright}{\LTleft}
\begin{longtable}{|p{4.10cm}|p{0.90cm}|p{0.90cm}|p{1.9cm}|p{1.75cm}|p{1.5cm}|p{1.4cm}|p{1.3cm}|}
\hline
Technologie & IM & ASM & RUM & Error-Montoring & Log-Mgmt. & Tracing & Session-Replay \\
\endhead
\hline
\hline
\multicolumn{8}{|l|}{APM-Plattformen} \\
\hline
AppDynamics & ja & ja & ja(*) & ja & eingeschr. & ja &  \\
\hline
Dynatrace & ja & ja & ja(*) & ja & ja & ja &  \\
\hline
New Relic & ja & ja & ja(*) & ja & ja & ja &  \\
\hline
Broadcom DX APM & ja & ja &  & teils & ja & ja &  \\
\hline
Splunk APM (SignalFX) & ja & ja &  & ja &  & ja &  \\
\hline
DataDog & ja & ja & ja(*) & ja & ja & ja &  \\
\hline
Azure Monitor & ja & ja &  & ja & ja & ja &  \\
\hline
Prometheus & ja & ja &  &  &  &  &  \\
\hline
\hline
\multicolumn{8}{|l|}{Log-Plattformen} \\
\hline
Papertrail & ja(*) & ja(*) & ja(*) & ja(*) & ja &  &  \\
\hline
Elastic Stack & ja(*) & ja(*) & ja(*) & ja(*) & ja &  &  \\
\hline
Fluentd &  &  &  &  & eingeschr. &  &  \\
\hline
Splunk \mbox{Enterprise} & ja(*) & ja(*) & ja(*) & ja(*) & ja &  &  \\
\hline
Graylog & ja(*) & ja(*) & ja(*) & ja(*) & ja &  &  \\
\hline
\hline
\multicolumn{8}{|l|}{Distributed-Tracing-Systeme} \\
\hline
Jaeger &  &  &  &  &  & ja &  \\
\hline
Zipkin &  &  &  &  &  & ja &  \\
\hline
\hline
\multicolumn{8}{|l|}{Error-Tracking} \\
\hline
Sentry &  &  & ja(*) & ja &  &  &  \\
\hline
TrackJS &  &  & ja(*) & ja &  &  &  \\
\hline
Rollbar &  &  & ja(*) & ja &  &  &  \\
\hline
Airbrake & ja & ja &  & ja &  &  &  \\
\hline
Bugsnag &  &  &  & ja &  &  &  \\
\hline
Raygun & ja & ja & ja(*) & ja &  &  &  \\
\hline
\hline
\multicolumn{8}{|l|}{Session-Replay-Dienste} \\
\hline
Inspectlet &  &  & ja & teils &  &  & ja \\
\hline
FullStory &  &  & ja & teils &  &  & ja \\
\hline
LogRocket &  &  & ja & ja &  &  & ja \\
\hline
\hline
\multicolumn{8}{|l|}{Web-Analytics} \\
\hline
Google Analytics & eing. & eing. & ja(*) & eingeschr. &  &  &  \\
\hline
Adobe Analytics & eing. & eing. & ja(*) & eingeschr. &  &  &  \\
\hline
\caption{Kategorisierung der untersuchten Technologien}
\label{tab:technologie-kategorisierung}
\end{longtable}
\endgroup


\subsection{Vorauswahl}
\label{subsec:technologie-vorauswahl}

Wie zuvor beschrieben eignen sich die unterschiedlichen Funktionskategorien teils mehr teils weniger für die, in dieser Arbeit angestrebte, Lösung: Ein Proof-of-Concept, welches die Nachvollziehbarkeit einer Webanwendung von Anwendungsverhalten und Nutzerinteraktionen für \textbf{Betreiber und Entwickler} verbessert.

Die beiden Kategorien Performance-Monitoring und Metriken bieten durch ihr APM sinnvolle Einsichten in die Leistung und Verfügbarkeit einer Anwendung, helfen aber nicht oder kaum bei der Aufdeckung von einzelnen Problemen. Sie bieten hilfreiche Informationen aus wirtschaftlicher und operativer Sichtweise, weniger aber bieten sie technische Informationen. Ein weiterer Grund gegen Werkzeuge des Performance-Monitorings ist die hier existierende Marktmacht von proprietären Lösungen, die teilweise sehr unflexibel in den Anpassungsmöglichkeiten sind aber dafür vorgefertigte Dashboards liefern, die Arbeit abnehmen können. Um diese Diskrepanz näher zu untersuchen wurden New Relic und Dynatrace beispielhaft näher evaluiert, indem jeweils die Testversion mit einer minimalen Beispielanwendung getestet wurde. Hierbei konnte festgestellt werden, dass die bereitgestellten Informationen, aus Sicht eines Entwicklern, nicht ausreichend Aufschluss bereiten. Aus diesen Gründen, gerade aufgrund der divergierenden Zielgruppen, werden die Funktionskategorien Performance-Monitoring und Metriken nicht näher betrachtet.

Hinzukommend und aus einem ähnlichen Grund, wird die Kategorie Web-Analytics nicht näher behandelt. Denn hierbei liegt der Fokus sehr stark auf einer Überprüfung von wirtschaftlichen und operativen Eigenschaften, nicht jedoch den in dieser Arbeit erwünschten Zielen. Die übrig gebliebenen Kategorien werden im nächsten Unterabschnitt näher betrachtet und kriteriengeleitet bewertet.

\subsection{Bewertung}

Um in dem Proof-of-Concept auf die hier vorgestellten Technologien zurückgreifen zu können, werden diese auf Basis verschiedener Kriterien bewertet. Diese Bewertung stützt sich auf öffentlich verfügbare Informationen, die die Hersteller der jeweiligen Technologie selber veröffentlicht haben. Die Kriterien werden folgend näher beschrieben.

\begin{enumerate}
	\item Kostenfrei
	\begin{itemize}
		\item Mit dem Kriterium \enquote{Kostenfrei} soll bewertet werden, ob eine kostenfreie Variante dieser Technologie existiert. Existiert eine kostenfreie Variante, wird \texttt{ja} angegeben und ansonsten \texttt{nein}. Existiert jedoch eine kostenfreie Version, die entweder von den Funktionalitäten oder der zeitlichen Nutzung beschränkt ist, wird diese mit \texttt{f. beschränkt} bzw. \texttt{z. beschränkt} angegeben.
	\end{itemize}

	\item Support für Webanwendungen
	\begin{itemize}
		\item Es ist zu bewerten, ob eine Unterstützung für das Senden von Daten von Webanwendungen existiert. Dies kann in der Form eines Agenten\footnotemark{} oder einer Schnittstelle sein. Ist die Schnittstelle nicht direkt aus einem Browserkontext ansprechbar, aber es ist eine Schnittstelle vorhanden, so ist diese Technologie mit \texttt{möglich} zu bewerten. Ist jedoch keine Schnittstelle, die das Senden von eigenen Daten ermöglicht, so ist die Technologie mit \texttt{nein} zu bewerten.
		\footnotetext{Ein Agent ist eine Bibliothek, die die jeweiligen Daten (wie Klicks, Ladezeiten für Ressourcen und DOM-Events, usw.) eigenständig sammelt und an ein Partnersystem übertragt \cite{SolvingBigDataChallengesForAPM}}
	\end{itemize}

	\item On Premise und SaaS
	\begin{itemize}
		\item Mit diesen zwei Kriterien soll bewertet werden, wie diese Technologie eingesetzt werden kann. Ist sie in einer eigenen Infrastruktur aufsetzbar, also \enquote{On Premise}, oder existiert die Technologie als buchbarer Dienst z. B. in der Cloud, also Software-as-a-Service (SaaS).
	\end{itemize}

	\item Standardisierung
	\begin{itemize}
		\item Setzt die jeweilige Technologie auf etablierte Standards, wie OpenTracing bei Distributed Tracing? Falls nicht, sind einzelne Komponenten (z. B. zur Instrumentalisierung) quelloffen oder öffentlich spezifiziert, sodass diese ausgetauscht oder angepasst werden können.
	\end{itemize}

	\item Multifunktional
	\begin{itemize}
		\item Mit dem Kriterium \enquote{Multifunktional} ist zu bewerten, ob neben der Kernfunktionalität einer Technologie diese auch eine Menge an weiteren Funktionalitäten vorweisen kann.
	\end{itemize}

	\item Zielgruppe
	\begin{itemize}
		\item Es ist einzuordnen welche Zielgruppen hauptsächlich von dieser Technologie profitieren, folgende wesentliche Zielgruppen werden differenziert:
		\begin{itemize}
			\item Projektmanager
			\item Fachabteilung
			\item Entwickler
		\end{itemize}
	\end{itemize}
\end{enumerate}

Auf Basis dieser Kriterien werden die Technologien der  übrig gebliebenen Kategorien in den Tabellen \ref{tab:technologie-bewertung-teil1} und \ref{tab:technologie-bewertung-teil2} bewertet.

\hvFloat[rotAngle=90,nonFloat=true,capWidth=w]%
{table}%
{
\begin{tabular}{|p{2.25cm}|p{2.0cm}|p{2.0cm}|p{2.0cm}|p{1.5cm}|p{2.0cm}|p{1.5cm}|p{2.5cm}|}
\hline
Technologie & Kostenfrei & Support f. Webanw. & On Premise & SaaS & Standard. & Multif. & Zielgruppe \\
\hline
\hline
\multicolumn{7}{|l|}{Log-Plattform} \\
\hline
Papertrail & f. begrenzt & möglich & nein & ja & nein & ja & Fachabteilung, Entwickler \\
\hline
Elastic Stack & ja & möglich & ja & ja & nein, aber quelloffen & ja & Fachabteilung, Entwickler \\
\hline
Fluentd & ja & möglich & ja & ja & nein, aber quelloffen & nein & Entwickler \\
\hline
Splunk \mbox{Enterprise} & f. begrenzt & möglich & ja & ja & nein & ja & Fachabteilung, Entwickler \\
\hline
Graylog & ja & möglich & ja & ja & nein, aber quelloffen & ja & Entwickler \\
\hline
\hline
\multicolumn{7}{|l|}{Tracing-Plattform} \\
\hline
Jaeger & ja & möglich & ja & nein & ja & nein & Entwickler \\
\hline
Zipkin & ja & möglich & ja & nein & nein, aber quelloffen & nein & Entwickler \\
\hline
\end{tabular}
}
{Bewertung der untersuchten Technologien, Teil 1}
{tab:technologie-bewertung-teil1}

\hvFloat[rotAngle=90,nonFloat=true,capWidth=w]%
{table}%
{
\begin{tabular}{|p{2.25cm}|p{2.0cm}|p{2.0cm}|p{2.0cm}|p{1.5cm}|p{2.0cm}|p{1.5cm}|p{2.5cm}|}
\hline
Technologie & Kostenfrei & Support f. Webanw. & On Premise & SaaS & Standard. & Multif. & Zielgruppe \\
\hline
\hline
\multicolumn{7}{|l|}{Error-Tracking} \\
\hline
Sentry & f. begrenzt & ja & ja & ja & nein, aber quelloffen & nein & Fachabteilung, Entwickler \\
\hline
TrackJS & z. und f. begrenzt & ja & nein & ja & nein & nein & Fachabteilung, Entwickler \\
\hline
Rollbar & z. und f. begrenzt & ja & ja & ja & nein & teils & Fachabteilung, Entwickler \\
\hline
Airbrake & z. und f. begrenzt & ja & nein & ja & nein & ja & Fachabteilung, Entwickler \\
\hline
Bugsnag & z. und f. begrenzt & ja & ja & ja & nein, aber teils quelloffen & nein & Fachabteilung, Entwickler \\
\hline
Raygun & z. und f. begrenzt & ja & nein & ja & nein & ja & Fachabteilung, Entwickler \\
\hline
\hline
\multicolumn{7}{|l|}{Session-Replay} \\
\hline
Inspectlet & f. begrenzt & ja & nein & ja & nein & teils & Projektmanager, Fachabteilung, Entwickler \\
\hline
FullStory & f. begrenzt & ja & nein & ja & nein & teils & Projektmanager, Fachabteilung, Entwickler \\
\hline
LogRocket & f. begrenzt & ja & ja & ja & nein & teils & Fachabteilung, Entwickler \\
\hline
\end{tabular}
}
{Bewertung der untersuchten Technologien, Teil 2}
{tab:technologie-bewertung-teil2}

\subsection{Auswahl}

Auf Basis der zuvor erstellten Bewertung wird schließlich je nach Funktionskategorie die präferierte Technologie ausgewählt.

In der Kategorie \enquote{Log-Management} findet sich auf Basis der Bewertung wenig Varianz zwischen den Technologien, lediglich Fluentd sticht hervor, aber dies ist dadurch erklärbar, dass Fluentd kein vollständiges Log-Management darstellt sondern eher einen Logaggregator ist \cite{FluentdAggregator}. Somit ist Fluentd nur als verwandte Technologie anzusehen und fällt somit als Präferenz aus. Der Elastic Stack eignet sich durch die hohe Flexiblität und die Komponente Logstash auch dazu, Log-Management mit ihr zu betreiben \cite{ThreatIdentificationFromAccessLogsUsingElasticStack}. Papertrail, Splunk sowie Graylog lassen sich als klassiche Log-Management-Werkzeuge verstehen, indem dass sie speziell auf diese Funktionskategorie angepasst sind. Graylog sowie der Elastic Stack sind quelloffen, aber auch als SaaS verfügbar. Bei einem gewünschten OnPremise-Deployment kann lediglich nur Papertrail nicht eingesetzt werden, denn dies wird nicht unterstüzt. Letztendlich lässt sich sagen, dass keine dieser 4 Technologien ausschließende Eigenschaften besitzt, sie sind allesamt geeignet für die in dieser Arbeit angestrebte Lösung. Es wird sich jedoch für \textbf{Splunk} entschieden, da der Kunde der OpenKnowledge dieses System bereits besitzt. Wie aber zuvor erwähnt eignen sich die anderen Technologien aber ähnlich gut und die Auswahl kann je nach Situation variieren.

Im Gebiet des Distributed Tracings gibt es auch nur wenige oberflächliche Unterschiede, sowohl Jaeger als auch Zipkin sind quelloffen sowie weit verbreitet im Einsatz \cite{AnalysisOfDistributedTracingSystemsEffectOnPerformance}. Jaeger scheint für neue Projekten attraktiver zu sein und findet dort mehr Einsatz, wie in StackShares Gegenüberstellung zu sehen ist \cite{StackShareJaegerVsZipkin}. Teilweise ist dies erklärbar durch die Ergebnisse, die Mart{\'i}nez \etal \cite{ComparisonOfE2ETestingToolsForMicroservices} herausfanden: Jaeger zeigt mehr hilfreiche Informationen an und kann diese schneller bereitstellen als Zipkin. Zudem entwickelt Jaeger aktiv eine Unterstützung des neuen OpenTelemetry Standards \cite{JaegerOpenTelemetry}, jedoch findet sich bei Zipkin keine vergleichbare Entwicklung. Aus diesen Gründen ist \textbf{Jaeger} hierbei das Werkzeug der Wahl.

Die Auswahl in der Kategorie \enquote{Error-Monitoring} ist etwas diverser, denn hier weisen manche Technologien Funktionalitäten auf, die sonst in dem Gebiet fremd sind. Bespielweise Airbrake und Raygun bieten neben einem Error-Monitoring zudem Aspekte eines APM, sodass die Anwendung/das System auch im Normalbetrieb überprüft werden kann. Jedoch sind diese APM-Funktionalitäten nicht so ausgereift, wie bei spezialisierten APM-Lösungen. Jedoch sind Airbrake und Raygun nur als SaaS-Produkte verfügbar, währenddessen Sentry, Rollbar und Bugsnag auch als OnPremise-Lösung verfügbar sind. Sentry ist zudem vollständig quelloffen verfügbar\footnote{Sentry GitHub Repo: \url{https://github.com/getsentry/sentry}} und entwickelt aktiv mit der Community auf GitHub \cite{GitHub}. Weiterhin ist Sentry das einzig identifizierte Werkzeug, welches eine nicht zeitlich begrenzte Version der SaaS-Lösung zur Verfügung stellt. Eine stark aussagekräftige Entscheidung kann jedoch nicht getroffen werden, da alle Werkzeuge hier adäquat die Bedingungen eines guten Error-Monitoring-Werkzeugs erfüllen. Dennoch wird sich an dieser Stelle für \textbf{Sentry} entschieden, auf Basis der zuvor nahe gelegten Gründe.

In der Beschreibung zur Kategorie \enquote{Session-Replay} wurde erwähnt, dass einige dieser Werkzeuge eine eher geschäfts- und andere eher eine entwicklerorientierte Session-Replay-Funktionalität vorweisen. Genauer ist FullStory fast ausschließlich für das Nachvollziehen von User-Experience konzipiert, währenddessen LogRocket sehr detaillierte und sehr technische Informationen liefert \cite{Webalyt}. Inspectlet lässt sich als Mischung dieser beiden Sichten verstehen, bietet aber z. B. nicht alle Informationen an, die LogRocket darstellt  \cite{Webalyt}. Da die hier angestrebte Lösung auf Betreiber und insbesondere Entwickler abzielt, wird sich hiermit für \textbf{LogRocket} entschieden.

Auf dieser Basis ist im folgenden Kapitel der eigentliche Proof-of-Concept zu entwerfen und zu implementieren. Da manche Technologien bzw. Kategorien Überschneidungen in den Funktionalitäten vorweisen (bpsw. Splunk und Sentry), kommt es ggf. dazu, dass nicht alle hier identifizierten Technologien im Proof-of-Concept zum Einsatz kommen. Vor dem Proof-of-Concept wird jedoch zunächst die die Demoanwendung, auf der das Konzept angewendet werden soll, vorgestellt.