% \chapter{Beispielhafte Integration}
	
\section{Vorstellung der Demoanwendung}

	\textit{In diesem Abschnitt soll die Demoanwendung vorgestellt werden, anhand dessen das Proof-of-Concept erstellt wird. Damit das Proof-of-Concept erstellt werden kann, muss die Demoanwendung die zuvor beschriebenen Probleme aufweisen, hierbei sollen die Probleme möglichst realitätsnah sein und nicht frei erfunden.}
	
\section{Konzept}
	
	\subsection{Architektur}

	\textit{Hier soll die grobe Architektur geplant werden, welche Komponente es gibt und wie diese kommunizieren sollen.}
	
	\subsection{Datenverarbeitung}
		
		\subsubsection{Erhebung}
		\textit{Wie werden die Daten erhoben (Nennung der verwendeten Methoden!)?}
		\textit{Wie gelangen die Daten an eine auswertende Komponente?}
		
		\subsubsection{Auswertung}
		\textit{Wie werden die Daten zusammengefasst und ausgewertet?}
		\textit{Wie gelangt das Ergebnis an die darstellende Komponente?}
		
		\subsubsection{Visualisierung}
		\textit{Wie werden den Stakeholdern die Informationen präsentiert?}

\section{Implementierung}

	\textit{Auf Basis des Konzeptes soll nun eine Implementierung erfolgen.}

	\subsection{Technologie-Stack}

\section{Demonstration}

	\textit{Nachdem nun eine Implementierung steht, soll die Erweiterung auf nicht-technische Weise veranschaulicht werden. Hier soll dargestellt werden, wie die Nachvollziehbarkeit nun verbessert worden ist.}