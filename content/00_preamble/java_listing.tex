% Farben fuer Programmlisting
\usepackage{listings,xcolor}
\definecolor{pl_background}{rgb}{0.95,0.95,0.95}
\definecolor{pl_comment}{rgb}{0.12, 0.38, 0.18 }
\definecolor{pl_ifelse}{rgb}{0.74,0.74,.29}
\definecolor{pl_keyword}{rgb}{0.37, 0.08, 0.25}
\definecolor{pl_string}{rgb}{0.06, 0.10, 0.98}

\definecolor{lstbackground}{RGB}{235,235,235}
\definecolor{lstkeyword}{RGB}{127,0,85}
\definecolor{lststring}{RGB}{42,0,255}
\definecolor{lstcomment}{RGB}{63,127,95}
\definecolor{lstannotation}{RGB}{127,159,191}

\definecolor{fh_grau}{rgb}{0.76,0.75,0.76}

% Vordefiniertes Programmlisting
\lstset{
	basicstyle = \small\sffamily,
	backgroundcolor = \color{pl_background},
	stringstyle = \color{pl_string},
	keywordstyle = \color{pl_keyword}\bfseries,
	commentstyle = \color{pl_comment}\itshape,
	frame = lrbt,
	numbers = left,
	captionpos=b,
	showstringspaces = false,
	breaklines = true,
	xleftmargin = 15pt,
	emph = [1]{php},
	emphstyle = [1]\color{black},
	emph = [2]{if,and,or,else},
	emphstyle = [2]\color{pl_ifelse}
}

% Quellcode Formatierung im normalen Stil 
\lstset{
	basicstyle={\footnotesize\fontfamily{pcr}\selectfont},
	backgroundcolor=\color{lstbackground},
	breaklines=true,
	frame=single,
	numbers=left,
	showstringspaces=false,
	tabsize=4
}

% Quellcde Formatierung im Eclipse Stil
\lstdefinestyle{java-eclipse}{
	commentstyle={\color{lstcomment}},
	keywordstyle={\color{lstkeyword}\bfseries},
	stringstyle={\color{lststring}},
	moredelim={[il][\textcolor{lstannotation}]{§§}},
	moredelim={[is][\textcolor{lstannotation}]{\%\%}{\%\%}}
}