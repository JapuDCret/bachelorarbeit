

% Schachtelungstiefe eine Nummerierung der Überschriften 
\setcounter{secnumdepth}{3}
\setcounter{tocdepth}{3}

% Die folgenden Befehle sind nüzulich bei Verwendung des alten nomencl.sty Pakets
\providecommand{\printnomenclature}{\printglossary}
\providecommand{\makenomenclature}{\makeglossary}
\makenomenclature

% Zusätzliche Konfigurationen 
\makeatletter
\setkomafont{disposition}{\normalcolor\bfseries}
\makeatother

% Farben fuer Programmlisting
\usepackage{listings,xcolor}
\definecolor{pl_background}{rgb}{0.95,0.95,0.95}
\definecolor{pl_comment}{rgb}{0.12, 0.38, 0.18 }
\definecolor{pl_ifelse}{rgb}{0.74,0.74,.29}
\definecolor{pl_keyword}{rgb}{0.37, 0.08, 0.25}
\definecolor{pl_string}{rgb}{0.06, 0.10, 0.98}

\definecolor{lstbackground}{RGB}{235,235,235}
\definecolor{lstkeyword}{RGB}{127,0,85}
\definecolor{lststring}{RGB}{42,0,255}
\definecolor{lstcomment}{RGB}{63,127,95}
\definecolor{lstannotation}{RGB}{127,159,191}

\definecolor{fh_grau}{rgb}{0.76,0.75,0.76}

% Vordefiniertes Programmlisting
\lstset{
	basicstyle = \small\sffamily,
	backgroundcolor = \color{pl_background},
	stringstyle = \color{pl_string},
	keywordstyle = \color{pl_keyword}\bfseries,
	commentstyle = \color{pl_comment}\itshape,
	frame = lrbt,
	numbers = left,
	captionpos=b,
	showstringspaces = false,
	breaklines = true,
	xleftmargin = 15pt,
	emph = [1]{php},
	emphstyle = [1]\color{black},
	emph = [2]{if,and,or,else},
	emphstyle = [2]\color{pl_ifelse}
}

% Quellcde Formatierung im normalen Stil 
\lstset{
	basicstyle={\footnotesize\fontfamily{pcr}\selectfont},
	backgroundcolor=\color{lstbackground},
	breaklines=true,
	frame=single,
	numbers=left,
	showstringspaces=false,
	tabsize=4
}

% Custom Zeugs

% Umbenennen von Listings und Nomuklatur 
\renewcommand{\nomname}{Abkürzungs- und Erklärungsverzeichnis}
\renewcommand\lstlistlistingname{Quellcodeverzeichnis}
\renewcommand{\lstlistingname}{Quellcode}
\renewcommand\appendixname{Anhang}

\addto\captionsngerman{
	\renewcommand{\figurename}{Abb.}
	\renewcommand{\tablename}{Tab.}
}

% Listing defintion for JavaScript and ECMAScript2015 (ES6)
%
% ECMAScript 2015 (ES6) definition by Gary Hammock
%

\lstdefinelanguage[ECMAScript2015]{JavaScript}[]{JavaScript}{
	morekeywords=[1]{await, async, case, catch, class, const,
		default, do, enum, export, extends, finally, from, implements,
		import, instanceof, let, static, super, switch, throw, try},
	morestring=[b]` % Interpolation strings.
}


%
% JavaScript version 1.1 by Gary Hammock
%
% Reference:
%	 B. Eich and C. Rand Mckinney, "JavaScript Language Specification
%		 (Preliminary Draft)", JavaScript 1.1.	1996-11-18.	[Online]
%		 http://hepunx.rl.ac.uk/~adye/jsspec11/titlepg2.htm
%

\lstdefinelanguage{JavaScript}{
	morekeywords=[1]{break, continue, delete, else, for, function,
		if, in, new, return, this, typeof, var, void, while, with, =>,
		constructor, window, class, interface, enum, declare, const,
		var, typeof},
	% Literals, primitive types, and reference types.
	morekeywords=[2]{false, null, true, boolean, number, undefined,
		Array, Boolean, Date, Math, Number, String, Object, void, any,
		string, private, public, protected},
	% Built-ins.
	morekeywords=[3]{eval, parseInt, parseFloat, escape, unescape,
		pipe, tap, subscribe},
	morekeywords=[4]{@Injectable, @NgModule, @Inject},
	sensitive,
	morecomment=[s]{/*}{*/},
	morecomment=[l]//,
	morecomment=[s]{/**}{*/}, % JavaDoc style comments
	morestring=[b]',
	morestring=[b]"
}[keywords, comments, strings]


\lstalias[]{ES6}[ECMAScript2015]{JavaScript}

% Requires package: color.
\definecolor{mediumgray}{rgb}{0.3, 0.4, 0.4}
\definecolor{mediumblue}{rgb}{0.0, 0.0, 0.8}
\definecolor{forestgreen}{rgb}{0.13, 0.55, 0.13}
\definecolor{darkviolet}{rgb}{0.58, 0.0, 0.83}
\definecolor{royalblue}{rgb}{0.25, 0.41, 0.88}
\definecolor{crimson}{rgb}{0.86, 0.8, 0.24}

\lstdefinestyle{JSES6Base}{
	backgroundcolor=\color{white},
	basicstyle=\ttfamily,
	breakatwhitespace=false,
	breaklines=true,
	captionpos=b,
	columns=fullflexible,
	commentstyle=\color{mediumgray}\upshape,
	emph={},
	emphstyle=\color{crimson},
	extendedchars=true,	% requires inputenc
	fontadjust=true,
	frame=single,
	identifierstyle=\color{black},
	keepspaces=true,
	keywordstyle=\color{mediumblue},
	keywordstyle={[2]\color{darkviolet}},
	keywordstyle={[3]\color{royalblue}},
	keywordstyle={[4]\color{mediumblue}},
	numbers=left,
	numbersep=5pt,
	numberstyle=\color{black},
	rulecolor=\color{black},
	showlines=true,
	showspaces=false,
	showstringspaces=false,
	showtabs=false,
	stringstyle=\color{forestgreen},
	tabsize=2,
	title=\lstname,
	upquote=true,	% requires textcomp
	literate={ä}{{\"a}}1 {ö}{{\"o}}1 {ü}{{\"u}}1 {Ä}{{\"A}}1 {Ö}{{\"O}}1 {Ü}{{\"U}}1 {ß}{{\ss}}1,
}

\lstdefinestyle{JavaScript}{
	language=JavaScript,
	style=JSES6Base
}
\lstdefinestyle{ES6}{
	language=ES6,
	style=JSES6Base
}

% Listing defintion for Java
% Farben fuer Programmlisting
\usepackage{listings,xcolor}
\definecolor{pl_background}{rgb}{0.95,0.95,0.95}
\definecolor{pl_comment}{rgb}{0.12, 0.38, 0.18 }
\definecolor{pl_ifelse}{rgb}{0.74,0.74,.29}
\definecolor{pl_keyword}{rgb}{0.37, 0.08, 0.25}
\definecolor{pl_string}{rgb}{0.06, 0.10, 0.98}

\definecolor{lstbackground}{RGB}{235,235,235}
\definecolor{lstkeyword}{RGB}{127,0,85}
\definecolor{lststring}{RGB}{42,0,255}
\definecolor{lstcomment}{RGB}{63,127,95}
\definecolor{lstannotation}{RGB}{127,159,191}

\definecolor{fh_grau}{rgb}{0.76,0.75,0.76}

% Vordefiniertes Programmlisting
\lstset{
	basicstyle = \small\sffamily,
	backgroundcolor = \color{pl_background},
	stringstyle = \color{pl_string},
	keywordstyle = \color{pl_keyword}\bfseries,
	commentstyle = \color{pl_comment}\itshape,
	frame = lrbt,
	numbers = left,
	captionpos=b,
	showstringspaces = false,
	breaklines = true,
	xleftmargin = 15pt,
	emph = [1]{php},
	emphstyle = [1]\color{black},
	emph = [2]{if,and,or,else},
	emphstyle = [2]\color{pl_ifelse}
}

% Quellcde Formatierung im normalen Stil 
\lstset{
	basicstyle={\footnotesize\fontfamily{pcr}\selectfont},
	backgroundcolor=\color{lstbackground},
	breaklines=true,
	frame=single,
	numbers=left,
	showstringspaces=false,
	tabsize=4
}

% Quellcde Formatierung im Eclipse Stil
\lstdefinestyle{java-eclipse}{
	commentstyle={\color{lstcomment}},
	keywordstyle={\color{lstkeyword}\bfseries},
	stringstyle={\color{lststring}},
	moredelim={[il][\textcolor{lstannotation}]{§§}},
	moredelim={[is][\textcolor{lstannotation}]{\%\%}{\%\%}}
}

% own commands
\newcommand{\etal}{\textit{et al}}
\newcommand{\citationneeded}{{\color{red}[cite]}}
\newcommand{\source}[1]{\vspace{-0.75\baselineskip}\caption*{Quelle: {#1}} }

% see https://tex.stackexchange.com/a/160035/221944
%\newcommand{\anforderung}[1]{%
%   \protected@write \@auxout {}{\string \newlabel {#1}{{#1}{\thepage}{#1}{Anforderung}{}} }%
%   \hypertarget{#1}{#1}
%}

\newcounter{anfcounter}
\renewcommand{\theanfcounter}{\arabic{anfcounter}}

\makeatletter
\newenvironment{anf}[6]{%
\refstepcounter{anfcounter}%
\protected@edef\@currentlabel{Anforderung #2}
\label{#1}%
	\begin{table}[H]
	\begin{tabular}{ |p{1.25cm}|p{5.5cm}|p{2.25cm}|p{2.1cm}|p{1.25cm}| }
\hline
Id   & Name          & Kano-Modell  & Funktionsart & Quelle       \\
#2 & #3 & #4 & #5 & #6 \\
\hline
}%
{
	\end{tabular}
	\end{table}%
}%
\makeatother
