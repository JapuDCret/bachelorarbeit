% twoside = false setzen, falls kein doppelseitiger Druck nötig
\documentclass[oneside, ngerman, final, 11pt, a4paper, 1.1headlines, headinclude=false, footinclude=false, mpinclude=false, pagesize, onecolumn, titlepage, parskip=half, headsepline, chapterprefix=false, version=first, listof=totoc, bibliography=totoc, toc=graduated, fleqn, twoside=true]{scrbook}

% Import file that contains configuration
% Die folgenden Pakete werden in der TEX-Vorlage verwendet.
% Sollten zusätzliche Pakete verwendet werden, sind sie an dieser Stelle hinzuzufügen 
\usepackage[T1]{fontenc}
\usepackage{listings}
\usepackage{refstyle}
\usepackage{float}
\usepackage{textcomp}
\usepackage{graphicx}
\usepackage{nomencl}
\usepackage{textpos} 
\usepackage{xcolor}
\usepackage{tocloft}
\usepackage{lipsum}
\usepackage[utf8]{inputenc}
\usepackage{color}
\usepackage{scrhack}
\usepackage{babel}
\usepackage{wrapfig}

% Custom packages
\usepackage{url}
% to allow the configuration of caption
\usepackage{caption}
\usepackage{hyperref}

\usepackage{tikz}
\usepackage{pgfplots}
% see https://tex.stackexchange.com/questions/1460/script-to-automate-externalizing-tikz-graphics
\usetikzlibrary{external}
\tikzexternalize[prefix=tikz-figures/]

\usepackage{tabularx}
\usepackage{makecell}
\usepackage{csquotes}

% Import file that contains configuration
% Schachtelungstiefe eine Nummerierung der Überschriften 
\setcounter{secnumdepth}{3}
\setcounter{tocdepth}{3}

% Die folgenden Befehle sind nüzulich bei Verwendung des alten nomencl.sty Pakets
\providecommand{\printnomenclature}{\printglossary}
\providecommand{\makenomenclature}{\makeglossary}
\makenomenclature

% Zusätzliche Konfigurationen 
\makeatletter
\setkomafont{disposition}{\normalcolor\bfseries}
\makeatother

% Custom Zeugs

% Umbenennen von Listings und Nomenklatur 
\renewcommand{\nomname}{Abkürzungs- und Erklärungsverzeichnis}
\renewcommand\lstlistlistingname{Quellcodeverzeichnis}
\renewcommand{\lstlistingname}{Quellcode}
\renewcommand\appendixname{Anhang}

\addto\captionsngerman{
	\renewcommand{\figurename}{Abb.\@}
	\renewcommand{\tablename}{Tab.\@}
}

% Listing defintion for JavaScript and ECMAScript2015 (ES6)
%
% ECMAScript 2015 (ES6) definition by Gary Hammock
%

\lstdefinelanguage[ECMAScript2015]{JavaScript}[]{JavaScript}{
  morekeywords=[1]{await, async, case, catch, class, const, default, do,
    enum, export, extends, finally, from, implements, import, instanceof,
    let, static, super, switch, throw, try},
  morestring=[b]` % Interpolation strings.
}


%
% JavaScript version 1.1 by Gary Hammock
%
% Reference:
%   B. Eich and C. Rand Mckinney, "JavaScript Language Specification
%     (Preliminary Draft)", JavaScript 1.1.  1996-11-18.  [Online]
%     http://hepunx.rl.ac.uk/~adye/jsspec11/titlepg2.htm
%

\lstdefinelanguage{JavaScript}{
  morekeywords=[1]{break, continue, delete, else, for, function, if, in,
    new, return, this, typeof, var, void, while, with, =>, constructor, window, class, interface, enum, declare, const, var, typeof},
  % Literals, primitive types, and reference types.
  morekeywords=[2]{false, null, true, boolean, number, undefined,
    Array, Boolean, Date, Math, Number, String, Object, void, any, string, private, public, protected},
  % Built-ins.
  morekeywords=[3]{eval, parseInt, parseFloat, escape, unescape, pipe, tap, subscribe},
  morekeywords=[4]{@Injectable, @NgModule, @Inject},
  sensitive,
  morecomment=[s]{/*}{*/},
  morecomment=[l]//,
  morecomment=[s]{/**}{*/}, % JavaDoc style comments
  morestring=[b]',
  morestring=[b]"
}[keywords, comments, strings]


\lstalias[]{ES6}[ECMAScript2015]{JavaScript}

% Requires package: color.
\definecolor{mediumgray}{rgb}{0.3, 0.4, 0.4}
\definecolor{mediumblue}{rgb}{0.0, 0.0, 0.8}
\definecolor{forestgreen}{rgb}{0.13, 0.55, 0.13}
\definecolor{darkviolet}{rgb}{0.58, 0.0, 0.83}
\definecolor{royalblue}{rgb}{0.25, 0.41, 0.88}
\definecolor{crimson}{rgb}{0.86, 0.8, 0.24}

\lstdefinestyle{JSES6Base}{
  backgroundcolor=\color{white},
  basicstyle=\ttfamily,
  breakatwhitespace=false,
  breaklines=true,
  captionpos=b,
  columns=fullflexible,
  commentstyle=\color{mediumgray}\upshape,
  emph={},
  emphstyle=\color{crimson},
  extendedchars=true,  % requires inputenc
  fontadjust=true,
  frame=single,
  identifierstyle=\color{black},
  keepspaces=true,
  keywordstyle=\color{mediumblue},
  keywordstyle={[2]\color{darkviolet}},
  keywordstyle={[3]\color{royalblue}},
  keywordstyle={[4]\color{mediumblue}},
  numbers=left,
  numbersep=5pt,
  numberstyle=\color{black},
  rulecolor=\color{black},
  showlines=true,
  showspaces=false,
  showstringspaces=false,
  showtabs=false,
  stringstyle=\color{forestgreen},
  tabsize=2,
  title=\lstname,
  upquote=true  % requires textcomp
}

\lstdefinestyle{JavaScript}{
  language=JavaScript,
  style=JSES6Base
}
\lstdefinestyle{ES6}{
  language=ES6,
  style=JSES6Base
}

% Listing defintion for Angular HTML
\lstdefinelanguage{AngularTemplateHTML}{
	language=html,
	sensitive=true,	
	alsoletter={<>=-},
	morecomment=[s]{<!-}{-->},
	tag=[s],
	otherkeywords={
	% General
	>,
	% Standard tags
	<!DOCTYPE,
	</html, <html, <head, <title, </title, <style, </style, <link, </head, <meta, />,
	% body
	</body, <body,
	% Divs
	</div, <div, </div>, 
	% Paragraphs
	</p, <p, </p>, </h1, <h1, </h2, <h2, </h3, <h3, </h4, <h4, </h5, <h5, </h6, <h6,
	% scripts
	</script, <script,
	% More tags...
	<canvas, /canvas>, <svg, <rect, <animateTransform, </rect>, </svg>, <video, <source, <iframe, </iframe>, </video>, <image, </image>, <header, </header, <article, </article, <button, </button
	},
	ndkeywords={
	% General
	=,
	% HTML attributes
	charset=, src=, id=, width=, height=, style=, type=, rel=, href=, class=, color=, (click)=,
	% SVG attributes
	fill=, attributeName=, begin=, dur=, from=, to=, poster=, controls=, x=, y=, repeatCount=, xlink:href=,
	% properties
	margin:, padding:, background-image:, border:, top:, left:, position:, width:, height:, margin-top:, margin-bottom:, font-size:, line-height:,
	% CSS3 properties
	transform:, -moz-transform:, -webkit-transform:,
	animation:, -webkit-animation:,
	transition:,	transition-duration:, transition-property:, transition-timing-function:,
	},
	% fix umlaute causing issues
	% https://tex.stackexchange.com/questions/24528/having-problems-with-listings-and-utf-8-can-it-be-fixed/24529
	extendedchars=true,
	literate={ä}{{\"a}}1 {ö}{{\"o}}1 {ü}{{\"u}}1 {Ä}{{\"A}}1 {Ö}{{\"O}}1 {Ü}{{\"U}}1 {ß}{{\ss}}1,
}

% Listing defintion for JSON
\colorlet{json_punct}{red!60!black}
\definecolor{json_background}{HTML}{EEEEEE}
\definecolor{json_delim}{RGB}{20,105,176}
\colorlet{json_numb}{magenta!60!black}

\lstdefinelanguage{JSON}{
	basicstyle=\normalfont\ttfamily,
	numbers=left,
	numberstyle=\scriptsize,
	stepnumber=1,
	numbersep=8pt,
	showstringspaces=false,
	breaklines=true,
	frame=lines,
	backgroundcolor=\color{json_background},
	literate=
	 *{0}{{{\color{json_numb}0}}}{1}
	  {1}{{{\color{json_numb}1}}}{1}
	  {2}{{{\color{json_numb}2}}}{1}
	  {3}{{{\color{json_numb}3}}}{1}
	  {4}{{{\color{json_numb}4}}}{1}
	  {5}{{{\color{json_numb}5}}}{1}
	  {6}{{{\color{json_numb}6}}}{1}
	  {7}{{{\color{json_numb}7}}}{1}
	  {8}{{{\color{json_numb}8}}}{1}
	  {9}{{{\color{json_numb}9}}}{1}
	  {:}{{{\color{json_punct}{:}}}}{1}
	  {,}{{{\color{json_punct}{,}}}}{1}
	  {\{}{{{\color{json_delim}{\{}}}}{1}
	  {\}}{{{\color{json_delim}{\}}}}}{1}
	  {[}{{{\color{json_delim}{[}}}}{1}
	  {]}{{{\color{json_delim}{]}}}}{1}
	  {ä}{{\"a}}1 {ö}{{\"o}}1 {ü}{{\"u}}1 {Ä}{{\"A}}1 {Ö}{{\"O}}1 {Ü}{{\"U}}1 {ß}{{\ss}}1,
}

% Listing defintion for Java
% Farben fuer Programmlisting
\usepackage{listings,xcolor}
\definecolor{pl_background}{rgb}{0.95,0.95,0.95}
\definecolor{pl_comment}{rgb}{0.12, 0.38, 0.18 }
\definecolor{pl_ifelse}{rgb}{0.74,0.74,.29}
\definecolor{pl_keyword}{rgb}{0.37, 0.08, 0.25}
\definecolor{pl_string}{rgb}{0.06, 0.10, 0.98}

\definecolor{lstbackground}{RGB}{235,235,235}
\definecolor{lstkeyword}{RGB}{127,0,85}
\definecolor{lststring}{RGB}{42,0,255}
\definecolor{lstcomment}{RGB}{63,127,95}
\definecolor{lstannotation}{RGB}{127,159,191}

\definecolor{fh_grau}{rgb}{0.76,0.75,0.76}

% Vordefiniertes Programmlisting
\lstset{
	basicstyle = \small\sffamily,
	backgroundcolor = \color{pl_background},
	stringstyle = \color{pl_string},
	keywordstyle = \color{pl_keyword}\bfseries,
	commentstyle = \color{pl_comment}\itshape,
	frame = lrbt,
	numbers = left,
	captionpos=b,
	showstringspaces = false,
	breaklines = true,
	xleftmargin = 15pt,
	emph = [1]{php},
	emphstyle = [1]\color{black},
	emph = [2]{if,and,or,else},
	emphstyle = [2]\color{pl_ifelse}
}

% Quellcode Formatierung im normalen Stil 
\lstset{
	basicstyle={\footnotesize\fontfamily{pcr}\selectfont},
	backgroundcolor=\color{lstbackground},
	breaklines=true,
	frame=single,
	numbers=left,
	showstringspaces=false,
	tabsize=4
}

% Quellcde Formatierung im Eclipse Stil
\lstdefinestyle{java-eclipse}{
	commentstyle={\color{lstcomment}},
	keywordstyle={\color{lstkeyword}\bfseries},
	stringstyle={\color{lststring}},
	moredelim={[il][\textcolor{lstannotation}]{§§}},
	moredelim={[is][\textcolor{lstannotation}]{\%\%}{\%\%}}
}

% Listing defintion for Gherkin
% Gherkin
% https://gist.github.com/nsommer/9a04f6ebc6ea8b9f5b816becb97e7d9b
\lstdefinelanguage{gherkin}{
	morekeywords = {
		Given,
		When,
		Then,
		And,
		Scenario,
		Feature,
		But,
		Background,
		Scenario Outline,
		Examples
	},
	sensitive=true,
	morecomment=[l]{\#},
	morestring=[b]",
	morestring=[b]',
	% fix umlaute causing issues
	% https://tex.stackexchange.com/questions/24528/having-problems-with-listings-and-utf-8-can-it-be-fixed/24529
	extendedchars=true,
	literate={ä}{{\"a}}1 {ö}{{\"o}}1 {ü}{{\"u}}1 {Ä}{{\"A}}1 {Ö}{{\"O}}1 {Ü}{{\"U}}1 {ß}{{\ss}}1,
}

% own commands
\newcommand{\etal}{\textit{et al.\@ }}
\newcommand{\citationneeded}{{\color{red}[cite]}}
\newcommand{\source}[1]{\vspace{-0.75\baselineskip}\caption*{Quelle: {#1}} }

% custom Anforderung command
\newcounter{anfcounter}
\renewcommand{\theanfcounter}{\arabic{anfcounter}}

\makeatletter
\newenvironment{anf}[6]{%
\refstepcounter{anfcounter}%
\protected@edef\@currentlabel{Anforderung #2}
\label{#1}%
	\begin{table}[H]
	\begin{tabular}{ |p{1.25cm}|p{5.5cm}|p{2.25cm}|p{2.1cm}|p{1.25cm}| }
\hline
Id	 & Name					& Kano-Modell	& Funktionsart & Quelle			 \\
#2 & #3 & #4 & #5 & #6 \\
\hline
}%
{
	\end{tabular}
	\end{table}%
}%
\makeatother

% allow breaking of urls
% see https://tex.stackexchange.com/a/344711/221944
\def\UrlBreaks{\do\/\do-}

\newcommand*\circled[1]{
	\tikz[baseline=(char.base)]{
		\node[shape=circle,draw,inner sep=2pt] (char) {#1};
	}
}

\pgfplotsset{compat=1.17}

%\usepackage{glossaries}
%\makeglossaries

\newglossaryentry{Bug Report}
{
    name=Bug Report,
    description={Fehlerbericht},
    plural={Bug Reports},
}


% Beginn des Dokumentes 
\begin{document}
	% Die Commands werden auf der Titelseite eingefügt, die Werte sind anzupassen
	\newcommand*{\thedockind}{Bachelorthesis}
	\newcommand*{\thetitle}{Nachvollziehbarkeit von Nutzerinteraktion und Anwendungsverhalten am Beispiel JavaScript-basierter Webapplikationen}
	\newcommand*{\thesubtitle}{}
	\newcommand*{\theauthor}{Marvin Kienitz}
	\newcommand*{\thematriculationnumber}{7097533}
	\newcommand*{\thebirthday}{26.04.1996}
	\newcommand*{\thedegree}{Bachelor of Science}
	\newcommand*{\themajor}{Software- und Systemtechnik, Vertiefung Softwaretechnik} % Studiengang
	\newcommand*{\thedate}{\today} % \today kann durch ein Datum erstetzt werden. 
	\newcommand*{\thebetreuer}{Prof. Dr. Sven Jörges} 
	\newcommand*{\thezweitbetreuer}{Dipl. Inf. Stephan Müller}

	\begin{titlepage}
	  \begin{textblock}{6.5}(-1,-3)
	    \begin{color}{fh_grau}
	      \rule{6.8cm}{33cm}    
	    \end{color}
	  \end{textblock}
	  \begin{textblock}{6.5}(-1.2,-0.7)
	  \end{textblock}
	  \begin{textblock}{6.5}(-0.8,1)
	    {\Large \textsf{\thedockind}}            
	  \end{textblock}
	
	  \begin{textblock}{6}(4.5,2)
	    {\noindent \huge 
	      \textsf{\textbf{\thetitle\\[0.3cm] 
	          \Large  \thesubtitle\\[0.05cm]
	          }} }
	  \end{textblock}
	
	
	  \begin{textblock}{8.5}(4.5,6.5)\noindent
	    \textsf{An der Fachhochschule Dortmund\\
	    im Fachbereich Informatik\\
	    Studiengang \themajor \\
	    erstellte \thedockind \\
	    zur Erlangung des akademischen Grades\\
	    \thedegree}
	  \end{textblock}
	
	  \begin{textblock}{6.5}(-0.4,10.0)
	    \noindent
	    \textsf{von \\
	      \theauthor \\
	      geb.\ am \thebirthday  \\
	      Matr.-Nr. \thematriculationnumber\\[0.7cm]
	      Betreuer:\\
	       \noindent\hspace*{6mm} \thebetreuer \\
	       \noindent\hspace*{6mm} \thezweitbetreuer\\ [0.5cm]
	      Dortmund, \today}    
	  \end{textblock}
		
	
	\end{titlepage}
	
	\newpage{}
	
	\section*{\thispagestyle{empty}Kurzfassung}
	
	\textit{\lipsum[1-4]}
	
	\newpage{}
	
	\section*{\thispagestyle{empty}Abstract}
	
	\textit{\lipsum[1-4]}
	
	\newpage{}
	
	% Römische Nummerierung der Seiten des Inhaltsverzeichnisses 
	\setcounter{page}{1}
	\pagenumbering{roman}
	
	\tableofcontents{}
	
	\newpage{}
	
	% Arabische Nummerierung aller anderen Seiten
	\setcounter{page}{1} 
	\pagenumbering{arabic}
	
	\chapter{Einleitung}
	% a hack, to make the Motivation fit into one page..
\vspace{-\baselineskip}

%In diesem Unterkapitel sollten folgende Punkte behandelt werden:
%\begin{itemize}
%	\item	Was ist das Problem
%	\item 	Problemgeschichte?
%\end{itemize}
\section{Motivation}

Die Open Knowledge GmbH ist als branchenneutraler Softwaredienstleister in einer Vielzahl von Branchen wie Automotive, Logistik, Telekommunikation und Versicherungs- und Finanzwirtschaft aktiv. Zu den zahlreichen Kunden der Open Knowledge GmbH gehört auch ein führender deutscher Direktversicherer. 

Ein Direktversicherer bietet Versicherungsprodukte seinen Kunden ausschließlich im Direktvertrieb, d. h. vor allem über das Internet und zusätzlich auch über Telefon, Fax oder Brief an. Im Unterschied zum klassischen Versicherer verfügt ein Direktversicherer jedoch über keinen Außendienst oder Geschäftsstellen, bei denen Kunden eine persönliche Beratung erhalten. Da das Internet der primäre Vertriebskanal ist, ist heute ein umfassender Online-Auftritt die Norm. Dieser besteht typischerweise aus einem Kundenportal mit der Möglichkeit Angebote für Versicherungsprodukte berechnen und abschließen zu können, sowie persönliche Daten und Verträge einsehen und ändern zu können.

\nomenclature[Fachbegriff]{Serverseitiges Rendering}{Die darzustellenden Inhalte, werden beim Server generiert und der Client stellt diese dar. Beispielsweise sind Anwendungen mit PHP oder auch eine Java Web Application}
\nomenclature[Fachbegriff]{Clientseitiges Rendering}{Der Server stellt dem Client lediglich die Logik und die notwendigen Daten bereit, die eigentliche Inhaltsgenerierung geschieht im Client. Für ein Beispiel siehe \autoref{subsec:singe-page-applications}}

% i.d.R. getrennt: https://www.scribbr.de/wissenschaftliches-schreiben/abkuerzungen/
Während in der Vergangenheit Online-Auftritte i. d. R. als Webanwendung mit serverseitigen Rendering realisiert wurden, sind heutzutage Javascript-basierte Webanwendung mit clientseitigem Rendering die Norm. Bei einer solchen Webanwendung befindet sich die gesamte Logik mit Ausnahme der Berechnung des Angebots und der Verarbeitung der Antragsdaten im Browser des Nutzers.

Im produktiven Einsatz kommt es auch bei gut getesteten Webanwendungen hin und wieder vor, dass es zu unvorhergesehenen Fehlern in der Berechnung oder Verarbeitung kommt. Liegt die Ursache für den Fehler im Browser, z. B. aufgrund einer ungültigen Wertkombination, ist dies eine Herausforderung. Während bei Server-Anwendungen Fehlermeldungen in den Log-Dateien einzusehen sind, gibt es für den Betreiber der Anwendung i. d. R. keine Möglichkeit die notwendigen Informationen über den Nutzer und seine Umgebung abzurufen. Noch wichtiger ist, dass er mitbekommt, wenn ein Nutzer ein Problem bei der Bedienung der Anwendung hat. Ohne eine aktive Benachrichtigung durch den Nutzer, sowie detaillierte Informationen, ist es dem Betreiber nicht möglich, Kenntnis über das Problem zu erlangen, geschweige denn dieses nachzustellen.

Dies stellt ein Kernproblem von  Webanwendungen dar \cite{ClientSideMonitoringOfDistributedSystems}. Im Rahmen der Arbeit soll daher ein Proof-of-Concept konzipiert und umgesetzt werden, welcher dieses Kernproblem am Beispiel einer Demoanwendung löst.

%\begin{itemize}
%	\item 	Was soll mit der Arbeit erreicht werden? Welche Ziele werden angestrebt?
%			Möglichst kurz und präzise geplante Ergebnisse umreißen. Daran werden
%			Ihre Resultate am Ende gemessen!
%\end{itemize}
\section{Zielsetzung}

Das grundlegende Ziel dieser Arbeit soll es sein, den Betreibern einer JavaScript-basierten Webanwendung die Möglichkeit zu geben das Verhalten ihrer Applikation und die Interaktionen von Nutzern. Diese Nachvollziehbarkeit soll insbesondere bei Fehlerfällen u. Ä. gewährleistet sein, aber auch in sonstigen Fällen soll eine Nachvollziehbarkeit möglich sein. Eine vollständige Überwachung der Applikation und des Nutzers (wie bspw. bei Werbe-Tracking) sind jedoch nicht vorgesehen. Daraus ergibt sich die Forschungsfrage:

\begin{quotation}
	Wie sieht ein Ansatz aus, um bei clientseitigen JavaScript-basierten Webanwendung den Betreibern eine Nachvollziehbarkeit zu gewährleisten?
\end{quotation}

Vom Leser wird eine Grundkenntnis der Informatik in Theorie oder Praxis erwartet, aber es sollen keine detaillierten Erfahrungen in der Webentwicklung vom Leser erwartet werden. Daher sind das Projektumfeld und seine besonderen Eigenschaften zu erläutern.

Die anzustrebende Lösung soll ein Proof-of-Concept sein, welches eine, zu erstellenden, Demoanwendung erweitert. Die Demoanwendung soll repräsentativ eine abgespeckte JavaScript-basierte Webanwendung darstellen, bei der die zuvor benannten Hürden zur Nachvollziehbarkeit bestehen.

Vor der eigentlichen Lösungserstellung soll jedoch die theoretische Seite beleuchtet werden, indem die Nachvollziehbarkeit sowie Methoden und Praktiken zur Erreichung dieser beschrieben werden. Es gilt aktuelle Literatur und den Stand der Technik zu erörtern, in Bezug auf die Forschungsfrage. Beim Stand der Technik sind Technologien aus Fachpraxis und Literatur näher zu betrachten und zu beschreiben.

Weiterhin gilt es zu beleuchten, wie die Auswirkungen für die Nutzer der Webanwendung sind. Wurde die Leistung der Webanwendung beeinträchtigt (erhöhte Ladezeit, erhöhte Datenlast)? Werden mehr Daten von ihm erhoben und zu welchem Zweck?

Am Ende der Ausarbeitung soll überprüft werden, ob und wie die Forschungsfrage beantwortet wurde. Auch die Übertragbarkeit der erstellten Lösung (PoC) und Ergebnisse gilt es hierbei näher zu betrachten.

\subsection{Abgrenzung}

\nomenclature[Fachbegriff]{PoC}{Proof-of-Concept}

Die Demoanwendung wird als Single-Page-Application (SPA) realisiert, denn hier bewegt sich das Projektumfeld von der Open Knowledge GmbH. Bei der Datenerhebung und -verarbeitung sind datenschutzrechtliche Aspekte nicht näher zu betrachten. Bei der Betrachtung von Technologien aus der Wirtschaft ist eine bewertende Gegenüberstellung nicht das Ziel.

\pagebreak

%\begin{itemize}
%	\item 	Wie wird vorgegangen, um das Ziel zu erreichen?
%	\item 	Warum ist die Arbeit so gegliedert, wie sie gegliedert ist?
%	\item 	Welche Aspekte werden nicht behandelt und warum?
%\end{itemize}
\section{Vorgehensweise}

\vspace{-0.5\baselineskip}

Zur Vorbereitung eines Proof-of-Concepts wird zunächst die Ausgangssituation geschildert. Speziell wird auf die Herausforderungen der Umgebung \enquote{Browser} eingegangen, besonders in Hinblick auf die Verständnisgewinnung zu Interaktionen eines Nutzers und des Verhaltens der Applikation. Des Weiteren wird die Nachvollziehbarkeit als solche formal beschrieben und was sie im Projektumfeld genau bedeutet.

Darauf aufbauend werden allgemeine Methoden vorgestellt, mit der die Betreiber und Entwickler eine bessere Nachvollziehbarkeit erreichen können. Dabei werden die Besonderheiten der Umgebung beachtet und es wird erläutert, wie diese Methoden in der Umgebung zum Einsatz kommen können. Hiernach sind Ansätze aus der Literatur und Fachpraxis zu erörtern, welche eine praktische Realisierung der zuvor vorgestellten Methoden darstellen.

Auf Basis des detaillierten Verständnisses der Problemstellung und der Methoden wird nun ein Proof-of-Concept erstellt. Ziel soll dabei sein, die Nachvollziehbarkeit einer Webanwendung zu verbessern. Der Proof-of-Concept erfolgt auf Basis einer Demoanwendung, die im Rahmen dieser Arbeit erstellt wird.

Ist ein Proof-of-Concept nun erstellt, wird analysiert, welchen Einfluss es auf die Nachvollziehbarkeit hat und ob die gewünschten Ziele erreicht wurden (vgl. Zielsetzung).

\vspace{-0.5\baselineskip}

\section{Open Knowledge GmbH}

\vspace{-0.5\baselineskip}

%{\color{red}TODO: Dieser Abschnitt muss noch überarbeitet werden}

Die Bachelorarbeit wird im Rahmen einer Werkstudententätigkeit innerhalb der Open Knowledge GmbH erstellt. Der Standortleiter des Standortes Essen, Dipl.-Inf Stephan Müller, übernimmt die Zweitbetreuung.

Die Open Knowledge GmbH ist ein branchenneutrales mittelständisches Dienstleistungsunternehmen mit dem Ziel bei der Analyse, Planung und Durchführung von Softwareprojekten zu unterstützen. Das Unternehmen wurde im Jahr 2000 in Oldenburg, dem Hauptsitz des Unternehmens, gegründet und beschäftigt heute 74 Mitarbeiter. Mitte 2017 wurde in Essen der zweite Standort eröffnet, an dem 13 Mitarbeiter angestellt sind.

Die Mitarbeiter von Open Knowledge übernehmen in Kundenprojekten Aufgaben bei der Analyse über die Projektziele und der aktuellen Ausgangssituationen, der Konzeption der geplanten Software, sowie der anschließenden Implementierung. Die erstellten Softwarelösungen stellen Individuallösungen dar und werden den Bedürfnissen der einzelnen Kunden entsprechend konzipiert und implementiert. Technisch liegt die Spezialisierung bei der Mobile- und bei der Java Enterprise Entwicklung, bei der stets moderne Technologien und Konzepte verwendet werden. Die Geschäftsführer als auch diverse Mitarbeiter der Open Knowledge GmbH sind als Redner auf Fachmessen wie der Javaland oder als Autoren in Fachzeitschriften wie dem Java Magazin vertreten.

\pagebreak

	
	\chapter{Instandhaltung und Support}
	
		\section{Phase in Softwareprojekten}

			\begin{wrapfigure}[12]{r}{0.45\linewidth}
				\centering
				\vspace{-\baselineskip}
				\includegraphics[width=\linewidth]{img/software-development-life-cycle.jpg}
				\caption{Lebenszyklus einer Software {\color{red}TODO: Replace with own diagram}}
				\label{fig:software-development-life-cycle}
			\end{wrapfigure}
			
			In vielen Modellen über den Lebenszyklus einer Software gibt es eine Phase, in der Instandhaltung und Support den Alltag bestimmen, sie wird oftmals ``Maintenance`` bezeichnet \cite{ManagingTheComplexityOfWebSystemsDevelopment} \cite{CostBenefitAnalaysisHumanFactorsSoftwareLifecycle}. Sie ist nach Zelkowitz \etal \cite{PrinciplesOfSoftwareEngineeringAndDesign} für rund zwei Drittel der Entwicklungskosten verantwortlich, begründet durch exponentielle Steigung \cite{ExtremeProgrammingExplained}.
			
			Es werden immer bessere Methoden entwickelt, um Probleme in Software - oder auch Bugs - zu verringern. Jedoch erhöht sich zugleich die Komplexität von Software, was zur Ursache hat, dass es mehr Nährboden für Bugs gibt \cite{TrackingDownSoftwareBugsAnomalyDetection}. De-facto sind Bugs ein unvermeidbarer Bestandteil einer Software und müssen daher erwartet und gehandhabt werden \cite{TheMythicalManMonth}.
			
			Wenn nun ein Bug auffällt, sei es durch einen Nutzer oder auch zufällig einem Stakeholder, muss entschieden werden, ob dieser zu beheben ist. Wenn eine Behebung angestrebt wird, benötigt der Stakeholder meistens Rahmeninformationen \citationneeded um den Bug ggf. zu reproduzieren und die Situation \textbf{nachzuvollziehen}. Desto mehr Verständnis der Stakeholder über das Problem erhält, desto schneller und präziser kann er die Ursache aufdecken. Die Ermöglichung der schnellen Verständnis über ein Problem, wird in dieser Arbeit Nachvollziehbarkeit genannt.
	
		\section{Nachvollziehbarkeit}
		
			Sie beschäftigt sich mit der Informationserfassung und -aufbereitung, um das Verhalten eines Systems und die Interaktionen der Nutzer verstehen zu können.
		
	\chapter{Methoden und Praktiken}

		In diesem Kapitel soll beschrieben werden, wie eine Nachvollziehbarkeit in Softwareprojekten erreicht werden kann. Spezielle Methoden und Praktiken sollen vorgestellt und beleuchtet werden.
		
		\section{Fehlerberichte}
		
		\section{Logging}
		
		\section{Metriken}
		
		\section{Tracing}
	
	\chapter{Beispielhafte Integration}
	
		\section{Konzept}
		
		Hier soll die grobe Architektur geplant werden, welche Komponente es gibt und wie diese kommunizieren sollen.
	
		\section{Implementierung}
		
		Auf Basis des Konzeptes soll nun eine Implementierung erfolgen.
	
		\section{Demonstration}
		
		Nachdem nun eine Implementierung steht, soll die Erweiterung auf nicht-technische Weise veranschaulicht werden. Hier soll dargestellt werden, wie die Nachvollziehbarkeit nun verbessert worden ist.
	
	\chapter{Abschluss}
	% \chapter{Abschluss}

\section{Zusammenfassung}

Ziel der Arbeit war es einen Ansatz zu erstellen, mit dem Betreibern von JavaScript-basierten Webanwendungen eine Nachvollziehbarkeit gewährleistet werden kann. Der Proof-of-Concept, konnte die Mehrheit der gestellten Anforderungen erfüllen sowie konnten zuvor definierte Fehlerszenarien aufgedeckt werden. Weiterhin weist die erstellte Lösung und dabei das Konzept, eine Übertragbarkeit auf andere ähnliche Softwareprojekte auf. Zudem konnte kein signifikanter negativer Einfluss für den Nutzer festgestellt werden. Somit wurde das grundlegende Ziel dieser Arbeit erreicht.

\section{Fazit}

Es konnte ein Kernproblem von Webanwendungen und speziell bei SPAs festgestellt werden, aber es konnten zudem Methoden und Technologien identifiziert werden, die dieses Kernproblem beheben. Weiterhin wurden diese Ansätze erfolgreich eingesetzt und boten einen Mehrwert für Entwickler und Betreiber, der sich nur marginal auf die Performance der Webanwendung niederschlug.

\section{Ausblick}

Wie bei der Recherche zum Stand der Technik zu sehen war, gibt es seit 2020 mit dem OpenTelemetry-Standard eine neue Entwicklung, die das Feld der Nachvollziehbarkeit bzw. der Observability in den nächsten Jahren nachhaltig beeinflussen wird. Sollte der Standard von Herstellern adaptiert werden, könnte dies zu einer höheren Auswahl an Technologien führen, die verwendet werden können, da sie miteinander kompatibel sind. Des Weiteren kann es hierdurch einfacher werden mehrere Observability-Systeme und -Konzepte miteinander zu kombinieren, um eine erhöhte Durchleuchtung zu erreichen. Diese Entwicklung ist in meinen Augen vielversprechend, besonders durch die Unterstützung von führenden Herstellern von Observability-Werkzeugen. Somit sollte dieses Feld in nächster Zeit verfolgt werden.
	
%	\chapter{Test-Kapitel}
%	\section{Zitate}

Im Literaturverzeichnis sollte zu jedem Zitat ein Eintrag mit dem entsprechenden Kürzel zu finden sein.

\subsection{Einzelner Autor}

Zur Überprüfung wird das Buch \enquote{The Hitchhiker's Guide to the Galaxy}  von Douglas Adams aus dem Jahr 1979 hinzugezogen. Die folgende Zitierung sollte mit \texttt{[Ada79]}  abgekürzt dargestellt werden. \textit{Zitat} \cite{TestCitation001}.

\subsection{Zwei Autoren}

Zur Überprüfung wird das Buch \enquote{Hard drive: Bill Gates and the making of the Microsoft empire} von James Wallace und Jim Erickson aus dem Jahr 1992 hinzugezogen. Die folgende Zitierung sollte mit \texttt{[WE92]} abgekürzt dargestellt werden. \textit{Zitat} \cite{TestCitation002}.

\subsection{Drei Autoren}

Zur Überprüfung wird der Artikel \enquote{Antioxidant activity of apple peels} von Kelly Wolfe, Xianzhong Wu und Rui Hai Liu aus dem Jahr 2003 hinzugezogen. Die folgende Zitierung sollte mit \texttt{[WWL03]} abgekürzt dargestellt werden. \textit{Zitat} \cite{TestCitation003}.

\subsection{Viele Autoren}

Zur Überprüfung wird der Artikel \enquote{Observation of top quark production in p p collisions with the Collider Detector at Fermilab} von Fumio Abe, H. Akimoto, A. Akopian, M.G. Albrow, S.R. Amendolia, D. Amidei, J. Antos, C. Anway-Wiese, S. Aota, G. Apollinari und Weitere aus dem Jahr 1995 hinzugezogen. Die folgende Zitierung sollte mit \texttt{[AAA+95]} abgekürzt dargestellt werden. \textit{Zitat} \cite{TestCitation004}.

\newpage

\section{Tabellen}

Eine Tabelle sollte immer eine Über- oder Unterschrift erhalten und mit dieser im Tabellenverzeichnis wiederzufinden sein. Weiterhin muss ein Zitat angegeben werden, wenn es sich um hinzugezogene Daten handelt.

\subsection{Simple Tabelle}

\begin{table}[H]
\centering
\begin{tabular}{|l|l|} 
\hline
Spezifische Wärmekapazität (J/(mol K)) & Temperatur (°C)  \\ 
\hline
12,2                                   & -200             \\ 
\hline
15,0                                   & -180             \\ 
\hline
17,3                                   & -160             \\ 
\hline
19,8                                   & -140             \\ 
\hline
24,8                                   & -100             \\ 
\hline
29,6                                   & -60              \\
\hline
\end{tabular}
\caption{Spezifische Wärmekapazität von Wasser \cite{TestCitation020}}
\end{table}

\subsection{Komplexere Tabelle}

\newcommand{\tabtitel}[1]{\multicolumn{2}{l|}{\textbf{#1}}}

\begin{table}[H]
\centering
\begin{tabular}{|l|l|l|l|l|} 
\hline
           & \tabtitel{Ostdeutschland} & \tabtitel{Westdeutschland}  \\ 
\hline
Geschlecht & Frauen & Männer           & Frauen & Männer                       \\ 
\hline
weiblich   & 100\%  & 0\%              & 100\%  & 0\%                          \\ 
\hline
männlich   & 0\%    & 100\%            & 0\%    & 100\%                        \\
\hline
\end{tabular}
\caption{Absurder Vergleich von Ost- und Westdeutschland}
\end{table}

\newpage

\section{Grafiken, Bilder, etc.}

Eine Tabelle sollte immer eine Über- oder Unterschrift erhalten und mit dieser im Tabellenverzeichnis wiederzufinden sein.

\subsection{Simple Grafik}

\begin{figure}[H]
	\centering
	\includegraphics[width=\linewidth]{content/xx_test/Phase_diagram_of_water_simplified.svg.png}
	\caption{Vereinfachtes Phasendiagramm von Wasser \cite{TestCitation020}}
\end{figure}

\newpage

\subsection{Innerhalb eines Textes (Floating)}

\begin{wrapfigure}[14]{r}{0.33\textwidth}
	\includegraphics[width=\linewidth]{content/xx_test/1032px-3D_model_hydrogen_bonds_in_water.svg.png}
	\caption{Verkettung der Wassermoleküle \cite{TestCitation021}}
\end{wrapfigure}

Lorem ipsum dolor sit amet, consectetur adipiscing elit. Praesent eu risus a erat auctor bibendum. Morbi eget aliquet nisl. Aenean vestibulum elit sed arcu condimentum euismod. Nunc quis ipsum sed augue maximus molestie. Pellentesque condimentum elit vitae justo tincidunt elementum. Aliquam erat volutpat. Vestibulum feugiat auctor fringilla.

Cras non arcu ante. Donec faucibus lectus risus, ac sagittis risus malesuada pretium. Cras elementum quis turpis accumsan faucibus. Vestibulum vitae volutpat nisl, et sodales nunc. Etiam egestas at magna a commodo. Mauris sagittis suscipit tempus. Duis rhoncus nec ligula eget viverra. Maecenas eu nisl orci. Proin dignissim laoreet libero in interdum. Lorem ipsum dolor sit amet, consectetur adipiscing elit.

Sed fringilla lectus non elit convallis, non eleifend sem porta. Proin aliquet urna ultrices metus blandit ornare. Duis nec ultricies ligula, quis volutpat ante. Suspendisse non lacus mauris.

\begin{wrapfigure}[14]{o}{0.45\textwidth}
	\includegraphics[width=\linewidth]{content/xx_test/Kleiner_Streichteich_Ilmenau.JPG}
	\caption{Spiegelung an der Wasseroberfläche \cite{TestCitation022}}
\end{wrapfigure}

Nullam quam diam, mattis non bibendum vel, iaculis sed leo. Phasellus condimentum auctor ante in mollis. Nullam rhoncus enim ac metus fermentum aliquam. Phasellus orci metus, tristique ac odio sed, fermentum faucibus mauris. Praesent vehicula risus aliquam nibh sodales, ac aliquam ante posuere. Praesent at semper sapien. In posuere augue vel tortor posuere rhoncus. Duis eget quam tempus diam posuere fringilla ut vitae lacus. Sed sagittis fringilla diam.

Suspendisse potenti. Ut pellentesque malesuada dolor vitae porta. Aliquam erat volutpat. Proin convallis mauris neque. Etiam et accumsan ex. Class aptent taciti sociosqu ad litora torquent per conubia nostra, per inceptos himenaeos. Suspendisse potenti. Cras ac magna enim. Praesent ultrices lacinia sem nec placerat. Sed tristique non sapien quis efficitur. Ut sit amet egestas metus. Mauris nec erat sodales, semper dolor eu, egestas justo. Ut sit amet urna ligula.

\section{Quellcode}

Der jeweilige Quellcode sollte mit Syntax-Highlighting versehen sein und im Quellcodeverzeichnis aufzufinden sein.

\subsection{Darstellung eines JavaScript Quellcodes (inline)}

\begin{lstlisting}[
  language = JavaScript,
     style = ES6,
   caption = Beispiel eines JavaScript Quellcodes (inline),
captionpos = b,
     label = lst:javascript-inline-example,
]
var fetch = require("node-fetch")

async function getCountries() {
  let res = await fetch("https://restcountries.eu/rest/v2/name/Indonesia?fullText=true")
  let json = await res.json()
  let code = json[0].alpha2Code
  let res2 = await fetch("http://country.io/phone.json")
  let json2 = await res2.json()
  console.log(json2[code])
}

getCountries()
\end{lstlisting}

\subsection{Darstellung eines JavaScript Quellcodes (importiert)}

\lstinputlisting[
  language = JavaScript,
     style = ES6,
   caption = Beispiel eines JavaScript Quellcodes (importiert),
captionpos = b,
     label = lst:javascript-import-example,
]{content/xx_test/import-example_javascript.js}

\subsection{Darstellung eines Java Quellcodes (inline)}

\begin{lstlisting}[
   caption = Beispiel eines Java Quellcodes (inline),
     label = lst:java-inline-example,
  language = java, 
     style = java-eclipse,
basicstyle = {\footnotesize\fontfamily{pcr}\selectfont}
]
	/*
	 * Main-Methode
	 */
	§§@LineAnnotation
	public class Main {
	  public static void main(%%@InlineAnnotation%% String[] args) {
	    System.out.println("Hallo Welt");
	  }
	}
\end{lstlisting}

\subsection{Darstellung eines Java Quellcodes (importiert)}

\lstinputlisting[
   caption = Beispiel eines Java Quellcodes (importiert),
     label = lst:java-import-example,
  language = java,
     style = java-eclipse,
basicstyle = {\footnotesize\fontfamily{pcr}\selectfont}
]{content/xx_test/import-example_java.js}
	
	\newpage{}
	
	\addcontentsline{toc}{chapter}{Eidesstattliche Erklärung}
	\chapter*{Eidesstattliche Erklärung}
	% siehe https://www.fh-dortmund.de/de/fb/9/personen/lehr/buechler/Buechler_2012_Leitfaden_zum_Anfertigen_wissenschaftlicher_Arbeiten.pdf?dev=ugbpucsk
Hiermit versichere ich an Eides statt, dass ich die vorliegende Arbeit selbständig und ohne die Benutzung anderer als der angegebenen Hilfsmittel angefertigt habe. Alle Stellen, die wörtlich oder sinngemäß aus veröffentlichten und nicht veröffentlichten Schriften entnommen wurden, sind als solche kenntlich gemacht. Die Arbeit ist in gleicher oder ähnlicher Form oder auszugsweise im Rahmen einer anderen Prüfung noch nicht vorgelegt worden.
		
\vspace{4cm}
		
% bases on https://www.math.tugraz.at/~grabner/Masterarbeit-template.tex
		\noindent
\begin{minipage}[h]{0.4\linewidth}
	Dortmund, am \dotfill\\
	\vspace*{2.5mm}
\end{minipage}
	\hspace*{0.1\linewidth}
	\begin{minipage}[h]{0.5\linewidth}
	\begin{center}
		\dotfill\\
		(Unterschrift)
	\end{center}
\end{minipage}
	
	\let\cleardoublepage\relax
	
	\newpage{}
	
	\addcontentsline{toc}{chapter}{Abkürzungsverzeichnis}
	
	\settowidth{\nomlabelwidth}{API}
	\printnomenclature{}
	
	\newpage{}
	
	\addcontentsline{toc}{chapter}{Abbildungsverzeichnis}
	
	\listoffigures
	
	\newpage{}
	
	\addcontentsline{toc}{chapter}{Tabellenverzeichnis}
	
	\listoftables
	
	\lstlistoflistings
	
	\newpage{}
	
	\bibliographystyle{alphadin}
	\bibliography{bib}
\end{document}
