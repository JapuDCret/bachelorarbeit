% twoside = false setzen, falls kein doppelseitiger Druck nötig
\documentclass[oneside, ngerman, final, 11pt, a4paper, 1.1headlines, headinclude=false, footinclude=false, mpinclude=false, pagesize, onecolumn, titlepage, parskip=half, headsepline, chapterprefix=false, version=first, listof=totoc, bibliography=totoc, toc=graduated, fleqn, twoside=true]{scrbook}

% Import file that contains configuration
% Die folgenden Pakete werden in der TEX-Vorlage verwendet.
% Sollten zusätzliche Pakete verwendet werden, sind sie an dieser Stelle hinzuzufügen 
\usepackage[T1]{fontenc}
\usepackage{listings}
\usepackage{refstyle}
\usepackage{float}
\usepackage{textcomp}
\usepackage{graphicx}
\usepackage{nomencl}
\usepackage{textpos} 
\usepackage{xcolor}
\usepackage{tocloft}
\usepackage{lipsum}
\usepackage[utf8]{inputenc}
\usepackage{color}
\usepackage{scrhack}
\usepackage{babel}
\usepackage{wrapfig}

% Custom packages
\usepackage{url}
% to allow the configuration of caption
\usepackage{caption}
\usepackage{hyperref}

\usepackage{tikz}
\usepackage{pgfplots}
% see https://tex.stackexchange.com/questions/1460/script-to-automate-externalizing-tikz-graphics
\usetikzlibrary{external}
\tikzexternalize[prefix=tikz-figures/]

\usepackage{tabularx}
\usepackage{makecell}
\usepackage{csquotes}

% Import file that contains configuration
% Schachtelungstiefe eine Nummerierung der Überschriften 
\setcounter{secnumdepth}{3}
\setcounter{tocdepth}{3}

% Die folgenden Befehle sind nüzulich bei Verwendung des alten nomencl.sty Pakets
\providecommand{\printnomenclature}{\printglossary}
\providecommand{\makenomenclature}{\makeglossary}
\makenomenclature

% Zusätzliche Konfigurationen 
\makeatletter
\setkomafont{disposition}{\normalcolor\bfseries}
\makeatother

% Custom Zeugs

% Umbenennen von Listings und Nomenklatur 
\renewcommand{\nomname}{Abkürzungs- und Erklärungsverzeichnis}
\renewcommand\lstlistlistingname{Quellcodeverzeichnis}
\renewcommand{\lstlistingname}{Quellcode}
\renewcommand\appendixname{Anhang}

\addto\captionsngerman{
	\renewcommand{\figurename}{Abb.\@}
	\renewcommand{\tablename}{Tab.\@}
}

% Listing defintion for JavaScript and ECMAScript2015 (ES6)
%
% ECMAScript 2015 (ES6) definition by Gary Hammock
%

\lstdefinelanguage[ECMAScript2015]{JavaScript}[]{JavaScript}{
  morekeywords=[1]{await, async, case, catch, class, const, default, do,
    enum, export, extends, finally, from, implements, import, instanceof,
    let, static, super, switch, throw, try},
  morestring=[b]` % Interpolation strings.
}


%
% JavaScript version 1.1 by Gary Hammock
%
% Reference:
%   B. Eich and C. Rand Mckinney, "JavaScript Language Specification
%     (Preliminary Draft)", JavaScript 1.1.  1996-11-18.  [Online]
%     http://hepunx.rl.ac.uk/~adye/jsspec11/titlepg2.htm
%

\lstdefinelanguage{JavaScript}{
  morekeywords=[1]{break, continue, delete, else, for, function, if, in,
    new, return, this, typeof, var, void, while, with, =>, constructor, window, class, interface, enum, declare, const, var, typeof},
  % Literals, primitive types, and reference types.
  morekeywords=[2]{false, null, true, boolean, number, undefined,
    Array, Boolean, Date, Math, Number, String, Object, void, any, string, private, public, protected},
  % Built-ins.
  morekeywords=[3]{eval, parseInt, parseFloat, escape, unescape, pipe, tap, subscribe},
  morekeywords=[4]{@Injectable, @NgModule, @Inject},
  sensitive,
  morecomment=[s]{/*}{*/},
  morecomment=[l]//,
  morecomment=[s]{/**}{*/}, % JavaDoc style comments
  morestring=[b]',
  morestring=[b]"
}[keywords, comments, strings]


\lstalias[]{ES6}[ECMAScript2015]{JavaScript}

% Requires package: color.
\definecolor{mediumgray}{rgb}{0.3, 0.4, 0.4}
\definecolor{mediumblue}{rgb}{0.0, 0.0, 0.8}
\definecolor{forestgreen}{rgb}{0.13, 0.55, 0.13}
\definecolor{darkviolet}{rgb}{0.58, 0.0, 0.83}
\definecolor{royalblue}{rgb}{0.25, 0.41, 0.88}
\definecolor{crimson}{rgb}{0.86, 0.8, 0.24}

\lstdefinestyle{JSES6Base}{
  backgroundcolor=\color{white},
  basicstyle=\ttfamily,
  breakatwhitespace=false,
  breaklines=true,
  captionpos=b,
  columns=fullflexible,
  commentstyle=\color{mediumgray}\upshape,
  emph={},
  emphstyle=\color{crimson},
  extendedchars=true,  % requires inputenc
  fontadjust=true,
  frame=single,
  identifierstyle=\color{black},
  keepspaces=true,
  keywordstyle=\color{mediumblue},
  keywordstyle={[2]\color{darkviolet}},
  keywordstyle={[3]\color{royalblue}},
  keywordstyle={[4]\color{mediumblue}},
  numbers=left,
  numbersep=5pt,
  numberstyle=\color{black},
  rulecolor=\color{black},
  showlines=true,
  showspaces=false,
  showstringspaces=false,
  showtabs=false,
  stringstyle=\color{forestgreen},
  tabsize=2,
  title=\lstname,
  upquote=true  % requires textcomp
}

\lstdefinestyle{JavaScript}{
  language=JavaScript,
  style=JSES6Base
}
\lstdefinestyle{ES6}{
  language=ES6,
  style=JSES6Base
}

% Listing defintion for Angular HTML
\lstdefinelanguage{AngularTemplateHTML}{
	language=html,
	sensitive=true,	
	alsoletter={<>=-},
	morecomment=[s]{<!-}{-->},
	tag=[s],
	otherkeywords={
	% General
	>,
	% Standard tags
	<!DOCTYPE,
	</html, <html, <head, <title, </title, <style, </style, <link, </head, <meta, />,
	% body
	</body, <body,
	% Divs
	</div, <div, </div>, 
	% Paragraphs
	</p, <p, </p>, </h1, <h1, </h2, <h2, </h3, <h3, </h4, <h4, </h5, <h5, </h6, <h6,
	% scripts
	</script, <script,
	% More tags...
	<canvas, /canvas>, <svg, <rect, <animateTransform, </rect>, </svg>, <video, <source, <iframe, </iframe>, </video>, <image, </image>, <header, </header, <article, </article, <button, </button
	},
	ndkeywords={
	% General
	=,
	% HTML attributes
	charset=, src=, id=, width=, height=, style=, type=, rel=, href=, class=, color=, (click)=,
	% SVG attributes
	fill=, attributeName=, begin=, dur=, from=, to=, poster=, controls=, x=, y=, repeatCount=, xlink:href=,
	% properties
	margin:, padding:, background-image:, border:, top:, left:, position:, width:, height:, margin-top:, margin-bottom:, font-size:, line-height:,
	% CSS3 properties
	transform:, -moz-transform:, -webkit-transform:,
	animation:, -webkit-animation:,
	transition:,	transition-duration:, transition-property:, transition-timing-function:,
	},
	% fix umlaute causing issues
	% https://tex.stackexchange.com/questions/24528/having-problems-with-listings-and-utf-8-can-it-be-fixed/24529
	extendedchars=true,
	literate={ä}{{\"a}}1 {ö}{{\"o}}1 {ü}{{\"u}}1 {Ä}{{\"A}}1 {Ö}{{\"O}}1 {Ü}{{\"U}}1 {ß}{{\ss}}1,
}

% Listing defintion for JSON
\colorlet{json_punct}{red!60!black}
\definecolor{json_background}{HTML}{EEEEEE}
\definecolor{json_delim}{RGB}{20,105,176}
\colorlet{json_numb}{magenta!60!black}

\lstdefinelanguage{JSON}{
	basicstyle=\normalfont\ttfamily,
	numbers=left,
	numberstyle=\scriptsize,
	stepnumber=1,
	numbersep=8pt,
	showstringspaces=false,
	breaklines=true,
	frame=lines,
	backgroundcolor=\color{json_background},
	literate=
	 *{0}{{{\color{json_numb}0}}}{1}
	  {1}{{{\color{json_numb}1}}}{1}
	  {2}{{{\color{json_numb}2}}}{1}
	  {3}{{{\color{json_numb}3}}}{1}
	  {4}{{{\color{json_numb}4}}}{1}
	  {5}{{{\color{json_numb}5}}}{1}
	  {6}{{{\color{json_numb}6}}}{1}
	  {7}{{{\color{json_numb}7}}}{1}
	  {8}{{{\color{json_numb}8}}}{1}
	  {9}{{{\color{json_numb}9}}}{1}
	  {:}{{{\color{json_punct}{:}}}}{1}
	  {,}{{{\color{json_punct}{,}}}}{1}
	  {\{}{{{\color{json_delim}{\{}}}}{1}
	  {\}}{{{\color{json_delim}{\}}}}}{1}
	  {[}{{{\color{json_delim}{[}}}}{1}
	  {]}{{{\color{json_delim}{]}}}}{1}
	  {ä}{{\"a}}1 {ö}{{\"o}}1 {ü}{{\"u}}1 {Ä}{{\"A}}1 {Ö}{{\"O}}1 {Ü}{{\"U}}1 {ß}{{\ss}}1,
}

% Listing defintion for Java
% Farben fuer Programmlisting
\usepackage{listings,xcolor}
\definecolor{pl_background}{rgb}{0.95,0.95,0.95}
\definecolor{pl_comment}{rgb}{0.12, 0.38, 0.18 }
\definecolor{pl_ifelse}{rgb}{0.74,0.74,.29}
\definecolor{pl_keyword}{rgb}{0.37, 0.08, 0.25}
\definecolor{pl_string}{rgb}{0.06, 0.10, 0.98}

\definecolor{lstbackground}{RGB}{235,235,235}
\definecolor{lstkeyword}{RGB}{127,0,85}
\definecolor{lststring}{RGB}{42,0,255}
\definecolor{lstcomment}{RGB}{63,127,95}
\definecolor{lstannotation}{RGB}{127,159,191}

\definecolor{fh_grau}{rgb}{0.76,0.75,0.76}

% Vordefiniertes Programmlisting
\lstset{
	basicstyle = \small\sffamily,
	backgroundcolor = \color{pl_background},
	stringstyle = \color{pl_string},
	keywordstyle = \color{pl_keyword}\bfseries,
	commentstyle = \color{pl_comment}\itshape,
	frame = lrbt,
	numbers = left,
	captionpos=b,
	showstringspaces = false,
	breaklines = true,
	xleftmargin = 15pt,
	emph = [1]{php},
	emphstyle = [1]\color{black},
	emph = [2]{if,and,or,else},
	emphstyle = [2]\color{pl_ifelse}
}

% Quellcode Formatierung im normalen Stil 
\lstset{
	basicstyle={\footnotesize\fontfamily{pcr}\selectfont},
	backgroundcolor=\color{lstbackground},
	breaklines=true,
	frame=single,
	numbers=left,
	showstringspaces=false,
	tabsize=4
}

% Quellcde Formatierung im Eclipse Stil
\lstdefinestyle{java-eclipse}{
	commentstyle={\color{lstcomment}},
	keywordstyle={\color{lstkeyword}\bfseries},
	stringstyle={\color{lststring}},
	moredelim={[il][\textcolor{lstannotation}]{§§}},
	moredelim={[is][\textcolor{lstannotation}]{\%\%}{\%\%}}
}

% Listing defintion for Gherkin
% Gherkin
% https://gist.github.com/nsommer/9a04f6ebc6ea8b9f5b816becb97e7d9b
\lstdefinelanguage{gherkin}{
	morekeywords = {
		Given,
		When,
		Then,
		And,
		Scenario,
		Feature,
		But,
		Background,
		Scenario Outline,
		Examples
	},
	sensitive=true,
	morecomment=[l]{\#},
	morestring=[b]",
	morestring=[b]',
	% fix umlaute causing issues
	% https://tex.stackexchange.com/questions/24528/having-problems-with-listings-and-utf-8-can-it-be-fixed/24529
	extendedchars=true,
	literate={ä}{{\"a}}1 {ö}{{\"o}}1 {ü}{{\"u}}1 {Ä}{{\"A}}1 {Ö}{{\"O}}1 {Ü}{{\"U}}1 {ß}{{\ss}}1,
}

% own commands
\newcommand{\etal}{\textit{et al.\@ }}
\newcommand{\citationneeded}{{\color{red}[cite]}}
\newcommand{\source}[1]{\vspace{-0.75\baselineskip}\caption*{Quelle: {#1}} }

% custom Anforderung command
\newcounter{anfcounter}
\renewcommand{\theanfcounter}{\arabic{anfcounter}}

\makeatletter
\newenvironment{anf}[6]{%
\refstepcounter{anfcounter}%
\protected@edef\@currentlabel{Anforderung #2}
\label{#1}%
	\begin{table}[H]
	\begin{tabular}{ |p{1.25cm}|p{5.5cm}|p{2.25cm}|p{2.1cm}|p{1.25cm}| }
\hline
Id	 & Name					& Kano-Modell	& Funktionsart & Quelle			 \\
#2 & #3 & #4 & #5 & #6 \\
\hline
}%
{
	\end{tabular}
	\end{table}%
}%
\makeatother

% allow breaking of urls
% see https://tex.stackexchange.com/a/344711/221944
\def\UrlBreaks{\do\/\do-}

\newcommand*\circled[1]{
	\tikz[baseline=(char.base)]{
		\node[shape=circle,draw,inner sep=2pt] (char) {#1};
	}
}

\pgfplotsset{compat=1.17}

%\usepackage{glossaries}
%\makeglossaries

\newglossaryentry{Bug Report}
{
    name=Bug Report,
    description={Fehlerbericht},
    plural={Bug Reports},
}


% Beginn des Dokumentes 
\begin{document}
	% Die Commands werden auf der Titelseite eingefügt, die Werte sind anzupassen
	\newcommand*{\thedockind}{Bachelorthesis}
	\newcommand*{\thetitle}{Analyse zum besseren\\Verständnis von Nutzer-\\problemen bei\\JavaScript-basierten\\Webapplikationen}
	\newcommand*{\thesubtitle}{}
	\newcommand*{\theauthor}{Marvin Kienitz}
	\newcommand*{\thematriculationnumber}{7097533}
	\newcommand*{\thebirthday}{26.04.1996}
	\newcommand*{\thedegree}{Bachelor of Science}
	\newcommand*{\themajor}{Software- und Systemtechnik, Vertiefung Softwaretechnik} % Studiengang
	\newcommand*{\thedate}{\today} % \today kann durch ein Datum erstetzt werden. 
	\newcommand*{\thebetreuer}{Prof. Dr. Sven Jörges} 
	\newcommand*{\thezweitbetreuer}{{\color{red}\textit{Zweitbetreuer: -ausstehend-}}}

	\begin{titlepage}
	  \begin{textblock}{6.5}(-1,-3)
	    \begin{color}{fh_grau}
	      \rule{6.8cm}{33cm}    
	    \end{color}
	  \end{textblock}
	  \begin{textblock}{6.5}(-1.2,-0.7)
	  \end{textblock}
	  \begin{textblock}{6.5}(-0.8,1)
	    {\Large \textsf{\thedockind}}            
	  \end{textblock}
	
	  \begin{textblock}{7}(4.5,2)
	    {\noindent \huge 
	      \textsf{\textbf{\thetitle\\[0.3cm] 
	          \Large  \thesubtitle\\[0.05cm]
	          }} }
	  \end{textblock}
	
	
	  \begin{textblock}{8.5}(4.5,6.5)\noindent
	    \textsf{An der Fachhochschule Dortmund\\
	    im Fachbereich Informatik\\
	    Studiengang \themajor \\
	    erstellte \thedockind \\
	    zur Erlangung des akademischen Grades\\
	    \thedegree}
	  \end{textblock}
	
	  \begin{textblock}{6.5}(-0.4,10.0)
	    \noindent
	    \textsf{von \\
	      \theauthor \\
	      geb.\ am \thebirthday  \\
	      Matr.-Nr. \thematriculationnumber\\[0.7cm]
	      Betreuer:\\
	       \noindent\hspace*{6mm} \thebetreuer \\
	       \noindent\hspace*{6mm} \thezweitbetreuer\\ [0.5cm]
	      Dortmund, \today}    
	  \end{textblock}
		
	
	\end{titlepage}
	
	\newpage{}
	
	\section*{\thispagestyle{empty}Kurzfassung}
	
	\textit{\lipsum[1-4]}
	
	\newpage{}
	
	\section*{\thispagestyle{empty}Abstract}
	
	\textit{\lipsum[1-4]}
	
	\newpage{}
	
	% Römische Nummerierung der Seiten des Inhaltsverzeichnisses 
	\setcounter{page}{1}
	\pagenumbering{roman}
	
	\tableofcontents{}
	
	\newpage{}
	
	% Arabische Nummerierung aller anderen Seiten
	\setcounter{page}{1} 
	\pagenumbering{arabic}
	
	\chapter{Einleitung}
	
	%In diesem Unterkapitel sollten folgende Punkte behandelt werden:
%\begin{itemize}
%	\item	Was ist das Problem
%	\item 	Problemgeschichte?
%\end{itemize}
\section{Motivation}

In der Welt der Systeme mit Benutzerinteraktionen gibt es stets die Hürde, dass ``im Feld`` unvorhergesehene Probleme auftreten. Diese Systeme können bspw. Graphical User Interfaces (GUI) sein. Donald Norman \cite{TheProblemOfAutomation} argumentierte bereits 1989, dass bei komplexen Aufgaben und Umgebungen das Unerwartete erwartet werden muss. Nutzerfeedback ist notwendig um diese Situation aufzuklären und beheben zu können \cite{AnErrorReportingAndFeedbackComponent}.

In der Webentwicklung ist dieses Problem noch prägnanter, denn hier sind..
\begin{itemize}
	\item ..die Systeme selber meist um ein Vielfaches komplexer \cite{ManagingTheComplexityOfWebSystemsDevelopment},
	\item ..die Umgebungen komplex und unterschiedlich,
	\item ..die Nutzer meist weniger gut oder gar nicht geschult,
	\item ..die Nutzer eher unwissend, wie das System funktioniert \cite{AnErrorReportingAndFeedbackComponent} und
	\item ..die Nutzer stehen meist nicht im direkten Kontakt mit den Entwicklern \cite{EndUsersAsUnwittingSoftwareDevelopers}.
\end{itemize}

\begin{wrapfigure}[13]{r}{0.33\linewidth}
	\centering
	\vspace{-10pt}
	\includegraphics[width=\linewidth]{img/instagram-feedback/instagram-feedback}
	\caption{Formular aus der Instagram \cite{Instagram} Android App}
	\label{fig:instagram-feedback-example}
\end{wrapfigure}

\section{Problemstellung}

Aufgrund dieser Bedingungen werden bei Webprojekten Probleme im Feld erwartet. Zur Behebung dieser Mängel benötigen die Entwickler Informationen. Für Nutzer gibt es daher oftmals Formulare um diese Auffälligkeiten zu melden (Beispiel siehe rechts). Die Einbindung solcher Formulare ist zeit- und kostengünstig, kann aber nur erfolgreich sein, wenn die Nutzer verständliches und informatives Feedback geben können und wollen.

Bettenburg \etal \cite{WhatMakesAGoodBugReport} fanden bei Fehlerberichten eine Dissonanz zwischen dem was Entwickler als hilfreich empfanden und dem was Nutzer ihnen als Bericht lieferten. Eine besser zugeschnittene Lösung ist anzustreben. % \textbf{Wie kann den Entwicklern die Möglichkeit geboten werden, diese Probleme zu beheben?}

%\begin{itemize}
%	\item 	Was soll mit der Arbeit erreicht werden? Welche Ziele werden angestrebt?
%			Möglichst kurz und präzise geplante Ergebnisse umreißen. Daran werden
%			Ihre Resultate am Ende gemessen!
%\end{itemize}
\section{Zielsetzung}

Ziel dieser Arbeit ist es, zu erörtern, wie Fehler und Probleme, die ``im Feld`` auftreten, effektiv von Entwicklern identifiziert und behoben werden können.

Im Zuge der Arbeit sollen folgende Fragen beantwortet werden:

\begin{enumerate}
	\item Was gibt es für Probleme bei Nutzern?
	\item Was sind häufige Fehlerursachen?
	\item Was ist der aktuelle Stand der Technik? \\ 
	Wenn es hier Ansätze gibt:
	\begin{enumerate}
		\item Welche Probleme werden hiermit behoben?
		\item Was sind die Kosten für den Einsatz?
		\item Was sind die Kosten für den Nutzer? (initiale Ladezeit, Datenlast)
		\item Wie wird mit den Daten gehandhabt? (in Hinblick auf die DSGVO)
	\end{enumerate}
\end{enumerate}

Am Ende der Arbeit soll dem Leser klar vermittelt worden sein, wie er diese Probleme identifiziert und ihnen nachhaltig begegnen kann. Risiken sowie Einschränkungen der Technologien und Methodiken sind dem Leser ebenso vermittelt worden.

%\subsection{Abgrenzung}
%
%Diese Arbeit soll sich mit der GUI Entwicklung in der Webentwicklung beschäftigen. Weiterhin soll sich nicht auf den anfänglichen Entwicklungsprozess konzentriert werden, sondern auf bereits bestehende Software. Es soll keine benutzerorientierte Entwicklung \cite{UserCenteredWebDesign} beleuchtet oder angestrebt werden.

%\begin{itemize}
%	\item 	Wie wird vorgegangen, um das Ziel zu erreichen?
%	\item 	Warum ist die Arbeit so gegliedert, wie sie gegliedert ist?
%	\item 	Welche Aspekte werden nicht behandelt und warum?
%\end{itemize}
\section{Vorgehensweise}

Anfangs soll identifiziert werden, was alles für einen Nutzer ein Problem darstellen kann. Es muss sich bei den Problemen nicht nur um Laufzeitfehler o.Ä. handeln, sondern Logikfehler oder auch Verständnisprobleme führen zu einer Einschränkung der Nutzbarkeit. Hier soll eine Analyse aus der Literatur und einer ggf. Umfrage erstellt werden, um daraufhin grobe Problembilder zu klassifizieren.

Danach soll erörtert werden, welche Ursachen es für häufige Fehlerszenarien gibt. Darauf aufbauend könnten Aussagen getroffen werden, wie man diese bereits vorher vermeidet/reduziert.

Da nun ein Basisverständnis gewonnen wurde, sollen etablierte Technologien wie Google Cloud \cite{GoogleCloudErrorReporting}, Dynatrace \cite{DynatraceDigitalExperienceMonitoring}, Sentry \cite{SentryForJavaScript} und LogRocket \cite{LogRocket} näher betrachtet werden. Durch die Erörterung des Stands der Technik sollen folgend Empfehlungen ausgesprochen werden, für welche Projekte welche Technologie oder Kombination von Technologien sinnvoll ist. Sollte keine Technologie als angemessen betrachtet werden, so soll basierend auf den Anforderungen ein Vorschlag gemacht werden, wie so eine Technologie aussehen könnte.

	
	\chapter{Probleme der Nutzer}
	
		\textit{\lipsum[1]}
		
		\section{Logische Probleme}
		\textit{\lipsum[1]}
		
		\section{Probleme in der Software}
		\textit{\lipsum[1]}
		
		\section{Klassifizierung}
		\textit{\lipsum[1]}
	
	\chapter{Problemursachen}
	
	\textit{\lipsum[1-3]}
	
	\chapter{Konzept einer Softwarelösung}
	
		\section{Stand der Technik}
		\textit{\lipsum[1]}
		
		\section{Datenerhebung}
		\textit{\lipsum[1]}
		
		\section{Auswertung und Visualisierung}
		\textit{\lipsum[1]}
	
	\chapter{Entwurf der Softwarelösung}
		
		\section{Grobarchitektur}
		\textit{\lipsum[1]}
	
		\section{Datenerhebende Komponente}
		\textit{\lipsum[1]}
		
		\section{Auswertende Komponente}
		\textit{\lipsum[1]}
	
	\chapter{Implementierung der Softwarelösung}
	
		\section{Datenerhebende Komponente}
		\textit{\lipsum[1]}
		
		\section{Auswertende Komponente}
		\textit{\lipsum[1]}
		
		\section{Kommunikation zwischen den Komponenten}
		\textit{\lipsum[1]}
	
	\chapter{Einsatz der Softwarelösung}
	
		\section{Integration in eine bestehende Applikation}
		\textit{\lipsum[1]}
		
		\section{Testen der Softwarelösung anhand der Applikation}
		\textit{\lipsum[1]}
		
		\section{Aufwände und Kosten}
		\textit{\lipsum[1]}
		
		\section{Nutzen}
		\textit{\lipsum[1]}
	
	\chapter{Fazit}
	
	\textit{\lipsum[1-3]}
	
	\chapter{Test-Kapitel}
	
	\section{Zitate}

Im Literaturverzeichnis sollte zu jedem Zitat ein Eintrag mit dem entsprechenden Kürzel zu finden sein.

\subsection{Einzelner Autor}

Zur Überprüfung wird das Buch \enquote{The Hitchhiker's Guide to the Galaxy}  von Douglas Adams aus dem Jahr 1979 hinzugezogen. Die folgende Zitierung sollte mit \texttt{[Ada79]}  abgekürzt dargestellt werden. \textit{Zitat} \cite{TestCitation001}.

\subsection{Zwei Autoren}

Zur Überprüfung wird das Buch \enquote{Hard drive: Bill Gates and the making of the Microsoft empire} von James Wallace und Jim Erickson aus dem Jahr 1992 hinzugezogen. Die folgende Zitierung sollte mit \texttt{[WE92]} abgekürzt dargestellt werden. \textit{Zitat} \cite{TestCitation002}.

\subsection{Drei Autoren}

Zur Überprüfung wird der Artikel \enquote{Antioxidant activity of apple peels} von Kelly Wolfe, Xianzhong Wu und Rui Hai Liu aus dem Jahr 2003 hinzugezogen. Die folgende Zitierung sollte mit \texttt{[WWL03]} abgekürzt dargestellt werden. \textit{Zitat} \cite{TestCitation003}.

\subsection{Viele Autoren}

Zur Überprüfung wird der Artikel \enquote{Observation of top quark production in p p collisions with the Collider Detector at Fermilab} von Fumio Abe, H. Akimoto, A. Akopian, M.G. Albrow, S.R. Amendolia, D. Amidei, J. Antos, C. Anway-Wiese, S. Aota, G. Apollinari und Weitere aus dem Jahr 1995 hinzugezogen. Die folgende Zitierung sollte mit \texttt{[AAA+95]} abgekürzt dargestellt werden. \textit{Zitat} \cite{TestCitation004}.

\newpage

\section{Tabellen}

Eine Tabelle sollte immer eine Über- oder Unterschrift erhalten und mit dieser im Tabellenverzeichnis wiederzufinden sein. Weiterhin muss ein Zitat angegeben werden, wenn es sich um hinzugezogene Daten handelt.

\subsection{Simple Tabelle}

\begin{table}[H]
\centering
\begin{tabular}{|l|l|} 
\hline
Spezifische Wärmekapazität (J/(mol K)) & Temperatur (°C)  \\ 
\hline
12,2                                   & -200             \\ 
\hline
15,0                                   & -180             \\ 
\hline
17,3                                   & -160             \\ 
\hline
19,8                                   & -140             \\ 
\hline
24,8                                   & -100             \\ 
\hline
29,6                                   & -60              \\
\hline
\end{tabular}
\caption{Spezifische Wärmekapazität von Wasser \cite{TestCitation020}}
\end{table}

\subsection{Komplexere Tabelle}

\newcommand{\tabtitel}[1]{\multicolumn{2}{l|}{\textbf{#1}}}

\begin{table}[H]
\centering
\begin{tabular}{|l|l|l|l|l|} 
\hline
           & \tabtitel{Ostdeutschland} & \tabtitel{Westdeutschland}  \\ 
\hline
Geschlecht & Frauen & Männer           & Frauen & Männer                       \\ 
\hline
weiblich   & 100\%  & 0\%              & 100\%  & 0\%                          \\ 
\hline
männlich   & 0\%    & 100\%            & 0\%    & 100\%                        \\
\hline
\end{tabular}
\caption{Absurder Vergleich von Ost- und Westdeutschland}
\end{table}

\newpage

\section{Grafiken, Bilder, etc.}

Eine Tabelle sollte immer eine Über- oder Unterschrift erhalten und mit dieser im Tabellenverzeichnis wiederzufinden sein.

\subsection{Simple Grafik}

\begin{figure}[H]
	\centering
	\includegraphics[width=\linewidth]{content/xx_test/Phase_diagram_of_water_simplified.svg.png}
	\caption{Vereinfachtes Phasendiagramm von Wasser \cite{TestCitation020}}
\end{figure}

\newpage

\subsection{Innerhalb eines Textes (Floating)}

\begin{wrapfigure}[14]{r}{0.33\textwidth}
	\includegraphics[width=\linewidth]{content/xx_test/1032px-3D_model_hydrogen_bonds_in_water.svg.png}
	\caption{Verkettung der Wassermoleküle \cite{TestCitation021}}
\end{wrapfigure}

Lorem ipsum dolor sit amet, consectetur adipiscing elit. Praesent eu risus a erat auctor bibendum. Morbi eget aliquet nisl. Aenean vestibulum elit sed arcu condimentum euismod. Nunc quis ipsum sed augue maximus molestie. Pellentesque condimentum elit vitae justo tincidunt elementum. Aliquam erat volutpat. Vestibulum feugiat auctor fringilla.

Cras non arcu ante. Donec faucibus lectus risus, ac sagittis risus malesuada pretium. Cras elementum quis turpis accumsan faucibus. Vestibulum vitae volutpat nisl, et sodales nunc. Etiam egestas at magna a commodo. Mauris sagittis suscipit tempus. Duis rhoncus nec ligula eget viverra. Maecenas eu nisl orci. Proin dignissim laoreet libero in interdum. Lorem ipsum dolor sit amet, consectetur adipiscing elit.

Sed fringilla lectus non elit convallis, non eleifend sem porta. Proin aliquet urna ultrices metus blandit ornare. Duis nec ultricies ligula, quis volutpat ante. Suspendisse non lacus mauris.

\begin{wrapfigure}[14]{o}{0.45\textwidth}
	\includegraphics[width=\linewidth]{content/xx_test/Kleiner_Streichteich_Ilmenau.JPG}
	\caption{Spiegelung an der Wasseroberfläche \cite{TestCitation022}}
\end{wrapfigure}

Nullam quam diam, mattis non bibendum vel, iaculis sed leo. Phasellus condimentum auctor ante in mollis. Nullam rhoncus enim ac metus fermentum aliquam. Phasellus orci metus, tristique ac odio sed, fermentum faucibus mauris. Praesent vehicula risus aliquam nibh sodales, ac aliquam ante posuere. Praesent at semper sapien. In posuere augue vel tortor posuere rhoncus. Duis eget quam tempus diam posuere fringilla ut vitae lacus. Sed sagittis fringilla diam.

Suspendisse potenti. Ut pellentesque malesuada dolor vitae porta. Aliquam erat volutpat. Proin convallis mauris neque. Etiam et accumsan ex. Class aptent taciti sociosqu ad litora torquent per conubia nostra, per inceptos himenaeos. Suspendisse potenti. Cras ac magna enim. Praesent ultrices lacinia sem nec placerat. Sed tristique non sapien quis efficitur. Ut sit amet egestas metus. Mauris nec erat sodales, semper dolor eu, egestas justo. Ut sit amet urna ligula.

\section{Quellcode}

Der jeweilige Quellcode sollte mit Syntax-Highlighting versehen sein und im Quellcodeverzeichnis aufzufinden sein.

\subsection{Darstellung eines JavaScript Quellcodes (inline)}

\begin{lstlisting}[
  language = JavaScript,
     style = ES6,
   caption = Beispiel eines JavaScript Quellcodes (inline),
captionpos = b,
     label = lst:javascript-inline-example,
]
var fetch = require("node-fetch")

async function getCountries() {
  let res = await fetch("https://restcountries.eu/rest/v2/name/Indonesia?fullText=true")
  let json = await res.json()
  let code = json[0].alpha2Code
  let res2 = await fetch("http://country.io/phone.json")
  let json2 = await res2.json()
  console.log(json2[code])
}

getCountries()
\end{lstlisting}

\subsection{Darstellung eines JavaScript Quellcodes (importiert)}

\lstinputlisting[
  language = JavaScript,
     style = ES6,
   caption = Beispiel eines JavaScript Quellcodes (importiert),
captionpos = b,
     label = lst:javascript-import-example,
]{content/xx_test/import-example_javascript.js}

\subsection{Darstellung eines Java Quellcodes (inline)}

\begin{lstlisting}[
   caption = Beispiel eines Java Quellcodes (inline),
     label = lst:java-inline-example,
  language = java, 
     style = java-eclipse,
basicstyle = {\footnotesize\fontfamily{pcr}\selectfont}
]
	/*
	 * Main-Methode
	 */
	§§@LineAnnotation
	public class Main {
	  public static void main(%%@InlineAnnotation%% String[] args) {
	    System.out.println("Hallo Welt");
	  }
	}
\end{lstlisting}

\subsection{Darstellung eines Java Quellcodes (importiert)}

\lstinputlisting[
   caption = Beispiel eines Java Quellcodes (importiert),
     label = lst:java-import-example,
  language = java,
     style = java-eclipse,
basicstyle = {\footnotesize\fontfamily{pcr}\selectfont}
]{content/xx_test/import-example_java.js}
	
	\addcontentsline{toc}{chapter}{Abkürzungsverzeichnis}
	
	\settowidth{\nomlabelwidth}{API}
	\printnomenclature{}
	
	\newpage{}
	
	\addcontentsline{toc}{chapter}{Abbildungsverzeichnis}
	
	\listoffigures
	
	\newpage{}
	
	\addcontentsline{toc}{chapter}{Tabellenverzeichnis}
	
	\listoftables
	
	\lstlistoflistings
	
	\newpage{}
	
	\bibliographystyle{alphadin}
	\bibliography{bib}
\end{document}
