% twoside = false setzen, falls kein doppelseitiger Druck nötig
\documentclass[oneside, ngerman, final, 11pt, a4paper, 1.1headlines, headinclude=false, footinclude=false, mpinclude=false, pagesize, onecolumn, titlepage, parskip=half, headsepline, chapterprefix=false, version=first, listof=totoc, bibliography=totoc, toc=graduated, fleqn, twoside=false]{scrbook}

% Import file that contains configuration
% Die folgenden Pakete werden in der TEX-Vorlage verwendet.
% Sollten zusätzliche Pakete verwendet werden, sind sie an dieser Stelle hinzuzufügen 
\usepackage[T1]{fontenc}
\usepackage{listings}
\usepackage{refstyle}
\usepackage{float}
\usepackage{textcomp}
\usepackage{graphicx}
\usepackage{nomencl}
\usepackage{textpos} 
\usepackage{xcolor}
\usepackage{tocloft}
\usepackage{lipsum}
\usepackage[utf8]{inputenc}
\usepackage{color}
\usepackage{scrhack}
\usepackage{babel}
\usepackage{wrapfig}

% Custom packages
\usepackage{url}
% to allow the configuration of caption
\usepackage{caption}
\usepackage{hyperref}

\usepackage{tikz}
\usepackage{pgfplots}
% see https://tex.stackexchange.com/questions/1460/script-to-automate-externalizing-tikz-graphics
\usetikzlibrary{external}
\tikzexternalize[prefix=tikz-figures/]

\usepackage{tabularx}
\usepackage{makecell}
\usepackage{csquotes}

% Import file that contains configuration
% Schachtelungstiefe eine Nummerierung der Überschriften 
\setcounter{secnumdepth}{3}
\setcounter{tocdepth}{3}

% Die folgenden Befehle sind nüzulich bei Verwendung des alten nomencl.sty Pakets
\providecommand{\printnomenclature}{\printglossary}
\providecommand{\makenomenclature}{\makeglossary}
\makenomenclature

% Zusätzliche Konfigurationen 
\makeatletter
\setkomafont{disposition}{\normalcolor\bfseries}
\makeatother

% Custom Zeugs

% Umbenennen von Listings und Nomenklatur 
\renewcommand{\nomname}{Abkürzungs- und Erklärungsverzeichnis}
\renewcommand\lstlistlistingname{Quellcodeverzeichnis}
\renewcommand{\lstlistingname}{Quellcode}
\renewcommand\appendixname{Anhang}

\addto\captionsngerman{
	\renewcommand{\figurename}{Abb.\@}
	\renewcommand{\tablename}{Tab.\@}
}

% Listing defintion for JavaScript and ECMAScript2015 (ES6)
%
% ECMAScript 2015 (ES6) definition by Gary Hammock
%

\lstdefinelanguage[ECMAScript2015]{JavaScript}[]{JavaScript}{
  morekeywords=[1]{await, async, case, catch, class, const, default, do,
    enum, export, extends, finally, from, implements, import, instanceof,
    let, static, super, switch, throw, try},
  morestring=[b]` % Interpolation strings.
}


%
% JavaScript version 1.1 by Gary Hammock
%
% Reference:
%   B. Eich and C. Rand Mckinney, "JavaScript Language Specification
%     (Preliminary Draft)", JavaScript 1.1.  1996-11-18.  [Online]
%     http://hepunx.rl.ac.uk/~adye/jsspec11/titlepg2.htm
%

\lstdefinelanguage{JavaScript}{
  morekeywords=[1]{break, continue, delete, else, for, function, if, in,
    new, return, this, typeof, var, void, while, with, =>, constructor, window, class, interface, enum, declare, const, var, typeof},
  % Literals, primitive types, and reference types.
  morekeywords=[2]{false, null, true, boolean, number, undefined,
    Array, Boolean, Date, Math, Number, String, Object, void, any, string, private, public, protected},
  % Built-ins.
  morekeywords=[3]{eval, parseInt, parseFloat, escape, unescape, pipe, tap, subscribe},
  morekeywords=[4]{@Injectable, @NgModule, @Inject},
  sensitive,
  morecomment=[s]{/*}{*/},
  morecomment=[l]//,
  morecomment=[s]{/**}{*/}, % JavaDoc style comments
  morestring=[b]',
  morestring=[b]"
}[keywords, comments, strings]


\lstalias[]{ES6}[ECMAScript2015]{JavaScript}

% Requires package: color.
\definecolor{mediumgray}{rgb}{0.3, 0.4, 0.4}
\definecolor{mediumblue}{rgb}{0.0, 0.0, 0.8}
\definecolor{forestgreen}{rgb}{0.13, 0.55, 0.13}
\definecolor{darkviolet}{rgb}{0.58, 0.0, 0.83}
\definecolor{royalblue}{rgb}{0.25, 0.41, 0.88}
\definecolor{crimson}{rgb}{0.86, 0.8, 0.24}

\lstdefinestyle{JSES6Base}{
  backgroundcolor=\color{white},
  basicstyle=\ttfamily,
  breakatwhitespace=false,
  breaklines=true,
  captionpos=b,
  columns=fullflexible,
  commentstyle=\color{mediumgray}\upshape,
  emph={},
  emphstyle=\color{crimson},
  extendedchars=true,  % requires inputenc
  fontadjust=true,
  frame=single,
  identifierstyle=\color{black},
  keepspaces=true,
  keywordstyle=\color{mediumblue},
  keywordstyle={[2]\color{darkviolet}},
  keywordstyle={[3]\color{royalblue}},
  keywordstyle={[4]\color{mediumblue}},
  numbers=left,
  numbersep=5pt,
  numberstyle=\color{black},
  rulecolor=\color{black},
  showlines=true,
  showspaces=false,
  showstringspaces=false,
  showtabs=false,
  stringstyle=\color{forestgreen},
  tabsize=2,
  title=\lstname,
  upquote=true  % requires textcomp
}

\lstdefinestyle{JavaScript}{
  language=JavaScript,
  style=JSES6Base
}
\lstdefinestyle{ES6}{
  language=ES6,
  style=JSES6Base
}

% Listing defintion for Angular HTML
\lstdefinelanguage{AngularTemplateHTML}{
	language=html,
	sensitive=true,	
	alsoletter={<>=-},
	morecomment=[s]{<!-}{-->},
	tag=[s],
	otherkeywords={
	% General
	>,
	% Standard tags
	<!DOCTYPE,
	</html, <html, <head, <title, </title, <style, </style, <link, </head, <meta, />,
	% body
	</body, <body,
	% Divs
	</div, <div, </div>, 
	% Paragraphs
	</p, <p, </p>, </h1, <h1, </h2, <h2, </h3, <h3, </h4, <h4, </h5, <h5, </h6, <h6,
	% scripts
	</script, <script,
	% More tags...
	<canvas, /canvas>, <svg, <rect, <animateTransform, </rect>, </svg>, <video, <source, <iframe, </iframe>, </video>, <image, </image>, <header, </header, <article, </article, <button, </button
	},
	ndkeywords={
	% General
	=,
	% HTML attributes
	charset=, src=, id=, width=, height=, style=, type=, rel=, href=, class=, color=, (click)=,
	% SVG attributes
	fill=, attributeName=, begin=, dur=, from=, to=, poster=, controls=, x=, y=, repeatCount=, xlink:href=,
	% properties
	margin:, padding:, background-image:, border:, top:, left:, position:, width:, height:, margin-top:, margin-bottom:, font-size:, line-height:,
	% CSS3 properties
	transform:, -moz-transform:, -webkit-transform:,
	animation:, -webkit-animation:,
	transition:,	transition-duration:, transition-property:, transition-timing-function:,
	},
	% fix umlaute causing issues
	% https://tex.stackexchange.com/questions/24528/having-problems-with-listings-and-utf-8-can-it-be-fixed/24529
	extendedchars=true,
	literate={ä}{{\"a}}1 {ö}{{\"o}}1 {ü}{{\"u}}1 {Ä}{{\"A}}1 {Ö}{{\"O}}1 {Ü}{{\"U}}1 {ß}{{\ss}}1,
}

% Listing defintion for JSON
\colorlet{json_punct}{red!60!black}
\definecolor{json_background}{HTML}{EEEEEE}
\definecolor{json_delim}{RGB}{20,105,176}
\colorlet{json_numb}{magenta!60!black}

\lstdefinelanguage{JSON}{
	basicstyle=\normalfont\ttfamily,
	numbers=left,
	numberstyle=\scriptsize,
	stepnumber=1,
	numbersep=8pt,
	showstringspaces=false,
	breaklines=true,
	frame=lines,
	backgroundcolor=\color{json_background},
	literate=
	 *{0}{{{\color{json_numb}0}}}{1}
	  {1}{{{\color{json_numb}1}}}{1}
	  {2}{{{\color{json_numb}2}}}{1}
	  {3}{{{\color{json_numb}3}}}{1}
	  {4}{{{\color{json_numb}4}}}{1}
	  {5}{{{\color{json_numb}5}}}{1}
	  {6}{{{\color{json_numb}6}}}{1}
	  {7}{{{\color{json_numb}7}}}{1}
	  {8}{{{\color{json_numb}8}}}{1}
	  {9}{{{\color{json_numb}9}}}{1}
	  {:}{{{\color{json_punct}{:}}}}{1}
	  {,}{{{\color{json_punct}{,}}}}{1}
	  {\{}{{{\color{json_delim}{\{}}}}{1}
	  {\}}{{{\color{json_delim}{\}}}}}{1}
	  {[}{{{\color{json_delim}{[}}}}{1}
	  {]}{{{\color{json_delim}{]}}}}{1}
	  {ä}{{\"a}}1 {ö}{{\"o}}1 {ü}{{\"u}}1 {Ä}{{\"A}}1 {Ö}{{\"O}}1 {Ü}{{\"U}}1 {ß}{{\ss}}1,
}

% Listing defintion for Java
% Farben fuer Programmlisting
\usepackage{listings,xcolor}
\definecolor{pl_background}{rgb}{0.95,0.95,0.95}
\definecolor{pl_comment}{rgb}{0.12, 0.38, 0.18 }
\definecolor{pl_ifelse}{rgb}{0.74,0.74,.29}
\definecolor{pl_keyword}{rgb}{0.37, 0.08, 0.25}
\definecolor{pl_string}{rgb}{0.06, 0.10, 0.98}

\definecolor{lstbackground}{RGB}{235,235,235}
\definecolor{lstkeyword}{RGB}{127,0,85}
\definecolor{lststring}{RGB}{42,0,255}
\definecolor{lstcomment}{RGB}{63,127,95}
\definecolor{lstannotation}{RGB}{127,159,191}

\definecolor{fh_grau}{rgb}{0.76,0.75,0.76}

% Vordefiniertes Programmlisting
\lstset{
	basicstyle = \small\sffamily,
	backgroundcolor = \color{pl_background},
	stringstyle = \color{pl_string},
	keywordstyle = \color{pl_keyword}\bfseries,
	commentstyle = \color{pl_comment}\itshape,
	frame = lrbt,
	numbers = left,
	captionpos=b,
	showstringspaces = false,
	breaklines = true,
	xleftmargin = 15pt,
	emph = [1]{php},
	emphstyle = [1]\color{black},
	emph = [2]{if,and,or,else},
	emphstyle = [2]\color{pl_ifelse}
}

% Quellcode Formatierung im normalen Stil 
\lstset{
	basicstyle={\footnotesize\fontfamily{pcr}\selectfont},
	backgroundcolor=\color{lstbackground},
	breaklines=true,
	frame=single,
	numbers=left,
	showstringspaces=false,
	tabsize=4
}

% Quellcde Formatierung im Eclipse Stil
\lstdefinestyle{java-eclipse}{
	commentstyle={\color{lstcomment}},
	keywordstyle={\color{lstkeyword}\bfseries},
	stringstyle={\color{lststring}},
	moredelim={[il][\textcolor{lstannotation}]{§§}},
	moredelim={[is][\textcolor{lstannotation}]{\%\%}{\%\%}}
}

% Listing defintion for Gherkin
% Gherkin
% https://gist.github.com/nsommer/9a04f6ebc6ea8b9f5b816becb97e7d9b
\lstdefinelanguage{gherkin}{
	morekeywords = {
		Given,
		When,
		Then,
		And,
		Scenario,
		Feature,
		But,
		Background,
		Scenario Outline,
		Examples
	},
	sensitive=true,
	morecomment=[l]{\#},
	morestring=[b]",
	morestring=[b]',
	% fix umlaute causing issues
	% https://tex.stackexchange.com/questions/24528/having-problems-with-listings-and-utf-8-can-it-be-fixed/24529
	extendedchars=true,
	literate={ä}{{\"a}}1 {ö}{{\"o}}1 {ü}{{\"u}}1 {Ä}{{\"A}}1 {Ö}{{\"O}}1 {Ü}{{\"U}}1 {ß}{{\ss}}1,
}

% own commands
\newcommand{\etal}{\textit{et al.\@ }}
\newcommand{\citationneeded}{{\color{red}[cite]}}
\newcommand{\source}[1]{\vspace{-0.75\baselineskip}\caption*{Quelle: {#1}} }

% custom Anforderung command
\newcounter{anfcounter}
\renewcommand{\theanfcounter}{\arabic{anfcounter}}

\makeatletter
\newenvironment{anf}[6]{%
\refstepcounter{anfcounter}%
\protected@edef\@currentlabel{Anforderung #2}
\label{#1}%
	\begin{table}[H]
	\begin{tabular}{ |p{1.25cm}|p{5.5cm}|p{2.25cm}|p{2.1cm}|p{1.25cm}| }
\hline
Id	 & Name					& Kano-Modell	& Funktionsart & Quelle			 \\
#2 & #3 & #4 & #5 & #6 \\
\hline
}%
{
	\end{tabular}
	\end{table}%
}%
\makeatother

% allow breaking of urls
% see https://tex.stackexchange.com/a/344711/221944
\def\UrlBreaks{\do\/\do-}

\newcommand*\circled[1]{
	\tikz[baseline=(char.base)]{
		\node[shape=circle,draw,inner sep=2pt] (char) {#1};
	}
}

\pgfplotsset{compat=1.17}

%\usepackage{glossaries}
%\makeglossaries

\newglossaryentry{Bug Report}
{
    name=Bug Report,
    description={Fehlerbericht},
    plural={Bug Reports},
}


% Beginn des Dokumentes 
\begin{document}
	% Die Commands werden auf der Titelseite eingefügt, die Werte sind anzupassen
	\newcommand*{\thedockind}{\textbf{Proposal} zur Bachelorarbeit}
	\newcommand*{\thetitle}{Bessere Nachvollziehbarkeit von JavaScript-Anwendungen im Browser}
	\newcommand*{\thesubtitle}{Ansatz zur Verbesserung der Nachvollziehbarkeit von Nutzerinteraktion und Anwendungsverhalten} % TODO: block align
	\newcommand*{\theauthor}{Marvin Kienitz}
	\newcommand*{\thematriculationnumber}{7097533}
	\newcommand*{\thebirthday}{26.04.1996}
	\newcommand*{\thedegree}{Bachelor of Science}
	\newcommand*{\themajor}{Software- und Systemtechnik, Vertiefung Softwaretechnik} % Studiengang
	\newcommand*{\thedate}{\today} % \today kann durch ein Datum erstetzt werden. 
	\newcommand*{\thebetreuer}{Prof. Dr. Sven Jörges} 
	\newcommand*{\thezweitbetreuer}{Dipl. Inf. Stephan Müller}

	\begin{titlepage}
	  \begin{textblock}{6.5}(-1,-3)
	    \begin{color}{fh_grau}
	      \rule{6.8cm}{33cm}    
	    \end{color}
	  \end{textblock}
	  \begin{textblock}{6.5}(-1.2,-0.7)
	  \end{textblock}
	  \begin{textblock}{6.5}(-0.8,1)
	    {\Large \textsf{\thedockind}}            
	  \end{textblock}
	
	  \begin{textblock}{7}(4.5,2)
	    {\noindent \huge 
	      \textsf{\textbf{\thetitle\\[0.3cm] 
	          \Large  \thesubtitle\\[0.05cm]
	          }} }
	  \end{textblock}
	
	
	  \begin{textblock}{8.5}(4.5,6.5)\noindent
	    \textsf{An der Fachhochschule Dortmund\\
	    im Fachbereich Informatik\\
	    Studiengang \themajor \\
	    erstelltes \thedockind \\
	    zur Erlangung des akademischen Grades\\
	    \thedegree}
	  \end{textblock}
	
	  \begin{textblock}{6.5}(-0.4,10.0)
	    \noindent
	    \textsf{von \\
	      \theauthor \\
	      geb.\ am \thebirthday  \\
	      Matr.-Nr. \thematriculationnumber\\[0.7cm]
	      Betreuer:\\
	       \noindent\hspace*{6mm} \thebetreuer \\
	       \noindent\hspace*{6mm} \thezweitbetreuer\\ [0.5cm]
	      Dortmund, \today}    
	  \end{textblock}
		
	
	\end{titlepage}
	
	\newpage{}
	
	\section*{\thispagestyle{empty}Kurzfassung}
	
	\textit{\lipsum[1-4]}
	
	\newpage{}
	
	\section*{\thispagestyle{empty}Abstract}
	
	\textit{\lipsum[1-4]}
	
	\newpage{}
	
	% Römische Nummerierung der Seiten des Inhaltsverzeichnisses 
	\setcounter{page}{1}
	\pagenumbering{roman}
	
	\tableofcontents{}
	
	\newpage{}
	
	% Arabische Nummerierung aller anderen Seiten
	\setcounter{page}{1} 
	\pagenumbering{arabic}
	
	\chapter{Einleitung}
	% a hack, to make the Motivation fit into one page..
\vspace{-\baselineskip}

%In diesem Unterkapitel sollten folgende Punkte behandelt werden:
%\begin{itemize}
%	\item	Was ist das Problem
%	\item 	Problemgeschichte?
%\end{itemize}
\section{Motivation}

Die Open Knowledge GmbH ist als branchenneutraler Softwaredienstleister in einer Vielzahl von Branchen wie Automotive, Logistik, Telekommunikation und Versicherungs- und Finanzwirtschaft aktiv. Zu den zahlreichen Kunden der Open Knowledge GmbH gehört auch ein führender deutscher Direktversicherer. 

Ein Direktversicherer bietet Versicherungsprodukte seinen Kunden ausschließlich im Direktvertrieb, d. h. vor allem über das Internet und zusätzlich auch über Telefon, Fax oder Brief an. Im Unterschied zum klassischen Versicherer verfügt ein Direktversicherer jedoch über keinen Außendienst oder Geschäftsstellen, bei denen Kunden eine persönliche Beratung erhalten. Da das Internet der primäre Vertriebskanal ist, ist heute ein umfassender Online-Auftritt die Norm. Dieser besteht typischerweise aus einem Kundenportal mit der Möglichkeit Angebote für Versicherungsprodukte berechnen und abschließen zu können, sowie persönliche Daten und Verträge einsehen und ändern zu können.

\nomenclature[Fachbegriff]{Serverseitiges Rendering}{Die darzustellenden Inhalte, werden beim Server generiert und der Client stellt diese dar. Beispielsweise sind Anwendungen mit PHP oder auch eine Java Web Application}
\nomenclature[Fachbegriff]{Clientseitiges Rendering}{Der Server stellt dem Client lediglich die Logik und die notwendigen Daten bereit, die eigentliche Inhaltsgenerierung geschieht im Client. Für ein Beispiel siehe \autoref{subsec:singe-page-applications}}

% i.d.R. getrennt: https://www.scribbr.de/wissenschaftliches-schreiben/abkuerzungen/
Während in der Vergangenheit Online-Auftritte i. d. R. als Webanwendung mit serverseitigen Rendering realisiert wurden, sind heutzutage Javascript-basierte Webanwendung mit clientseitigem Rendering die Norm. Bei einer solchen Webanwendung befindet sich die gesamte Logik mit Ausnahme der Berechnung des Angebots und der Verarbeitung der Antragsdaten im Browser des Nutzers.

Im produktiven Einsatz kommt es auch bei gut getesteten Webanwendungen hin und wieder vor, dass es zu unvorhergesehenen Fehlern in der Berechnung oder Verarbeitung kommt. Liegt die Ursache für den Fehler im Browser, z. B. aufgrund einer ungültigen Wertkombination, ist dies eine Herausforderung. Während bei Server-Anwendungen Fehlermeldungen in den Log-Dateien einzusehen sind, gibt es für den Betreiber der Anwendung i. d. R. keine Möglichkeit die notwendigen Informationen über den Nutzer und seine Umgebung abzurufen. Noch wichtiger ist, dass er mitbekommt, wenn ein Nutzer ein Problem bei der Bedienung der Anwendung hat. Ohne eine aktive Benachrichtigung durch den Nutzer, sowie detaillierte Informationen, ist es dem Betreiber nicht möglich, Kenntnis über das Problem zu erlangen, geschweige denn dieses nachzustellen.

Dies stellt ein Kernproblem von  Webanwendungen dar \cite{ClientSideMonitoringOfDistributedSystems}. Im Rahmen der Arbeit soll daher ein Proof-of-Concept konzipiert und umgesetzt werden, welcher dieses Kernproblem am Beispiel einer Demoanwendung löst.

%\begin{itemize}
%	\item 	Was soll mit der Arbeit erreicht werden? Welche Ziele werden angestrebt?
%			Möglichst kurz und präzise geplante Ergebnisse umreißen. Daran werden
%			Ihre Resultate am Ende gemessen!
%\end{itemize}
\section{Zielsetzung}

Das grundlegende Ziel dieser Arbeit soll es sein, den Betreibern einer JavaScript-basierten Webanwendung die Möglichkeit zu geben das Verhalten ihrer Applikation und die Interaktionen von Nutzern. Diese Nachvollziehbarkeit soll insbesondere bei Fehlerfällen u. Ä. gewährleistet sein, aber auch in sonstigen Fällen soll eine Nachvollziehbarkeit möglich sein. Eine vollständige Überwachung der Applikation und des Nutzers (wie bspw. bei Werbe-Tracking) sind jedoch nicht vorgesehen. Daraus ergibt sich die Forschungsfrage:

\begin{quotation}
	Wie sieht ein Ansatz aus, um bei clientseitigen JavaScript-basierten Webanwendung den Betreibern eine Nachvollziehbarkeit zu gewährleisten?
\end{quotation}

Vom Leser wird eine Grundkenntnis der Informatik in Theorie oder Praxis erwartet, aber es sollen keine detaillierten Erfahrungen in der Webentwicklung vom Leser erwartet werden. Daher sind das Projektumfeld und seine besonderen Eigenschaften zu erläutern.

Die anzustrebende Lösung soll ein Proof-of-Concept sein, welches eine, zu erstellenden, Demoanwendung erweitert. Die Demoanwendung soll repräsentativ eine abgespeckte JavaScript-basierte Webanwendung darstellen, bei der die zuvor benannten Hürden zur Nachvollziehbarkeit bestehen.

Vor der eigentlichen Lösungserstellung soll jedoch die theoretische Seite beleuchtet werden, indem die Nachvollziehbarkeit sowie Methoden und Praktiken zur Erreichung dieser beschrieben werden. Es gilt aktuelle Literatur und den Stand der Technik zu erörtern, in Bezug auf die Forschungsfrage. Beim Stand der Technik sind Technologien aus Fachpraxis und Literatur näher zu betrachten und zu beschreiben.

Weiterhin gilt es zu beleuchten, wie die Auswirkungen für die Nutzer der Webanwendung sind. Wurde die Leistung der Webanwendung beeinträchtigt (erhöhte Ladezeit, erhöhte Datenlast)? Werden mehr Daten von ihm erhoben und zu welchem Zweck?

Am Ende der Ausarbeitung soll überprüft werden, ob und wie die Forschungsfrage beantwortet wurde. Auch die Übertragbarkeit der erstellten Lösung (PoC) und Ergebnisse gilt es hierbei näher zu betrachten.

\subsection{Abgrenzung}

\nomenclature[Fachbegriff]{PoC}{Proof-of-Concept}

Die Demoanwendung wird als Single-Page-Application (SPA) realisiert, denn hier bewegt sich das Projektumfeld von der Open Knowledge GmbH. Bei der Datenerhebung und -verarbeitung sind datenschutzrechtliche Aspekte nicht näher zu betrachten. Bei der Betrachtung von Technologien aus der Wirtschaft ist eine bewertende Gegenüberstellung nicht das Ziel.

\pagebreak

%\begin{itemize}
%	\item 	Wie wird vorgegangen, um das Ziel zu erreichen?
%	\item 	Warum ist die Arbeit so gegliedert, wie sie gegliedert ist?
%	\item 	Welche Aspekte werden nicht behandelt und warum?
%\end{itemize}
\section{Vorgehensweise}

\vspace{-0.5\baselineskip}

Zur Vorbereitung eines Proof-of-Concepts wird zunächst die Ausgangssituation geschildert. Speziell wird auf die Herausforderungen der Umgebung \enquote{Browser} eingegangen, besonders in Hinblick auf die Verständnisgewinnung zu Interaktionen eines Nutzers und des Verhaltens der Applikation. Des Weiteren wird die Nachvollziehbarkeit als solche formal beschrieben und was sie im Projektumfeld genau bedeutet.

Darauf aufbauend werden allgemeine Methoden vorgestellt, mit der die Betreiber und Entwickler eine bessere Nachvollziehbarkeit erreichen können. Dabei werden die Besonderheiten der Umgebung beachtet und es wird erläutert, wie diese Methoden in der Umgebung zum Einsatz kommen können. Hiernach sind Ansätze aus der Literatur und Fachpraxis zu erörtern, welche eine praktische Realisierung der zuvor vorgestellten Methoden darstellen.

Auf Basis des detaillierten Verständnisses der Problemstellung und der Methoden wird nun ein Proof-of-Concept erstellt. Ziel soll dabei sein, die Nachvollziehbarkeit einer Webanwendung zu verbessern. Der Proof-of-Concept erfolgt auf Basis einer Demoanwendung, die im Rahmen dieser Arbeit erstellt wird.

Ist ein Proof-of-Concept nun erstellt, wird analysiert, welchen Einfluss es auf die Nachvollziehbarkeit hat und ob die gewünschten Ziele erreicht wurden (vgl. Zielsetzung).

\vspace{-0.5\baselineskip}

\section{Open Knowledge GmbH}

\vspace{-0.5\baselineskip}

%{\color{red}TODO: Dieser Abschnitt muss noch überarbeitet werden}

Die Bachelorarbeit wird im Rahmen einer Werkstudententätigkeit innerhalb der Open Knowledge GmbH erstellt. Der Standortleiter des Standortes Essen, Dipl.-Inf Stephan Müller, übernimmt die Zweitbetreuung.

Die Open Knowledge GmbH ist ein branchenneutrales mittelständisches Dienstleistungsunternehmen mit dem Ziel bei der Analyse, Planung und Durchführung von Softwareprojekten zu unterstützen. Das Unternehmen wurde im Jahr 2000 in Oldenburg, dem Hauptsitz des Unternehmens, gegründet und beschäftigt heute 74 Mitarbeiter. Mitte 2017 wurde in Essen der zweite Standort eröffnet, an dem 13 Mitarbeiter angestellt sind.

Die Mitarbeiter von Open Knowledge übernehmen in Kundenprojekten Aufgaben bei der Analyse über die Projektziele und der aktuellen Ausgangssituationen, der Konzeption der geplanten Software, sowie der anschließenden Implementierung. Die erstellten Softwarelösungen stellen Individuallösungen dar und werden den Bedürfnissen der einzelnen Kunden entsprechend konzipiert und implementiert. Technisch liegt die Spezialisierung bei der Mobile- und bei der Java Enterprise Entwicklung, bei der stets moderne Technologien und Konzepte verwendet werden. Die Geschäftsführer als auch diverse Mitarbeiter der Open Knowledge GmbH sind als Redner auf Fachmessen wie der Javaland oder als Autoren in Fachzeitschriften wie dem Java Magazin vertreten.

\pagebreak

	
	\chapter{Ausgangssituation}
	\section{Browserumgebung}

\textit{Hier soll eine Beschreibung der Umgebung erstellt werden, mit Nennung der speziellen Eigenschaften (Keywords: Sandbox, CORS, Logging)}

Als JavaScript 1997 veröffentlicht und in den NetScape Navigator integriert wurde, gab es die berechtigen Bedenken, dass das Öffnen einer Webseite dem Betreiber erlaubt Code auf dem System eines Nutzers auszuführen. Damit dies nicht eintritt, wurde der JavaScript Ausführungskontext in eine virtuelle Umgebung  integriert - einer sogenannten Sandbox. \cite{LearningJavaScript}

Zusätzlich hierzu gibt es weitere Einschränkungen, welche definieren was für Daten innerhalb der JavaScript Umgebung abgerufen werden dürfen und mit welchen Diensten kommuniziert werden darf \cite{LearningJavaScript}. Zwei wichtige dieser Einschränkungen, die eine Webapplikationen nutzen kann, werden folgend erklärt.

% Erster Eindruck zu 2.1 ist schon ganz gut. Ich würde noch die Themen "Sandbox", "Logdaten" und "Fernzugriff (nicht möglich)" ergänzen.
% Beim Thema "Fernzugriff" kann Christian Wansart dir ein bisschen was zum S&B Projekt erzählen. Dort haben wir an einer Kassensoftware gearbeitet, wo die Herausforderung existierte, dass man sich bei Bedarf auf der Kasse aufschalten kann.

\subsection{Content-Security-Policy}

\nomenclature[Fachbegriff]{CSP}{Content-Security-Policy}
\nomenclature[Fachbegriff]{XSS}{cross-site scripting}
\nomenclature[Fachbegriff]{CDN}{Content Delivery Network}

Eine Content-Security-Policy definiert, welche Ressourcen (also Bilder, Skripte, etc.) von der Webapplikation aus geladen werden dürfen und über welches Protokoll. Dies dient dem Schutz vor cross-site scripting, indem eine Webapplikation beschränken kann, welche Aufrufe von ihr aus getätigt werden dürfen. Wird beispielsweise eine JavaScript-Bibliothek aus einer externen Quelle (wie z.B. ein CDN) benutzt und die Bibliothek wird zu späterer Zeit böswillig ausgetauscht oder modifiziert, kann durch die Härtungsmaßnahme einer Content-Security-Policy gewährleistet werden, dass keine Daten an unbekannte Server geschickt werden dürfen \cite{MDNContentSecurityPolicy}.

\subsection{Cross-Origin Resource Sharing (CORS)}

\subsection{Sandbox}

Wie zuvor angerissen, wird die JavaScript Umgebung in Browsern in einer virtuellen Umgebung bereitgestellt - einer sogenannten Sandbox. Speziell bei JavaScript heißt dies, dass unter anderem kein Zugriff auf das Dateisystem erfolgen kann. Auch Zugriff auf native Bibliotheken oder Ausführung von nativem Code ist nicht möglich \cite{TheSpyInTheSandbox}.

\subsection{Logdaten}

Ähnlich wie andere Umgebungen gibt es eine standardisierte Log- bzw. Konsolenausgabe für die JavaScript Umgebung \cite{MDNConsole}. Diese Ausgabe ist aber für den Standard-Benutzer eher unbekannt und der Zugriff darauf und die Funktionen dessen können je nach Browser variieren. Weiterhin sind die Daten, welche in die Ausgabe geschrieben wurden, aus der JavaScript Umgebung nicht lesbar.

Die Daten selber zu Handhaben wäre eine Möglichkeit, Zugriff zu ihnen zu haben - aber hier besteht weiterhin das Problem, wie die Daten an die Stakeholder gelangen. Weiterhin muss dann mit dem Problem des limitierten Speichers der JavaScript Umgebung umgegangen werden.

\subsection{Fernzugriff}

Ein weiterer Punkt, der die Umgebung ``Browser`` von anderen unterscheidet, ist dass die Stakeholder sich normalerweise nicht auf die Systeme der Nutzer schalten können. Bei Expertenanwendungen ginge dies vielleicht, aber wenn eine Webapplikation für den offenen Markt geschaffen ist, sind die Nutzer zahlreich und unbekannt.

\section{JavaScript-basierte Webapplikationen}
\textit{Hier soll beschrieben werden, was JavaScript-basierte Webapplikationen sind.}

Diese Ausarbeitung konzentriert sich, wie im Titel beschrieben, auf JavaScript-basierte Webapplikationen. Weiterhin wird sich auf sogenannte Single-Page-Applications (SPAs) konzentriert, welche eine Submenge der JavaScript-basierten Webapplikationen darstellen. Um allen Lesern eine gleiche Grundkenntnis zu ermöglichen, werden diese Konzepte nun kurz vorgestellt.

\textit{..Erklärung zu JavaScript-basierten Webapplikationen..}

\textit{..Erklärung zu SPAs..}

Bekannte Frameworks, um SPAs zu Erstellen, sind beispielsweise Angular \cite{AngularHomepage}, React \cite{ReactHomepage} oder Vue.js \cite{VueJSHomepage}.
	
\newpage

% \section{Instandhaltung und Support}

\section{Softwarebetrieb}

Diese Arbeit konzentriert sich auf Software, die sich in der Betriebsphase befindet. Gängige Software-Entwicklungszyklen und ihre Definition dieser Phase werden folgend beschrieben.

\subsection{Klassisches Vorgehen}

	\begin{wrapfigure}[14]{r}{0.45\linewidth}
		\centering
		\vspace{-\baselineskip}
		\includegraphics[width=\linewidth]{img/02_theorie/software-life-cycle.png}
		\caption{Lebenszyklus einer Software}
		\label{fig:software-development-life-cycle}
		\source{Eigene Darstellung von \cite{ASimulationModelWaterfallSoftware}}
	\end{wrapfigure}
	
	In vielen Modellen über den Lebenszyklus einer Software wird die Phase während der Betreibung oftmals \enquote{Maintenance} genannt \cite{ManagingTheComplexityOfWebSystemsDevelopment} \cite{ASimulationModelWaterfallSoftware}, in der Instandhaltung und Support den Alltag bestimmen. Sie ist nach Zelkowitz \etal \cite{PrinciplesOfSoftwareEngineeringAndDesign} für rund zwei Drittel der Entwicklungskosten verantwortlich, begründet durch exponentielle Steigung \cite{ExtremeProgrammingExplained}.
	
	Das Wasserfallmodell \cite{ASimulationModelWaterfallSoftware} sowie das V-Modell XT \cite{WaterfallVsVModelVsAgile} sehen vor, dass in dieser Phase die Software funktionstüchtig gehalten wird und dass die Anforderungen an die Software erfüllt sind. Bei nicht-erfüllten Anforderungen oder Fehlern, werden diese behoben. Jedoch ein kontinuierlicher Verbesserungsprozess ist in dieser Phase nicht vorgesehen.
	
%	Es werden immer bessere Methoden entwickelt, um Probleme in Software - oder auch Bugs - zu verringern. Jedoch erhöht sich zugleich die Komplexität von Software, was zur Ursache hat, dass es mehr Nährboden für Bugs gibt \cite{TrackingDownSoftwareBugsAnomalyDetection}. De-facto sind Bugs ein unvermeidbarer Bestandteil einer Software und müssen daher erwartet und gehandhabt werden \cite{TheMythicalManMonth}.
%	
%	Wenn nun ein Bug auffällt, sei es durch einen Nutzer oder auch zufällig einem Stakeholder, muss entschieden werden, ob dieser zu beheben ist. Wenn eine Behebung angestrebt wird, benötigt der Stakeholder meistens Rahmeninformationen \cite{WhatMakesAGoodBugReport} um den Bug ggf. zu reproduzieren und die Situation nachzuvollziehen. Desto mehr Verständnis der Stakeholder über das Problem erhält, desto schneller und präziser kann er die Ursache aufdecken. Die Ermöglichung der schnellen Verständnis über ein Problem, wird in dieser Arbeit \textbf{Nachvollziehbarkeit} genannt.

\subsection{Agiles Vorgehen}

	\begin{wrapfigure}[11]{l}{0.45\linewidth}
		\centering
		\vspace{-\baselineskip}
		\includegraphics[width=\linewidth]{img/02_theorie/devops-life-cycle.png}
		\caption{DevOps Toolchain}
		\label{fig:devops-life-cycle}
		\source{Wikimedia Commons \cite{DevOpsLifeCycle}}
	\end{wrapfigure}
	
	Bei agilen Ansätzen wird der Betrieb meist nicht abgegrenzt von der normalen Entwicklung. In dieser Phase werden weiterhin Anforderungen erhoben und diese Stück für Stück umgesetzt \cite{WaterfallVsVModelVsAgile}. Vorteilhaft dabei ist, dass auf neue Wünsche oder Auffälligkeiten sehr einfach reagiert werden kann. Es wird eine kontinuierliche Verbesserung angestrebt.
	
	Um den kontinuierlichen Entwicklungs- und Deploymentprozess reibungslos ablaufen zu lassen, werden Ansätze wie DevOps \cite{DevOps} verfolgt (vgl. \autoref{fig:devops-life-cycle}).

\newpage

\section{Nachvollziehbarkeit}

	\textit{Hier soll die Nachvollziehbarkeit allgemein beschrieben werden und warum sie erstrebenswert ist.}

	Sie beschäftigt sich mit der Informationserfassung und -aufbereitung, um das Verhalten eines Systems und die Interaktionen der Nutzer für die Stakeholder verständlich zu machen. Sie ist getrennt von der Anstrebung einer Reproduzierbarkeit nach der wissenschaftlichen Methode anzusehen.
	
	{\color{red}\textit{\lipsum[1]}}
	
	Tritt ein Problem bei einem Nutzer auf, aber die Stakeholder erhalten nicht ausreichende Informationen, so kann der Bug ignoriert werden oder in Vergessenheit geraten. Dies geschah im Jahr 2013, als Khalil Shreateh eine Sicherheitslücke bei Facebook fand und bei Facebooks Bug-Bounty-Projekt Whitehat meldete \cite{FacebookBugBounyHunt}. Sein Fehlerreport wurde aufgrund mangelnder Informationen abgelehnt:
	
	\begin{quotation}
	Unfortunately your report [...] did not have enough technical information for us to take action  on  it. We  cannot  respond  to  reports  which  do  not contain enough detail to allow us to reproduce an issue.
	\end{quotation}

\subsection{Nachvollziehbarkeit bei Webapplikationen}

	\textit{Hier sollen die Besonderen Hürden bei Webapplikationen hervorgehoben werden (indirekte Kommunikation, keinen Zugriff auf Logs, etc.)}

	\textit{Weiterhin soll beschrieben werden, }
	
	\chapter{Methoden und Praktiken}
	% \chapter{Methoden und Praktiken}

\textit{In diesem Kapitel soll beschrieben werden, wie eine Nachvollziehbarkeit in Webapplikationen erreicht werden kann. Spezielle Methoden und Praktiken sollen vorgestellt und beleuchtet werden.}
% \textit{Hier könnte unter anderem \textbf{OpenTelemetry} betrachtet werden.}

\section{Methoden}

\subsection{Logging}

\textit{Folgende Fragen sollen zur Methode beantwortet werden}
\begin{enumerate}
	\item \textit{Gibt es Besonderheiten zu Logging in anderen Projekten (Backend vs. Frontend)?}
	\item \textit{Wie können Logs an einen auswertenden Stakeholder gelangen??}
	\item \textit{Welches Verhalten kann hiermit aufgedeckt/nachvollziehbar gemacht werden?}
\end{enumerate}

%\subsection{Monitoring}
%
%\textit{Folgende Fragen sollen zur Methode beantwortet werden}
%\begin{enumerate}
%	\item \textit{Welche Anwendungseigenschaften sind zu monitoren?}
%	\item \textit{Welches Verhalten kann hiermit aufgedeckt/nachvollziehbar gemacht werden?}
%\end{enumerate}

\subsection{Metriken}

\textit{Folgende Fragen sollen zur Methode beantwortet werden}
\begin{enumerate}
	\item \textit{Welche Metriken können definiert?}
	\item \textit{Wie können Metriken definiert werden?}
	\item \textit{Welches Verhalten kann hiermit aufgedeckt/nachvollziehbar gemacht werden?}
\end{enumerate}

\subsection{Tracing}

\textit{Folgende Fragen sollen zur Methode beantwortet werden}
\begin{enumerate}
	\item \textit{Welche Nutzerinteraktionen sind zu tracen?}
	\item \textit{Welches Verhalten kann hiermit aufgedeckt/nachvollziehbar gemacht werden?}
\end{enumerate}

\subsection{Fehlerberichte}

\textit{Folgende Fragen sollen zur Methode beantwortet werden}
\begin{enumerate}
	\item \textit{Was genau sind Fehlerberichte (=Bug-Reports) }
	\item \textit{Welches Verhalten kann hiermit aufgedeckt/nachvollziehbar gemacht werden?}
\end{enumerate}

\section{Werkzeuge und Technologien}

%\textit{Basierend auf dem Grundwissen über die Methoden und Praktiken, soll nun der Stand der Technik erörtert werden. Hierbei sollen Werkzeuge und Technologien und ihre Ansätze hervorgehoben werden und mit Hilfe welcher Methoden sie welches Ziel erreichen.}
%
%\textit{Wie in der Zielsetzung definiert sollen hier zwei bis drei Technologien näher vorgestellt werden.}
%
%\textit{Weiterhin könnte beleuchtet werden, wie ähnliche Herausforderungen bei anderen „Fat-Client“-Lösungen (also nicht SPAs) angegangen werden, und kann man hier vielleicht etwas lernen oder übertragen (und wenn nicht, warum nicht)?}

In der Fachpraxis haben sich einige Technologien über die Jahre entwickelt und etabliert, die eine verbesserte Nachvollziehbarkeit als Ziel haben. Es lassen sich zudem verschiedene Funktionskategorien festlegen, auf die sich die jeweiligen Technologien konzentrieren. Die einzelnen Technologien lassen sich jedoch nicht strikt kategorisieren und weisen unterschiedliche Funktionsumfänge für dieselben Kategorien auf. Deshalb werden die Kategorien folgend beschrieben, aber bei der Vorstellung der Technologien erfolgt keine Kategorisierung oder direkte Gegenüberstellung.

\subsection{Kategorien}

\subsubsection{System Monitoring}

System Monitoring beschäftigt sich mit der Überwachung der notwendigen Systeme und Dienste in Bezug auf Hardware- und Softwareressourcen. Es handelt sich hierbei um ein projektunabhängiges Monitoring, welches sicherstellen soll, dass die Infrastruktur funktionstüchtig bleibt.

\subsubsection{Log Management}

Log Management umfasst die Erfassung, Speicherung, Verarbeitung und Analyse von Logdaten von Anwendungen. Weiterhin bieten Werkzeuge hierbei oftmals Such- und Meldefunktionen.

\subsubsection{Application Performance Monitoring (APM)}

Beim Application Performance Monitoring werden Daten innerhalb von Applikationen gesammelt, die Rückschlüsse auf die Perfomanz von bspw. Transaktionen geben sollen \cite{StudyingTheEffectivenessOfAPMTools}. Mit diesen Daten können dann Regressionen der Performanz, in Aspekten wie Zeitaufwand oder Ressourcennutzung, festgestellt werden.

\subsubsection{Real User Monitoring (RUM)}

Real User Monitoring beschäftigt sich mit dem Mitschneiden von allen Nutzerinteraktionen mit bspw. einer Webapplikation. Hiermit lässt sich nachvollziehen, wie ein Nutzer die Anwendung verwednet. RUM kann dazu verwendet werden um Herauszufinden, wie ein Nutzer zu einem Zustand gelangt ist. Aber es können auch ineffiziente Klickpfade hierdurch festgestellt werden und darauf basierend UX Verbesserungen vorgenommen werden.

\subsubsection{Synthetic Monitoring}

Beim Synthetic Monitoring werden Endnutzerszenarien simuliert, um zu prüfen und sicherzustellen, dass diese Szenarien wie gewünscht ablaufen. Hierbei kann auf Aspekte wie Funktionalität, Verfügbarkeit und auch verstrichene Zeit kontrolliert werden.

\subsubsection{Tracing}

Tracing beschäftigt sich mit dem Aufzeichnen von Kommunikationsflüssen. Hierbei können einerseits die Kommunikationsflüsse innerhalb einer Applikation oder innerhalb eines Systems erfasst werden, aber auch andererseits die Kommunikationsflüsse bei verteilten Systemen erfasst werden, um diese meist komplexen Zusammenhänge zu veranschaulichen. Ein herstellerunabhängiger Standard, der sich aus diesem Gebiet entwickelt hat, ist OpenTracing \cite{OpenTracing}.

\subsubsection{Error/Crash Monitoring}

Das Error Monitoring konzentriert sich auf das Erfassen und Melden von Fehlern. Es werden oftmals neben dem eigentlichen Fehler auch Aspekte vom RUM und Logging gemeldet, um mehr Kontextinformationen zu liefern.

\subsubsection{Session Replay}

Session Replay bedeuted, dass eine Sitzung eines Nutzers nachgestellt wird, so als ob sie gerade passiert. Hierbei können einzelne Aspekte der Anwendung nachgestellt werden, bspw. der Kommunikationsablauf, bei dem die tatsächliche zeitliche Abfolge von Kommunikationen nachvollzogen werden können. Desto mehr Aspekte nachgestellt werden, desto realitätsnaher ist die Simulation und entsprechend hilfreich ist sie beim Nachvollziehen.

\subsection{Fachpraxis}

\subsubsection{OpenTelemetry}

\begin{wrapfigure}[20]{r}{0.45\textwidth}
\centering
\includegraphics[width=\linewidth]{img/03_methoden/otel_unified-collection.png}
\caption{Schaubild einer Lösung auf Basis von OTel \cite{OpenTelemetryUnifiedCollection}}
\label{fig:otel-unified-collection}
\end{wrapfigure}

OpenTelemetry (OTel) \cite{OpenTelemetry} ist ein sich derzeit entwickelnder herstellerunabhängiger Standard, um Tracing-, Metrik- und Logdaten\footnotemark zu erfassen, verarbeiten, analysieren und zu visualisieren. Der Standard fasst die beiden Standards OpenTracing und OpenCensus \cite{OpenCensus} zusammen und hat sich als Ziel gesetzt diese zu erweitern. Hinter dem Standard stehen u. A. die Cloud Native Computing Foundation (CNCF), Google, Microsoft, und führende Hersteller in Tracing und Monitoring-Lösungen. Ein erster Release ist für Ende 2020/Anfang 2021 geplant. Ziel ist es, dass Entwickler Tools und Werkzeuge benutzen können, ohne jedesmal eine hochspezifische Anbindung schreiben und konfigurieren zu müssen. Stattdessen definiert der Standard Komponenten, die spezielle Aufgabengebiete haben und mit einer allgemeinen API angesprochen werden können. Die technische Infrastruktur einer Lösung basierend auf OTel kann in \autoref{fig:otel-unified-collection} betrachtet werden. Im groben definiert OTel folgende Komponenten: API, SDK, Exporter, Collector, Backend (vgl. \autoref{fig:otel-components}).

\begin{figure}[H]
	\centering
	\includegraphics[width=0.75\linewidth]{img/03_methoden/dynatrace_otel-components.png}
	\caption{OTel Komponenten \cite{DynatraceOTelComponents}}
	\label{fig:otel-components}
\end{figure}

\nomenclature[Fachbegriff]{OTel}{OpenTelemetry}
\nomenclature[Fachbegriff]{CNCF}{Cloud Native Computing Foundation}
\footnotetext{Logging wird derzeit noch nicht unterstützt, es wird jedoch daran gearbeitet \cite{OpenTelemetryLoggingSpecification}}

\subsubsection{New Relic}

New Relic \cite{NewRelic} ist ein Dienst der gleichnamigen Firma, welcher Betreiber von Softwareprojekten dabei unterstützt das Verhalten ihrer Anwendungen zu überwachen. Der Dienst konzentriert sich auf System Monitoring, APM und RUM und erfasst die notwendigen Daten mit proprietären Lösungen. Neben den Kernfunktionalitäten unterstützt New Relic auch Log Management, Synthetic Monitoring, Tracing und Error Monitoring.

New Relic gibt an, dass Daten nach dem OpenTelemetry Standard selber erfasst und an New Relic gesendet werden können, ohne eine proprietäre Software \cite{NewRelicAnnoundOTelBetaSupport}. Leider ist dieses Feature in der Testversion, die zum evaluieren benutzt wurde, nicht enthalten und kann somit nicht bestätigt werden. Es sind jedoch offizielle und quelloffen veröffentlichte Exporter für New Relic verfügbar für .NET, Python und Java \cite{OpenTelemetryRegistry}.

\subsubsection{Dynatrace}

Dynatrace \cite{Dynatrace} ist ein Dienst der gleichnamigen Firma, welcher Betreiber von Softwareprojekten dabei unterstützt das Verhalten ihrer Anwendungen zu überwachen. Der Dienst konzentriert sich auf System Monitoring, APM und RUM und erfasst die notwendigen Daten mit proprietären Lösungen, dem OneAgent. Ganz ähnlich wie New Relic unterstützt Dynatrace neben den Kernfunktionalitäten auch Log Management, Synthetic Monitoring, Tracing und Error Monitoring.

Dynatrace ist dem OpenTelemetry Team beigetreten und hat angegeben, an der Weiterentwicklung mitzuhelfen \cite{DynatraceJoinOTelProject}. Eine Integration des Dienstes Dynatrace ins Ökosystem von OTel gibt es jedoch noch nicht.

\subsubsection{Sentry}

{\color{red}\textit{Beschreibung zu grundlegenden Eigenschaften von Sentry}}

\subsubsection{Splunk}

{\color{red}\textit{Beschreibung zu grundlegenden Eigenschaften von Splunk}}

\subsubsection{LogRocket}

{\color{red}\textit{Beschreibung zu grundlegenden Eigenschaften von LogRocket}}

\subsubsection{Honeycomb}

{\color{red}\textit{Beschreibung zu grundlegenden Eigenschaften von Honeycomb}}

\subsubsection{Jaeger}

Jaeger wurde 2017 als ein Projekt der CNCF gestartet \cite{Jaeger}. Es ist ein System für verteiltes Tracing von der Datensammlung bis hin zur Visualisierung, es unterstützt und implementiert den Standard OpenTracing.  Eine Unterstützung des OpenTelemetry Standards ist derzeit im Gange. Weiterhin kann Jaeger dazu benutzt werden, Metriken nach Prometheus \cite{Prometheus} zu exportieren, einem weiteren CNCF Projekt zur Speicherung und Visualisierung von Daten.

\subsubsection{Weiteres}

Bei meiner Recherche und Evaluierung wurden nicht alle auf dem Markt verfügbaren Werkzeuge und Technologien tiefergehend betrachtet. Deshalb werden weitere Funde, die nicht betrachtet wurden, hier kurz notiert:

\begin{itemize}
	\item APM \& RUM: AppDynamics, DataDog
	\item Error Monitoring: Airbrake, Instabug, Rollbar, Bugsnag, TrackJS
	\item Tracing: Google Cloud Trace, Zipkin
\end{itemize}

Auch diese Auflistung stellt nicht die komplette Bandbreite an Werkzeugen und Technologien dar und eine vorherige Betrachtung ist nicht als direkte Empfehlung zu verstehen.

\subsection{Literatur}
	
	\chapter{Erstellung Proof-of-Concept}
	% \chapter{Beispielhafte Integration}
	
\section{Anforderungen}
%\section{Anforderungen}
%\textit{Hier soll definiert werden, welche Kriterien der PoC erfüllen soll und wenn möglich sollen zusätzlich Möglichkeiten zur Überprüfung dieser Kriterien definiert werden.}
	
\subsection{Definitionen}
	
Um die Anforderungen systematisch zu kategorisieren, werden folgend zwei Modelle vorgestellt, mit denen die Anforderungen kategorisiert werden.

Beim ersten handelt es sich um das Kano-Modell \cite{KanoModell} der Kundenzufriedenheit kategorisiert (vgl. \autoref{tab:merkmale-nach-dem-kano-modell}).
	
\begin{table}[H]
\begin{tabular}{ |p{1.2cm}|p{2.75cm}|p{9.55cm}| }
	\hline
	Kürzel & Titel & Beschreibung \\
	\hline
	\textbf{B} & Basis\-merkmal & Merkmale, die als selbstverständlich angesehen werden. Eine Erfüllung erhöht kaum die Zufriedenheit, jedoch eine Nichterfüllung führt zu starker Unzufriedenheit \\
	\hline
	\textbf{L} & Leistungs\-merkmal & Merkmale, die der Kunde erwartet und bei nicht Vorhandensein in Unzufriedenheit äußert. Ein Vorhandensein erzeugt Zufriedenheit, beim übertreffen umso mehr. \\
	\hline
	\textbf{S} & Begeisterungs\-merkmal & Merkmale, mit der man sich von der Konkurrenz herabsetzen kann und die den Nutzenfaktor steigern. Sind sie vorhanden, steigern sie die Zufriedenheit merklich. \\
	\hline
	\textbf{U} & Unerhebliches Merkmal & Für den Kunden belanglos, ob vorhanden oder nicht \\
	\hline
	\textbf{R} & Rückweisungs\-merkmal & Diese Merkmale führen bei Vorhandensein zu Unzufriedenheit, sind jedoch beim Fehlen unerheblich \\
	\hline
\end{tabular}
 \captionsetup{justification=centering}
  \caption{Merkmale nach dem Kano-Modell der Kundenzufriedenheit}
   \label{tab:merkmale-nach-dem-kano-modell}
\end{table}

Neben der Unterscheidung nach dem Kano-Modell werden die Anforderungen in funktionale und nicht-funktionale Anforderungen \cite{FunktionaleUndNichtFunktionaleAnforderungen} aufgeteilt (vgl. \autoref{tab:kategorien-der-anforderungsliste}).

\begin{table}[H]
\begin{tabular}{ |p{1.25cm}|p{3cm}|p{9.25cm}| }
	\hline
	Kürzel & Titel & Beschreibung \\
	\hline
	f & funktional & Beschreiben Anforderungen, welche ein Produkt ausmachen und von anderen differenzieren (``Was soll das Produkt können?{``}). Sie sind sehr spezifisch für das jeweilige Produkt. Ein Beispiel: Das Frontend fragt Daten für X vom Partnersystem\#1 über eine SOAP-API ab, etc.\\
	\hline
	nf & nicht-funktional & Beschreiben Leistungs- sowie Qualitätsanforderungen und Randbedingungen (``Wie soll das Produkt sich verhalten?{``}). Sie sind meist unspezifisch und in gleicher Form auch in unterschiedlichsten Produkten vorzufinden. Beispiele sind: Benutzbarkeit, Verfügbarkeit, Antwortzeit, etc. Zur Überprüfung sind oftmals messbare, vergleichbare und reproduzierbare Definitionen notwendig. \\
	\hline
\end{tabular}
 \captionsetup{justification=centering}
  \caption{Kategorien der Anforderungen}
   \label{tab:kategorien-der-anforderungsliste}
\end{table}
	
\subsection{Anforderungsanalyse}

\textit{Wie werden die Anforderungen erhoben?}

Die primäre Quelle von Anforderungen, stellen die Stakeholder selbst dar. Die Stakeholder dieser Arbeit sind Christian Wansart und Stephan Müller von Open Knowledge. Sie betreuen die Arbeit und haben ein starkes Interesse, dass ein sinnvolles und übertragbares Ergebnis aus der Arbeit entspringt, um es z.B. bei Kunden anwenden zu können.

\textit{{\color{red}Hier: Erstellung von einer Klassifizierung der Anforderungsquellen}}
	
\subsection{Anforderungsliste}

\textit{Was sind die Anforderungen?}

\begin{table}[H]
\begin{tabular}{ |p{1.25cm}|p{5.5cm}|p{2.25cm}|p{2.1cm}|p{1.25cm}| }
\hline
Id           & Name         & Kano-Modell  & Funktionsart & Quelle       \\
\textit{1234} & \textit{Dummy} & \textit{Dummy} & \textit{nf} & \textit{S} \\
\hline
\multicolumn{5}{|l|}{\textit{Hier soll eine Beschreibung hin.}} \\
\hline
\end{tabular}
\end{table}
	
\section{Vorstellung der Demoanwendung}
%\section{Anforderungen}
%\textit{In diesem Abschnitt soll die Demoanwendung vorgestellt werden, anhand dessen das Proof-of-Concept erstellt wird. Damit das Proof-of-Concept erstellt werden kann, muss die Demoanwendung die zuvor beschriebenen Probleme aufweisen, hierbei sollen die Probleme möglichst realitätsnah sein und nicht frei erfunden.}

Wie in \autoref{anf:1020} beschrieben, ist eine Demoanwendung zu erstellen, auf Basis dessen das Konzept anzuwenden ist und somit praktisch umgesetzt werden kann. Dieser Abschnitt beschäftigt sich mit der Vorstellung der Demoanwendung und der repräsentativen Aufgabe, die diese übernimmt.

In der Motivation wurde ein konkretes Problem eines Kunden der Open Knowledge genannt. Damit die Demoanwendung realistisch eine , wird sie in Grundzügen die Webanwendung des Direktversicherers nachahmen. Bei der Webanwendung handelt es sich um eine mit Angular erstellte SPA, die den Nutzer verschiedene teils dynamische Formulare ausfüllen lässt und am Ende diese Daten an einen Dienst sendet und ein Ergebnis erhält, welches dargestellt wird. Während der Eingabe der Formulare werden einzelne Werte gegen Dienste validiert. Um die gewünschte Demoanwendung zu definieren, wird im folgenden Abschnitt das Verhalten festgelegt.

\subsection{Verhaltensdefinition}

Mit den beiden Stakeholdern, also Christian Wansart und Stephan Müller, die beide am Projekt für den Kunden involviert sind, wurde diese Verhaltensdefinition erstellt. Diesen Ansatz der Definition der Software anhand des Verhaltens nennt man Behavior Driven Development (BDD). Um die BDD-Definition festzuhalten wurde sie in der gängigen Gherkin \cite{Gherkin} Syntax geschrieben. Die Syntax ist natürlich zu lesen, und folgend werden alle gewünschen Features der Demoanwendung in der Gherkin-Syntax beschrieben.

\lstinputlisting[
  language = gherkin,
   caption = Demoanwendung: Gherkin Definition zum Feature \enquote{Warenkorb},
captionpos = b,
     label = lst:demoanwendung-gherkin-warenkorb
]{content/04_erstellung-poc/warenkorb-gherkin/1-warenkorb.feature}

\lstinputlisting[
  language = gherkin,
   caption = Demoanwendung: Gherkin Definition zum Feature \enquote{Rechnungsadresse},
captionpos = b,
     label = lst:demoanwendung-gherkin-rechnungsadresse
]{content/04_erstellung-poc/warenkorb-gherkin/2-rechnungsadresse.feature}

\lstinputlisting[
  language = gherkin,
   caption = Demoanwendung: Gherkin Definition zum Feature \enquote{Lieferadresse},
captionpos = b,
     label = lst:demoanwendung-gherkin-lieferadresse
]{content/04_erstellung-poc/warenkorb-gherkin/3-lieferadresse.feature}

\lstinputlisting[
  language = gherkin,
   caption = Demoanwendung: Gherkin Definition zum Feature \enquote{Zahlungsdaten},
captionpos = b,
     label = lst:demoanwendung-gherkin-zahlungsdaten
]{content/04_erstellung-poc/warenkorb-gherkin/4-zahlungsdaten.feature}

\lstinputlisting[
  language = gherkin,
   caption = Demoanwendung: Gherkin Definition zum Feature \enquote{Bestellung abschließen},
captionpos = b,
     label = lst:demoanwendung-gherkin-rechnungsadresse
]{content/04_erstellung-poc/warenkorb-gherkin/5-bestellung_abschließen.feature}

\subsection{Frontend}

\begin{figure}[H]
	\centering
	\includegraphics[width=0.75\linewidth]{img/04_erstellung-poc/demoanwendung_vorstellung_01-warenkorb.png}
	\caption{Demoanwendung: Startseite \enquote{Warenkorb}}
	\label{fig:demoanwendung_vorstellung_01-warenkorb}
\end{figure}

\begin{figure}[H]
	\centering
	\includegraphics[width=0.75\linewidth]{img/04_erstellung-poc/demoanwendung_vorstellung_02-rechnungsadresse.png}
	\caption{Demoanwendung: Seite \enquote{Rechnungsadresse}}
	\label{fig:demoanwendung_vorstellung_02-rechnungsadresse}
\end{figure}

\begin{figure}[H]
	\centering
	\includegraphics[width=0.75\linewidth]{img/04_erstellung-poc/demoanwendung_vorstellung_03-lieferdaten.png}
	\caption{Demoanwendung: Seite \enquote{Lieferdaten}}
	\label{fig:demoanwendung_vorstellung_03-lieferdaten}
\end{figure}

\begin{figure}[H]
	\centering
	\includegraphics[width=0.75\linewidth]{img/04_erstellung-poc/demoanwendung_vorstellung_04-zahlungsdaten.png}
	\caption{Demoanwendung: Seite \enquote{Zahlungsdaten}}
	\label{fig:demoanwendung_vorstellung_04-zahlungsdaten}
\end{figure}

\begin{figure}[H]
	\centering
	\includegraphics[width=0.75\linewidth]{img/04_erstellung-poc/demoanwendung_vorstellung_05-bestelluebersicht.png}
	\caption{Demoanwendung: Seite \enquote{Bestellübersicht}}
	\label{fig:demoanwendung_vorstellung_05-bestelluebersicht}
\end{figure}

\begin{figure}[H]
	\centering
	\includegraphics[width=0.75\linewidth]{img/04_erstellung-poc/demoanwendung_vorstellung_06-bestellbestaetigung.png}
	\caption{Demoanwendung: Finale Seite \enquote{Bestellbestätigung}}
	\label{fig:demoanwendung_vorstellung_06-bestellbestaetigung}
\end{figure}

\subsection{Backend}

\begin{figure}[H]
	\centering
	\includegraphics[width=1.0\linewidth]{img/04_erstellung-poc/demoanwendung_k8s-deployment.png}
	\caption{Demoanwendung: Kubernetes-Architektur-Diagramm, Quelle: Eigene Darstellung}
	\label{fig:demoanwendung_k8s-deployment}
\end{figure}
	
% \newpage
	
\section{Konzept}
	
	\subsection{Datenverarbeitung}

	Auf Basis der zuvor vorgestellten Methoden und Praktiken wird nun eine sinnvolle Kombination für das Frontend konzeptioniert, die als Ziel hat, die Nachvollziehbarkeit nachhaltig zu erhöhen. Es werden die Grunddisziplinen Datenerhebung, -auswertung und -präsentation unterschieden und nacheinander beschrieben. Danach und darauf aufbauend wird eine grobe Architektur vorgestellt, die diese Ansätze in ein Gesamtbild bringt.
		
	\subsubsection{Erhebung}
	
	Wie zuvor in \autoref{sec:nachvollziehbarkeit-bei-spas} beschrieben, erhalten Betreiber und Entwickler im Normalfall nur unzureichende Information über das Anwendungsverhalten oder die getätigten Nutzerinteraktionen bei einer SPA. Aus diesem Grund sollen explizit weitere Daten erhoben werden, um die Nachvollziehbarkeit zu erhöhen.
	
	Wie aus den Erkenntnissen von FAME (\autoref{sec:fame}) und Kaiju (\autoref{sec:kaiju}) zu deuten ist, gibt es durch die Verknüpfung von verschiedenen Datenkategorien einen Mehrwert für die Verständnis von Betreibern und Entwicklern. Deshalb sollen in der Lösung die 4 Datenkategorien \enquote{Logs}, \enquote{Metriken}, \enquote{Traces} und \enquote{Fehler} erhoben und an Partnersysteme weitergeleitet werden.
	
	Neben diesen Daten sollen auch die Benutzerinteraktionen aufgezeichnet werden. Hierfür soll jedoch kein tiefer gehendes Real-User-Monitoring zur Verwendung kommen, stattdessen soll ein Session-Replay-Mechanismus eingesetzt werden. RUM wird nicht gefordert, da es für die Verständnisgewinnung der Benutzerinteraktionen weniger aussagekräftig ist als Session-Replay. Bei Session-Replay werden die Benutzerinteraktionen im Kontext dargestellt und nicht gesondert oder abstrahiert, was für eine Nachvollziehbarkeit hinderlich sein kann. Beim Session-Replay wird jedoch jedwede Ein- und Ausgaben der Webanwendung aufgezeichnet, damit dies nicht ein zu großes Datenvolumen erzeugt und um den Nutzer nicht konstant zu überwachen, sollte sie standardmäßig abgeschaltet sein und nur auf expliziten Nutzerwunsch aktiviert werden.
	
	\vspace{-0.75\baselineskip}
	\subsubsection{Auswertung}
	\vspace{-0.50\baselineskip}
	
	Die genaue Auswertung ist Teil der Implementierung und diese Disziplin sieht keine direkten Vorgaben vor. Jedoch sind die gemeldeten Daten mit Kontextinformationen anzureichern, wenn möglich. Diese umfassen bspw. Zeitstempel, User-Agent, IP, Browser.
	
	\vspace{-0.75\baselineskip}
	\subsubsection{Präsentation}
	\vspace{-0.50\baselineskip}
	
	Um ein zufriedenstellendes Ergebnis zu gewährleisten, sollte die Lösung die Daten auf folgende Weise den Betreibern und Entwicklern präsentieren:
	
	\begin{enumerate}
		\item Logdaten..
		\begin{enumerate}
			\item ..lassen sich einsehen.
			\item ..lassen sich basierend auf ihren Eigenschaften filtern.
		\end{enumerate}
		\item Fehler..
		\begin{enumerate}
			\item ..lassen sich einsehen,
			\item ..lassen sich basierend auf ihren Eigenschaften filtern,
			\item ..lassen sich gruppieren.
			\item Fehlergruppen lassen sich in Graphen visualisieren (bspw. Histogramm der Häufigkeit).
		\end{enumerate}
		\item Metriken..
		\begin{enumerate}
			\item ..lassen sich in Graphen visualisieren.
		\end{enumerate}
		\item Traces..
		\begin{enumerate}
			\item ..lassen sich einsehen,
			\item ..lassen sich basierend auf ihren Eigenschaften filtern,
			\item ..lassen sich als ein Trace-Gantt-Diagramm darstellen.
		\end{enumerate}
		\item Session-Replay-Daten
		\begin{enumerate}
			\item Mithilfe der Session-Replay-Daten soll eine videoähnliche Nachstellung einer Sitzung erstellt werden.
		\end{enumerate}
	\end{enumerate}
	
	\subsection{Architektur}
	
	Auf Basis der zuvor beschriebenen Grunddisziplinen wird nun eine beispielhafte Umsetzung dessen konzipiert. Genauer wird eine Architektur vorgeschlagen, welche auf die zuvor betrachteten Methoden und Praktiken zurückgreift, um eine verbesserter Nachvollziehbarkeit zu erreichen. Speziell wird im Folgeabschnitt zudem vorgeschlagen, welche Werkzeuge oder Technologien zum Einsatz kommen sollen und wie diese Komponenten miteinander kommunizieren.
	
	 Die genaue Erhebung der Daten ist Teil der Implementierung und wird hier nicht näher bestimmt. Jedoch ergeben sich aus der zuvor definierten Anforderungen zur Erhebung bereits Datenkategorien, welche von einem entsprechenden Partnersystem zu konsumieren und verarbeiten sind. Es wird zwar nicht auf spezielle Werkzeuge oder Technologien eingegangen, aber es lassen sich bereits Partnersysteme bestimmen, auf Basis der zuvor identifizierten Funktionsbereiche. Hierbei wurde zudem versucht möglichst viele Bereiche über die gleichen Partnersysteme abzubilden (vgl. \autoref{anf:3100}), die Machbarkeit einer solchen Verknüpfung basiert auf den Ergebnissen von \autoref{chap:methoden-und-praktiken}.
	 
	 So soll für die Verarbeitung von Fehler-, Log-, und Metrikdaten ein einzelnes Partnersystem verantwortlich sein, welches ermöglicht diese zu konsumieren, speichern, durchsuchen und diese zu visualisieren. Grund hierfür ist, dass die Daten gemeinsame Eigenschaften besitzen, die eine gemeinsame Verarbeitung erlauben. In der \autoref{fig:grobe-architektur} ist dieses System als \enquote{Log- und Monitoringplattform} vorzufinden.
	
	Um Traces zu konsumieren und den Betreibern und Entwicklern aufbereitet zu visualisieren, soll ein weiteres Partnersystem eingesetzt werden. Dieses System wurde als notwendig empfunden, da kein Werkzeug identifiziert werden konnte, welches neben Traces noch andere Datenkategorien zufriedenstellend abdecken kann. Auf Basis von OpenTelemetry könnten sich jedoch in Zukunft Technologien entwickeln, welches alle 3 Datenkategorien von OpenTelemetry unterstützt: Metriken, Traces und Logs. Es wurde sich zudem gegen eine weitreichende Monitoringplattform, wie z. B. New Relic oder Dynatrace, entschieden, denn hier wurde identifiziert, dass diese nicht ausreichend flexibel für verschiedene Projekte sind und zudem auch nicht erlauben, dass einzelne Komponenten ausgetauscht oder entfernt werden können.
	
	Es ist zudem ein drittes System notwendig, um die gewünschte Funktionalität des Session-Replays einzubinden. Session-Replay ist ein spezielles und sehr konkretes Aufgabengebiet und es konnte kein Werkzeug identifiziert werden, welches sich nicht nur auf dieses Gebiet spezialisiert.
	
\begin{figure}[H]
	\centering
	\includegraphics[width=1.00\linewidth]{img/04_erstellung-poc/konzept-simple.png}
	\caption{Grobe Architektur}
	\label{fig:grobe-architektur}
\end{figure}

%Wie in der Datenerfassung erwähnt, werden die einzelnen Datentypen unterschiedlich erhoben und besitzen somit auch andere Eigenschaften. Wie bei Big Data \cite{ZikopoulosUnderstandingBigData}, lassen sich auch hier die Eigenschaften Volume, Velocity und Variety identifizieren. Der Aspekt Volume ist weniger präsent, denn die Datenmengen sind nicht vergleichbar mit echten Big-Data-Anwendungen. Genau ist dies nicht prognostizierbar, aber in der Evaluierung des Stands der Technik, ließ sich ein Datendurchsatz von 1 MB/min feststellen - somit stellt dies im Frontend keine Herausforderung dar, jedoch in den verarbeitenden Systemen kann dies natürlich durch eine große Menge an Frontends multipliziert werden, was jedoch nicht im Fokus der Arbeit steht.
%
%Eine Variety der Daten, also Unterschiedlichkeit der Datenstruktur, ist definitiv vorhanden und entspringt den verschiedenen Funktionsgebieten. Auch innerhalb derselben Datenströme kann eine Variety identifiziert werden, denn bspw. sind Logmeldungen sehr individuell, sie folgen meist nicht streng einem Format und enthalten unterschiedliche Mengen an Informationen.
%
%Der Aspekt des Velocity ist zudem auch sehr wichtig und eine Visualisierung dessen für das vorhergehende Konzept findet sich in \autoref{fig:grobe-architektur-datendurchsatz}.
%	
%\begin{figure}[H]
%	\centering
%	\includegraphics[width=0.75\linewidth]{img/04_erstellung-poc/konzept-datendurchsatz.png}
%	\caption{Grobe Architektur mit hervorgehobenem Datendurchsatz}
%	\label{fig:grobe-architektur-datendurchsatz}
%\end{figure}

	\subsection{Technologie-Stack}
	\label{sec:technologie-stack}
	
	Auf Basis der zuvor erstellten Architektur wird sich nun für spezielle Technologien entschieden, mit der diese Architektur umgesetzt werden soll. Weiterhin wird behandelt, wie die Daten vom Frontend aus erhoben werden sollen und wie sie an die Partnersysteme gelangen.
	
	Für die \enquote{Log- und Monitoringplattform} wurde sich für Splunk entschieden, denn auf Basis der Evaluierung konnte festgestellt werden, dass Splunk die drei gewünschten Datenkategorien Logs, Metriken und Fehler zufriedenstellend unterstützt. Es wurde sich gegen New Relic und Dynatrace entschieden, da es diesen Werkzeugen an Flexibilität fehlt und sie viele Funktionen anbieten, die für die Lösung nicht notwendig sind. Eine weitere Alternative ist der Elastic Stack, welcher jedoch nicht näher evaluiert wurde. Somit wird Splunk eingesetzt, aber für eine äquivalente Lösung kann Splunk durch ein gleichwertiges Werkzeug ausgetauscht werden.
	
	Für das Partnersystem, welches sich mit den Tracedaten befasst, wurde sich für Jaeger entschieden. Die Entscheidung wurde auf der Basis getroffen, dass Jaeger einen moderner Tracingdienst darstellt, welcher zudem quelloffen entwickelt wird. Weiterhin ist in Zukunft eine Unterstützung des OpenTelemetry-Standards geplant, was durch die standardisierte Schnittstelle die Anbindung an bestehende Systeme vereinfachen wird. Des Weiteren erfüllt Jaeger alle aufgestellten Kriterien und erzeugt zudem ein Abbild der Systemarchitektur auf Basis der Traces.
	
	Um die Session-Replay-Funktionalität einzubringen wird in diesem Konzept LogRocket vorgeschlagen. LogRockets Nachstellung einer Sitzung ist nicht nur wie gefordert videoähnlich, sondern auch interaktiv. Die gesamte HTML-Struktur wird nachgestellt und kann so zu jedem Zeitpunkt begutachtet werden.
	
	Der sich daraus ergebende Technologiestack, angewandt auf die Architektur, ist in \autoref{fig:architektur-technologien} zu betrachten. Wie bei Splunk erwähnt sind auch die anderen Werkzeuge durch gleichwertige Werkzeuge ersetzbar, so können individuelle Anpassungen erfolgen und betriebliche Gegebenheiten zu berücksichtigen.
	
\begin{figure}[H]
	\centering
	\includegraphics[width=1.00\linewidth]{img/04_erstellung-poc/konzept-technologien.png}
	\caption{Architektur mit speziellen Technologien}
	\label{fig:architektur-technologien}
\end{figure}

	\subsection{Übertragbarkeit}
	
	Übertragbarkeit beschäftigt sich mit der Eigenschaft eines Ansatzes in verschiedene Situationen anwendbar zu sein. Ein Ansatz ist nicht übertragbar, wenn zu vielen Annahmen über die Situation getroffen werden.
	
	Das zuvor definierte Konzept wurde getrennt von der Demoanwendung erstellt und ist somit nicht auf dessen Eigenschaften beschränkt. Das Konzept nimmt dennoch an, dass es sich um eine Webanwendung handelt, auf das es anzuwenden ist. Weiterhin wird angenommen, dass Quellcodeänderungen vorgenommen werden können und das Partnersysteme hinzugefügt sowie angebunden werden können.
	
	Es wird jedoch nicht angenommen, dass die Partnersysteme exakt von den vorgeschlagenen Technologien realisiert werden. Vielmehr wurde die Funktionsgruppen definiert, zusammengefasst und darauf basierend eine Auswahl getroffen. Sowohl die Zusammenfassung der Funktionsgruppen als auch die Auswahl der eigentlichen Technologien sind individuell änderbar, sodass ein angepasstes aber äquivalent hilfreiches Konzept resultiert.
	
	Somit lässt sich abschließend betrachten, dass das Konzept eine akzeptable Übertragbarkeit aufweist. Eine tiefergehende Nachbetrachtung der Übertragbarkeit erfolgt in \autoref{sec:uebertragbarkeit}, im Anschluss an die Implementierung. Hierbei kann es zu Abweichungen zu dieser Bewertung der Übertragbarkeit kommen, aufgrund von Implementierungsdetails oder einem geänderten Vorgehen. Auf Basis des beleuchteten Konzeptes, wird im nächsten Abschnitt die eigentliche Umsetzung beschrieben.

\section{Implementierung}
\subsection{Backend}

Wie in \autoref{subsec:demoanwendung-backend} beschrieben, wurden die Dienste mit Eclipse MicroProfile umgesetzt. Neben den standardmäßig enthaltenen Bibliotheken, gibt es hierbei aber auch unterstützte optionale Bibliotheken, wie Implementierungen von \texttt{OpenAPI}, \texttt{OpenTracing}, \texttt{Fault Tolerance} und vieler weiterer \cite{EclipseMicroprofile}.

Um Traces von den Microservices zu sammeln, wurde die OpenTracing Implementierung sowie ein Jaeger-Client \cite{JaegerClient} zum Exportieren der Daten hinzugezogen. Mit dieser Anbindung lassen sich per Annotation (vgl. \autoref{lst:implementierung-traced-example}) alle zu tracenden Businessmethoden definieren, die dann automatisch getraced und über den Jaeger-Client an Jaeger gesendet werden. Bei jedem Microservice wurde diese Annotation dann an relevante Methoden geschrieben und der Jaeger-Client konfiguriert, was automatisch zu der Übertragung von verteilten Traces in Jaeger führte.

\lstinputlisting[
  language = java,
     style = java-eclipse,
basicstyle = {\footnotesize\fontfamily{pcr}\selectfont},
   caption = Beispielhafter Einsatz der @Traced-Annotation,
captionpos = b,
     label = lst:implementierung-traced-example
]{content/04_erstellung-poc/implementierung-code/TracedExample_OrderService.java}

\begin{wrapfigure}[8]{r}{0.45\textwidth}
\centering
\vspace{-\baselineskip}
\includegraphics[width=\linewidth]{img/04_erstellung-poc/implementierung_jaeger-trace-example.png}
\caption{Ausschnitt des Traces zu \autoref{lst:implementierung-traced-example}}
\label{fig:implementierung_jaeger-trace-example}
\end{wrapfigure}

In Jaeger erzeugt der o. g. Quellcode die in \autoref{fig:implementierung_jaeger-trace-example} zu sehenden Spans. Neben den Traces werden keine weiteren Daten von Backend-Komponenten erhoben, da das Hauptaugenmerk der Arbeit im Frontend liegt.

\subsection{Frontend}

\subsubsection{Traces und Metriken}

Das Frontend erhebt ebenso wie das Backend Traces, aber zusätzlich werden auch Metriken, Logmeldungen und Fehler erhoben und gemeldet. Traces und Metriken werden auf Basis von OpenTelemetry JavaScript Komponenten \cite{OpenTelemetryJS} erhoben. Genauer werden diese Komponenten in einem Angular Modul (siehe \autoref{lst:app-observability}) initialisiert und der restlichen Anwendung über \enquote{providers} zur Verfügung gestellt. Hierbei wird der SPA ein \texttt{Tracer} bereitgestellt, mit dem Spans aufgezeichnet werden können, ein \texttt{Meter}, welcher es erlaubt Metriken zu erstellen, und eine \texttt{requestCounter}-Metrik, welches die Aufzeichnung der Aufrufanzahl schnittstellenübergreifend erlaubt.

\lstinputlisting[
  language = JavaScript,
     style = ES6,
basicstyle = {\footnotesize\fontfamily{pcr}\selectfont},
   caption = Quellcode des Moduls \enquote{app-observability.module.ts},
captionpos = b,
     label = lst:app-observability
]{content/04_erstellung-poc/implementierung-code/app-observability.module.ts}

Im \autoref{lst:shopping-cart-datasource} ist die Benutzung des zur Verfügung gestellten \texttt{Tracers} zu sehen, hierbei wird ein Span erstellt und bei Schnittstellenaufrufen an die jeweiligen Services übergeben.

\lstinputlisting[
  language = JavaScript,
     style = ES6,
basicstyle = {\footnotesize\fontfamily{pcr}\selectfont},
   caption = Datenquelle zum Abrufen und Zusammenführen der Artikeldaten,
captionpos = b,
     label = lst:shopping-cart-datasource
]{content/04_erstellung-poc/implementierung-code/tracing_shopping-cart-datasource.ts}

Beispielhaft im Dienst zum Abrufen der Übersetzungsdaten (vgl. \autoref{lst:localization.service}) wird der übergebene Span als Elternspan benutzt. Bei dem eigentlichen HTTP-Aufruf wird zudem ein HTTP-Header \texttt{uber-trace-id} angereichert, den der dort laufende Jaeger-Client interpretiert \cite{JaegerClient} und daraus die Beziehung zu den Frontend-Spans herstellt. Zusätzlich zum Tracing wird hierbei auch die Metrik \texttt{requestCounter} inkrementiert.

\lstinputlisting[
  language = JavaScript,
     style = ES6,
basicstyle = {\footnotesize\fontfamily{pcr}\selectfont},
   caption = Service zum Abrufen der Übersetzungsdaten,
captionpos = b,
     label = lst:localization.service
]{content/04_erstellung-poc/implementierung-code/tracing_localization.service.ts}

Es wurde sich für die OpenTelemetry Implementierung für Tracing und Metriken im Frontend entschieden, da wie in \autoref{subsec:opentelemetry} beschrieben, OpenTelemetry einen vielversprechenden Standard darstellt. Weiterhin konnte keine Bibliothek identifiziert werden, die die Traces erhebt und direkt nach Jaeger sendet. Es gibt zwar einen Jaeger-Client für Node.js\footnote{Jaeger-Client für Node.js: https://github.com/jaegertracing/jaeger-client-node}, jedoch befindet sich das browserkompatible Pendant\footnote{Jaeger-Client für Browser: https://github.com/jaegertracing/jaeger-client-javascript/} seit 2017 in den Startlöchern. Weiterhin existiert ein OTel Exporter für Jaeger\footnote{OTel Jaeger Exporter: https://github.com/open-telemetry/opentelemetry-js/tree/main/packages/opentelemetry-exporter-jaeger}, welcher jedoch auch nur mit Node.js funktioniert.

Die gesammelten OTel Tracingdaten werden über einen Standard-Exporter an das \enquote{Backend4Frontend} gesendet, welcher diese dann in ein Jaeger-konformes Format umwandelt und sie dann subsequent an Jaeger übertragt. Die Metrikdaten werden jedoch bereits im Frontend konvertiert, in ein Splunk-kompatibles Logformat. Nach der Konvertierung werden die Daten an den \texttt{SplunkForwardingService} übergeben, welcher im folgenden Abschnitt näher beschrieben wird.

\subsubsection{Logging}

Das Logging im Frontend wurde über das npm \cite{npm} Paket \texttt{ngx-logger}\footnote{ngx-logger auf GitHub: https://github.com/dbfannin/ngx-logger} realisiert, welches eine speziell an Angular angepasste Logging-Lösung darstellt. Da dieses Pakete extra an Angular angepasst ist, lässt es sich ohne großen Aufwand als Modul einbinden, vgl. \autoref{lst:logging_app.module}.

\lstinputlisting[
  language = JavaScript,
     style = ES6,
basicstyle = {\footnotesize\fontfamily{pcr}\selectfont},
   caption = Ausschnitt des Hauptmoduls \texttt{app.module.ts},
captionpos = b,
     label = lst:logging_app.module
]{content/04_erstellung-poc/implementierung-code/logging_app.module.ts}

Wie in den vorherigen Codebeispielen zum Tracing zu sehen war, kann ein \texttt{NGXLogger} im Konstruktor von Komponenten und Diensten injected werden. Logmeldungen die hiermit erfasst werden, werden je nach Konfiguration und Loglevel der jeweiligen Meldung in die Browserkonsole geschrieben. Über einen \texttt{NGXLoggerMonitor} lassen sich die Logmeldungen anzapfen, wie in \autoref{lst:splunk-logging-monitor} zu sehen ist. Hierbei werden die Logmeldungen in ein Splunkformat übertragen und dann über den \texttt{SplunkForwardingService} an das \enquote{Backend4Frontend} übertragen. Eine direkte Übertragung an Splunk ist nicht möglich, da Splunk nicht mit kompatiblen CORS-Headern antwortet. Das Backend4Frontend reichert neben dem Weiterleiten auch die Meldungen mit Kontextinformationen, wie der User-IP, der Browserversion usw. an.

\lstinputlisting[
  language = JavaScript,
     style = ES6,
basicstyle = {\footnotesize\fontfamily{pcr}\selectfont},
   caption = Implementierung des \texttt{NGXLoggerMonitor}-Interfaces,
captionpos = b,
     label = lst:splunk-logging-monitor
]{content/04_erstellung-poc/implementierung-code/splunk-logging-monitor.ts}

\subsubsection{Fehler}

Die ErrorHandler-Hook von Angular übermittelt aufgetretene und unbehandelte Fehler an den \texttt{SplunkForwardingErrorHandler}. Weiterhin ist der ErrorHandler \texttt{Injectable} in andere SPA Klassen, wo er bspw. bei den Schnittstellen dazu benutzt wird, dass auch behandelte Fehler an Splunk zu übermitteln.

Wird ein Fehler gemeldet, werden zunächst die Fehlerinformationen in einen Splunkdatensatz konvertiert und dann über den zuvor behandelten \texttt{SplunkForwardingService} an Splunk weitergeleitet. Neben diesem Verhalten wird zusätzlich auch der Fehler an LogRocket übermittelt, damit dieser im Video des Session-Replays gesondert angezeigt wird.

\lstinputlisting[
  language = JavaScript,
     style = ES6,
basicstyle = {\footnotesize\fontfamily{pcr}\selectfont},
   caption = ErrorHandler zum Abfangen und Weiterleiten von aufgetretenen Fehlern,
captionpos = b,
     label = lst:splunk-forwarding-error-handler
]{content/04_erstellung-poc/implementierung-code/splunk-forwarding-error-handler.ts}

\subsubsection{Session-Replay}

LogRocket wird standardmäßig nicht aktiv, außer wenn der Nutzer explizit der Aufzeichnung zustimmt \citationneeded{}. Dies bedeutet jedoch auch, dass bis zur Zustimmung keine Sitzungsdaten aufgenommen wurden. Wenn der Nutzer zustimmt, wird, wie im \autoref{lst:session-replay_checkout.component} zu sehen ist, LogRocket initialisiert und mit identifizierenden Daten angereichert. Nach dem Warenkorbdialog und nach der Eingabe der Rechnungsadresse werden zudem weitere identifizierende Daten an LogRocket übermittelt. Die Aufnahme der Sitzung läuft größtenteils autonom, lediglich behandelte Fehler müssen LogRocket manuell übermittelt werden.

\lstinputlisting[
  language = JavaScript,
     style = ES6,
basicstyle = {\footnotesize\fontfamily{pcr}\selectfont},
   caption = Initialisierung von LogRocket in der Hauptkomponente,
captionpos = b,
     label = lst:session-replay_checkout.component
]{content/04_erstellung-poc/implementierung-code/session-replay_checkout.component.ts}


	\textit{Auf Basis des Konzeptes soll nun eine Implementierung erfolgen.}
	
	\chapter{Ergebnis}
	% \chapter{Ergebnis}

\section{Demonstration}
\label{sec:demonstration}
% \section{Demonstration}

Da die Implementierung nun erfolgt und beschrieben ist, wird nun beispielhaft der Nutzen dessen vorgestellt. Auf Basis der eingebauten Fehlerszenarien in der Demoanwendung soll der Mehrwert der erstellten Lösung evaluiert werden. Folgend wird näher beleuchtet, wie die drei Fehlerszenarien aufgedeckt werden können.

\subsection{Aufdecken des Szenarios \enquote{Keine Übersetzungen}}

In diesem Fehlerszenario sind für den Nutzer in der Webanwendung die Übersetzungsschlüssel zu sehen, statt der tatsächlichen Produktnamen (vgl. \autoref{subsec:keine-uebersetzungen}). Da es sich hierbei um einen Fallback handelt, wurde in der Webanwendung darauf verzichtet einen Fehler zu werfen. Dieser würde in Splunk hervorgehoben werden. Jedoch lassen sich in Splunk die Logdaten durchsuchen und so Sitzungen herausfinden, bei denen dieses Problem aufgetreten ist (vgl. \autoref{fig:keine-uebersetzungen_splunk-logs}). Darauf basierend konnten verwandte Logs derselben \enquote{Sitzung} inspiziert werden, jedoch konnte anhand dessen das Fehlverhalten nicht weiter nachvollzogen werden. Jedoch bietet die nun gefundene \texttt{shoppingCartId} die Möglichkeit, damit in weiteren Systemen nachzuforschen.
	
\begin{figure}[H]
	\centering
	\includegraphics[width=1.00\linewidth]{img/05_ergebnis/keine-uebersetzungen_splunk-logs.png}
	\caption{Suche nach Logs zu fehlenden Übersetzungen in Splunk}
	\label{fig:keine-uebersetzungen_splunk-logs}
\end{figure}

\begin{wrapfigure}[13]{r}{0.50\textwidth}
\centering
\includegraphics[width=\linewidth]{img/05_ergebnis/keine-uebersetzungen_jaeger_search-cropped.png}
\caption{Suchergebnisse in Jaeger zu spezieller \texttt{shoppingCartId}}
\label{fig:uebersetzungen_jaeger_search-cropped}
\end{wrapfigure}

In Jaeger lassen sich anhand der \texttt{shopping\-Cart\-Id} Traces finden (vgl. \autoref{fig:uebersetzungen_jaeger_search-cropped}). Einer dieser Traces ist mit einem Fehler markiert und entsprang der Funktion \texttt{getAndMapShoppingCart} im Frontend. Dabei handelt es sich um eine Funktion, die die Warenkorb- sowie Übersetzungsdaten abruft und diese zusammengeführt.

Bei Klick auf den Trace wird das entsprechende Trace-Gantt-Diagramm geöffnet (vgl. \autoref{fig:keine-uebersetzungen_jaeger_detail}). Bei dem Aufruf ist im Übersetzungsdienst ein Fehler aufgetreten, welcher vermutlich die fehlenden Übersetzungen verursachte. Weiterhin wurde ein Log im Span hinterlegt, welches die genaue Ursache beschreibt: \texttt{configuration is invalid}. Zudem ist auch der genaue Kubernetes-Pod (\texttt{localization-svc-003}) zu sehen, der die Abfrage durchführte. Somit sind dem problemlösenden Entwickler viele notwendige Informationen geboten, die ihm dabei helfen das Problem zu lösen.

\begin{figure}[H]
	\centering
	\includegraphics[width=1.00\linewidth]{img/05_ergebnis/keine-uebersetzungen_jaeger_detail.png}
	\caption{Auschnitt des Traces zur Übersetzungsanfrage in Jaeger}
	\label{fig:keine-uebersetzungen_jaeger_detail}
\end{figure}

\subsection{Aufdecken des Szenarios \enquote{Ungültige Adressen sind gültig}}

Das Backend bietet die Möglichkeit, dass die SPA eingegebene Adresse zuvor validiert. Bei der Rechnungs- sowie der Lieferadresse sollte dies auch der Fall sein. Jedoch kommt es dazu, dass Nutzer ungültige Adressen angeben können und dies bei der Aufgabe einer Bestellung zu einem Fehler führt (vgl. \autoref{subsec:ungueltige-adressen-sind-gueltig}). Durch die Suche in Splunk nach Logmeldungen der Angular-Komponente \texttt{Finalize\-Checkout\-Component} findet sich eine Fehlermeldung und eine entsprechende \texttt{shopping\-Cart\-Id}.

Bis auf die Fehlermeldung liefert Splunk jedoch nicht sprechende Informationen, deshalb wird erneut auf Jaeger zurückgegriffen. In Jaeger findet ein entsprechender Trace (vgl. \autoref{fig:ungueltige-adressen-sind-gueltig_jaeger-detail}), der über die Situation Aufschluss gibt. Es ist dabei zu erkennen, dass im Bestelldienst beide Adressen erneut validiert werden und dabei die Anfrage bzgl. der Lieferadresse fehlschlägt. Auf Basis dieser Informationen kann nun der Entwickler feststellen, dass in der SPA keine Validierung für die Lieferadresse durchgeführt wird.

\begin{figure}[H]
	\centering
	\includegraphics[width=1.00\linewidth]{img/05_ergebnis/ungueltige-adressen-sind-gueltig_jaeger-detail.png}
	\caption{Auschnitt des Traces zur Bestellanfrage in Jaeger}
	\label{fig:ungueltige-adressen-sind-gueltig_jaeger-detail}
\end{figure}

\subsection{Aufdecken des Szenarios \enquote{Lange Verarbeitung}}

In diesem Szenario kommt es bei der Anzeige des Warenkorbes zu einer starken Verzögerung, bis dieser sichtbar ist (vgl. \autoref{subsec:lange-verarbeitung}). Hierbei ist Splunk wenig hilfreich, da eine zeitliche Abfolge alleine durch Logs schwer nachvollziehbar ist. Durch LogRocket ist das resultierende Verhalten gut nachvollzuziehen und ein Video dessen ist im Anhang vorzufinden (siehe \autoref{sec:demo-logrocket}).

Hierbei hilft erneut Jaeger, denn durch Traces lassen sich nicht nur hierarchische Beziehungen nachvollziehen, sondern auch zeitliche. Eine Trace-Gantt-Visualisierung eines Traces, bei dem der Warenkorb abgerufen wird, ist in \autoref{fig:lange-verarbeitung_jaeger} zu betrachten. Dabei sticht hervor, dass der überwiegende Anteil der Zeit nicht in den Backendanfragen verloren geht, sondern durch die Operation \texttt{mapShoppingCart} im Frontend. Anhand dieser Informationen kann ein Entwickler nun hergehen und diese Operation verbessern.

\begin{figure}[H]
	\centering
	\includegraphics[width=1.00\linewidth]{img/05_ergebnis/lange-verarbeitung_jaeger.png}
	\caption{Auschnitt des Traces zur Warenkorbanfrage in Jaeger}
	\label{fig:lange-verarbeitung_jaeger}
\end{figure}

\section{Bewertung der Anforderungen}
\label{sec:bewertung-der-anforderungen}
% \section{Bewertung der Anforderungen}

Vor der Erstellung des Konzeptes und der eigentlichen Implementierung wurden dafür Anforderungen definiert. Diese sollen nun überprüft werden, ob und in welchem Umfang sie umgesetzt wurden. Dafür werden die Anforderungen tabellarisch und in kurzer Form dargestellt. Um den Grad der Erfüllung zu beschreiben, existiert zudem die Spalte \textit{Erfüllungsgrad}. Dabei handelt es sich um einen Prozentwert von 0\%--100\%, wobei 0\% keine Umsetzung der Anforderung bedeutet und 100\% eine vollständige. Werte, die dazwischen liegen, werden nachgehend genauer erläutert, da hierfür keine allgemeine Beschreibung gewählt werden kann.
	
% textidote: ignore begin
\begingroup
\centering
\setlength{\LTleft}{-20cm plus -1fill}
\setlength{\LTright}{\LTleft}
\begin{longtable}{|p{0.85cm}|p{6.2cm}|p{1.55cm}|p{1.75cm}|p{1.1cm}|p{1.8cm}|}
\hline
Id & Titel & Kano-Modell & Funktions\-art & Quelle & Erfüllungs\-grad \\
\endhead
\hline
\multicolumn{6}{|l|}{Funktionsumfang} \\
\hline
2110 & Schnittstellen-Logging & Basis. & f. & S & 100\% \\
\hline
2111 & Use-Case-Logging & Basis. & f. & S & 100\% \\
\hline
2120 & Übertragung von Logs & Basis. & f. & S & 100\% \\
\hline
2210 & Error-Monitoring & Basis. & f. & S & 100\% \\
\hline
2220 & Übertragung von Fehlern & Basis. & f. & S & 100\% \\
\hline
2310 & Tracing & Basis. & f. & S & 100\% \\
\hline
2311 & Tracing-Standard & Leistungs. & n. f. & A & 100\% \\
\hline
2320 & Übertragung von Tracingdaten & Basis. & f. & S & 100\% \\
\hline
2410 & Metriken & Basis. & f. & S & 100\% \\
\hline
2411 & Metrik-Standard & Begeist. & n. f. & A & 100\% \\
\hline
2420 & Übertragung von Metrikdaten & Basis. & f. & S & 100\% \\
\hline
2510 & Session-Replay & Basis. & f. & S & 100\% \\
\hline
2511 & Schalter für Session-Replay & Basis. & f. & A & 100\% \\
\hline
2520 & Übertragung von Session-Replay-Daten & Basis. & f. & S & 100\% \\
\hline
\multicolumn{6}{|l|}{Eigenschaften} \\
\hline
3010 & Resilienz der Übertragung & Begeist. & f. & S & 75\% \\
\hline
3020 & Batchverarbeitung & Begeist. & f. & S & 100\% \\
\hline
3100 & Anzahl Partnersysteme & Basis. & n. f. & K & 70\% \\
\hline
3200 & Structured Logging & Leistungs. & f. & A+S & 100\% \\
\hline
\multicolumn{6}{|l|}{Partnersysteme} \\
\hline
5100 & Partnersystem \textit{Log-Management} & Basis. & f. & A+S & 100\% \\
\hline
5110 & Manuelle Analyse \textit{Log-Management} & Basis. & f. & A+S & 100\% \\
\hline
5200 & Partnersystem \textit{Error-Monitoring} & Basis. & f. & A+S & 100\% \\
\hline
5210 & Manuelle Analyse \textit{Error-Monitoring} & Basis. & f. & A+S & 100\% \\
\hline
5220 & Visualisierung \textit{Error-Monitoring} & Leistungs. & f. & S & 100\% \\
\hline
5230 & Alerting  \textit{Error-Monitoring} & Begeist. & f. & A+S & 0\% \\
\hline
5300 & Partnersystem \textit{Tracing} & Basis. & f. & A+S & 100\% \\
\hline
5310 & Manuelle Analyse \textit{Tracing} & Basis. & f. & A+S & 100\% \\
\hline
5320 & Visualisierung \textit{Tracing} & Basis. & f. & A+S & 100\% \\
\hline
5400 & Partnersystem \textit{Metriken} & Leistungs. & f. & A+S & 100\% \\
\hline
5410 & Visualisierung \textit{Metriken} & Leistungs. & f. & A+S & 100\% \\
\hline
5420 & Alerting \textit{Metriken} & Begeist. & f. & S & 0\% \\
\hline
5500 & Partnersystem \textit{Session-Replay} & Basis. & f. & A+S & 100\% \\
\hline
5510 & Nachstellung \textit{Session-Replay} & Basis. & f. & A+S & 100\% \\
\hline
\caption{Tabellarische Bewertung der Anforderungen}
\label{tab:anforderungsbewertung}
\end{longtable}
\endgroup

% textidote: ignore end

Die \autoref{anf:3010} konnte nicht vollständig erfüllt, da nicht alle Daten, die der Nachvollziehbarkeit dienen, resilient erfasst und übertragen werden. Genauer werden Logs, Metriken und Fehler so resilient behandelt, wie in \autoref{subsec:weiterleitung-an-splunk} beschrieben, jedoch nicht Traces. Grund hierfür ist, dass durch den Einsatz von OpenTelemetry-Komponenten dies nicht oder nicht einfach möglich war. Da somit 3 von 4 Datentypen eine Resilienz aufweisen, wurde die Anforderung mit einem Erfüllungsgrad von 75\% bewertet.

Die \autoref{anf:3100}, die Anzahl der Systeme gering zu halten, konnte nicht vollständig umgesetzt werden. Grund hierfür ist, dass zwar auf ein weiteres System zum Error-Monitoring verzichtet werden konnte, aber dennoch 3 zusätzliche Partnersysteme notwendig sind. Somit wurde diese Anforderung als teilweise erfüllt bewertet.

Die Anforderungen \hyperref[anf:5230]{5230} und \hyperref[anf:5420]{5420} wurden nicht erfüllt, da kein Alerting mithilfe von Splunk umgesetzt wurde. Grund hierfür war eine Netzwerkeinschränkung des Kuber\-netes\-clusters, die das Übertragen von Daten an außenstehende Systeme nicht einfach möglich machte. Dadurch, dass es sich um zwei Begeisterungsmerkmale handelte, wurde ein Fehlen dieser \enquote{Alerting}-Funktionalität akzeptiert.

\section{Übertragbarkeit}
\label{sec:uebertragbarkeit}
% \section{Übertragbarkeit}

\textit{Wie gut lassen sich die ermittelten Ergebnisse im PoC auf andere Projekte im selben Umfeld übertragen?}

\section{Einfluss für den Nutzer}
\label{sec:einfluss-fuer-den-nutzer}
% \section{Einfluss für den Nutzer}


	
	\chapter{Abschluss}
	% \chapter{Abschluss}

\section{Zusammenfassung}

Ziel der Arbeit war es einen Ansatz zu erstellen, mit dem Betreibern von JavaScript-basierten Webanwendungen eine Nachvollziehbarkeit gewährleistet werden kann. Der Proof-of-Concept, konnte die Mehrheit der gestellten Anforderungen erfüllen sowie konnten zuvor definierte Fehlerszenarien aufgedeckt werden. Weiterhin weist die erstellte Lösung und dabei das Konzept, eine Übertragbarkeit auf andere ähnliche Softwareprojekte auf. Zudem konnte kein signifikanter negativer Einfluss für den Nutzer festgestellt werden. Somit wurde das grundlegende Ziel dieser Arbeit erreicht.

\section{Fazit}

Es konnte ein Kernproblem von Webanwendungen und speziell bei SPAs festgestellt werden, aber es konnten zudem Methoden und Technologien identifiziert werden, die dieses Kernproblem beheben. Weiterhin wurden diese Ansätze erfolgreich eingesetzt und boten einen Mehrwert für Entwickler und Betreiber, der sich nur marginal auf die Performance der Webanwendung niederschlug.

\section{Ausblick}

Wie bei der Recherche zum Stand der Technik zu sehen war, gibt es seit 2020 mit dem OpenTelemetry-Standard eine neue Entwicklung, die das Feld der Nachvollziehbarkeit bzw. der Observability in den nächsten Jahren nachhaltig beeinflussen wird. Sollte der Standard von Herstellern adaptiert werden, könnte dies zu einer höheren Auswahl an Technologien führen, die verwendet werden können, da sie miteinander kompatibel sind. Des Weiteren kann es hierdurch einfacher werden mehrere Observability-Systeme und -Konzepte miteinander zu kombinieren, um eine erhöhte Durchleuchtung zu erreichen. Diese Entwicklung ist in meinen Augen vielversprechend, besonders durch die Unterstützung von führenden Herstellern von Observability-Werkzeugen. Somit sollte dieses Feld in nächster Zeit verfolgt werden.
	
%	\chapter{Test-Kapitel}
%	\section{Zitate}

Im Literaturverzeichnis sollte zu jedem Zitat ein Eintrag mit dem entsprechenden Kürzel zu finden sein.

\subsection{Einzelner Autor}

Zur Überprüfung wird das Buch \enquote{The Hitchhiker's Guide to the Galaxy}  von Douglas Adams aus dem Jahr 1979 hinzugezogen. Die folgende Zitierung sollte mit \texttt{[Ada79]}  abgekürzt dargestellt werden. \textit{Zitat} \cite{TestCitation001}.

\subsection{Zwei Autoren}

Zur Überprüfung wird das Buch \enquote{Hard drive: Bill Gates and the making of the Microsoft empire} von James Wallace und Jim Erickson aus dem Jahr 1992 hinzugezogen. Die folgende Zitierung sollte mit \texttt{[WE92]} abgekürzt dargestellt werden. \textit{Zitat} \cite{TestCitation002}.

\subsection{Drei Autoren}

Zur Überprüfung wird der Artikel \enquote{Antioxidant activity of apple peels} von Kelly Wolfe, Xianzhong Wu und Rui Hai Liu aus dem Jahr 2003 hinzugezogen. Die folgende Zitierung sollte mit \texttt{[WWL03]} abgekürzt dargestellt werden. \textit{Zitat} \cite{TestCitation003}.

\subsection{Viele Autoren}

Zur Überprüfung wird der Artikel \enquote{Observation of top quark production in p p collisions with the Collider Detector at Fermilab} von Fumio Abe, H. Akimoto, A. Akopian, M.G. Albrow, S.R. Amendolia, D. Amidei, J. Antos, C. Anway-Wiese, S. Aota, G. Apollinari und Weitere aus dem Jahr 1995 hinzugezogen. Die folgende Zitierung sollte mit \texttt{[AAA+95]} abgekürzt dargestellt werden. \textit{Zitat} \cite{TestCitation004}.

\newpage

\section{Tabellen}

Eine Tabelle sollte immer eine Über- oder Unterschrift erhalten und mit dieser im Tabellenverzeichnis wiederzufinden sein. Weiterhin muss ein Zitat angegeben werden, wenn es sich um hinzugezogene Daten handelt.

\subsection{Simple Tabelle}

\begin{table}[H]
\centering
\begin{tabular}{|l|l|} 
\hline
Spezifische Wärmekapazität (J/(mol K)) & Temperatur (°C)  \\ 
\hline
12,2                                   & -200             \\ 
\hline
15,0                                   & -180             \\ 
\hline
17,3                                   & -160             \\ 
\hline
19,8                                   & -140             \\ 
\hline
24,8                                   & -100             \\ 
\hline
29,6                                   & -60              \\
\hline
\end{tabular}
\caption{Spezifische Wärmekapazität von Wasser \cite{TestCitation020}}
\end{table}

\subsection{Komplexere Tabelle}

\newcommand{\tabtitel}[1]{\multicolumn{2}{l|}{\textbf{#1}}}

\begin{table}[H]
\centering
\begin{tabular}{|l|l|l|l|l|} 
\hline
           & \tabtitel{Ostdeutschland} & \tabtitel{Westdeutschland}  \\ 
\hline
Geschlecht & Frauen & Männer           & Frauen & Männer                       \\ 
\hline
weiblich   & 100\%  & 0\%              & 100\%  & 0\%                          \\ 
\hline
männlich   & 0\%    & 100\%            & 0\%    & 100\%                        \\
\hline
\end{tabular}
\caption{Absurder Vergleich von Ost- und Westdeutschland}
\end{table}

\newpage

\section{Grafiken, Bilder, etc.}

Eine Tabelle sollte immer eine Über- oder Unterschrift erhalten und mit dieser im Tabellenverzeichnis wiederzufinden sein.

\subsection{Simple Grafik}

\begin{figure}[H]
	\centering
	\includegraphics[width=\linewidth]{content/xx_test/Phase_diagram_of_water_simplified.svg.png}
	\caption{Vereinfachtes Phasendiagramm von Wasser \cite{TestCitation020}}
\end{figure}

\newpage

\subsection{Innerhalb eines Textes (Floating)}

\begin{wrapfigure}[14]{r}{0.33\textwidth}
	\includegraphics[width=\linewidth]{content/xx_test/1032px-3D_model_hydrogen_bonds_in_water.svg.png}
	\caption{Verkettung der Wassermoleküle \cite{TestCitation021}}
\end{wrapfigure}

Lorem ipsum dolor sit amet, consectetur adipiscing elit. Praesent eu risus a erat auctor bibendum. Morbi eget aliquet nisl. Aenean vestibulum elit sed arcu condimentum euismod. Nunc quis ipsum sed augue maximus molestie. Pellentesque condimentum elit vitae justo tincidunt elementum. Aliquam erat volutpat. Vestibulum feugiat auctor fringilla.

Cras non arcu ante. Donec faucibus lectus risus, ac sagittis risus malesuada pretium. Cras elementum quis turpis accumsan faucibus. Vestibulum vitae volutpat nisl, et sodales nunc. Etiam egestas at magna a commodo. Mauris sagittis suscipit tempus. Duis rhoncus nec ligula eget viverra. Maecenas eu nisl orci. Proin dignissim laoreet libero in interdum. Lorem ipsum dolor sit amet, consectetur adipiscing elit.

Sed fringilla lectus non elit convallis, non eleifend sem porta. Proin aliquet urna ultrices metus blandit ornare. Duis nec ultricies ligula, quis volutpat ante. Suspendisse non lacus mauris.

\begin{wrapfigure}[14]{o}{0.45\textwidth}
	\includegraphics[width=\linewidth]{content/xx_test/Kleiner_Streichteich_Ilmenau.JPG}
	\caption{Spiegelung an der Wasseroberfläche \cite{TestCitation022}}
\end{wrapfigure}

Nullam quam diam, mattis non bibendum vel, iaculis sed leo. Phasellus condimentum auctor ante in mollis. Nullam rhoncus enim ac metus fermentum aliquam. Phasellus orci metus, tristique ac odio sed, fermentum faucibus mauris. Praesent vehicula risus aliquam nibh sodales, ac aliquam ante posuere. Praesent at semper sapien. In posuere augue vel tortor posuere rhoncus. Duis eget quam tempus diam posuere fringilla ut vitae lacus. Sed sagittis fringilla diam.

Suspendisse potenti. Ut pellentesque malesuada dolor vitae porta. Aliquam erat volutpat. Proin convallis mauris neque. Etiam et accumsan ex. Class aptent taciti sociosqu ad litora torquent per conubia nostra, per inceptos himenaeos. Suspendisse potenti. Cras ac magna enim. Praesent ultrices lacinia sem nec placerat. Sed tristique non sapien quis efficitur. Ut sit amet egestas metus. Mauris nec erat sodales, semper dolor eu, egestas justo. Ut sit amet urna ligula.

\section{Quellcode}

Der jeweilige Quellcode sollte mit Syntax-Highlighting versehen sein und im Quellcodeverzeichnis aufzufinden sein.

\subsection{Darstellung eines JavaScript Quellcodes (inline)}

\begin{lstlisting}[
  language = JavaScript,
     style = ES6,
   caption = Beispiel eines JavaScript Quellcodes (inline),
captionpos = b,
     label = lst:javascript-inline-example,
]
var fetch = require("node-fetch")

async function getCountries() {
  let res = await fetch("https://restcountries.eu/rest/v2/name/Indonesia?fullText=true")
  let json = await res.json()
  let code = json[0].alpha2Code
  let res2 = await fetch("http://country.io/phone.json")
  let json2 = await res2.json()
  console.log(json2[code])
}

getCountries()
\end{lstlisting}

\subsection{Darstellung eines JavaScript Quellcodes (importiert)}

\lstinputlisting[
  language = JavaScript,
     style = ES6,
   caption = Beispiel eines JavaScript Quellcodes (importiert),
captionpos = b,
     label = lst:javascript-import-example,
]{content/xx_test/import-example_javascript.js}

\subsection{Darstellung eines Java Quellcodes (inline)}

\begin{lstlisting}[
   caption = Beispiel eines Java Quellcodes (inline),
     label = lst:java-inline-example,
  language = java, 
     style = java-eclipse,
basicstyle = {\footnotesize\fontfamily{pcr}\selectfont}
]
	/*
	 * Main-Methode
	 */
	§§@LineAnnotation
	public class Main {
	  public static void main(%%@InlineAnnotation%% String[] args) {
	    System.out.println("Hallo Welt");
	  }
	}
\end{lstlisting}

\subsection{Darstellung eines Java Quellcodes (importiert)}

\lstinputlisting[
   caption = Beispiel eines Java Quellcodes (importiert),
     label = lst:java-import-example,
  language = java,
     style = java-eclipse,
basicstyle = {\footnotesize\fontfamily{pcr}\selectfont}
]{content/xx_test/import-example_java.js}
	
	\newpage{}
	
	\addcontentsline{toc}{chapter}{Anhang}
	\chapter{Anhang}
	\section{Studien zur Browserkompatibilität}
\label{sec:studien-zur-browser-kompatibilitaet}

Im \autoref{sec:browserprodukte} wurde die Anzahl an Studien zur Browserkompatiblität dargestellt. Dabei wurde in der Suche jeweils bei den Optionen \texttt{von} und \texttt{bis} des Erscheinungsjahres jedes Jahr einzeln eingegeben und eine Suche durchgeführt, die Trefferanzahl wurde dann notiert. Die Daten hierfür wurden über die Literatursuchmaschine \enquote{Google Scholar} am 25.02.2021 abgerufen. Für die Suche wurde folgender Suchterm benutzt:
\begin{verbatim}
"cross browser" compatibility|incompatibility|inconsistency|XBI
\end{verbatim}

\begin{wraptable}[10]{r}{0.42\linewidth}
\centering
\vspace{-\baselineskip}
\begin{tabular}{|l|l|}
  \hline
  Jahr & Suchtreffer \\
  \hline
  2015 & 272 \\
  \hline
  2016 & 228 \\
  \hline
  2017 & 208 \\
  \hline
  2018 & 204 \\
  \hline
  2019 & 172 \\
  \hline
  2020 & 167 \\
  \hline
\end{tabular}
\caption{Suchtreffer zu Studien über Browserkompatibilität}
	\label{tab:suchtreffer-metastudie}
\end{wraptable}

\def\lc{\left\lceil}   
\def\rc{\right\rceil}

Die jahresbezogene Trefferanzahl (vgl. \autoref{tab:suchtreffer-metastudie}) soll aufdecken, ob ein Trend in der Literatur zu erkennen ist. Ein schwacher, aber vorhandener Negativtrend konnte festgestellt werden.

Weiterhin lässt sich die selbe Thematik bei Google Trends \cite{GoogleTrendsCrossBrowserCompatibility} untersuchen. Hierbei wurden die Suchtrends für den Suchterm \texttt{cross browser compatability} abgerufen und geplotted (vgl. \autoref{fig:google-search-trends_cross-browser-compatability}). Dabei lässt sich ebenso ein Negativtrend erkennen.

\begin{figure}[H]
	\centering
	\includegraphics[width=0.84\linewidth]{img/99_postscript/google-trends_cross-browser-compatability.png}
	\caption{Google Trends zur Browserkompatibilität, angereichert mit \cite{MicrosoftIEandEdgeLifecycleFAQ}}
	\label{fig:google-search-trends_cross-browser-compatability}
\end{figure}

	
	\clearpage
	
	\let\cleardoublepage\relax
	
	\newpage{}
	
	\addcontentsline{toc}{chapter}{Eidesstattliche Erklärung}
	\chapter*{Eidesstattliche Erklärung}
	% siehe https://www.fh-dortmund.de/de/fb/9/personen/lehr/buechler/Buechler_2012_Leitfaden_zum_Anfertigen_wissenschaftlicher_Arbeiten.pdf?dev=ugbpucsk
Hiermit versichere ich an Eides statt, dass ich die vorliegende Arbeit selbständig und ohne die Benutzung anderer als der angegebenen Hilfsmittel angefertigt habe. Alle Stellen, die wörtlich oder sinngemäß aus veröffentlichten und nicht veröffentlichten Schriften entnommen wurden, sind als solche kenntlich gemacht. Die Arbeit ist in gleicher oder ähnlicher Form oder auszugsweise im Rahmen einer anderen Prüfung noch nicht vorgelegt worden.
		
\vspace{4cm}
		
% bases on https://www.math.tugraz.at/~grabner/Masterarbeit-template.tex
		\noindent
\begin{minipage}[h]{0.4\linewidth}
	Dortmund, am \dotfill\\
	\vspace*{2.5mm}
\end{minipage}
	\hspace*{0.1\linewidth}
	\begin{minipage}[h]{0.5\linewidth}
	\begin{center}
		\dotfill\\
		(Unterschrift)
	\end{center}
\end{minipage}
	
	\let\cleardoublepage\relax
	
	\newpage{}
	
	\addcontentsline{toc}{chapter}{Abkürzungsverzeichnis}
	
	\settowidth{\nomlabelwidth}{API}
	\printnomenclature{}
	
	\newpage{}
	
	\addcontentsline{toc}{chapter}{Abbildungsverzeichnis}
	
	\listoffigures
	
	\newpage{}
	
	\addcontentsline{toc}{chapter}{Tabellenverzeichnis}
	
	\listoftables
	
	\lstlistoflistings
	
	\newpage{}
	
	\bibliographystyle{my-alphadin}
	\bibliography{bib}
\end{document}
